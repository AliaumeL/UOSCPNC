
\documentclass[a4paper,UKenglish]{lipics-v2018}
%This is a template for producing LIPIcs articles. 
%See lipics-manual.pdf for further information.
%for A4 paper format use option "a4paper", for US-letter use option "letterpaper"
%for british hyphenation rules use option "UKenglish", for american hyphenation rules use option "USenglish"
% for section-numbered lemmas etc., use "numberwithinsect"

\usepackage{microtype}%if unwanted, comment out or use option "draft"
\usepackage{tikz}
\usetikzlibrary{arrows,positioning,decorations.pathreplacing,automata,shadows}
\usepackage{tikz-cd}
\usepackage{stmaryrd}
\usepackage{bussproofs}
\usepackage{todonotes}

%\graphicspath{{./graphics/}}%helpful if your graphic files are in another directory

\bibliographystyle{plainurl}% the recommnded bibstyle


% Syntax 
%\newcommand{\Nat}{\textnormal{Nat}}
\newcommand{\Nat}{\mathbb{N}}

\newcommand{\Succ}{\operatorname{S}}
\newcommand{\Zero}{\textnormal{Z}}
\newcommand{\Fix}{\operatorname{fix}}
\newcommand{\Unroll}{\operatorname{unroll}}
\newcommand{\Pos}{\operatorname{Pos}}
\newcommand{\In}{\operatorname{in}}
\newcommand{\Inr}{\operatorname{inl}}
\newcommand{\Inl}{\operatorname{inr}}
\newcommand{\Out}{\operatorname{out}}
\newcommand{\Ret}{\operatorname{return}}
\newcommand{\Bind}[3]{\operatorname{let} #2 \Leftarrow #1 \operatorname{in} #3}
\newcommand{\lcase}[3]{\operatorname{case} #1 \operatorname{of} \, \Zero \Rightarrow #2; \Succ(x) \Rightarrow #3}
\newcommand{\scase}[3]{\operatorname{case} #1 \operatorname{of} \, \Inl(x) \Rightarrow #2; \Inr (x) \Rightarrow #3}
\newcommand{\pcase}[2]{\operatorname{case} #1 \operatorname{of} \, (x,y) \Rightarrow #2}

% Trees

\newcommand{\Trees}{\textnormal{Trees}}
\newcommand{\Tree}{\textnormal{Trees}} %%%!!!

\newcommand{\Treeleq}{\sqsubseteq}

\newcommand{\Basicleq}{\preccurlyeq}

% Results
\newcommand{\res}{\operatorname{results}}

% Effects 
\newcommand{\prEff}{\operatorname{\textsf{pr}}}
\newcommand{\orEff}{\operatorname{\textsf{or}}}
\newcommand{\exEff}{\operatorname{\textsf{raise}}}

\newcommand{\prnd}{\text{pr/nd}}
\newcommand{\pr}{\text{pr}}
\newcommand{\nd}{\text{nd}}
\newcommand{\ang}{\text{ang}}
\newcommand{\dem}{\text{dem}}
\newcommand{\prang}{\text{pr/ang}}
\newcommand{\prdem}{\text{pr/dem}}

\newcommand{\Op}{\text{op}}
\newcommand{\Den}{\text{den}}
\newcommand{\Ax}{\text{ax}}

% Denotational

\newcommand{\Sem}[1]{\llbracket #1 \rrbracket}

\newcommand{\Set}{\mathbf{Set}}
\newcommand{\ContAlg}{\mathbf{ContAlg}}

\newcommand{\wCPO}{\mathbf{{\omega}CPO}}

\newcommand{\Conv}{\text{conv}}

% Axiomatic

\newcommand{\Vars}{\mathsf{Vars}}

\newcommand{\Iterx}[1]{(1\!-\!2^{- #1})x}
\newcommand{\Iter}[2]{(1\!-\!2^{- #2})#1}


% Reduction
\newcommand{\reduces}{\rightarrowtail}
\newcommand{\reductionPair}[2]{#1 \leadsto #2}
\newcommand{\sqeq}{\sqsubseteq}


% Mixed non-determinism
\newcommand{\drop}{\operatorname{drop}}
\newcommand{\odd}{\operatorname{odd}}
\newcommand{\even}{\operatorname{even}}

% Theorem environments
\theoremstyle{plain}
\newtheorem{proposition}[theorem]{Proposition}

\title{Basic operational preorders  for algebraic effects in general, and for
combined probability and nondeterminism in particular}

\titlerunning{Basic operational preorders for probability and nondeterminism}%optional, please use if title is longer than one line

\author{Aliaume Lopez}{\'Ecole Normale Sup\'erieure Paris-Saclay\\{Universit\'e Paris-Saclay, France}}{aliaume.lopez@ens-paris-saclay.fr}{}{}%mandatory, please use full name; only 1 author per \author macro; first two parameters are mandatory, other parameters can be empty.

\author{Alex Simpson}{Faculty of Mathematics and Physics\\{University of Ljubljana, Slovenia}}{Alex.Simpson@fmf.uni-lj.si}{}{}



\authorrunning{A. Lopez and A. Simpson}%mandatory. First: Use abbreviated first/middle names. Second (only in severe cases): Use first author plus 'et. al.'

\Copyright{Aliaume Lopez and Alex Simpson}%mandatory, please use full first names. LIPIcs license is "CC-BY";  http://creativecommons.org/licenses/by/3.0/

\subjclass{Dummy classification}% mandatory: Please choose ACM 2012 classifications from https://www.acm.org/publications/class-2012 or https://dl.acm.org/ccs/ccs_flat.cfm . E.g., cite as "General and reference $\rightarrow$ General literature" or \ccsdesc[100]{General and reference~General literature}. 

\keywords{contextual equivalence, algebraic effects, operational semantics, domain theory, nondeterminism, probabilistic choice, Markov decision process}%mandatory

\category{}%optional, e.g. invited paper

\relatedversion{}%optional, e.g. full version hosted on arXiv, HAL, or other respository/website

\supplement{}%optional, e.g. related research data, source code, ... hosted on a repository like zenodo, figshare, GitHub, ...

\funding{}%optional, to capture a funding statement, which applies to all authors. Please enter author specific funding statements as fifth argument of the \author macro.

\acknowledgements{We thank Gordon Plotkin, Matija Pretnar  and Niels Voorneveld for helpful discussions.}%optional

%Editor-only macros:: begin (do not touch as author)%%%%%%%%%%%%%%%%%%%%%%%%%%%%%%%%%%
\EventEditors{John Q. Open and Joan R. Access}
\EventNoEds{2}
\EventLongTitle{42nd Conference on Very Important Topics (CVIT 2016)}
\EventShortTitle{For review}
\EventAcronym{CVIT}
\EventYear{2016}
\EventDate{December 24--27, 2016}
\EventLocation{Little Whinging, United Kingdom}
\EventLogo{}
\SeriesVolume{42}
\ArticleNo{23}
%\nolinenumbers %uncomment to disable line numbering
%\hideLIPIcs  %uncomment to remove references to LIPIcs series (logo, DOI, ...), e.g. when preparing a pre-final version to be uploaded to arXiv or another public repository
%%%%%%%%%%%%%%%%%%%%%%%%%%%%%%%%%%%%%%%%%%%%%%%%%%%%%%

\begin{document}

\maketitle

\begin{abstract}
The ``generic operational metatheory'' of  Johann, Simpson and Voigtl\"{a}nder (LiCS 2010) defines
contextual equivalence, 
in the presence of algebraic effects, in terms of a
\emph{basic operational preorder} on ground-type effect trees. We propose three general approaches to 
specifying such preorders: (i)~operational (ii)~denotational, and (iii)~axiomatic; coinciding with the three major styles of program semantics. We illustrate these via a nontrivial case study: the combination of probabilistic choice with nondeterminism, for which we show that  natural instantiations of the three specification methods (operational in terms of Markov decision processes, denotational using  a powerdomain, and axiomatic) all determine the same canonical preorder. We do this in the case of both angelic and demonic nondeterminism. 
%I DO NOT THINK WE WILL HAVE SPACE FOR:
%We further show how basic operational preorders support a manageable theory of contextual equivalence for call-by-value languages, complementing the call-by-name case previously treated in the literature.
 \end{abstract}


\section{Introduction}
\label{section:intro}

%\todo[inline]{OBVIOUSLY INTRO NEEDS REVISING. BUT WE SHOULD WRITE THIS TOWARDS THE END}

%Contextual equivalence, in the style of Morris,
%has imposed itself as a very simple and powerful
%way to express what an equivalence on programs should be. 
%Two programs are said to be contextually equivalent if 
%they \emph{behave} equivalently under any context. 
%However this operational notion of equivalence is rarely usable 
%as-is because of the quantification over potentially complex 
%contexts, which can lead to unintuitive results \cite{pitts1997operationally}.

\emph{Contextual equivalence}, in the style of Morris,
is a powerful and general method for defining program equivalence, applicable to many 
programming languages. 
Two programs are said to be contextually equivalent if 
they `behave' equivalently when embedded in any suitable context that leads to `observable' behaviour. 
%However this operational notion of equivalence is rarely usable 
%as-is because of the quantification over potentially complex 
%contexts, which can lead to unintuitive results \cite{pitts1997operationally}.
More generally,\footnote{It is more general, since every equivalence relation is a preorder.} one can define \emph{contextual preorder} in the same manner. Let $P_1$ and $P_2$ be comparable programs (for example, in a typed language, $P_1$ and $P_2$ would  have the same type in order to be comparable). Suppose further  that we have some {basic preorder} $\Basicleq$, defined on `observable' computations, according to appropriate behavioural considerations. Then the contextual preorder is defined by
\begin{equation}
\label{equation:contextual-preorder}
P_1 \sqsubseteq_\text{ctxt} P_2 ~ \iff ~
\text{for all observation contexts $C[-]$, $~ |C[P_1]| \Basicleq |C[P_2]|$} \enspace . 
\end{equation}
This method of definition has important consequences. For example, the relation
$\sqsubseteq_\text{ctxt}$ is guaranteed to be a precongruence with respect 
to the constructors of the programming language.
However, the quantification over contexts makes the definition awkward to work with directly.
So various more manageable techniques for reasoning about contextual preorder relations have been developed, including:
%The choices of citations below all seem quite arbitrary, so I am removing them
bisimulations 
and their refinements (applicative/environmental bisimulations, 
bisimulations up-to), %\cite{koutavas2011applicative}, 
denotational interpretation in domains, % \cite{scott1982domains}, 
game semantics, %\cite{abramsky1999game}, 
program logics,
%higher order logic on  programs \cite{honda2005observationally} 
and logical relations. %\cite{Pitts2000}. [[EXPAND REFERENCES]].
These techniques are all reasonably general, in the sense that they adapt to different styles of programming languages, and combinations of programming features. Nonetheless, they are usually studied on a language-by-language basis.
%A major reason for this is that it is difficult to give mathematical definitions that apply uniformly across a range of languages and features. 

One direction for the systematisation of a range of programming features has been provided by Plotkin and Power through their work on
 \emph{algebraic effects}~\cite{plotkin2001adequacy,PlotkinPower2002}. Broadly speaking, effects are interactions between a  program and/or its environment (including the machine state), and include features such as
 error raising, global/local state, input/output, nondeterminism and probabilistic choice. 
 Plotkin and Power realised that the majority of effects (including all the aforementioned ones) are \emph{algebraic}, in the sense that the operations that trigger them %can be given in an algebraic signature, and also 
 satisfy a certain natural behavioural constraint.\footnote{In operational terms, the constraint  is that the behaviour of the operation does not depend on the content of the continuation at the time the operation is triggered.} 
 %THIS PARAGRAPH: \cite{plotkin2001adequacy} [OTHER REFS]

The algebraic effects  in a programming language can  be supplied via an algebraic signature $\Sigma$ of effect-triggering operations,
and the operational semantics of the language can then be defined parametrically in $\Sigma$. 
This is achieved by effectively splitting the semantics of 
the language into two steps. In the first step, operational rules specify how any program $P$ evaluates 
to an associated \emph{effect tree} $|P|$, 
which documents  all the effects that might potentially occur during execution. % and their interdependencies.
In an effect tree, the effects themselves are uninterpreted, in the sense that no specific execution behaviour is imposed upon them. 
As the second step, an interpretation is given to effect trees, by one means or another, from which a semantics for the whole language is extrapolated.
This methodology was first followed in \cite{plotkin2001adequacy}, where the operational reduction to effect trees (there called \emph{infinitary effect values}) is used as a method for proving the computational adequacy of denotational semantics. 
In \cite{gom}, effect trees (there called \emph{computation trees}) are used to give a uniform definition of 
contextual preorder, %for a language with algebraic effects, 
and to characterise it as a logical relation.
Effect trees also allow a general definition of applicative (bi)similarity for effects~\cite{SV2018,Voor2018} (but see \cite{Ugo2017} for another approach).

%In~\cite{Ugo2017}, effect trees implicitly underpin the definition of relator used to give a general definition of applicative (bi)similarity for effects.
%And in~\cite{SV2018,Voor2018} effect trees are used to specify program logics that characterise applicative (bi)similarity.

In this paper, as in  \cite{gom}, our aim is to exploit the notion of effect tree for the purpose of 
giving a unified theory of contextual preorders for programming languages with algebraic effects.
In~\cite{gom}, this was carried out in the context of a specific polymorphically-typed call-by-name functional language with general recursion, to which algebraic effects were added. In this paper, we build on the technical work of~\cite{gom}, but an important departure is that we detach the development from any fixed choice of background programming language. This is based on the following general considerations. In order to define contextual preorder via~(\ref{equation:contextual-preorder}) above, one needs to specify what constitutes an observation context, and also the basic behavioural relation $\Basicleq$ on the computations such contexts induce. In the case of a language with algebraic effects, we can observe two things about a computation. Firstly, we can observe any discrete return value. 
In any sufficiently expressive language, discrete values should be convertible to natural numbers. So it is a not unreasonable  restriction to restrict observation contexts to \emph{ground contexts} whose return values (if any) are natural numbers.  The process of comptation for such a context can be  modelled as an effect tree with natural-number-labelled leaves.
%We call such contexts \emph{ground contexts}.
Secondly, we can also potentially observe aspects of effectful behaviour of such  computations, with exactly what is observable very much depending on the effects in question. 
We thus specify a  \emph{basic operational preorder} 
$\Basicleq$ on the set
%$\Trees(\Nat)$ 
of effect trees with natural-number-labelled leaves, which implements a desired behavioural preorder on
effectful computations with natural number return values.
% Irrespective of how we actually interpret these effects, (\ref{equation:contextual-preorder}) requires that we specify a relation $\Basicleq$ between effectful computations with natural number return values.
 We are thus 
led to the following general formulation of contextual equivalence. Given a chosen basic operational preorder
$\Basicleq$, we define the induced contextual preorder on programs by:
\begin{equation}
\label{equation:contextual-preorder-via-trees}
P_1 \sqsubseteq_\text{ctxt} P_2 ~ \iff ~
\text{for all ground contexts $C[-]$, $~ |C[P_1]| \Basicleq |C[P_2]|$} \enspace . 
\end{equation}

In~\cite{gom}, this general approach was developed in detail for 
a polymorphically typed call-by-name functional language with algebraic effects. 
The main result was that the resulting contextual preorder, defined by (\ref{equation:contextual-preorder-via-trees}), is well behaved if the basic operational preorder satisfies two technical properties, \emph{admissibility} and \emph{compositionality}. In particular, it follows from these conditions that 
the contextual preorder is characterisable as a {logical relation} (and hence amenable to an important proof technique), and also that, on ground type programs $P_1,P_2$,
the contextual and basic operational preorders coincide (i.e., $P_1 \sqsubseteq_\text{ctxt} P_2$ if and only if
$|P_1| \Basicleq |P_2|$). 
%These results turn out not to be specific to the call-by-name language of~\cite{gom}.
Recently, we have carried out a similar programme for a call-by-value language,
similar to the language in~\cite{plotkin2001adequacy}, and obtained analogous results.%
% :  if the basic operational preorder is admissible and compositional then it is characterisable as a logical relation, 
% and coincides with the basic preorder at ground type.
\footnote{\label{footnote:unpublished}Unfortunately, there is no space to include these results, which were obtained while the first author was on an internship in Ljubljana in 2017, in this paper.}
It seems likely that similar results hold for other language variants.
% (call-by-push-value~\cite{LevyCBPV}, untyped or with additional type constructors, etc.). %are obtainable for other  %We had hoped to include it in this paper but 



The
notion of admissible and  compositional basic operational preorder thus provides a uniform and well-behaved definition of contextual preorder, for different languages with algebraic effects. Furthermore,
as is argued in~\cite[\S{V}]{gom}, it can also be given an intrinsic, more conceptually motivated, justification in terms of an explicit  notion of \emph{observation}. 
Our general position is that the notion of admissible and  compositional basic operational preorder is a fundamental one. 
\emph{For any given combination of algebraic effects, one need only define a corresponding admissible and compositional basic operational preorder.} Once this has been done,  one obtains, via (\ref{equation:contextual-preorder-via-trees}), a definition of contextual preorder that can be applied to many programming languages containing those effects, and which will enjoy good properties.

In this paper, we describe three different approaches to defining basic operational preorders. 
The first is an \emph{operational} approach. One explicitly models the execution of the effects in question, and uses this model to determine the preorder. This is the approach that was followed 
in~\cite{gom}. Under this approach, admissibility and compositionality do not hold automatically, and so need to be explicitly verified. The second is a  \emph{denotational} approach. One builds a suitable domain-based model of the relevant effect operations. This induces a basic operational preorder on  effect trees that is automatically admissible and compositional. 
The third is \emph{axiomatic}. One finds a set of (possibly infinitary) Horn-clause axioms asserting desired properties of the intended preorder. The basic operational preorder is then taken to be the smallest admissible preorder satisfying the axioms. In addition to being admissible by definition, the resulting preorder is automatically compositional. 
It will not have escaped the readers attention that our three approaches to defining preorders  parallel the three main styles of program semantics: \emph{operational}, \emph{denotational} and \emph{axiomatic}. Nonetheless, irrespective of how they are defined, we view basic operational preorders as a part of operational semantics, for their purpose is to define the operational notion of contextual preorder. 
%Thus our triptych of approaches shows that operational semantics can itself come in
% \emph{operational}, \emph{denotational} and \emph{axiomatic} flavours. 

The general identification of these three approaches is the first main contribution of the paper.
%which is a contribution of a methodological nature. 
Our second contribution is more technical. We illustrate the three approaches
with a nontrivial case study: the combination of (finitary) nondeterminism with probabilistic choice, which is a combination of effects that enjoys a certain notoriety 
for some of the technical complications it incurs~\cite{Mislove2000,mislove2004axioms,VW06,tix2009semantic,JGL15,JGL-mscs16,KeimelP2016}.
%We treat this combination of effects from each of the three viewpoints. 
On the operational side, we consider effect trees as Markov decision processes (MDPs), and we define a basic operational preorder based on the comparison of values of MDPs. On the denotational side, we make use of recently developed domain-theoretic models of combined nondeterministic and probabilistic choice~\cite{tix2009semantic,JGL-mscs16,KeimelP2016}.
On the axiomatic side, we give a simple axiomatisation, similar to axiomatisations in~\cite{mislove2004axioms,KeimelP2016}. 
Our main result is that the
operationally, denotationally and axiomatically-defined basic operational preorders all coincide with each other.
In fact, we give this result in two different versions.
The first is for an \emph{angelic} interpretation of nondeterminism, in which nondeterministic choices are resolved by a cooperative scheduler. The second is for \emph{demonic} nondeterminism, where an antagonistic scheduler is assumed. 
In each case, our coincidence theorem suggests the canonicity of the  preorder we obtain for the form of nondeterminism in question, with each of the three methods of definition providing a distinct perspective on it. 

%The paper is structured as follows. 
In Sections~\ref{section:trees} and~\ref{section:basic}, we review the 
definition of effect trees and basic operational preorders, largely following~\cite{gom}. Our main contribution starts in Sections~\ref{section:operational}, \ref{section:denotational} and~\ref{section:axiomatic}, which discuss the 
operational, denotational and axiomatic approaches to defining basic operational preorders. The discussion is illustrated using the example of combined nondeterminism and probabilistic choice. 
The main coincidence theorem, for this example, is then proved in Section~\ref{section:equivalence}. Finally, in Section~\ref{section:conclusions}, we briefly discuss related and further work.






%In this section we are going to fix a specific signature $\Sigma$
%containing two binary operators \texttt{pr} and \texttt{or}. The two
%operators are used to model a language where both probabilistic choice 
%and non-determinism coexist. Combining these specific effects has been 
%the subject of numerous papers, and even when restricting ourselves to the 
%denotational setting, the work of Regina Tix on powercones \cite{tix2009semantic} 
%continued afterwards by Plotkin and Keimel \cite{KeimelP2016} on Kegelspitze
%shows the interest of such combination.
%A more functional version of theses domains can also be found in the work of Jean-Goubault Larrecq 
%\cite{JGL-mscs16}.
%

%% This should be in conclusion ! 
%This setting is purposely restricted, and several improvements can 
%be added without technical issue. For instance, more complex types 
%such as sum, products and even type polymorphism can be studied 
%in this context. In fact, logical relations excel in proving parametricity 
%results \cite{wadler1989theorems}.
%In the spirit of simplicity 
%and to allow comparison with the work on bisimulations by
%Ugo Dal Lago, Francesco Gavazzo and Paul Blain Levy
%\cite{Ugo2017} we take the same kind of effect signature 
%as they do. Technically, it means that compared to 
%the paper from Johann et al. \cite{gom} it lacks 
%three of the four effect constructions, but as noticed 
%in the said paper, all of the constructions share the
%same pattern of proof, so that they actually treated 
%only one of the four cases in their proofs.
%


\section{Effect trees}
\label{section:trees}

The general scenario this paper addresses is that of a programming language whose programs may perform effects as they compute. In this paper, we assume that the available effects are  specified 
by  an \emph{effect signature}: a set $\Sigma$ of operation symbols, each with an associated finite arity. We call the operations in $\Sigma$ \emph{effect operations}. This setting is explicitly that of \cite{plotkin2001adequacy}.
More general effect signatures appear in the literature, e.g., allowing parameterised operations and infinite arities
\cite{gom}  [[OTHER REFERENCES]]. The technical development in this paper can be generalised to such
more general signatures. Since, however, the main running example considered in this paper has only binary operations, we restrict ourselves to finite arity operations 
for the sake of presentational convenience. %We now present this example.
\begin{example}[Signature for combined probabilistic and non-deterministic choice]
\label{example:prnd}
    Consider a programming language that can perform two effects: probabilistic and nondeterministic choice.
    An appropriate signature for such a language is 
    $\Sigma_{\prnd} = \{ (\prEff,2), (\orEff,2) \}$ containing two binary operations:
    nondeterministic choice $\orEff$, 
    and fair probabilistic choice $\prEff$. (As is well known, in programming languages with general recursion, all computable discrete probability distributions can be 
     simulated using fair probabilistic choice.)
 \end{example}

During the execution of a program with effects, three different situations can arise. Firstly, the computation process may
trigger an effect, represented by some $o \in \Sigma$. The execution will then continue along one of the $n$ possible continuation processes given as arguments to the operation $o$. Secondly, the execution may terminate, 
in which case it may produce a resulting value. 
Thirdly, the execution may continue forever without terminating and without invoking any effects. We call this last situation
 \emph{silent nontermination} to distinguish it from \emph{noisy nontermination}, which occurs
 when the computation process computes for ever while performing an infinite sequence of effects along the way.

The global behaviour of such a program is captured by the notion of an \emph{effect tree}: a  finitely branching tree, whose 
internal nodes represent effect operations, and whose leaves represent either termination with a result, or silent nontermination. The branches of the tree represent potential execution sequences of the program. 
Trees are allowed to be infinitely deep, with their infinite branches representing noisy nontermination.
Such trees were introduced as \emph{infinitary effect values} in  \cite{plotkin2001adequacy}, and used extensively in \cite{gom}, where they are called
\emph{computation trees}. Two example trees, for computations that return natural number values, are drawn in
Figure~\ref{fig:exampletrees} below. The left-hand tree $\orEff (\prEff (1,2), 3)$ represents a program that first makes a nondeterministic choice and then a potential probabilistic choice, with the choices determining the resulting number. In the second tree $\prEff (\orEff (1,3), \orEff (2,3))$, the probabilistic choice is made first, followed by the relevant nondeterministic choice.

\begin{figure}[h]
\small
    \begin{center}
        \begin{tikzpicture}
            \node [circle,draw] (z){$\orEff$}
                child { 
                    node [circle,draw] (a) {$\prEff$}
                    child { node[circle,draw] (b) {$1$} } 
                    child { node[circle,draw] (c) {$2$} }
                }
                child {
                    node [circle,draw] (d) {$3$}    
                };
        \end{tikzpicture}
        \hspace*{8ex}
        \begin{tikzpicture}[level 1/.style={sibling distance=3cm},
                            level 2/.style={sibling distance=1.5cm}]
            \node [circle,draw] (z){$\prEff$}
                child { 
                    node [circle,draw] (h) {$\orEff$}
                    child { node[circle,draw] (b) {$1$} } 
                    child { node[circle,draw] (c) {$3$} }
                }
                child {
                    node [circle,draw] (g) {$\orEff$}
                    child { node[circle,draw] (e) {$2$} } 
                    child { node[circle,draw] (f) {$3$} }
                };
        \end{tikzpicture}
    \end{center}
%  \begin{equation*}
%        \orEff (\prEff (1,2), 3) \quad \quad \prEff (\orEff (1,3), \orEff (2,3))
 %   \end{equation*}
    \caption{Two effect trees}
    \label{fig:exampletrees}
\end{figure}






\begin{definition}
The set $\Trees(X)$ of \emph{effect trees} with values from  the set $X$ is coinductively defined so that
every tree has one of the following  forms.
\begin{itemize}
\item The root of the tree is labelled with an operation $o \in \Sigma$, and the tree has the form
         $o(t_1, \dots, t_n)$ where $n$ is the arity of $o$ and $t_1, \dots, t_n \in \Trees(X)$; or
\item the tree is a leaf labelled with a value $x \in X$; or
\item the tree is a leaf labelled with $\bot$.
\end{itemize}
\end{definition}
As this is a coinductive definition, $\Trees(X)$ contains trees of both finite and infinite depth.

We define a partial order on  $\Trees(X)$ by
$t_1 \Treeleq t_2$ if and only if $t_2$ can be obtained from $t_1$ by replacing (possibly infinitely many)
$\bot$-leaves appearing in $t_1$ with arbitrary replacement trees (rooted where the leaves were located). With this ordering,  $\Trees(X)$  is an $\omega$-complete 
partial order ($\omega$CPO) with least element $\bot$. Furthermore, by considering it as
a tree constructor,
every operation $o \in \Sigma$  defines a continuous (i.e., $\omega$-continuous) function $o : \Trees(X)^n \to \Trees(X)$, where $n$ is the arity of $o$.
(For notational convenience, we use $o$ for both operation symbol and function. The ambiguity can be resolved from the context.) 

The properties described above state that $\Trees(X)$ is a continuous $\Sigma$-algebra. In general,
a \emph{continuous $\Sigma$-algebra} is a pointed (i.e., with least element) $\omega$CPO $A$ with associated
continuous functions $o_A : A^n \to A$ for every $o \in \Sigma$ of arity $n$. 
As morphisms between continuous $\Sigma$-algebras
$A$ and $B$, we consider functions $h: A \to B$  that are strict (i.e., preserve least element) continuous homomorphisms with respect to the $\Sigma$-algebra structure. 
We refer to such  functions $h: A \to B$ as \emph{continuous homomorphisms}, leaving the strictness property implicit.
We write $\ContAlg_\Sigma$ for the category of continuous $\Sigma$-algebras and continuous homomorphisms.
The characterisation of $\Trees(X)$ below is standard.
%\todo[inline]{[[REFERENCES]]}
\begin{proposition}
$\Trees(X)$ is the free
    continuous $\Sigma$-algebra over the set $X$.
   \begin{center}
        \begin{tikzcd}
            X
            \arrow[r, "f"] 
            \arrow[d, hook, "i"]
            & A \\
            \Trees(X) \arrow[ru, dashrightarrow, "\hat{f}" below]
        \end{tikzcd}
    \end{center}
    That is, for every function $f : X \to A$, where 
    $A$ is a continuous $\Sigma$-algebra,
    there exists a unique continuous homomorphism $$\hat{f} : \Trees(X) \to A$$
    such that $
        f = \hat{f} \circ i $, where $i : X \to \Trees(X)$ is the function mapping every $x \in X$ to the 
        leaf-tree labelled $x$.
 \end{proposition}
 
 We use the above proposition to define a substitution operation on trees. For any tree $t \in \Trees(X)$, 
 every function $f \colon X \to \Trees(Y)$ determines a tree $t[f]$ in $\Trees(Y)$ defined by substitution, \emph{viz}:
 $t[f] ~ := ~ \hat{f}(t)\,$.
 


\section{Basic operational preorders}
\label{section:basic}

As discussed in Section~\ref{section:intro}, out  interest in effect trees is that they provide a 
uniform template for defining 
 \emph{contextual preorders} for programming languages with algebraic effect operations
specified by signature $\Sigma$. 
As in~\cite{gom}, the crucial data is provided by a 
preorder  $\Basicleq$  on  $\Trees(\Nat)$, called the \emph{basic operational preorder}.  
In order for the resulting contextual preorder to be well behaved, we ask for the 
the basic operational preorder satisfy two properties: \emph{admissibility} and \emph{compositionality}.
In this section, we review the definitions of these and related notions. 

%We consider how to define a notion of \emph{contextual preorder} between programs in some unspecified programming language with algebraic effects from signature $\Sigma$. (Notions of \emph{contextual equivalence}
%are included as a special cases, as any equivalence relation is a preorder.) The general idea of a contextual preorder, is to relate comparable programs $P_1$ and $P_2$ (in a typed language, programs of different type would not be considered comparable), by considering how they behave in any program context $C[-]$ that produces `observable' behaviour. One thing that can certainly be observed is the return value (if any) of a computation, as long as this return value is discrete. In any sufficiently expressive language, such discrete values should be convertible to natural numbers. So it is a not unreasonable  restriction to consider contexts whose return values (if any) are natural numbers.  We call such contexts \emph{ground contexts}.
%
%We thus compare the behaviour of $P_1$ and $P_2$ by wrapping them in ground contexts $C[-]$. Since the programming language contains effects, the `observable'  behaviour of  $C[P_1]$ and $C[P_2]$ will, in general, involve not only return values, but also  the effects performed by the two computations. 
%%Moreover, exactly what counts as  `observable' and how it affects the comparison of ground type computations
%% will be highly dependent on the specific effects present. 
%Independently of the specific effects present, the execution behaviour of any ground type computation $M$ can be captured by an effect tree $|M|$ that represents all the effect operations that $M$ might potentially perform, and their dependencies. This suggests the following uniform approach to defining contextual equivalence.  As long as we have a specified \emph{basic operational preorder}  $\Basicleq$  on ground  computation trees, that is, on the set $\Trees(\Nat)$, we can define 
%contextual preorder on comparable programs $P_1,P_2$, by:
%\[ P_1 \sqsubseteq_\text{ctxt} P_2 ~ \iff ~
%\text{for all ground contexts $C[-]$, $~ |C[P_1]| \Basicleq |C[P_2]|$} \enspace . \]
%
%As in~\cite{gom}, we view the crucial data is provided by a 
%preorder  $\Basicleq$  on  $\Trees(\Nat)$, called the \emph{basic operational preorder}.  
%Any effect-specific behaviour resulting from  the particular effect operations present is captured in the relation $\Basicleq$ itself. Given this, the remaining definition of the contextual preorder $\sqsubseteq_\text{ctxt}$ is generic, relying only on a general operational semantics that transforms any computation $M$ to its effect tree $|M|$.
%
%In~\cite{gom}, this general approach was developed in detail for 
%a polymorphically typed call-by-name functional language with algebraic effects. 
%The main result was that the resulting contextual preorder is well-behaved if the basic operational preorder satisfies two properties, called \emph{admissibility} and \emph{compositionality}. Such well-behaved properties include that it the contextual preorder is definable as a {logical relation} (and hence amenable to an important proof technique), and also that, on ground type programs $P_1,P_2$,
%the contextual and basic operational preorders coincide (i.e., $P_1 \sqsubseteq_\text{ctxt} P_2$ iff 
%$|P_1| \Basicleq |P_2|$).
%
%It is our belief that the conditions of admissibility and compositionality are fundamental ones that  are applicable
%across a range of programming  languages. Indeed, we believe that it holds in broad generality that any
%admissible, compositional basic operational preorder  give rise  to a well-behaved notion of contextual preorder. In addition to the 
%call-by-name language of~\cite{gom}, we have verified this explicitly in the case of a call-by-value functional language with effects,
%similar to the language in~\cite{plotkin2001adequacy}. More generally, it seems  that the crucial language-independent property required, for this to hold, is that all effect operations are \emph{algebraic} in the sense of Plotkin and Power ~\cite{plotkin2001adequacy} [[CHECK THIS REFERENCE CONTAINS THE RIGHT USE OF ALGEBRAIC]].
%
%\todo[inline]{PERHAPS PUT MOST OF ABOVE IN INTRO}
%
%In summary, defining a \emph{basic operational preorder} on  $\Trees(\Nat)$ provides a uniform language-independent template for defining contextual preorder. Moreover, if the basic operational is \emph{admissible} and \emph{compositional} then the 
%contextual preorder is well behaved. The rest of this section reviews the technical definitions of these notions. 



\begin{definition}[Admissibility]
    A binary relation $R$ on $\Trees(X)$ is \emph{admissible} if,
    for every ascending chain $(t_i)_{i \geq 0}$ and 
    $(t_i')_{i\geq 0}$, we have:
    \[ \text{($\,t_i  R \, t_i'$ for all $i\,$)} ~ \implies~
        \left(\bigsqcup_{i \geq 0} t_i\right) \, R \, \left(\bigsqcup_{i \geq 0} t_i'\right) \enspace .
    \]
\end{definition}

\begin{definition}[Compatibility]
    A binary relation $R$ on $\Trees(X)$ is  \emph{compatible} if,
    for every $o \in \Sigma$ of arity $n$, and for all trees 
     $t_1,\dots, t_n$ and $t'_1, \dots, t'_n$, we have:
    \[ \text{($\,t_i  R \, t'_i$ for all $i = 1, \dots, n\,$)} ~ \implies ~ 
        o(t_1, \dots, t_n) \, R \; o(t'_1, \dots, t'_n) \enspace .
    \]
\end{definition}
If a compatible relation is a preorder then it is called a \emph{precongruence}. If it is an equivalence relation it is called a \emph{congruence}.


The next two definitions make use of the substitution operation on trees defined at the end of
Section~\ref{section:trees}.
\begin{definition}[Substitutivity]
    A binary relation $R$ on $\Trees(X)$ is  \emph{substitutive} if,
    for all trees $t$, $t'$ and $\{t_x\}_{x \in X}$, we have:
    \[ \text{$\,t\, R \, t'$} ~ \implies ~ 
       t[ x \mapsto t_x] \, R \, t'[ x \mapsto t_x] \enspace .
    \]
\end{definition}



\begin{definition}[Compositionality]
    A binary relation $R$ on $\Trees(X)$ is \emph{compositional} if, for all 
    trees $t$, $t'$,  $\{t_x\}_{x \in X}$,  and $\{t'_x\}_{x \in X}$, we have:
        \[ \text{($\,t \, R \, t'$ and $t_x \, R \, t'_x$ for all $x \in X\,$)} ~ \implies ~ 
        t[ x \mapsto t_x] \, R \, t'[ x \mapsto t'_x] \enspace .
    \]
\end{definition}



\begin{proposition} 
\label{proposition:substitutive}
Let $\Basicleq$ be a preorder  on $\Trees(\mathbb{N})$.
\begin{enumerate} 
\item If  $\Basicleq$ is compositional then it is a substitutive precongruence.
\item If $\Basicleq$ is an admissible substitutive precongruence then it is compositional.
\end{enumerate}
\end{proposition}

%\noindent
%\todo[inline]{Only give the ideas of the proof to gain space}
\begin{proof}
    \begin{enumerate}
        \item 
             Suppose $\Basicleq$ is compositional. 
            %It is substitutive because if $t \Basicleq t'$ and $\{ t_n \}_{n
            %\in \mathbb{N}}$ is a family of trees then, since 
            %$t_n \Basicleq t_n$,  we have 
            $t[ n \mapsto t_n] \Basicleq t'[n \mapsto t_n]$ by compositionality.
            For compatibility, if $o$ is an operation of arity $n$, and $t_i \Basicleq t'_i$ for $i = 1 \dots n$, then since $o(1,\dots, n) \Basicleq o(1,\dots, n)$,
             we indeed have $o(t_1, \dots, t_n) \, \Basicleq \; o(t'_1, \dots, t'_n)$, by compositionality.
             Substitutivity is by a similar argument, also using reflexivity.

        \item 
            Suppose $\Basicleq$ is admissible, substitutive and compatible. 
            Suppose also that $t \Basicleq t'$ and $t_n \Basicleq t_n'$, for all $n \in \mathbb{N}$.
            By substitutivity, we have $t[n \mapsto t_n] \Basicleq t'[n \mapsto t_n]$.
             We would like to use compatibility to derive 
            that also $t'[n \mapsto t_n] \Basicleq t'[n \mapsto
            t_n']$, however this is only possible if $t'$ is finite. 
            The solution is to use finite approximations $(s_i')$ of $t'$
            satisfying $\sqcup_i s_i' = t'$. For each finite tree $s_i'$
            we have that $s_i'[n \mapsto t_n] \Basicleq s_i'[n \mapsto t_n']$, by compatibility.
            Hence, by admissibility, $t'[n \mapsto t_n] \Basicleq t'[n \mapsto
            t_n']$, whence $t[n \mapsto t_n] \Basicleq t'[n \mapsto t_n']$ by transitivity.
        \end{enumerate}
\end{proof}


%
%\todo[inline]{[[MOTIVATE]]}
%Following~\cite{gom}, we consider the basic ingredient for specifying  a notion of
%contextual equivalence for a programming language to be a preorder $\Basicleq$ on $\Trees(\mathbb{N})$.
%As long as the preorder is both admissible and compositional (equivalently an
%admissible substitutive precongruence), the mathematical tools of~\cite{gom} and
%Sections \textbf{[[WHICH]]} of the present paper are applicable. These allow fundamental properties of contextual equivalence to be proved.
%
%We observe that admissible compositional preorders are closed under arbitrary intersection. That is, if
%$\mathcal{R} \subseteq \mathcal{P}(\mathbb{N} \times \mathbb{N})$ is a family of admissible compositional preorders
%then so is $\bigcap \mathcal{R}$.
%
%Every family $\mathcal{O} \subseteq \mathcal{P}(\Trees(\mathbb{N}))$ determines a preorder $\Basicleq_\mathcal{O}$ on 
%$\Trees(\mathbb{N})$ by
%\begin{equation}
%\label{equation:observational-preorder}
%t \Basicleq_\mathcal{O} t' ~ \Leftrightarrow ~ \forall U \in \mathcal{O}~ (t \in U ~ \Rightarrow ~ t' \in U) \enspace .
%\end{equation}
%
%\begin{proposition} 
%The following are equivalent for
%any admissible  preorder  $\Basicleq$ on $\Trees(\mathbb{N})$.
%\begin{enumerate} 
%\item $\bot \Basicleq t$,  for every $t \in \Trees(\mathbb{N})$.
%\item $t \Treeleq t'$ implies $t \Basicleq t'$ for all $t,t' \in \Trees(\mathbb{N})$.
%\end{enumerate}
%For an arbitrary preorder $\Basicleq$, the following are equivalent.
%\begin{enumerate}
%\setcounter{enumi}{2}
%\item $\Basicleq$ is admissible and satisfies 1 (equivalently 2) above.
%\item $\Basicleq$ arises as $\Basicleq_\mathcal{O}$ for some family $\mathcal{O}$ of Scott-open subsets
%of $\Trees(\mathbb{N})$.
%\end{enumerate}
%\end{proposition}
%
%It is possible to characterise the compositionality property for relations of the form $\Basicleq_\mathcal{O}$, using a notion of \emph{decomposability} of $\mathcal{O}$, see~\cite{gom}. 
%\todo[inline]{[[IT WOULD POSSIBLY BE NICE TO REPEAT THIS AND REFINE IT, E.G., TO
%REFLECT THE SUBSTITUTIVE PRECONGRUENCE DEFINITION OF COMPOSITIONALITY.]]}
%




\section{Operationally-defined preorders}

In this section, we consider our first approach to defining an admissible and compositional basic operational
preorder $\Basicleq$ on $\Trees(\mathbb{N})$. We call this method \emph{operational}. Its characteristic is that the preorder
 $\Basicleq$ is directly defined using a mathematical model of the way that an effect tree in $\Trees(\mathbb{\Nat})$ will be executed.

This approach is illustrated for several examples of effects in~\cite{gom}. 
The main goal of the section is to demonstrate the approach using a different example, the signature
$\Sigma_\prnd = \{\prEff,\orEff\}$ from Example~\ref{example:prnd}, which is of interest because of the 
interplay between probabilistic and nondeterministic effects. 

\todo[inline]{I THINK FOR SPACE REASONS WE WILL NEED TO JUMP IN STRAIGHT AWAY
    WITH THE MIXED CASE. SO THE FOLLOWING DISCUSSION NEEDS TO BE CUT OUT WITH
SOME REQUIRED PARTS MOVED TO THE MIXED CASE}

However, we first
briefly review the simpler cases, $\Sigma_\pr = \{\prEff\}$ and $\Sigma_\nd = \{\orEff\}$, of purely probabilistic and 
purely nondeterministic computation. These two cases are already covered in~\cite{gom}, so our treatment will be brief. 

First, we  consider the case $\Sigma_\pr = \{\prEff\}$. Every internal node in a  tree $t \in \Trees(\mathbb{N})$  is a binary branching node labelled with $\prEff$ representing a probabilistic choice with each branch having probability $\frac{1}{2}$. The tree can thus be interpreted as a (countable state) Markov-chain, that determines a discrete subprobability distribution over the set $\mathbb{N}$.
(It is a subprobability distribution because there can be a positive probability of nontermination, both silent and noisy.) We write
$\mathbf{P}_t (H)$ for the probability assigned to a set $H \subseteq \mathbb{N}$
by the tree $t$, and $\mathbf{E}_t (h)$ for the (possibly infinite) expectation of a function $h\colon \mathbb{N} \to [0,\infty]$ under the subprobability distribution on $\mathbb{N}$ determined by $t$.
The basic operational preorder can be defined using any of the three equivalent properties below.
\begin{align*}
t \Basicleq_\pr t' ~~ \Leftrightarrow ~~ &  \forall n ~ ~ \mathbf{P}_t (\{n\}) \leq \mathbf{P}_{t'}(\{n\})  \\
 ~ \Leftrightarrow ~~ & \forall H \subseteq \mathbb{N} ~~ \mathbf{P}_t (H) \leq \mathbf{P}_{t'}(H) \\
 ~ \Leftrightarrow ~~ & \forall h\colon \mathbb{N} \to [0,\infty] ~~ \mathbf{E}_t (h) \leq \mathbf{E}_{t'}(h) \enspace .
\end{align*}
Here one can view $H$ as a test set of desirable results, and $h$ as a function that assigns (possibly infinite) desirability weights 
to result values in $\mathbb{N}$. The first alternative above has the advantage of simplicity. The second strengthens the definition by testing the probability of arbitrary properties. The third further strengthens it by using quantitative properties.
\begin{proposition}[\cite{gom}]
The preorder $\Basicleq_\pr$ is admissible and compositional. 
\end{proposition}
%\noindent
%It is also possible to define $\Basicleq_\pr$ cas $\Basicleq_{\mathcal{O}_\pr}$ for the a family of Scott-open subsets of $\Trees(\mathbb{N})$, \emph{viz}:
%\[
%\mathcal{O}_\pr ~ = ~ \{\,  \{t \mid \mathbf{P}_t (n) > q\} \mid n \in \mathbb{N}, \,q \in \mathbb{Q}\, \}  \enspace .
%\]

Next, we consider the case $\Sigma_\nd = \{\orEff\}\,$, in which every internal node in a tree is a binary nondeterministic choice.
Compared with~\cite{gom}, we give a more directly operational treatment in terms of 
 \emph{schedulers}, namely external agents that resolve nondeterministic choices.
We model the scheduler as a 
function $s: \{l,r\}^* \to \{l,r\}$. The idea is that a word $w \in \{l,r\}^*$ represents a finite path of left/right choices from the root of a 
tree $t \in \Trees(\mathbb{N})$. If the computation reaches a nondeterministic choice at the node indexed by 
$w$ then it takes the left/right branch according to the value of $s(w)$. This way of representing choices has some redundancy
(in every tree other than the complete infinite binary tree, there will be words $w$ that do not index nodes in the tree; if $s(\varepsilon) = l$ then the value of $s$ on words beginning with $r$ is immaterial; etc.), but it is simple and convenient for future purposes. 
Given such a scheduling function $s$, we write $t@s$ for the result of the computation as scheduled by $s$. This is defined by:
\[
t@s ~ = ~ \begin{cases} 
 n & \text{if there exists $w \in \{l,r\}^*$ indexing an $n$ node in $t$} \\
    & ~~~\text{such that, for every $i < |w|$, $~w_{i+1} = s(w\!\restriction_i)\,$;} \\
  \bot & \text{otherwise.}
 \end{cases}
\]
Here we write $|w|$ for the length of a word, $w_i$ for the $i$-th symbol in a word, and $w \!\restriction_i$ for the prefix of $w$ that has length $i$.

The  \emph{angelic} interpretation of nondeterminism takes into account the possibility of a nondeterministic computation achieving a specified goal, given a cooperative scheduler.  The  \emph{demonic} interpretation, 
models the {necessity} that a goal will be achieved, however adversarial the scheduler. 
For angelic nondeterminism, one again has three possible definitions of the basic operational preorder
$\Basicleq_\ang$, paralleling the three alternatives above. In the definitions below, $s$ ranges over schedulers, the statement $t@s \in H$, where $H \subseteq \mathbb{N}$, implies in particular that $t@s \neq\bot$, and we write $h_\bot$ for the unique strict function from $\mathbb{N}_\bot$ to
$[0,\infty]$ extending $h$.
\begin{align*}
t \Basicleq_\ang t' ~ \Leftrightarrow ~ ~& \forall n ~~ (\exists s ~\; t@s =n)~ \Rightarrow~ (\exists s ~ \; t'@s =n) 
\\
~ \Leftrightarrow ~ ~& \forall H \subseteq \mathbb{N}~~ (\exists s ~\; t@s \in H)~ \Rightarrow~ (\exists s ~ \; t'@s \in H) 
\\
~ \Leftrightarrow ~ ~& \forall h\colon \mathbb{N}\to [0,\infty]~~ \sup_s h_\bot(t@s)~ \leq ~ \sup_s h_\bot(t'@s)
\end{align*}
For demonic nondeterminism, the first pattern of definition is not available, because, although  the relation defined by the formula below is admissible, it is  \emph{not} compositional.
\[
\forall n ~~ (\forall s ~\; t@s =n)~ \Rightarrow~ (\forall s ~ \; t'@s =n) \enspace .
\]
To define the basic preorder $\Basicleq_\dem$ for demonic nondeterminism one can use either of the other two patterns of definition.
\begin{align*}
t \Basicleq_\dem t' ~ \Leftrightarrow ~ ~& \forall H \subseteq \mathbb{N}~~ (\forall s ~ \; t@s \in H)~ \Rightarrow~ (\forall s ~ \; t'@s \in H) 
\\
~ \Leftrightarrow ~ ~& \forall h\colon \mathbb{N}\to [0,\infty]~~ \inf_s  h_\bot(t@s)~ \leq ~ \inf_s h_\bot(t'@s)
\end{align*}
\begin{proposition}[\cite{gom}]
The preorders $\Basicleq_\ang$  and $\Basicleq_\dem$ are admissible and compositional. 
\end{proposition}
\noindent
Because admissible compositional preorders are closed under intersection, 
the relation ${\Basicleq_\ang} \cap {\Basicleq_\dem}$ is an admissible compositional preorder too. This gives a basic operational preorder corresponding to a neutral view of nondeterminism that takes account of both possibility and necessity.


We now turn to the main example of interest in this paper, the signature $\Sigma_\prnd = \{\prEff,\orEff\}$.
In this case, trees in $\Trees(\mathbb{N})$ have both probabilistic and nondeterministic branching nodes.
It is thus natural to consider them as (countable state) Markov Decision Processes. Once again the 
scheduler resolving nondeterministic
choices can be modelled by a 
function $s: \{l,r\}^* \to \{l,r\}$. For any given $t \in \Trees(\mathbb{N})$, such a function
$s: \{l,r\}^* \to \{l,r\}$ models a (deterministic) \emph{strategy} for the scheduler, in which the choice of direction at a nondeterministic node  
can take into account the outcomes of probabilistic nodes higher up the tree.
% (since the position of a node in the tree determines the entire sequence of choices that need to be made for that node to be reached during the process of execution).
(There is now further degree of redundancy in defining strategies as functions $s: \{l,r\}^* \to \{l,r\}$, because the value of $s(w)$ on words $w$ that index 
probabilistic nodes in $t$ is irrelevant.) A strategy $s$ and a tree $t$ in combination determine a subtree $t\#s$, defined by 
removing, at every nondeterministic node in $t$ with index $w$, the child tree that is not selected by $s(w)$. So $t\#s$ is a tree that has binary branching at probabilistic nodes, and unary branching at nondeterministic nodes. It is thus, in effect, a purely probabilistic tree, and so defines a Markov chain determining a subprobability distribution over $\mathbb{N}$. 

It is now natural to interpret the combination of probability with angelic nondeterminism 
by giving the scheduler the aim of maximising the probability of landing in any specified goal set. Similarly, in the demonic case, the scheduler should minimise the probability. However, as the proposition below shows, 
this does not work in the form stated. 
\begin{proposition}
Neither of the preorders below between $t,t' \in \Trees(\mathbb{N})$ is compositional.
\begin{align*}
& \forall H \subseteq \mathbb{N}  ~~ \sup_s  \mathbf{P}_{t\#s} (H)~ \leq~ \sup_s \mathbf{P}_{t'\#s} (H)
\\
& \forall H \subseteq \mathbb{N}  ~~ \inf_s  \mathbf{P}_{t\#s} (H)~ \leq~ \inf_s \mathbf{P}_{t'\#s} (H)
\end{align*}
\end{proposition}

\begin{proof}
    We use the two trees presented figure~\ref{fig:exampletrees},
    representing the expressions $A = 3 \orEff (1 \prEff 2)$
    and $B = (3 \orEff 1) \prEff (3 \orEff 2)$.

    To prove that both expressions behave the same on every subset of $\mathbb{N}$,
    it suffices to check it on every subsets of $\{ 1, 2, 3\}$, and it is clear
    with both preorders.

    However, we build a family $\{ t_1, t_2, t_3\}$ such that 
    $A[ i \mapsto t_i] = t_3 \orEff (t_1 \prEff t_3) = C$ is not equivalent to 
    $B[ i\mapsto t_i] = (t_3 \orEff t_1) \prEff (t_3 \orEff t_2) = D$,
    which contradicts the compositionality.

    Let $t_1 = 0 \prEff (0 \prEff (0 \prEff (0 \prEff 1)))$,
    $t_2 = 1$ et $t_3 = 0 \prEff (0 \prEff (0 \prEff 1))$. The distinguishing 
    factor will be the probability associated with the subset $\{ 1 \}$.
    We dress a table of the minimal/maximal said probability on the different 
    trees.
    
    \todo[inline]{Remove the table and only give the final probabilities in the
    four cases}
    \begin{equation*}
        \begin{array}{c|c|c}
                & \textbf{Min} & \textbf{Max} \\ \hline
            t_1 & 1/8 & - \\
            t_2 & 1   & -   \\
            t_3 & 1/4 & - \\
            t_1 \prEff t_2 & 9 / 16 & - \\
            t_3 \orEff t_1 & 1/8    & 1/4 \\
            t_3 \orEff t_2 & 1/4    & 1   \\
            t_3 \orEff (t_1 \prEff t_3) & 1/4 & 9/16 \\
            (t_3 \orEff t_1) \prEff (t_3 \orEff t_2) & 3 / 16 & 5/8 \\
        \end{array}
    \end{equation*}

    In both preorders, we have $\neg (C = D)$,
    giving the expected contradiction.
\end{proof}

The situation is rescued by generalising the test properties $H$ to 
quantitative properties $h \colon \mathbb{N} \to [0,\infty]$. 
Accordingly, we define:
\begin{align*}
t \Basicleq^\Op_\prang t' ~ \Leftrightarrow ~ ~& \forall h \colon \mathbb{N} \to [0,\infty]~~ \sup_s  \mathbf{E}_{t\#s} (h)~ \leq~ \sup_s \mathbf{E}_{t'\#s} (h)
\\
t \Basicleq^\Op_\prdem t' ~ \Leftrightarrow ~ ~& \forall g \colon \mathbb{N} \to [0,\infty]~~ \inf_s  \mathbf{E}_{t\#s} (h)~ \leq~ \inf_s \mathbf{E}_{t'\#s} (h)
\end{align*}
\begin{proposition}
The preorders $\Basicleq^\Op_\prang$ and $\Basicleq^\Op_\prdem$ are admissible and compositional.
\end{proposition}


\begin{proof}
    \begin{description}
        \item[Admissibility]
            We prove that the function $F_h : t \mapsto \inf_s  \mathbf{E}_{t\#s} (h)$
            is scott-continuous from $\Tree(\Nat)$ to $[0,+\infty]$ with the
            scott-topology. Therefore, if $t_i \Basicleq^\Op_\prang t_i'$,
            then $F_h(t_i) \leq F_h (t_i')$, and using the scott-continuity
            of $F_h$ we deduce $F_h (\sqcup_i t_i) \leq F_h (\sqcup_i t_i')$
            leading to $\sqcup_i t_i \Basicleq^\Op_\prang \sqcup_i t_i'$.

            We use the natural topology on the set $S$ of strategies, where 
            $S$ is compact. The functions $(s,t) \mapsto t \# s$  
            and $t \mapsto \mathbf{E}_t (h)$ are clearly continuous.
            We write $G_h : (s,t) \mapsto \mathbf{E}_{t \# s} (h)$ their composition.
            \todo[inline]{It is clear if we know that a function continuous in
                both its argument having values in a continuous domain is 
            "globally" continuous} 

            It suffices to prove that $t \mapsto \inf_s G(s,t)$ is
            scott-continuous. This is a consequence of the compactness of $S$ 
            \cite{AndreaShalk}
            (Theorem 7.31). 

            \todo[inline]{Choose one but keeping both is weird}
            \textbf{Alternate Proof}

            It is clear that if $(t_i)$ is an ascending chain of trees
            then 
            $\inf_s \sup_i G(s,t_i) \geq \sup_i \inf_s G(s,t_i)$,
            however, $\inf_s G(s,t_i) = G(s_i, t_i)$ because $S$ is compact.
            Now, using the fact that $S$ is metric and compact
            \textbf{(It would be best to specify this beforehand)}, we can extract 
            a subsequence such that $s_i \rightarrow s_\infty$ in $S$. Then
            it is clear that the following inequalities hold:

            \begin{equation*}
                \inf_s \sup_i G(s,t_i)
                \geq 
                \sup_i \inf_s G(s,t_i)
                \geq 
                \sup_i G(s_i, t_i)
                \geq 
                G(s_\infty, \sup_i t_i)
                \geq
                \inf_s \sup_i G(s,t_i)
            \end{equation*}

            Therefore $t \mapsto \inf_s G(s,t)$ is scott-continuous.
        \item[Compositionality]

            The proof relies on the following key lemma. 
            If $\sigma$ is a substitution on $\Tree(\Nat)$,
            $t$ is a tree, $s$ a strategy and $h$ a weighting function then
            the following equality holds:

            \todo[inline]{I am using the notation $\sigma$ for a substitution 
                instead of giving a family $\{ t_x \}_{x \in X}$ which is 
                \emph{way more convenient} in this proof but not consistent 
                with the rest of the paper}

            \begin{equation*}
                \inf_s \mathbf{E}_{t\sigma \# s } (h)
                = 
                \inf_s \mathbf{E}_{t \# s} (h_\sigma)
            \end{equation*}

            Where $h_\sigma (n) = \inf_s \mathbf{E}_{\sigma(n) \# s} (h)$.

            \todo[inline]{The proof of this lemma is long and not interesting, 
            should we mention it ?}
            
            Using this key lemma the proof can go smoothly because (by
            definition)
            if $\sigma \Basicleq^\Op_\prang \sigma'$ then 
            $h_\sigma \leq h_\sigma'$.

            \begin{align*}
                \inf_s \mathbf{E}_{t\sigma \# s} (h)
                 & = \inf_s \mathbf{E}_{t \# s} (h_\sigma)       & \text{ key lemma }  \\
                 & \leq \inf_s \mathbf{E}_{t \# s} (h_{\sigma'}) & \text{ monotonicity wrt the weighting function } \\
                 & \leq \inf_s \mathbf{E}_{t'\# s} (h_{\sigma'}) & \text{ monotonicity wrt the tree } \\
                 & = \inf_s \mathbf{E}_{t'\sigma' \# s} (h )     & \text{ key lemma } \\
            \end{align*}

            Therefore $ t \Basicleq^\Op_\prang t'$ and $\sigma
            \Basicleq^\Op_\prang \sigma'$ implies $t\sigma \Basicleq^\Op_\prang
            t'\sigma'$.
    \end{description}

\end{proof}

We view the fact that the use of quantitative properties is required to obtain a compositional preorder as being
a mathematical analogue of the situation found in the work of Kozen
\textbf{[[REF]]} and McIver and Morgan \textbf{[[REF]]}, where 
logics of quantitative properties are seen to be necessary to obtain compositional reasoning principles for 
probabilistic program logics.


\section{Denotationally-defined preorders}
\label{section:denotational}

Our second approach to defining an admissible and compositional basic operational
preorder $\Basicleq$ on $\Trees(\mathbb{N})$ is to make use of established constructions from domain theory.
Under this approach, admissibility and compositionality of the defined preorder $\Basicleq$ hold automatically,
for general reasons. Since this approach essentially amounts to giving a denotational semantics to effect trees, we call it the \emph{denotational} method of defining a basic operational preorder.



In order to define a basic operational preorder using the denotational method, one needs to merely to provide
a continuous $\Sigma$-algebra $D$ (see Section~\ref{section:trees}),  together with a function
% with a distinguished function 
$j\colon \mathbb{N} \to D$. 
Then let   $\llbracket \cdot \rrbracket \colon \Trees(\mathbb{N}) \to D$ be the unique continuous homomorphism that makes the diagram below commute.
   \begin{center}
        \begin{tikzcd}
            \mathbb{N}
            \arrow[r, "j"] 
            \arrow[d, hook, "i"]
            & D \\
            \Trees(\mathbb{N}) \arrow[ru, dashrightarrow, "\llbracket \cdot \rrbracket" below]
        \end{tikzcd}
    \end{center}
\noindent
The map $\llbracket \cdot \rrbracket \colon \Trees(\mathbb{N}) \to D$ is used to induce
the basic operational preorder $\Basicleq_D$ from the partial order relation on the $\omega$CPO $D$.
\[
t \Basicleq_D t' ~~ \Leftrightarrow ~~ \Sem{t} \sqsubseteq \Sem{t' } \enspace .
\]
\begin{proposition}
The relation $\Basicleq_D$ is admissible pregongruence.
\end{proposition}
%
The proof is immediate: admissibility follows from the continuity of 
$\llbracket \cdot \rrbracket$, and compatibility because  $\llbracket \cdot \rrbracket$ is a homomorphism.

In order to obtain substitutivity, hence compositionality, a further property is required.

\begin{definition}[Factorisation property]
    The map $j\colon \mathbb{N} \to D$ is said to have  the \emph{factorisation property} if,
    for every function $f \colon \mathbb{N} \to D$, there exists a 
    continuous homomorphism $h_{\!f} : D \to D$ such that $f = h_{\!f} \circ j$.
    \begin{center}
        \begin{tikzcd}
            \mathbb{N} \arrow[r, "j"] 
                 \arrow[rr, bend right, "f"] &
            D \arrow[r, "h_{\!f}", dashed] & 
            D  
        \end{tikzcd}
    \end{center}
\end{definition}
\begin{proposition}
If $j\colon \mathbb{N} \to D$ has the factorisation property then 
the relation $\Basicleq_D$ is substitutive, hence it is an admissible compositional precongruence.
\end{proposition}

\begin{proof}
Suppose $\sigma: \Nat \to \Tree(\Nat)$ is any  substitution.
Let $\hat{\sigma} : \Tree(\Nat) \to \Tree(\Nat)$ be the continuous homomorphism
such that $\hat{\sigma} \circ i = \sigma$. Consider the map $g := \llbracket \cdot \rrbracket \circ \hat{\sigma} \circ i : \Nat \to D$. By the factorisation property, there exists $h_g : D \to D$ such that
$g = h_g \circ j$. Expanding this, and using the definition of $\Sem{\cdot}$, we have:
\[
 \llbracket \cdot \rrbracket \circ \hat{\sigma} \circ i ~ = ~ h_g \circ j ~ = ~  h_g \circ  \llbracket \cdot \rrbracket \circ i \enspace .
\]
It then follows from  the uniqueness property of Proposition~\ref{proposition:free} that
\begin{equation}
\label{equation:before-pizza}
\llbracket \cdot \rrbracket \circ \hat{\sigma} ~ = ~ h_g \circ  \llbracket \cdot \rrbracket \enspace ,
\end{equation}
because both maps are continuous homomorphisms.

Now, for substitutivity, suppose  that $t \Basicleq_D t'$, i.e., $\Sem{t} \leq \Sem{t'}$. Then 
$h_g (\Sem{t})  \leq h_g(\Sem{t'})$ by monotonicity. That is
$\Sem{ \hat{\sigma}(t)} \leq \Sem{ \hat{\sigma}(t')}$, by~(\ref{equation:before-pizza}). 
This says that $\Sem{ t[\sigma]} \leq \Sem{t'[\sigma]}$. That is
$t[\sigma] \Basicleq_D t'[\sigma]$, as required.
\end{proof}

In practice, it is usually not necessary to prove the factorisation property directly. Instead  it holds as a consequence of the continuous algebra $D$ and map $j: \Nat \to D$ being derived from a suitable monad. The next result establishes general conditions under which this holds.
\begin{proposition}
\label{proposition:monad}
Let $\mathbf{S}$ be a category with a faithful functor $U : \mathbf{S} \to \Set$. Suppose also that 
$\mathbf{S}$ has an object $N$ such that $UN = \mathbb{N}$, and every hom set $\mathbf{S}(N,X)$
is mapped bijectively by $U$ to $\Set(\Nat,UX)$. Suppose also that $(T,\eta,\mu)$ is a monad on $\mathbf{S}$
with the following properties: there is a continuous $\Sigma$-algebra structure on $UTN$; and, for 
every map $f \colon N \to TN$, the induced function $Uf^*$, where 
$f^* \colon TN \to TN$ is the Kleisli lifting, is a continuous  homomorphism.
Then defining $D$ to be the continuous $\Sigma$-algebra on $U T N$, and
$j$ to be $U\eta \colon \mathbb{N} \to UTN$, it follows that $j$ has the factorisation property.
\end{proposition}

\begin{proof}
Consider any function $f \colon \mathbb{N} \to UTN$. This is the $U$ image of a unique morphism 
$g : N \to TN$ in $\mathbf{S}$. Let $g^*: TN \to TN$ be the Kleisli lifting of $g$, which satisfies $g^* \circ \eta = g$. Defining $h_f = Ug^*$, we indeed have  that $h_f$ is a continuous algebra homomorphism, and
$h_f \circ j = f$.
\end{proof}


\noindent
Although the statement of the proposition is verbose, the result is relatively easy to apply in practice, as the examples we consider next will show.

%
%\begin{lemma}
%Let $\mathcal{A}$ be any category given together with a functor $U' \colon \mathcal{A} \to \ContAlg_\Sigma$.
%Suppose it holds that that the
%composite functor $UU' : \mathcal{A} \to \Set$ has a left adjoint $F$, where $U: \ContAlg_\Sigma \to \Set$ is the forgetful functor.
%Then, defining $D$ to be the continuous $\Sigma$-algebra $U' F \, \mathbb{N}$, the unit of the adjunction defines a function
%$j \colon \mathbb{N} \to D$ which has the factorisation property.
%\end{lemma}
%
%\begin{lemma}
%Let $T$ be a monad on the category $\wCPO$.  Suppose that for every $\omega$CPO $X$, we have a 
%continuous $\Sigma$-algebra structure on TX. Suppose also that all Kleisli liftings of 
%$f^* \colon TX \to TY$, of maps $f \colon X \to TY$, are  continuous  homomorphisms. Then,
%defining $D$ to be the continuous $\Sigma$-algebra $T \mathbb{N}$, the unit of the adjunction defines a function
%$j \colon \mathbb{N} \to D$ which has the factorisation property.
%\end{lemma}

In the remainder of the section, we return to our main example, and again define basic operational preorders for the combination of probabilistic choice and nondeterminism (both angelic and demonic), but this time we use the denotational method. Accordingly, we call the defined preorders
$\Basicleq^\Den_\prang$ and $\Basicleq^\Den_\prdem$

We use the powerdomains combining probabilistic choice and nondeterminism
defined in~\cite[\S3.4]{KeimelP2016}, although our setting is simpler because we only need to apply them to sets.
Let $\mathcal{V}_{\leq 1} \,X$ be the $\omega$CPO of (discrete) subprobability distributions on a set $X$.
We write $\mathcal{H}\mathcal{V}_{\leq 1} \,X$ for the $\omega$CPO of nonempty Scott-closed convex subsets
of  $\mathcal{V}_{\leq 1} \,X$  ordered by subset inclusion $\subseteq$. 
We write $\mathcal{S}\mathcal{V}_{\leq 1} \,X$ for the $\omega$CPO of nonempty Scott-compact convex upper-closed subsets
of  $\mathcal{V}_{\leq 1} \,X$  ordered by reverse inclusion $\supseteq$.
The $\omega$CPOs $\mathcal{H}\mathcal{V}_{\leq 1} \,X$ and $\mathcal{S}\mathcal{V}_{\leq 1} \,X$ are both continuous algebras for $\Sigma_\prnd$. In both cases, the operations are defined by:
\begin{align*}
\orEff(A,B) ~ = ~ & \Conv(A \cup B)  
% & & \text{(definition for $\mathcal{H}\mathcal{V}_{\leq 1} \,X$)}
% \\ \orEff(A,B) ~ = ~ & & & \text{(definition for $\mathcal{S}\mathcal{V}_{\leq 1} \,X$)}
& 
\prEff(A,B) ~ = ~ & \{\frac{1}{2}a + \frac{1}{2}b \mid a \in A, b \in B\} \enspace ,
% & & \text{(definition for both cases)}
\end{align*}
where $\Conv$ is the convex closure operation. 
We remark that these straightforward uniform definitions are possible because of the simple structure of the 
domains  $\mathcal{H}\mathcal{V}_{\leq 1} \,X$ and $\mathcal{S}\mathcal{V}_{\leq 1} \,X$, over a set $X$. For the more general 
lower and upper `Kegelspitze' considered in \cite{KeimelP2016}, additional order-theoretic and topological closure operations need to be applied.

To apply the above in the angelic case, we use the fact that $\mathcal{H}\mathcal{V}_{\leq 1} \,X$  is the free 
Kegelspitze join semilattice over a set $X$  \cite[Corollary 3.15]{KeimelP2016} (where the result is proved more generally for domains). It follows that $\mathcal{H}\mathcal{V}_{\leq 1} $ is
a monad on  $\Set$ itself  satisfying the conditions
of Proposition~\ref{proposition:monad}. Thus defining 
$D_\prang = \mathcal{H}\mathcal{V}_{\leq 1} \,\mathbb{N}$, and
$j(n) = \downarrow\!\delta(n)$ (where $\delta(n)$ is the Dirac probability distribution that assigns probability 1 to $n$ and $0$ to all other numbers, and $\downarrow\!x$ is the down-closure $\{y \mid y \leq x\}$), the induced $\Sem{\cdot}_\prang : \Tree(\Nat) \to  D_\prang$ defines an admissible and compositional preorder
\[
t \Basicleq^\Den_\prang t' ~~ \Leftrightarrow ~~ \Sem{t}_\prang  \leq \Sem{t'}_\prang \enspace .
\]

Similarly, in the demonic case, we use \cite[Corollary 3.15]{KeimelP2016} that characterises $\mathcal{S}\mathcal{V}_{\leq 1} \,X$ as the free Kegelspitze meet semilattice over $X$. Again $\mathcal{S}\mathcal{V}_{\leq 1}$ is a monad
on  $\Set$ to which Proposition~\ref{proposition:monad} applies. In this case, we define 
$D_\prang = \mathcal{S}\mathcal{V}_{\leq 1} \,\mathbb{N}$, and
$j(n) = \{\delta(n)\}$. Then the induced $\Sem{\cdot}_\prdem : \Tree(\Nat) \to  D_\prdem$ defines an admissible and compositional preorder
\[
t \Basicleq^\Den_\prdem t' ~~ \Leftrightarrow ~~ \Sem{t}_\prang  \leq \Sem{t'}_\prdem \enspace .
\]





\section{Axiomatically-defined preorders}

In this section, we consider a third way of defining a basic operational preorder, by axiomatising
properties of the operations in the effect signature $\Sigma$.
Since we are defining a preorder, it is appropriate for the axiomatisation to involve inequalities
specifying desired properties of the operational preorder. As the technical framework for this, we allow
Horn clause axiomatisations of inequalities between infinitary terms.  This provides a flexible general setting for
axiomatising admissible and compositional preorders on 
$\Trees(\mathbb{N})$. 

Let $\Vars$ be a set of countably many distinct variables. By an  \emph{expression}, we mean a
tree $e \in \Trees(\Vars)$. The use of trees incorporates infinitary non-well-founded terms alongside the usual finite
algebraic terms. By an \emph{inequality} we mean a statement $e_1 \leq e_2$, where $e_1, e_2$ are expressions.
By an \emph{(infinitary) Horn clause} we mean an implication of the form:
\begin{equation}
\label{equation:horn-clause}
\left( \bigwedge_{i \in I} e_i \leq e'_i \right)~ \implies ~ e \leq e' \enspace ,
\end{equation}
An \emph{effect theory} $T$  is a set of Horn clauses.

A precongruence $\Basicleq$ on $\Trees(X)$ is said to \emph{satisfy} a Horn clause (\ref{equation:horn-clause}) if,
for every environment $\rho \colon \Vars \to \Trees(X)$, the implication below holds (recall the notation for tree substitution from Section~\ref{section:trees}).
\[
\left( \bigwedge_{i \in I} e_i[\rho] \Basicleq e'_i[\rho] \right) ~ \implies ~  e[\rho] \Basicleq e'[\rho] 
\]
We say that a precongruence $\Basicleq$ is a \emph{model} of a Horn clause theory $T$ if it satisfies every Horn clause in $T$.
We consider models as subsets of $\Trees(X) \times \Trees(X)$, partially ordered by inclusion. Note that models are precongruences by assumption.
\begin{proposition}
Every Horn clause theory $T$ defines a smallest admissible model  ${\Basicleq_T} \subseteq \Trees(X) \times \Trees(X)$. The smallest model
is substitutive. In the case that $X = \mathbb{N}$, the smallest admissible model is thus an
admissible compositional preorder.
\end{proposition}


\begin{figure}[h!]
    \begin{equation*}
        \begin{array}{lrl}
            \text{Probability} & a \prEff a &= a \\
                        & a \prEff b &= b \prEff a \\
                        & (a \prEff b) \prEff (c \prEff d) &= (a \prEff c) \prEff (b \prEff d) \\
                        & a \prEff b \leq b &\implies a \leq b  \\
            %\hline
            \\
        \end{array}
        \begin{array}{lrl}
            \text{Non-Determinism} & a \orEff a &= a \\
                        & a \orEff b &= b \orEff a \\
                        & (a \orEff b) \orEff c &= a \orEff (b \orEff c) \\
            \\
        \end{array}
    \end{equation*}
    \begin{equation*}
        \begin{array}{lrl}

            \text{Angelic} & a \orEff b &\leq a \\ 
            \text{Demonic} & a \orEff b &\geq a \\ 
            %\hline 
            \text{Distributivity}
            & (a \orEff b) \prEff c &= (a \prEff c) \orEff (b \prEff c)
        \end{array}
    \end{equation*}
    \caption{Inequational theory for mixed probability and non
    determinism}
    \label{fig:axiomsmixed}
\end{figure}

\todo[inline]{AXIOMS FOR PROBABILITY PLUS TWO THEORIES OF NONDETERMINISM}

\todo[inline]{USE THE ABOVE TO DEFINE $\Basicleq^\Ax_\prang$ and
$\Basicleq^\Ax_\prdem$}

\todo[inline]{NOTE EXPLICITLY THAT ADMISSIBLE AND COMPOSITIONAL}

\begin{definition}
    \label{def:probaApproxConstruct}
    Let $t$ be a tree, $t^n$ is inductively defined as 
    $t^0 = \bot$ and $t^{n+1} = t \prEff t^n$. The
    tree $t^\infty$ is defined as $\sqcup_n t^n$.
\end{definition}

\begin{proposition}[Removing Horn-Clauses]
    The preorder $\Basicleq^{\Ax'}_\prdem$ obtained 
    by replacing the Horn-clause $a \prEff b \leq b \implies a \leq b$ by 
    the infinitary axiom $t^\infty = t$.
\end{proposition}

\begin{proof}
    It is clear that $t^n \Basicleq^\Ax_\prdem t$ forall $n$,
    and therefore $t^\infty \Basicleq^\Ax_\prdem t$. 
    We have $t \prEff t^n \Basicleq^\Ax_\prdem t^{n+1}$,
    and by admissibility we then deduce $t \prEff t^\infty \Basicleq^\Ax_\prdem
    t^\infty$, which using the Horn-clause axiom allows us to derive $t
    \Basicleq^\Ax_\prdem t^\infty$. The rule $t = t^\infty$ is therefore admissible
    in $\Basicleq^\Ax_\prdem$.

    For the converse inclusion assume $t \prEff t' \Basicleq^{\Ax'}_\prdem t'$.
    Then it is possible to derive $t \prEff (t \prEff t') \Basicleq^{\Ax'}_\prdem t \prEff t' \Basicleq^{\Ax'}_\prdem t'$.
    By induction and admissibility one deduces $t^\infty \Basicleq^{\Ax'}_\prdem t'$,
    and then because $t = t^\infty$ the equation becomes $t
    \Basicleq^{\Ax'}_\prdem t'$. The rule $a \prEff b \leq b \implies a \leq b$
    is therefore admissible in $\Basicleq^{\Ax'}_\prdem$.
\end{proof}


\section{The equivalence theorem}

Our main theorem is that our operational, denotational and axiomatic preorders for combined probability and nondeterminism all coincide, in the case of both angelic and demonic nondeterminism.
\begin{theorem}[Equivalence theorem] \leavevmode
\begin{enumerate} 
\item The preorders $\Basicleq^\Op_\prang$, $\Basicleq^\Den_\prang$ and $\Basicleq^\Ax_\prang$, for mixed probability and angelic nondeterminism, coincide.

\item Similarly, the preorders $\Basicleq^\Op_\prdem$, $\Basicleq^\Den_\prdem$ and $\Basicleq^\Ax_\prdem$,
for mixed probability and demonic nondeterminism, coincide.
\end{enumerate}
\end{theorem}

\begin{proof}
    We give an outline of the proof in the case of $\Basicleq^\Op_\prdem$.
    It is easily checked that $\Basicleq^\Op_\prdem$ satisfies the Horn-clause
    theory of $\Basicleq^\Ax_\prdem$, and we know that $\Basicleq^\Op_\prdem$ is 
    both admissible and compositional. This gives us the inclusion 
    ${\Basicleq^\Ax_\prdem} \subseteq {\Basicleq^\Op_\prdem}\,$.
    
    For the other inclusion, the proof sketch is the following one~:
    \todo[inline]{Explain what is not working on finite trees}
    \begin{enumerate}
        \item Prove that both preorders coincide 
            on trees without $\orEff$ nodes
        \item Prove that both preorders coincide 
            on trees with a \emph{finite} number 
            of $\orEff$ nodes.
        \item Use finite approximations and admissibility
            of both preorders to conclude in the general case.
    \end{enumerate}

    \begin{description}
        \item[Trees with no disjunction]

    \todo[inline]{Remove this part of the proof, it is only here to show that it
    works for probabilities only ... Which is going to be assumed. However
    it is clearly the same proof in some sense when extending from finite 
    trees to infinite trees. It could be a more general lemma}

    For such trees, $\inf_s \mathbf{E}_{ t \# s } (h)
    = \mathbf{E}_{ t } (h)$.
    If both trees are finite, they can be put into 
    a normal form using the axioms concerning the 
    probability constructor: two complete binary trees 
    of same height,
    with leafs ordered by increasing number, and 
    such that a given leaf is $\bot$ the same 
    in both trees.

    It is easy to prove that two trees with 
    only $\prEff$ nodes are ordered for $\Basicleq^\Op_\prdem$
    if and only if their normal forms are ordered,
    and it only takes a simple induction to 
    deduce that they are ordered for $\Basicleq^\Ax_\prdem$.

    Now, if $t$ and $t'$ are infinite, 
    we approximate them with finite trees $(t_i)$ and 
    $(t_i')$.
    
    Given a tree $T$ we build $T^n$ as $\bot$ if $n = 0$, and $T \prEff T^{n-1}$
    otherwise. 

    It is clear that $T^n \Basicleq^\Op_\prdem T$ in a \emph{strict fashion}:
    the inequality is strict for all test function $h$.

    Therefore $t_i \prEff (t_i')^n \Basicleq^\Op_\prdem t_i$
    with a \emph{strict inequality for all tests functions h}.
    
    Let us fix $i \in \mathbb{N}$, and a $n \in \mathbb{N}$: 

    \begin{equation*}
        t_i \prEff (t_i')^n \Basicleq^\Op_\prdem t_i \prEff t_i'
        \Basicleq^\Op_\prdem t \prEff t' \Basicleq^\Op_\prdem t'
    \end{equation*}

    For any given test function $h$ we therefore have a strict 
    inequality between $t_i \prEff (t_i')^n$ and $t'$, however 
    $t'$ is the supremum of the family $(t_i')$ and the function 
    $ t \mapsto (h \mapsto \mathbf{E}_{ t } (h))$ is scott-continuous.

    Therefore, there exists \emph{a finite $j$} such that 
    $t_i \prEff (t_i')^n \Basicleq^\Op_\prdem t_j'$.
    All of the above inequalities are between finite trees, and 
    therefore are also true using $\Basicleq^\Ax_\prdem$.
    Using admissibility with a supremum over $n$ 
    one deduces $t_i \prEff t_i' \Basicleq^\Ax_\prdem t_j'$
    and using admissibility with a supremum over $i$ we can finally conclude
    $t \prEff t' \Basicleq^\Ax_\prdem t'$.
    
      

        \item[Trees with finite disjunction]

    Let $t \Basicleq^\Op_\prdem t'$ 
    where $t$ and $t'$ have only a finite number 
    of disjunction nodes.
    The proof is done using the following 
    remarks
    \begin{enumerate}
        \item We can first use the axioms of the theory 
            to put both trees in the form of 
            a \emph{finite disjunction} over 
            \emph{possibly infinite probability trees}.

        \item 
            If for all $t_i'$ 
            a probability
            tree of $t'$ there exists 
            a corresponding tree $t_i$ in $t$ 
            such that $t_i \Basicleq^\Op_\prdem t_i'$
            then it is clear that $t \Basicleq^\Ax_\prdem t'$.

        \item 
            If $t_i$ and $t_j$ are two 
            probability trees of $t$ 
            then $t$ is equivalent to 
            $t \orEff (t_i \prEff t_j)$.
            This can be extended to any infinite 
            $\prEff$ combination 
            of probability trees of $t$.

        \item 
            If $t_i'$ is a probability tree of $t'$,
            then there exists an infinite $\prEff$
            combination of probability trees of $t$
            noted $\hat{t_i'}$
            such that $\hat{t_i'} \Basicleq^\Op_\prdem t_i'$.

        \item Because there is only a finite number 
            of probability trees in $t'$, we can 
            conclude by induction that 
            $t \Basicleq^\Ax_\prdem t'$.
    \end{enumerate}


        \item[Extension]
    Let $t \Basicleq^\Op_\prdem t'$,
    such that $t = \sqcup_i t_i$ and $t' = \sqcup_i t_i'$ 
    where both families are composed of \emph{finite} trees.

    The basic idea is to prove that for all $i$ there exists 
    a $j > i$ such that
    $t_i \prEff t_i' \Basicleq^\Ax_\prdem t_j'$ $(\dagger)$. Using admissibility
    this will allow us to conclude that $t \prEff t' \Basicleq^\Ax_\prdem t'$
    and then use the Horn-clause axiom to conclude $t \Basicleq^\Ax_\prdem t'$

    To obtain $(\dagger)$ we use the equivalence of $\Basicleq^\Op_\prdem$
    and $\Basicleq^\Ax_\prdem$ on trees with a finite number of $\orEff$ nodes.
    Given a tree $T$ we build $T^n$ as $\bot$ if $n = 0$, and $T \prEff T^{n-1}$
    otherwise. 

    It is clear that $T^n \Basicleq^\Op_\prdem T$ in a \emph{strict fashion}:
    the inequality is strict for all test function $h$.

    Let us fix $i \in \mathbb{N}$, and a $n \in \mathbb{N}$: 

    \begin{equation*}
        t_i \prEff (t_i')^n \Basicleq^\Op_\prdem t_i \prEff t_i'
        \Basicleq^\Op_\prdem t \prEff t' \Basicleq^\Op_\prdem t'
    \end{equation*}

    For any given test function $h$ we therefore have a strict 
    inequality between $t_i \prEff (t_i')^n$ and $t'$, however 
    $t'$ is the supremum of the family $(t_i')$ and the function 
    $ t \mapsto (h \mapsto \inf_s \mathbf{E}_{ t \# s} (h))$ is scott-continuous.
    \todo[inline]{Was that proven somewhere ?}
    Therefore, there exists \emph{a finite $j$} such that 
    $t_i \prEff (t_i')^n \Basicleq^\Op_\prdem t_j'$.
    All of the above inequalities are between finite trees, and 
    therefore are also true using $\Basicleq^\Ax_\prdem$.
    Using admissibility with a supremum over $n$ 
    one deduces $t_i \prEff t_i' \Basicleq^\Ax_\prdem t_j'$
    and using admissibility with a supremum over $i$ we can finally conclude
    $t \prEff t' \Basicleq^\Ax_\prdem t'$.
    \end{description}
    

\end{proof}




\section{Related and future work}
\label{section:conclusions}

The results in this paper concern three methods of defining \emph{basic operational preorders} on effect trees, which we claim to be a useful abstraction for defining contextual preorder for programming languages with algebraic effects. This has been verified for simple call-by-name~\cite{gom} and call-by-value\footnotemark[3] languages, but needs further substantiation.

The axiomatic approach to defining basic operational preorders in Section~\ref{section:axiomatic} is close in spirit to the algebraic axiomatisation of effects of Plotkin and Power~\cite{PlotkinPower2002}, but with a different focus. 
In~\cite{PlotkinPower2002}, (in)equational axiomatisations are required in order to determine a
free-algebra monad modelling denotational equality of programs. Such axiomatisations have also been used to combine effects~\cite{hyland2006combining},
 and to induce a logic of effects~\cite{PlotkinPretnarLogic}; but they have not 
 hitherto been explicated as a method for defining  contextual preorder/equivalence. 
 In this paper, we have used  infinitary Horn clause axioms between infinitary terms for this purpose, with the notion of admissible model playing an important role.
 
 The main coincidence theorem in Section~\ref{section:equivalence} has some precursors in the literature. 
 The characterisations of  $\mathcal{H}\mathcal{V}_{\leq 1} D$
 and  $\mathcal{S}\mathcal{V}_{\leq 1} D$ as free Kegelspitze in~\cite{KeimelP2016} can be viewed as completeness theorems for inequational axiomatisations with respect to
 \emph{domains} $D$. In the special case $D = \mathbb{N}$, this is implied by our results, for it can be derived from Lemma~\ref{lemma:completeness} that the partial-order quotients of 
 $\Tree(\mathbb{N})$ by $\Basicleq^\Ax_\prang$ and $\Basicleq^\Ax_\prdem$ are isomorphic to
  $\mathcal{H}\mathcal{V}_{\leq 1} \mathbb{N}$
 and  $\mathcal{S}\mathcal{V}_{\leq 1} \mathbb{N}\,$.  Another related completeness result is given in~\cite{mislove2004axioms},  where inequational axioms for 
 a simple process algebra with nondeterministic and probabilistic choice are proved
 complete with respect to a domain-theoretic semantics. Translated into our setting, this process algebra corresponds to  \emph{regular trees} in a signature that combines
 the operations $\orEff$, $\prEff$ with an additional prefix operation and zero constant.
In~\cite{mislove2004axioms}, the semantics uses the convex powerdomain, rather than the upper $\mathcal{S}$ and lower $\mathcal{H}$ that we consider. In  the present paper, we have 
not considered  convex powerdomains and the associated \emph{neutral} (as opposed to angelic or demonic) nondeterminism. However, it would be a natural extension to do so.

The main limitation we see of the present paper is the restriction throughout to \emph{admissible} basic operational preorders. The admissibility condition plays a fundamental role in almost everything we do. It is, however, violated by some natural operational preorders; for example,  for countable demonic nondeterminism. 
It is an open question how to incorporate such more general preorders into our theory.








%%
%% Bibliography
%%

%% Please use bibtex, 

\bibliography{bibliographie}

%\newpage

%\appendix

%\section{A functional language with algebraic effects}

We use a  simply-typed  functional language, with a base type of natural numbers, general recursion, and a call-by-value evaluation strategy. In addition, 
the language is 
parameterised by a collection $\Sigma$ of \emph{effect} operations, resulting in a 
language similar to that considered in.~\cite{plotkin2001adequacy}.
Fine-grained call-by-value [[REFERENCES]]  is used, instead of regular call-by-value,
because it simplifies the technical development by devolving all sequencing of effects to a single
language construct. 
This is a purely stylistic choice, as there are easy translations between fine-grained and regular call-by-value
[[REFERENCES]].

The syntax of types and terms is given in in Figure \ref{fig:language}. 
Note that there is a syntactic distinction between \emph{values} $V$ and \emph{computations} $M$.
The latter depend on the 
parameter set $\Sigma$, which  is 
an arbitrary set of operation symbols each with an associated finite arity.



\begin{figure}[h!]
    \begin{align*}
        \tau :=& ~\Nat ~|~ \tau \to \tau \\
        V :=& ~x ~|~ \lambda x:\tau. M ~|~ \Zero ~|~ \Succ V \\
        M :=& ~\Ret V ~|~ V V ~|~ \Fix V \\
                    &|~ \lcase{V}{M}{M} \\
                    &|~ \Bind{M}{x:\tau}{M} \\
                    &|~ \sigma(\underbrace{M, \dots, M}_n ) \quad \text{ where } (\sigma,n)  \in \Sigma
    \end{align*}
    \caption{Refined Call-By-Value PCF with effects}
    \label{fig:language}
\end{figure}
%
%\begin{figure}[h!]
%    \begin{equation*}
%        \begin{array}{rl}
%            (\lambda x:\tau. N)M &\leftarrow \Bind{M}{x:\tau}{N} \\
%            M N &\rightarrow \Bind{M}{f:\sigma \to \tau}{\Bind{N}{x:\sigma}{ f x }} 
%        \end{array}
%    \end{equation*}
%    \caption{Translation between refined and regular call-by-value}
%    \label{fig:refinedNormal}
%\end{figure}
%

\begin{figure}[h]
    \begin{center}
        % Identity on variables 
            \AxiomC{}
            \UnaryInfC{$\Gamma, x:\tau \vdash_V x : \tau$}
            \DisplayProof 
        \hskip 1.5em
        % Return a value 
            \AxiomC{$\Gamma \vdash_V V : \tau$}
            \UnaryInfC{$\Gamma \vdash_C \Ret V : \tau$}
            \DisplayProof 
        \hskip 1.5em
        % Lambda abstraction 
            \AxiomC{$\Gamma, x : \tau \vdash_C M : \tau'$}
            \UnaryInfC{$\Gamma \vdash_V \lambda x:\tau. M : \tau \to \tau'$}
            \DisplayProof 
        \hskip 1.5em
        % Zero 
            \AxiomC{}
            \UnaryInfC{$\Gamma \vdash_V \Zero : \Nat$}
            \DisplayProof
        \hskip 1.5em
        \vskip 1em
        % Succ 
            \AxiomC{$\Gamma \vdash_V V : \Nat$}
            \UnaryInfC{$\Gamma \vdash_V \Succ V : \Nat$}
            \DisplayProof
        \hskip 1.5em
        % Fixed point 
            \AxiomC{$\Gamma \vdash_V V : (\tau \to \tau') \to \tau \to \tau'$}
            \UnaryInfC{$\Gamma \vdash_C \Fix V : \tau \to \tau'$}
            \DisplayProof
        \hskip 1.5em
        % Application  
            \AxiomC{$\Gamma \vdash_V V : \tau \to \tau'$} 
            \AxiomC{$\Gamma \vdash_V W : \tau$}
            \BinaryInfC{$\Gamma \vdash_C V W : \tau'$}
            \DisplayProof
        \hskip 1.5em
        \vskip 1em
        % Case   
            \AxiomC{$\Gamma \vdash_V V : \Nat$} 
            \AxiomC{$\Gamma \vdash_C M_1 : \tau$}
            \AxiomC{$\Gamma, x:\Nat \vdash_C M_2 : \tau$}
            \TrinaryInfC{$\Gamma \vdash_C \lcase{V}{M_1}{M_2} : \tau$}
            \DisplayProof
        \hskip 1.5em
        % Bind 
            \AxiomC{$\Gamma \vdash_C M : \tau$} 
            \AxiomC{$\Gamma, x:\tau \vdash_C N : \tau'$}
            \BinaryInfC{$\Gamma \vdash_C \Bind{M}{x:\tau}{N} : \tau'$}
            \DisplayProof
        \hskip 1.5em
        \vskip 1em
        % Effect 
            \AxiomC{$(\sigma,n) \in \Sigma$}
            \AxiomC{$\forall 1 \leq i \leq n, \quad \Gamma \vdash_C M_i : \tau$}
            \BinaryInfC{$\Gamma \vdash_C \sigma(M_1, \dots, M_n) : \tau$}
            \DisplayProof
    \end{center}
    \caption{Inference rules for typing}
    \label{fig:inference:typing}
\end{figure}


The type inference rules are given  in Figure \ref{fig:inference:typing}.
It is easily shown that the set of values of type $\Nat$ is isomorphic to $\mathbb{N}$,
with $\Zero$ as $0$ and $\Succ$ as successor.
We use $\underline{n}$ to denote 
the closed value corresponding to $n$.


%% \subsection{Uniform Small-Step Semantics}

%The first step to define contextual equivalence 
%is to define an operational semantics.
%Because we are considering a \emph{class} of languages 
%we are going to have a two steps approach: first 
%interpret the core language independently of the parameter $\Sigma$
%and only then \emph{refine} this semantics to give one to the effects.

[[THE REFERENCES NEED REORGANISING IN THIS PARAGRAPH]]
We give a small-step operational semantics using 
stacks and frames \cite{gom}.
The link between this small-step semantics 
and its big-step counterpart is 
made both in \cite{plotkin2001adequacy} 
and in \cite{Pitts2000}. This method
can be traced back to \cite{Amadio2008} (1998) page 184 and 
\cite{Felleisen1992}. 

The syntax of stacks and frames is given in Figure \ref{fig:stacks:frames}. 
A type system for stacks is also defined Figure \ref{fig:stacks:types}.
Because stacks are just a \texttt{let}-binding with one hole, we can define 
application of a stack to a \emph{computation} which returns a computation
obtained by substitution as defined in Figure \ref{fig:stackapplication}.
We say that a pair $(S,M)$ of a stack and a computation term is 
\emph{well typed} when $S : \sigma \multimap \tau$ and $M : \sigma$, 
this is consistent with the following type-safety lemma.

\begin{figure}[h]
    \begin{center}
        \begin{align*}
            E &:= \Bind{\square}{x:\tau}{M} \\
            S &:= Id ~|~ S \circ E
        \end{align*}
    \end{center}
    \caption{Stacks and Frames}
    \label{fig:stacks:frames}
    \begin{center}
        \AxiomC{$x:\tau \vdash_C M : \tau'$}
        \UnaryInfC{$\vdash \Bind{\square}{x:\tau}{M} : 
            \tau \multimap \tau'$}
        \DisplayProof
        \vskip 1.5em
        \AxiomC{$\vdash E : \tau \multimap \tau'$}
        \AxiomC{$\vdash S : \tau' \multimap \tau''$}
        \BinaryInfC{$\vdash S \circ E : \tau \multimap \tau''$}
        \DisplayProof
        \vskip 1.5em
        \AxiomC{}
        \UnaryInfC{$\vdash Id : \tau \multimap \tau$}
        \DisplayProof
    \end{center}
    \caption{Typing of Stacks and Frames}
    \label{fig:stacks:types}
    \begin{align*}
        Id\{ M \} &= M \\
        (S \circ E) \{ M \} &= S \{ E \{ M \} \} \\
        (\Bind{\square}{x : \tau}{N} \{ M \}) &= \Bind{M}{x : \tau}{N}
    \end{align*}
    \caption{Stack application}
    \label{fig:stackapplication}
\end{figure}

%\begin{definition}[Stack application]
    %Given a stack $S$ and a computation term $M$
    %we can define $S\{M\}$ to be the application 
    %of the stack to the term by induction on $S$
    %following the rules in Figure \ref{fig:stackapplication}.
%\end{definition}


\begin{lemma}
    The typing of stacks and frames is consistent
    with the typing of terms and stack application.

    \begin{equation*}
        \vdash S : \tau \multimap \tau'  ~ \wedge ~ 
        \vdash_C M : \tau
        ~ \implies ~
        \vdash_C S\{M\} : \tau'
    \end{equation*}
\end{lemma}

%Now that evaluation stacks are defined, we can use them to define 
%the operational semantics of our language. First of all, on 
%simple computations, we can define a 

A basic reduction relation $\leadsto$, on computation terms, 
is defined in Figure \ref{fig:termred}. Using this, a reduction relation $\reduces$ is
defined, on well-typed stack-computation configurations, in 
Figure \ref{fig:evalstep}.


\begin{figure}[h]
    \begin{center}
        \begin{equation*}
            \begin{array}{lcl}
                (\lambda x:\tau. M) V & \leadsto & M[x := V] \\
                \lcase{\Zero}{M_1}{M_2} & \leadsto & M_1 \\
                \lcase{\Succ(V)}{M_1}{M_2} & \leadsto & M_2[x := V] \\
                \Bind{\Ret V}{x : \tau}{M} & \leadsto & M[x := V] \\
                \Fix V & \leadsto & V (\lambda x. \Bind{(\Fix V)}{g}{g x})
            \end{array}
        \end{equation*}
    \end{center}
    \caption{Term reduction}
    \label{fig:termred}
    \begin{center}
        \begin{equation*}
            \begin{array}{lclr}
                (S,E\{M\}) & \reduces & (S \circ E, M) \\
                (S \circ E, \Ret V) & \reduces & (S, E \{ \Ret V \}) \\
                (S,M) & \reduces & (S,M') & \text{ when } M \leadsto M'
            \end{array}
        \end{equation*}
    \end{center}
    \caption{Evaluation steps}
    \label{fig:evalstep}
\end{figure}

\begin{lemma}[Safety]
    If a pair $(S,M)$ is well-typed then the
    evaluation relation $\reduces$ preserves this property 
    and the co-domain of the stack never changes.
\end{lemma}

%\begin{example}[Reduction for combination of non-determinism and probabilities]
%    We can consider the following term in our language with signature $\Sigma =
%    \{ \prEff, \orEff \}$:
%
%    \begin{equation*}
%        M = (\lambda x:\Nat. \prEff (x,\underline{1})) \underline{0}
%    \end{equation*}
%    
%    This term corresponds to applying \underline{0} to a function 
%    that tosses a coin and return either the input given to the 
%    function or \underline{1}.
%    It can be shown that $(Id,M)$ reduces to $(Id, \prEff (\underline{0},
%    \underline{1}))$
%    as one could expect. However there is no more reduction possible afterwards: the 
%    evaluation is stuck, because there is no rule to evaluate 
%    the effect of a coin toss.
%\end{example}


\subsection{Interpretation}

Because of the injection of values of type $\tau$ into $\Tree_\tau$,
and to avoid unnecessary clutter in the equations, we are going to treat 
this injection as an inclusion and write $V$ instead of $i(V)$.

%The idea of this definition is to interpret terms "as most as we can"
%without having a semantics for the effects. When encountering an effect operation,
%the idea is to take advantage of the "continuation passing"
%and instead of having a specific rule for dealing with the effect, 
%create a branch for every possible input of the remaining computation 
%(see \cite{plotkin2001adequacy} and \cite{gom}). 

\begin{definition}[Computation tree]
    Given a well-typed pair $(S,M)$ and an integer $n$ we can 
    compute the tree $|S,M|_n$ by induction on $n$ 
    following the rules in Figure \ref{fig:treecalcul}.
    The sequence $|S,M|_n$ is ascending in $n$ given a fixed 
    pair $(S,M)$ and we write $|S,M|$ for its least upper bound.
\end{definition}

\begin{figure}[h!]
    \begin{center}
        \begin{equation*}
            \begin{array}{llr}
                |S,M|_{n+1} &= |S',M'|_n & (S,M) \reduces (S',M') \\
                |Id,\Ret V|_{n+1} &= i(V) & i \text{ is the injection from values into
                trees} \\
                |S,\sigma(M_1, \dots, M_l)|_{n+1} &= \sigma (|S, M_1|_n, \dots,|S,M_l|_n) \\
                |S,M|_0 &= \bot \\
            \end{array}
        \end{equation*}
    \end{center}
    \caption{Computation tree construction}
    \label{fig:treecalcul}
\end{figure}

This construction can be tested on a non terminating term 
that never uses effects:
because the computation tree is going to be $\bot$ at any 
step, the global computation tree is going to be $\bot$.


\begin{example}[Non termination]
    The following term does not terminate, and no effects occurs 
    during the evaluation of this term:
    
    \begin{equation*}
        \Omega_\tau = \Bind{\Fix (\lambda f : \Nat \to \tau. \Ret f
        )}{g : \Nat \to \tau}{g \underline{0}}
    \end{equation*}

    It is therefore possible to conclude that the computation tree
    associated to this term is always the least element of $\Tree_\tau$,
    independently of the stack $\vdash S : \tau \to \tau'$:

    \begin{equation*}
        \forall S : \sigma \multimap \tau, |S, \Omega_\sigma| = \bot
    \end{equation*}
\end{example}


One key result about this tree construction is the relationship between 
substitution on trees and application of stacks. To compute a computation 
tree for the pair $(S,M)$ it suffices to compute the computation tree 
for $(Id,M)$ and then substitute leaves with the computation tree obtained 
by $(S,\Ret V)$ where $V$ is the corresponding value of the leaf.

\begin{lemma}[Stack commutation]
    \label{lem:stackcom}
    Let $S : \tau \multimap \tau'$ and $M : \tau$, we always have:

    \begin{equation*}
        |S, M| = |Id,M| \sigma_S
    \end{equation*}

    Where $\sigma_S (V) = |S, \Ret V|$.
\end{lemma}

\begin{proof}
    We prove by induction on $n$ that for any well-typed pair $(S,M)$ we have 
    such that for every $m \geq n$ and $m' \geq n$ 
    we have $|S,M|_n \sqeq |M|_m \sigma^{m'}_S$, where 
    $\sigma^{m'}_S (V) = |S, \Ret V|_{m'}$.

    When $n=0$ the result is obvious because $\bot$ is under any other tree.

    \begin{itemize}
        \item If $|S,\sigma(M_1, \dots, M_k)|_{n+1} = \sigma (|S,M_1|_n, \dots,
            |S, M_k|_n)$ then we can use the induction hypothesis on all 
            subtrees and have $|S, M_i|_n \sqeq |M_i|_m \sigma^{m'}_S$ when 
            $m$ and $m'$ are above $n$. By continuity of $\sigma$ and 
            compatibility of substitution we have therefore when $m$ and $m'$
            are above $n$ we have:

            \begin{equation*}
                |S, \sigma (M_1, \dots, M_k)|_{n+1} \sqeq |\sigma(M_1, \dots,
                M_k)|_m \sigma^{m'}_S
            \end{equation*}

        \item If $|S, M|_{n+1} = |S, M'|_n$ because $M \leadsto M'$, and 
            therefore $|M|_{n+1} = |M'|_n$, and we can use this to trivially
            obtain the desired result.

        \item If $|S \circ E, M|_{n+1} = |S, E \{M\}|_n$ then 
            $M = \Ret V$ for some value $V$ and the result is obvious.

        \item If $|S, E\{M\}|_{n+1} = |S \circ E, M|_n$ then 
            we can use the induction hypothesis and the fact that 
            $\sigma^m_{S \circ E} \sqeq \sigma^m_S \circ \sigma^m_E$ pointwise
            to conclude.
    \end{itemize}
    
    Taking the limit we can conclude:

    \begin{equation*}
        |S,M| \sqeq |M| \sigma_S
    \end{equation*}
    
    The other inequality is obtained using a similar reasoning. 
\qed\end{proof}


\subsection{Iteration Approximation}

The last thing to check with our construction is that approximation on trees 
is indeed capturing the construction of infinite trees obtained by fixed points 
in the language. This equivalence 
is precisely given by the Theorem \ref{thm:unrolling} stating that 
the least upper bound of the semantics of the approximations is 
exactly the semantics of the fixed point. An equivalent
result also exists in call-by-name \cite{gom}, note that 
because of the evaluation strategy, the theorem is stated 
on trees over values of type $\Nat$ to ensure full 
evaluation of the fixed-point.

Unrolling fixed points is a very common operation \cite{plotkin2001adequacy}
and this approximation result is a basic one in papers defining logical 
relations \cite{pitts1997operationally} \cite{Pitts2000}.

\begin{figure}[h]
    \begin{align*}
        \Unroll_0 V     &= \Omega_{\sigma \to \tau} \\
        \Unroll_{n+1} V &= 
        V (\lambda x : \sigma. \Bind{(\Unroll_n V)}{g : \sigma \to \tau}{gx})
    \end{align*}
    \caption{Unrolling fixed point}
    \label{fig:unrolling}
\end{figure}

\begin{theorem}[Unrolling]
    \label{thm:unrolling}
    Let $\vdash S : (\sigma \to \tau) \multimap \Nat$ be a stack and 
    $\vdash_V V : (\sigma \to \tau) \to \sigma \to \tau$ be a 
    value term.

    \begin{equation*}
        \bigsqcup_{n \geq 0} |S, \Unroll_n V| = |S, \Fix V| 
    \end{equation*}
\end{theorem}

\begin{proof}
    \textbf{Alternative proof:}

    By induction on $n$ we prove that 

    \begin{equation*}
        \forall S,M,n,
        \exists m_0, \forall m \geq m_0, 
        |S[\Fix V], M[\Fix V]|_n = |S[\Unroll_m V], M[\Unroll_m V]|_n
    \end{equation*}


    If $n = 0$ then the result is obvious because all trees are $\bot$.
    When looking at $n+1$, we do a case analysis on the production rule 
    of the tree, and use the induction hypothesis. The only interresting 
    cases are: 

    \begin{enumerate}
        \item A fixed point evaluation, where it suffices to increase $m_0$ 
            by one
        \item An effect unfolding, where we take the maximal $m_0$ for each
            branch
    \end{enumerate}

    This proof is way simpler than what can be found in the work of Andrew Pitts
    but it does not scale to countable branches. For the sake of the argument 
    the outline of a more operational (and generic) proof is the following one.

    By induction on the derivation we can prove that 
    if 
    \begin{equation*}
        (S[x:=\Omega_\tau],M[x:=\Omega_\tau]) \reduces^* 
        (S'[x:=\Omega_\tau], M'[x:=\Omega_\tau])
    \end{equation*}
    then for any computation term $F$ of type $\tau$ 
    we have 
    \begin{equation*}
        (S[x:=F],M[x:=F]) \reduces^* 
        (S'[x:=F], M'[x:=F])
    \end{equation*}

    Using this result, we can prove by induction on $n$ 
    that if 
    \begin{equation*}
        (S[x:=\Unroll_n V],M[x:=\Unroll_n V]) \reduces^* 
        (S'[x:=\Unroll_n V], M'[x:=\Unroll_n V])
    \end{equation*}

    Then 
    \begin{equation*}
        (S[x:=\Unroll_{n+1} V],M[x:=\Unroll_{n+1} V]) \reduces^* 
        (S'[x:=\Unroll_{n+1} V], M'[x:=\Unroll_{n+1} V])
    \end{equation*}

    This allows us to prove by inudction on the derivation 
    that if 
    \begin{equation*}
        (S[x:=\Fix V],M[x:=\Fix V]) \reduces^* 
        (S'[x:=\Fix V], M'[x:=\Fix V])
    \end{equation*}
    Then there exists an $n_0$ such that for all $n \geq n_0$
    we have 
    \begin{equation*}
        (S[x:=\Unroll_n V],M[x:=\Unroll_n V]) \reduces^* 
        (S'[x:=\Unroll_n V], M'[x:=\Unroll_n V])
    \end{equation*}

    Using another induction we can prove the converse statement. 

    Now by induction on $k$ we can prove that for any depth 
    $k$ the truncation of $|S[x:= \Fix V, M[x := \Fix V]|$ 
    at depth $k$ is equal to the truncation of 
    the supremum of the chain $|S[x:= \Unroll_n V], M[x := \Unroll_n V]|$
    at depth $k$.

    This gives us the expected result because a computation tree 
    is the limit of it's finite approximations.

\qed\end{proof}


There is now an operational semantics for the language, respecting 
effects and capturing approximations. Up to this point, everything 
has been done uniformly over all signatures $\Sigma$. 
Given a computation term $M$ of type $\tau$, one can compute 
$|Id,M|$ (abbreviated $|M|$) which is a tree labelled with effects 
and has \emph{values} of type $\tau$ as leafs. 

The next goal is to define contextual equivalence for a specific signature 
$\Sigma$. This is usually done by fixing a relation on computation terms of type 
$\Nat$. However we can separate the
semantics of the effects contained in this relation from the semantics 
of the ground language: it suffices to give a relation on \emph{trees} of 
natural numbers to have one over computation terms of type $\Nat$.
The information required is therefore a simple relation $\sqeq_b$
over $\Tree_\Nat$.


%\section{Contextual Preorder}

Because we are going to study the contextual \emph{preorder},
the ground relation $\sqeq_b$ is going to be a \emph{preorder}: a reflexive 
and transitive relation. By building computation trees and using this basic 
preorder, we are going to define an abstract contextual preorder 
as the largest relation satisfying some axioms \cite{gom} \cite{Ugo2017}.
The definition is borrowed from the work of Andrew Pitts \cite{Pitts2000}, but can be
found in several other papers.

\newcommand{\CE}{\operatorname{\mathcal{E}}}

\begin{definition}[Formalisation of relations respecting type]
    Let $(\CE_V,\CE_C)$ be a pair of relations, the first 
    one on values and the second one on computations.
    If $\bullet$ is $V$ or $C$ then the relation 
    $\CE_\bullet$ is a set of tuples of the form $(\Gamma_\bullet, M,M', \tau)$
    and for every such tuple inside $\CE$ we have 
    $\Gamma \vdash_\bullet M : \tau$ and $\Gamma \vdash_\bullet M' : \tau$. 

    \begin{enumerate}[(i)]
        \item We say that $\CE$ is \emph{compatible} when
            $\CE$ is closed under rules in Figure
            \ref{fig:ax:compatibility}

        \item We say that $\CE$ is $\sqeq_b$-adequate 
            when for every pair of closed computation 
            terms $M$ and $M'$
            of ground type $\Nat$ we have $M \CE_C M'$ 
            implies $|M| \sqeq_b |M'|$
    \end{enumerate}

    We write $\Gamma \vdash M \CE M' : \tau$ 
    instead of $(\Gamma, M, M', \tau) \in \CE$ 
    to simplify reading.
\end{definition}

\begin{figure}[h]
    \begin{center}
        % Identity on variables 
            \AxiomC{}
            \UnaryInfC{$\Gamma, x:\tau \vdash x \CE_V x : \tau$}
            \DisplayProof 
        \hskip 1.5em
        % Return a value 
            \AxiomC{$\Gamma \vdash V \CE_V V': \tau$}
            \UnaryInfC{$\Gamma \vdash \Ret V \CE_C \Ret V' : \tau$}
            \DisplayProof 
        \hskip 1.5em
        \vskip 1em
        % Lambda abstraction 
            \AxiomC{$\Gamma, x : \tau \vdash M \CE_C M' : \tau'$}
            \UnaryInfC{$\Gamma \vdash (\lambda x:\tau. M) \CE_V
                (\lambda x:\tau. M'): \tau \to \tau'$}
            \DisplayProof 
        \hskip 1.5em
        % Zero 
            \AxiomC{}
            \UnaryInfC{$\Gamma \vdash \Zero \CE_V \Zero : \Nat$}
            \DisplayProof
        \hskip 1.5em
        % Succ 
            \AxiomC{$\Gamma \vdash V \CE_V V' : \Nat$}
            \UnaryInfC{$\Gamma \vdash \Succ V \CE_V \Succ V' : \Nat$}
            \DisplayProof
        \hskip 1.5em
        \vskip 1em
        % Fixed point 
            \AxiomC{$\Gamma \vdash V \CE_V V' :
                (\tau \to \tau') \to \tau \to \tau'$}
            \UnaryInfC{$\Gamma \vdash 
                \Fix V \CE_C \Fix V' : \tau \to \tau'$}
            \DisplayProof
        \hskip 1.5em
        % Application  
            \AxiomC{$\Gamma \vdash V \CE_V V': \tau \to \tau'$} 
            \AxiomC{$\Gamma \vdash W \CE_V W' : \tau$}
            \BinaryInfC{$\Gamma \vdash V W \CE_C V'W': \tau'$}
            \DisplayProof
        \hskip 1.5em
        \vskip 1em
        % Case   
            \AxiomC{$\Gamma \vdash V \CE_V V' : \Nat$} 
            \AxiomC{$\Gamma \vdash M_1 \CE_C M_1': \tau$}
            \AxiomC{$\Gamma, x:\Nat \vdash M_2 \CE_C M_2' : \tau$}
            \TrinaryInfC{$\Gamma \vdash 
                \left(\lcase{V}{M_1}{M_2}\right) 
                \CE_C
                \left(\lcase{V'}{M_1'}{M_2'}\right) : \tau$}
            \DisplayProof
        \hskip 1.5em
        \vskip 1em
        % Bind 
            \AxiomC{$\Gamma \vdash M \CE_C M' : \tau$} 
            \AxiomC{$\Gamma, x:\tau \vdash N \CE_C N' : \tau'$}
            \BinaryInfC{$\Gamma \vdash
                (\Bind{M}{x:\tau}{N})
                \CE_C
                (\Bind{M'}{x:\tau}{N'})
                : \tau'$}
            \DisplayProof
        \hskip 1.5em
        \vskip 1em
        % Effect 
            \AxiomC{$(\sigma,n) \in \Sigma$} 
            \AxiomC{$\forall 1 \leq i \leq n, 
            \quad \Gamma \vdash M_i \CE_C M_i' : \tau$}
            \BinaryInfC{$\Gamma \vdash
            \sigma(M_1, \dots, M_n)
            \CE_C
            \sigma(M_1', \dots, M_n')
            : \tau$}
            \DisplayProof
    \end{center}
    \caption{Compatibility rules}
    \label{fig:ax:compatibility}
\end{figure}


\begin{definition}[Contextual Preorder]
    There exists a largest compatible and $\sqeq_b$-adequate 
    relation called $\sqeq_{ctx}$
\end{definition}

In order to derive some of the theorems, we need to have more information 
about the $\sqeq_b$ preorder. One of the first properties is that 
it should behave nicely with approximation of trees (admissibility) and 
that it should behave nicely with composition of trees (compositionality).

\vspace{1em}

The compositionality and admissibility requirements are 
natural ones, and they automatically have good properties 
on natural numbers (lemma \ref{lem:admcompnat}),
are nicely related to $\sqeq$ (lemma \ref{lem:coarserpreorder})
and can be constructed as the smallest preorder 
satisfying some inequational theory (lemma \ref{lem:freepreo}).

\begin{lemma}[Behaviour on natural numbers]
    \label{lem:admcompnat}
    If the preorder $\sqeq_b$ is  
    compositional, then on natural 
    numbers we have either that two distinct 
    natural numbers are not comparable, 
    or that every pair of tree is equated  
    by $\sqeq_b$.
\end{lemma}



\begin{proof}
    Assume that there exists two distinct numbers $\underline{m}$
    and $\underline{n}$ such that $\underline{n} \sqeq_b \underline{m}$.
    Let $t$ and $t'$ be 
    two arbitrary trees and define:

    \begin{equation*}
        \sigma(k) = \begin{cases}
            t  & \text{ if } k = n \\
            t' & \text{ if } k = m \\
            \bot  & \text{ otherwise } 
        \end{cases}
    \end{equation*}

    We have $\sigma \sqeq_b \sigma$ because $\sqeq_b$
    is reflexive, and therefore using compositionality 
    of $\sqeq_b$ we have:

    \begin{equation*}
        \underline{n}\sigma \sqeq_b \underline{m}\sigma
    \end{equation*}

    Hence for every trees $t$ and $t'$ we have 
    $t \sqeq_b t'$.
\qed\end{proof}

\begin{lemma}[The preorder is coarser than $\sqeq$]
    \label{lem:coarserpreorder}
    Assume that $\sqeq_b$ is admissible and compositional,
    then $(\sqeq) \subseteq (\sqeq_b)$ if and only if
    $\bot$ is a least element for $\sqeq_b$.
\end{lemma}

    
\begin{example}[Simple counter example]
    There exists an admissible and compositional 
    preorder that does not extend $\sqeq$.
\end{example}

\begin{proof}
    Define $t \sqeq_b t' \iff \bot \in t \implies \bot \in t'$ where
    the $\bot \in t$ means that there exists a leaf of $t$ 
    which is $\bot$ \emph{or} that there exists an infinite branch 
    in $t$.

    Compositionality and admissibility are simple, and we note 
    that $\bot \neq \sqeq_b \underline{n}$, which proves 
    the claim.
\qed\end{proof}


We can build free preorders respecting $\sqeq$
given an inequational theory $\mathcal{T}$ as shown
in the following lemma \ref{lem:freepreo}.

\begin{definition}[Horn Clause Inequational Theory]
    A theory $\mathcal{T}$ is a horn-clause inequational 
    theory over $\Tree_\Nat$ if and only if it consists 
    in a list of formulas obtained with the following grammar:

    \begin{align*}
        t    &:= x ~|~ \sigma (\overbrace{t,\dots,t}^n) \quad (\sigma,n) \in
        \Sigma\\
        \phi &:= t \leq t \\
        \psi &:= \phi \vee \dots \vee \phi \vee \neg \phi
    \end{align*}
\end{definition}

\begin{example}[Angelic non-determinism]
    It can be shown that the inequational theory 
    characterising the powerdomain for angelic 
    non-determinism is the following one:

    \begin{align*}
        x \leq \orEff (x,y) & & \orEff (x,x) \leq x & & \orEff (x,\orEff (y,z))
        \leq \orEff(\orEff (x,y),z) \\
        \orEff (x,y) \leq \orEff(y,x) & & & & \orEff(\orEff (x,y),z) \leq \orEff
        (x,\orEff (y,z))
    \end{align*}
\end{example}

\begin{lemma}[Free preorder construction]
    \label{lem:freepreo}
    Given an inequational theory with horn-clauses $\mathcal{T}$
    there exists a smallest admissible 
    and compositional preorder $\sqeq_\mathcal{T}$ 
    on $\Tree_\Nat$ satisfying the inequational theory 
    such that for any tree $t$, $\bot \sqeq_\mathcal{T} t$.
\end{lemma}

\begin{proof}
    It is clear that satisfying some inequational 
    theory is stable by arbitrary intersection. Because 
    admissibility, compositionality and being 
    a preorder are also 
    stable properties by such intersection, one 
    can take the intersection of all such preorders.
\qed\end{proof}


%\section{Logical Relation}

This section consists in defining a \emph{relational interpretation}
of types by induction on their structure, and proving that 
this relational interpretation characterises the contextual 
preorder $\sqeq_{ctx}$. 

This is a method that goes back to \cite{Reynolds83} 
(see \cite{wadler1989theorems}) [TODO CITE FROM PITTS] where 
instead of interpreting terms as elements of 
a set, and types as sets, terms are left 
uninterpreted, and types are interpreted as \emph{relations}.

To benefit from theses results, one need to parametrize 
over a class of \emph{well-behaved} relations \cite{Pitts2000} 
and it turns out that \emph{biorthogonality} 
is a generic construction that allows to construct them 
\cite{mellies2005recursive}.

\vspace{1em}

In the following section, we will fix a preorder $\sqeq_b$
and assume it admissible and compositional. This 
preorder can be used to define an antitone Galois Connection 
written $\top$ between relations on stacks and relations on closed 
computation terms, meaning 
that if $r$ is a relation on closed computation terms of type $\tau$ and $s$ is 
a relation on stacks of type $\tau \multimap \Nat$:

\begin{equation*}
    r^\top \subseteq s \iff s^\top \subseteq r
\end{equation*}

\begin{definition}[The $\top$ operation]
    Let $s$ be a relation on stacks of type $\tau \to \Nat$ 
    and $r$ be a relation on closed computation terms 
    of type $\tau$:

    \begin{equation*}
        \begin{array}{rl}
            (S,S') \in r^\top &\iff
            \forall (M,M') \in r, |S,M| \sqeq_b |S',M'|\\ 
            (M,M') \in s^\top &\iff
            \forall (S,S') \in s, |S,M| \sqeq_b |S',M'|
        \end{array}
    \end{equation*}
\end{definition}

The main interpretation is that it is possible to 
relate stacks 
when we apply them to closed computation terms we can relate to each other and 
we can relate closed computation terms when we apply them to stacks we can relate to
each other. 
It can be intuited that a relation $r$ satisfying $r = r^{\top\top}$ going 
to be a relation preserving observational equivalence 
in some way, which is exactly the kind of relation 
we are looking for.
In fact, the function $r \mapsto r^{\top\top}$ is a closure operator
and is going to be used in the definition 
of the logical relation to guarantee its adequacy and compatibility.


Because the $\top$ operation is an antitone Galois Connection, 
all the properties in Figure \ref{fig:galois} are automatically true.


\begin{figure}[h]
        \begin{equation*}
            r^{\top\top\top} = r^\top
        \end{equation*}
        \begin{equation*}
            r \subseteq s \implies s^\top \subseteq r^\top
        \end{equation*}
        \begin{equation*}
            r \subseteq r^{\top\top}
        \end{equation*}
        \begin{equation*}
            (r^{\top\top})^{\top\top} = r^{\top\top}
        \end{equation*}
        \begin{equation*}
            r \subseteq s \implies r^{\top\top} \subseteq s^{\top\top}
        \end{equation*}
    \caption{Properties of the $\top$ operation}
    \label{fig:galois}
\end{figure}


\begin{lemma}[Saturation for $\top\top$-closed relations and $\sqeq_{ctx}$]
    \label{lem:saturation}
    Let $r$ be a $\top\top$-closed relation on 
    computations, we always have:
    \begin{equation*}
        (\sqeq_{ctx} r \sqeq_{ctx}) \subseteq r 
    \end{equation*}
\end{lemma}

This closure operator is going to be used in the definition 
of the logical relation to guarantee it's adequacy and compatibility.

\subsection{Definition of the relation}

We combine ideas from \cite{gom} 
for the treatment of effects and 
\cite{pitts1998existential} for the 
adaptation to the call-by-value setting.
The evaluation strategy can also be 
seen in \cite{dagand2015normalization}
with an equivalent definition and 
interesting description of the
actual way things are going to compute.

As usual, we define an operation on relations on terms 
for each type constructor. Because there is only one 
constructor it suffices to define what is the \emph{arrow}
relation between two relations.

\begin{definition}[Arrow relation]
    Let $r_1$ be a relation on values 
    and $r_2$ be a relation on computations 
    of type respectively $\tau_1$ and $\tau_2$, we define 
    $r_1 \to r_2$ a relation on 
    values of type $\tau_1 \to \tau_2$ as:

    \begin{equation*}
        r_1 \to r_2 = 
        \left\{ 
            (V,V')  
            ~|~
            \forall (W,W') \in r_1, 
            (VW, V'W') \in r_2 
        \right\}
    \end{equation*}

    Note that values of type $\tau_1 \to \tau_2$
    are all of the form $\lambda x:\tau_1. M$
    where $x:\tau_1 \vdash_C M : \tau_2$.
\end{definition}


It is now possible to write in a very simple way the logical 
relation on closed terms, which is going to be automatically $\top\top$-closed 
on computation terms at any type.

\begin{definition}[Logical relation on closed terms]
    The logical relation on closed terms is defined in
    Figure \ref{fig:logicalrel}. For every type 
    $\tau$ it defines a relation $\| \tau \|_V$ 
    on values of type $\tau$ and $\| \tau \|_C$ 
    on computations of type $\tau$.
\end{definition}

\begin{figure}[h]
    \begin{align*}
        \| \Nat \|_V &= \  \sqeq_b \\
        \| \tau \to \tau' \|_V &= \| \tau \|_V \to \| \tau' \|_C \\
        \| \tau \|_C &= \left\{ (\Ret V, \Ret V') ~|~ (V,V') \in \|\tau\|_V
        \right\}^{\top\top}
    \end{align*}
    \caption{Logical relation}
    \label{fig:logicalrel}
\end{figure}

If one of the key results in 
a logical relation argument is the reflexivity of the relation, 
it is very common to have a similar reflexivity result on stacks \cite{Pitts2000}.
We are going to use several times the fact that $(Id,Id) \in \| \Nat \|_C^\top$
which is a consequence of the stack reflexivity at ground type $\Nat$.

\begin{lemma}[Stack reflexivity at ground type $\Nat$]
    \label{lem:stackrefl}
    For any stack $\vdash S : \Nat \multimap \Nat$ the pair $(S,S)$ is inside $\| \Nat \|_C^\top$
\end{lemma}

\begin{proof}
    Let $S$ be a stack such that $\vdash S : \Nat \multimap \Nat$. Let 
    $V$ and $V'$ be two values of type $\Nat$ such that $V \sqeq_b V'$.
    Using the stack commutation lemma \ref{lem:stackcom}, $|S, \Ret V| = |\Ret
    V| \sigma_S$. But we know that $|\Ret V| = V$, and now using
    compositionality and reflexivity of $\sqeq_b$ it can be shown 
    that $|S, \Ret V| \sqeq_b |S, \Ret V'|$. This proves that $(S,S) \in \| \Nat
    \|_C^\top$ because:

    \begin{equation*}
        \| \Nat \|_C^\top = \left\{ (\Ret V, \Ret V') ~|~ (V,V') \in \| \Nat \|_V
        \right\}^\top
    \end{equation*}
\qed\end{proof}

Now in order to compare the logical relation to the contextual preorder,
it is needed to be able to relate \emph{open} terms. The usual 
open extension of a relation is used, while being careful to only 
substitute \emph{value} terms for the free variables.

\begin{definition}[Generalisation to open terms]
    If $M$ and $M'$ are two open terms with variables 
    typed by $\Gamma$ then $\Gamma \vdash M \, \| \tau \|_C \, M'$ if 
    and only if for any pair of substitutions $\vec{V}$ and $\vec{V'}$
    for the free variables such that $\vdash V_x \| \tau_x \|_V V_x'$
    for any $x : \tau_x \in \Gamma$ we have:

    \begin{equation*}
        \vdash M[ x := V_x ] \, \| \tau \|_C M' [ x := V_x' ]
    \end{equation*}

    The definition can be adapted to value terms in the obvious way.
\end{definition}


One direct consequence of this defininition as an open extension is 
that two related computation terms with one free variable can be used 
to extend a stack.

\begin{lemma}[Stack extension]
    \label{lem:stackexten}
    Let $(S,S') \in \| \tau \|_C^\top$ be two related stacks,
    and $x: \sigma \vdash M \| \tau \|_C M'$. One can construct 
    new stacks $S \circ (\Bind{\square}{x:\sigma}{M})$
    and $S' \circ (\Bind{\square}{x:\sigma}{M'})$ that 
    have type $\sigma \multimap \Nat$. Theses two new 
    stacks are related for $\| \sigma \|_C^\top$.
\end{lemma}

\begin{proof}
    Let $(V,V') \in \| \sigma \|_V$, we can use the definition 
    of the computation tree to reduce the stack:

    \begin{equation*}
        | S \circ (\Bind{\square}{x:\sigma}{M}, \Ret V| 
            = 
        | S, M[x := V]|
    \end{equation*}

    This equation is also valid for the second computation 
    tree built using $S'$, $M'$ and $V'$.

    But we know that $\vdash M[x := V] \| \tau \|_C M'[x := V']$
    by definition of the open extension of the relation. This proves 
    that $|S, M[x:= V]| \sqeq_b |S', M'[x:= V']|$ and therefore 
    that the two stacks are indeed related for $\| \sigma \|_C^\top$.
\qed\end{proof}

\subsection{Inclusion in contextual preorder}

The first step is to prove a soundness result, mainly that 
if two terms are logically related then they are contextually 
related. This soundness is proven in two steps: first prove 
adequacy of the logical relation, then prove compatibility 
with the constructions of the language.

\begin{lemma}[Adequacy]
    The logical relation is adequate.
\end{lemma}

\begin{proof}
    Let $M$ and $M'$ be two computation terms of type $\Nat$
    such that $\emptyset \vdash M \| \Nat \|_C M'$. 
    We know that for every pair $(S,S') \in \| \Nat \|_C^\top$
    we have:

    \begin{equation*}
        | S, M | \sqeq_b |S',M'|
    \end{equation*}

    Therefore it suffices to show that $(Id,Id) \in \| \Nat \|_C^\top$ 
    to conclude, but this is a direct consequence of lemma \ref{lem:stackrefl}.
\qed\end{proof}

We are now going to prove compatibility of the logical relation with 
the different constructions of the language. To simplify the proofs, 
compatibility is proven in an empty typing context $\Gamma$, this is allowed 
because the logical relation is defined on open terms using closing 
substitutions.

\begin{lemma}[Compatibility for variables]
    If $\Gamma$ is a typing context and $x$ a variable name then:
    \begin{equation*}
        \Gamma, x:\tau \vdash x \| \tau \|_V x 
    \end{equation*}
\end{lemma}

\begin{proof}
    The compatibility property holds by definition of the open extension
\qed\end{proof}

\begin{lemma}[Compatibility for $\Ret$]
    If $\Gamma$ is a typing context and $\Gamma \vdash V \| \tau \|_V V'$ then:
    \begin{equation*}
        \Gamma \vdash \Ret V \| \tau \|_C \Ret V
    \end{equation*}
\end{lemma}

\begin{proof}
    The compatibility property holds because the $\top\top$ closure 
    is a monotone operator
\qed\end{proof}

\begin{lemma}[Compatibility for function application]
    If $\Gamma$ is a typing context,
    $\Gamma \vdash V \| \tau \to \tau' \|_V V'$ 
    and $\Gamma \vdash W \| \tau \|_V W'$:

    \begin{equation*}
        \Gamma \vdash VW \| \tau' \|_C V'W'
    \end{equation*}
\end{lemma}

\begin{proof}
    The compatibility property holds by defintion of the arrow relation.
\qed\end{proof}

\begin{lemma}[Compatibility for $\Zero$]
    If $\Gamma$ is a typing context then:
    \begin{equation*}
        \Gamma \vdash \Zero \| \Nat \|_V \Zero 
    \end{equation*}
\end{lemma}

\begin{proof}
The compatibility property holds 
because $\sqeq_b$ is reflexive. 
\qed\end{proof}

\begin{lemma}[Compatibility for $\Succ$]
    If $\Gamma$ is a typing context and $\Gamma \vdash V \| \Nat \|_V V'$. 
    then:
    \begin{equation*}
        \Gamma \vdash \Succ V \| \Nat \|_V \Succ V'
    \end{equation*}
\end{lemma}
\begin{proof}
The compatibility property
holds because of lemma \ref{lem:admcompnat}. Indeed, 
when two natural numbers are related, either they 
are equal, and then their successors are also related 
because $\sqeq_b$ is reflexive, or they are not 
equal and every number is equated by $\sqeq_b$ 
making the property trivial. 
\qed\end{proof}

\begin{lemma}[Compatibility for $\lambda$-abstraction]
    If $\Gamma$ is a tying context and 
    $\Gamma, x :\tau \vdash M \| \tau' \|_C M'$ then:
    \begin{equation*}
        \Gamma \vdash (\lambda x:\tau. M) \| \tau \to \tau' \|_V (\lambda
        x:\tau. M')
    \end{equation*}
\end{lemma}
\begin{proof}
Assume without loss of generality 
that the computation terms 
$M$ and $M'$ only have one free variable $x$,
and that $x : \sigma \vdash M \| \tau \|_C M'$.

We want to prove that $(\lambda x:\sigma .M, \lambda x:\sigma. M')$
is in $\| \sigma \to \tau\|_V$. Let $(V,V') \in \| \sigma \|_V$,
we have to prove that $((\lambda x:\sigma. M)V, (\lambda x:\sigma.
M')V')$ is in $\| \tau \|_C$.

But using the hypothesis and 
the definition of the open closure, 
we already know that 
$(M[x := V], M'[x := V']) \in \| \tau \|_C$.

Let $(S,S') \in \| \tau \|_C^\top$, we know that 
$|S, M[x := V]| \sqeq_b |S', M'[x := V']|$ by definition 
of the $\top\top$-closure. 

It suffices to see that $|S, M[x := V]| = |S, (\lambda x:\sigma. M)
V|$ by definition of $|-,-|$ to have the conclusion:

\begin{equation*}
    | S, (\lambda x:\sigma. M) V| \sqeq_b |S', (\lambda x:\sigma.
    M') V'| 
\end{equation*}

Therefore we do have $(\lambda x:\sigma. M, \lambda x:\sigma. M')
\in \| \sigma \to \tau \|_V$.
\qed\end{proof}

\begin{lemma}[Compatibility for computation binding]
    If $\Gamma$ is a typing context,
    $\Gamma \vdash M \| \tau \|_C M'$ and 
    $\Gamma, x : \tau \vdash N \| \tau' \|_C M'$
    then:

    \begin{equation*}
        \Gamma \vdash (\Bind{M}{x:\tau}{N}) \| \tau' \|_C 
        (\Bind{M'}{x:\tau}{N'})
    \end{equation*}
\end{lemma}
\begin{proof}
Let $M = \Bind{M_1}{x:\sigma}{M_2}$,
$M' = \Bind{M_1'}{x:\sigma}{M_2'}$ and assume 
that $\emptyset \vdash M_1 \| \sigma \|_C M_1'$
and $x : \sigma \vdash M_2 \| \tau \|_C M_2'$.

Let $(S,S') \in \| \tau \|_C^\top$, we know that

\begin{equation*}
    \begin{cases}
        | S, M | &=
        | S \circ \Bind{\square}{x : \sigma}{M_2}, M_1| \\
        | S', M' | &=
        | S' \circ \Bind{\square}{x : \sigma}{M_2'}, M_1'| \\
    \end{cases}
\end{equation*}

Therefore to prove the inequality for $\sqeq_b$ it suffices 
to show that

\begin{equation*}
    (S \circ \Bind{\square}{x : \sigma}{M_2},
     S' \circ \Bind{\square}{x : \sigma}{M_2'})
     \in \| \sigma \|_C^\top
\end{equation*}

But because of the properties of the 
$\top\top$-closure we know that 
$\| \sigma \|_C^\top = (\Ret \| \sigma \|_V)^\top$.

Let $(V,V') \in \| \sigma \|_V$, we can prove that 
applying the stacks to $(\Ret V, \Ret V')$ gives 
related trees, by using the fact that $M_2$ and $M_2'$
are related and that $(S,S') \in \| \tau \|_C^\top$.

\begin{align*}
    | S \circ \Bind{\square}{x : \sigma}{M_2}, \Ret V| 
    &= |S,  M_2[ x := V] | \\
    & \sqeq_b |S', M_2' [ x := V'] | \\
    &= 
    | S' \circ \Bind{\square}{x : \sigma}{M_2'}, \Ret V'| 
\end{align*}

Therefore we have the expected conclusion.
\qed\end{proof}

\begin{lemma}[Compatibility for the case distinction]
    If $\Gamma$ is a typing context, 
    $\Gamma \vdash V \| \Nat \|_V V'$,
    $\Gamma \vdash M_1 \| \tau \|_C M_1'$
    and $\Gamma, x:\Nat \vdash M_2 \| \tau \|_C M_2'$
    then:

    \begin{equation*}
        \Gamma \vdash (\lcase{V}{M_1}{M_2}) \| \tau \|_C
            (\lcase{V'}{M_1'}{M_2'})
    \end{equation*}
\end{lemma}
\begin{proof}
Let $M = \lcase{V_1}{M_2}{M_3}$ and $M' = \lcase{V_1'}{M_2'}{M_3'}$,
and assume without loss of generality that the terms are closed. 
If $\emptyset \vdash V_1 \| \Nat \|_V V_1'$, 
$\emptyset \vdash M_2 \| \tau \|_C M_2'$
and $ x: \Nat \vdash M_3 \| \tau \|_C M_3'$ 
then we are going to show that $\emptyset \vdash M \| \tau \|_C M'$.
Let $(S,S') \in \| \tau \|_C^\top$.

We can make a case distinction on $V_1$:
\begin{itemize}
    \item When $V_1 = \Zero$ then $|S,M| = |S,M_2|$,
        but because of lemma \ref{lem:admcompnat} 
        we can assume that $V_1 = V_2$ and therefore 
        $|S',M'| = |S', M_2'|$. But by hypothesis
        we know that $(M_2,M_2') \in \| \tau \|_C$ and therefore
        we have $| S, M| \sqeq_b |S',M'|$. When in the second 
        case of the lemma \ref{lem:admcompnat} 
        the result is obvious.

    \item When $V_1 = \Succ V'$, we have the same 
        disjunction as in the previous case, and 
        the same conclusion, except that we use 
        the open closure.
\end{itemize}
\qed\end{proof}



\begin{lemma}[Compatibility for the fixed-point construction]
    If $\Gamma$ is a typing context and $\Gamma \vdash V \| (\tau \to \tau') \to
\tau \to \tau' \|_V V'$ then:
    \begin{equation*}
        \Gamma \vdash \Fix V \| \tau \to \tau' \|_C \Fix V'
    \end{equation*}
\end{lemma}
\begin{proof}
Assume that $\emptyset \vdash V \| (\sigma \to \tau) \to \sigma \to
\tau \|_V V'$. Let $(S,S') \in \| \sigma \to \tau\|_C^\top$.
Using the unrolling theorem \ref{thm:unrolling} we know that:

\begin{equation*}
    \begin{cases}
        |S, \Fix V| &= \bigsqcup_i | S, \Unroll_i V |     \\
        |S', \Fix V'| &= \bigsqcup_i | S', \Unroll_i V' |
    \end{cases}
\end{equation*}

Using admissibility of $\sqeq_b$, we know that it suffices 
to show that $|S, \Unroll_i V| \sqeq_b |S', \Unroll_i V'|$
when $i \in \mathbb{N}$. This can be 
understood as a proof by fixed point induction, and we are 
going to prove that $(\Unroll_i V, \Unroll_i V') \in \| \sigma \to \tau \|_C$
which will give the expected conclusion.

\begin{itemize}
    \item We know that $\Unroll_0 V = \Unroll_0 V'$ and 
        that for every stack they give the tree $\bot$ 
        because they don't terminate. Therefore 
        we have obviously $(\Unroll_0 V, \Unroll_0 V') \in \|
        \sigma \to \tau \|_C$.

    \item Suppose that $(\Unroll_i V, \Unroll_i V') \in \| \sigma
        \to \tau \|_C$. By definition, we know that:

        \begin{equation*}
            \begin{cases}
                \Unroll_{i+1} V &= V (\lambda x:\sigma.
                        \Bind{(\Unroll_i V)}{g : \sigma \to \tau}{g
                    x}) \\
                \Unroll_{i+1} V' &= V' (\lambda x:\sigma.
                        \Bind{(\Unroll_i V')}{g : \sigma \to \tau}{g
                    x}) \\
            \end{cases}
        \end{equation*}

        But we can use the previous compatibility propeties 
        to simplify this proof. By application and variable 
        compatibility we know that 
        
        \begin{equation*}
            g : \sigma \to \tau, x : \sigma \vdash g x \| \tau \|_C g x
        \end{equation*}

        Then using a binding construction, because we know that
        $\Unroll_i V$ and $\Unroll_i V'$ are related we can derive:
        
        \begin{equation*}
            x : \sigma \vdash (\Bind{\Unroll_i V}{g : \sigma \to
                \tau}{g x}) \| \tau \|_C (\Bind{\Unroll_i V'}{g : \sigma
            \to \tau}{g x})
        \end{equation*}
        
        Then we use the lambda-abstraction compatibility to 
        relate the two functions terms.
        Afterwards using the fact that $V$ and $V'$ are related 
        and again function application we can derive:

        \begin{equation*}
            \emptyset \vdash (\Unroll_{i+1} V) \| \tau \|_C
            (\Unroll_{i+1} V')
        \end{equation*}

\end{itemize}
\qed\end{proof}

\begin{lemma}[Compatibility for the effect construction]
    If $\Gamma$ is a typing context, $(\sigma,n) \in \Sigma$
    and $\Gamma \vdash M_i \| \tau \|_C M_i'$ for $i \in \{ 1, \dots, n\}$
    then:
    \begin{equation*}
        \Gamma \vdash \sigma(M_1, \dots, M_n) \| \tau \|_C \sigma(M_1', \dots,
        M_n')
    \end{equation*}
\end{lemma}
\begin{proof}
        Let $M = \sigma (M_1, \dots, M_n)$ 
        and $M' = \sigma (M_1', \dots, M_n')$
        such that $(M_i,M_i') \in \| \tau \|_C$.

        Let $(S,S') \in \| \tau \|_C^\top$.

        \begin{equation*}
            \begin{cases}
                |S,M|   &= \sigma (|S,M_1|, \dots |S,M_n|) \\
                |S',M'| &= \sigma (|S',M_1'|, \dots |S',M_n'|) \\
            \end{cases}
        \end{equation*}

        But we know that $\sqeq_b$ is reflexive, and 
        therefore
        \begin{equation*}
            \sigma(\underline{1}, \dots, \underline{n})
            \sqeq_b \sigma(\underline{1}, \dots, \underline{n})
        \end{equation*}

        The hypothesis tells us that $|S,M_i| \sqeq_b |S',M_i'|$
        for every $1 \leq i \leq n$. We can then design
        two substitutions (the $\bullet$ represents either 
        nothing or a quote):

        \begin{equation*}
            \tau^\bullet (i) = \begin{cases}
                |S^\bullet, M_i^\bullet| & \text{ when } 1 \leq i \leq n \\
                \underline{i}            & \text{ otherwise } 
            \end{cases}
        \end{equation*}

        We know that $\tau \sqeq_b \tau'$ pointwise, and therefore
        by compositionality of $\sqeq_b$ we can deduce:

        \begin{equation*}
            \sigma (|S,M_1|, \dots, |S,M_n|)
            = \sigma(\underline{0}, \dots, \underline{n}) \tau
            \sqeq_b
             \sigma(\underline{0}, \dots, \underline{n}) \tau'
            =
            \sigma (|S',M_1'|, \dots, |S',M_n'|)
        \end{equation*}
\qed\end{proof}

We therefore have the compatibility with respect to the language constructions, 
allowing us to state the inclusion of the logical relation inside the contextual
preorder.

\begin{theorem}[Inclusion of the preorders]
    \label{thm:inclusionpreorder}
    The logical relation is included in the contextual preorder.
\end{theorem}

\begin{proof}
    The logical relation is adequate and compatible, and 
    therefore is included in the largest relation that 
    is adequate and compatible.
\qed\end{proof}

\subsection{Equality with contextual preorder}

Because the language is call-by-value, the logical 
relation satisfies a very strong property that 
can be found in \cite{pitts1998existential} in a slightly different form:
the relation can be recovered from its restriction to values.
Separating values from computation makes this result easier to 
prove and to grasp as seen in lemma \ref{lem:valuerelation}.
This result is necessary to prove completeness of the logical 
relation with respect to contextual equivalence, and can easily 
be extended when adding new types to the language like sum types 
or product types.

\begin{lemma}[Value relation]
    \label{lem:valuerelation}
    For any type $\tau$ we have 

    \begin{equation*}
        (\Ret V, \Ret V') \in \| \tau \|_C
        \iff 
        (V,V') \in \| \tau \|_V
    \end{equation*}
\end{lemma}

\begin{proof}
    The right-to-left implication is a consequence of the definition 
    and the monotonicity of the $\top\top$-closure.
    The other implication can be proven by case analysis on $\tau$.

    \begin{itemize}
        \item When $\tau = \Nat$. Let $(\Ret V, \Ret V') \in \| \Nat \|_C$,
            using lemma \ref{lem:stackrefl} we know that $(Id,Id) \in \| \Nat
            \|_C^\top$ and therefore that $|\Ret V| \sqeq_b |\Ret V'|$, thus 
            proving the result.
        
        \item When $\tau = \tau_1 \to \tau_2$.
            Assume that $(\Ret V, \Ret V') \in 
            \| \tau_1 \to \tau_2 \|_C$.

            Let $(W,W') \in \| \tau_1 \|_V$, $(S,S') \in \| \tau_2 \|_C^\top$.
            We want to prove $|S, VW| \sqeq_b |S',V'W'|$, because this would
            imply that $(VW,V'W')$ is an element of $\| \tau_2 \|_C^{\top\top}$
            and therefore that 
            $(V,V')$ is in $\| \tau_1 \to \tau_2 \|_V$.

            It suffices to notice that $x: \tau_1 \to \tau_2 \vdash xW \| \tau_2
            \|_C x W'$. Indeed, this can be proven using compatibility
            properties. Now using lemma \ref{lem:stackexten} we can extend the
            stacks $S$ and $S'$ into the stacks $S \circ
            (\Bind{\square}{x:\tau_1 \to \tau_2}{xW})$ and $S' \circ
            (\Bind{\square}{x:\tau_1 \to \tau_2}{xW'})$. The lemma 
            states that the two new stacks are related for $\| \tau_1 \to \tau_2
            \|_C^\top$, and therefore:

            \begin{align*}
                |S \circ \Bind{\square}{x:\tau_1 \to \tau_2}{xW}, \Ret V| &\sqeq_b \\
                |S' \circ \Bind{\square}{x:\tau_1 \to \tau_2}{xW'}, \Ret V'| 
            \end{align*}

            Which after reduction gives exactly the expected inequality.

    \end{itemize}
\qed\end{proof}


\begin{lemma}[Largest adequate compatible and substitutive relation]
    The logical relation is the largest adequate, compatible 
    and substitutive relation. Where being substitutive is 
    being compatible with the rules in Figure \ref{fig:substitutive}.
\end{lemma}

\begin{proof}
    First of all, it is clear that the logical relation is substitutive 
    by definition as an open extension.

    Let $\CE$ be an adequate compatible and substitutive relation.
    By definition of the contextual preorder, we know that $\CE$
    is included into the contextual preorder. 

    Let $\Gamma \vdash M \CE_C M' : \tau$, we are going 
    to prove that $\Gamma \vdash M \| \tau \|_C M'$. 
    
    Let $\vec{V}$ and $\vec{V'}$ be
    sequences of values related for the logical relation and
    types agreeing with $\Gamma$. 

    Because the logical relation is reflexive, we know that 
    we have $\Gamma \vdash M \| \tau \|_C M$, and using the
    fact that the logical relation is substitutive, we then 
    can deduce that

    \begin{equation*}
        \emptyset \vdash M[\vec{\Gamma} := \vec{V}] \| \tau \|_C 
                         M[\vec{\Gamma} := \vec{V'}]
    \end{equation*}
    
    On the other hand, we know that $\CE$ is reflexive 
    and therefore $\vec{V'}$ is pointwise related 
    to $\vec{V'}$ for this relation. We can therefore 
    conclude using substitutivity:

    \begin{equation*}
        \emptyset \vdash M[\vec{\Gamma} := \vec{V'}] 
        \CE_C
                         M'[\vec{\Gamma} := \vec{V'}]
    \end{equation*}

    But we know that the relation $\CE$ is included into 
    the contextual preorder, and therefore the 
    equation can be written:

    \begin{equation*}
        M[\vec{\Gamma} := \vec{V}] 
        \| \tau \|_C
        M[\vec{\Gamma} := \vec{V'}]
        \sqeq_{ctx}
        M'[\vec{\Gamma} := \vec{V'}]
    \end{equation*}

    Using the saturation lemma (lemma \ref{lem:saturation})
    we can then deduce that for every pair of
    logically related $(\vec{V},\vec{V'})$ we have:

    \begin{equation*}
        \emptyset \vdash M[\vec{\Gamma} := \vec{V}] \| \tau \|_C
        M'[\vec{\Gamma} := \vec{V'}] 
    \end{equation*}

    Which by definition means that $\Gamma \vdash M \| \tau \|_C M'$. 
    We now have proven that:


    \begin{equation*}
        \Gamma \vdash M \CE_C M' : \tau \implies \Gamma \vdash M
    \| \tau \|_C M'
    \end{equation*}
    
    Therefore to prove that $\CE$ is included in the logical 
    relation it suffices to prove the same for value terms.

    Let $\Gamma \vdash V \CE_V V' : \tau$, because the relation is compatible 
    we know that $\Gamma \vdash \Ret V \CE_C \Ret V' : \tau$, using the previous 
    result we know that $\Gamma \vdash \Ret V \| \tau \|_C \Ret V'$, but
    using lemma \ref{lem:valuerelation} we can deduce that $\Gamma \vdash V \|
    \tau \|_V V'$.

\qed\end{proof}

\begin{figure}[h]
    \begin{center}
        \AxiomC{$\Gamma, x : \tau \vdash W \CE_V W' : \tau'$}
        \AxiomC{$\Gamma \vdash V \CE_V V' : \tau$}
        \BinaryInfC{$\Gamma \vdash W[x := V] \CE_C W'[x := V'] : \tau'$}
        \DisplayProof
    \end{center}
    \caption{Substitutivity}
    \label{fig:substitutive}
\end{figure}

\begin{lemma}[Contextual preorder is substitutive]
\end{lemma}

\begin{proof}
    Because contextual preorder is transitive, it suffices 
    to show that we have the following equality when 
    $M$ is a computation term of type $\tau$ and $V$
    a value term of type $\sigma$ to have the 
    substitutive property on computation terms:

    \begin{equation*}
        \Gamma \vdash M[ x := V] \equiv_{ctx} (\lambda x:\sigma. M) V : \tau
    \end{equation*}

    But we have already seen that this equivalence is true for the logical 
    relation, and we know that the logical relation is included in 
    the contextual preorder, allowing us to conclude.
\qed\end{proof}


\begin{theorem}[Contextual preorder equals the logical relation]
\end{theorem}

\begin{proof}
    We already have one inclusion with the theorem \ref{thm:inclusionpreorder}
    and the second one is given because 
    contextual preorder is an adequate, compatible and substitutive relation,
    therefore included into the largest one (the logical relation).
\qed\end{proof}

\subsection{Meta properties}

Using the abstract equality between the contextual preorder
and the logical relation we defined, we can prove generic 
theorems about contextual equivalence independently 
of the properties of effects.

The first one is that a relation between effects is true 
at ground type if and only if they are true at all type. This 
justifies the fact that we only ask for a preorder on $\Tree_\Nat$.

\begin{theorem}[Inequalities between effects are seen at ground type]
    Meaning that if $M$ and $M'$ are terms with only effects and free 
    variables, a polymorphic inequality between $M$ and $M'$ is 
    admissible if and only if it is admissible on natural numbers. 
\end{theorem}

\begin{proof}
    Assume that we have $M$ and $M'$ terms constructed 
    only as trees of effect constructors with 
    a finite number of 
    place holders for leaves, and that for any type $\tau$
    we have the following admissible rule:


    \begin{prooftree}
        \AxiomC{$\forall i, \Gamma \vdash M_i \, \| \tau \|_C \, M_i'$}
        \UnaryInfC{$ \Gamma \vdash M[(M_i)] \,  \| \tau \|_C \,  M'[(M_i')]$}
    \end{prooftree}
    
    It is trivial to see that the rule can be seen 
    at type $\Nat$.

    Going the other way around, we only assume that 
    we have the following admissible rule:

    \begin{prooftree}
        \AxiomC{$\forall i, \Gamma \vdash M_i \, \| \Nat \|_C \, M_i'$}
        \UnaryInfC{$ \Gamma \vdash M[(M_i)] \,  \| \Nat \|_C \,  M'[(M_i')]$}
    \end{prooftree}

    Let $\tau$ be a given type, 
    and a set of computation terms 
    of type $\tau$ such that we have 
    derivations of $\Gamma \vdash M_i \, \| \tau \|_C \, M_i'$.

    Let $(S,S') \in \| \tau \|_C^\top$, because 
    $M$ is merely a tree, we can calculate $|S, M[(M_i)]|$:

    \begin{equation*}
        |S, M[(M_i)]| = 
        M[(i)] [ i \mapsto |S,M_i| ]
    \end{equation*}

    Now, using the given admissible rule on $\Ret i$ 
    and compositionality of the preorder we can 
    deduce that:

    \begin{equation*}
        |S, M[(M_i)]| \sqeq_b |S', M'[(M_i')] |
    \end{equation*}
\qed\end{proof}


\begin{theorem}[Contextual preorder can be reduced to closed terms]
    Two open computation terms $M$ and $M'$ are contextually related 
    if and only if for any closing substitution of contextually related 
    values, the two closed terms obtained are related.

    This is stating that $\sqeq_{ctx}$ is the open extension of 
    $\sqeq_{ctx}$ restricted to closed terms.
\end{theorem}

\subsection{Importance of hypothesis}

We can now ask ourselves to what extend we could change 
the hypothesis on $\sqeq_b$. The answer is that they 
are in some sense minimal because removing admissibility 
allows to "break" the behaviour of fixed-points, and 
removing compositionality allows to "break" the 
behaviour of effects.

The following results are \emph{not} using the meta-theorems 
that are extensively using compositionality and admissibility 
of $\sqeq_b$.

\begin{lemma}[Admissibility and good behaviour on $\Tree_\Nat$ implies compositionality]
    Namely, we are going to show that if $\sqeq_b$ is admissible then 
    \begin{equation*}
        \left[(M,M') \in \| \Nat \|_C \iff
        |M| \sqeq_b |M'| \right]
        \iff
        \sqeq_b \text{ is compositional }
    \end{equation*}
\end{lemma}

\begin{proof}
    \begin{itemize}
        \item Let's assume that $\sqeq_b$ is compositional. For 
            one direction it suffices to notice that
            $(Id,Id) \in \| \Nat \|_C^\top$ to conclude. For the other 
            direction, assume that $|M| \sqeq_b |M'|$, 
            we know that if $(S,S') \in \| \Nat \|_C^\top$ we have:

            \begin{equation*}
                |S,M| = |M|[i \mapsto |S, \Ret i|]
            \end{equation*}

            Now using compositionality we can conclude that $|S,M| \sqeq_b
            |S',M'|$ and therefore $(M,M') \in \| \Nat \|_C$.

        \item The other way around, we use the fact that 
            every tree is the least upper bound of it's finite approximations,
            and that any finite tree on $\Nat$ has a trivial representation as a 
            finite term\footnote{Note that it still holds if 
            we allow countable branching controled by a lambda-abstraction}.
            
            Without loss of generality (using admissibiltiy) we can 
            therefore restrict to finite trees represented by terms 
            $M,M'$.

            To prove compositionality, we are going to consider 
            two terms $M$ and $M'$ such that $|M| \sqeq_b |M'|$, two substitutions 
            $(\sigma,\sigma')$ such that $\sigma^\bullet (i)$ is always 
            a finite tree, represented by a term $N_i^\bullet$, and 
            $|N_i| \sqeq_b |N_i'|$.

            Compositionality claims that $|M|[ i \mapsto |N_i|] \sqeq_b |M'|[ i
            \mapsto |N_i'|]$, to obtain that result we are going to consider 
            stacks $(S_k,S_k') \in \| \Nat \|_C^\top$ that satisfies:

            \begin{equation*}
                | S_k^\bullet, \Ret i | = 
                \begin{cases}
                    |N_i| & \text{ when } i \leq k \\
                    \bot  & \text{ otherwise } 
                \end{cases}
            \end{equation*}

            Building theses and checking the relation is easy as it is just 
            writing a finite case disjunction and using the non terminating 
            term when we fall in the last case.

            Now, because $\| \Nat \|_C = \| \Nat \|_C^{\top\top}$ we can 
            deduce:
            \begin{equation*}
                |S_k, M| \sqeq_b |S_k', M'|
            \end{equation*}

            But we can rewrite the terms using the substitution lemma
            \ref{lem:stackcom}: 
            \begin{equation*}
                |M|[i \mapsto |S_k, \Ret i|] \sqeq_b |M'|[ i \mapsto |S_k', \Ret
                i|]
            \end{equation*}

            It is clear that the trees obtained are ascending chains in $k$
            with least upper bound $|M^\bullet|[i \mapsto |N_i^\bullet|]$ 
            giving the expected result.

            To adapt this to any tree, it suffices to see that the chain 
            is also ascending in $M$ and $N_i$, therefore allowing for 
            simultaneous approximaton on $k,M,N_i$.
    \end{itemize} 

\qed\end{proof}





%
\section{Domain theoretic preorders}

The goal of this section is to link the domain theoretic semantics 
with our setting. The first remark is that if 
$\llbracket \cdot \rrbracket$ is a semantic map from $\Tree_\Nat$ to 
an $\omega$CPPO $(D,\leq)$, one can define:

\begin{equation*}
    t \sqeq_b t' \iff \llbracket t \rrbracket \leq \llbracket t' \rrbracket 
\end{equation*}

This preorder is not necessarily admissible or compositional, therefore 
we can ask ourselves what conditions on the semantic map $\llbracket \cdot
\rrbracket$ are sufficient to obtain the desired properties.


\subsection{Admissibility}


One natural requirement for a semantic map is to be scott-continuous, and 
it turns out that it automatically gives an admissible preorder.

\begin{lemma}[Admissibility]
    \label{lem:continuousadm}
    If the function is scott-continuous, then the relation defined 
    is admissible.
\end{lemma}


\begin{proof}
    Let $(t_i)_i$ and $(t_i')_i$ be two ascending chains for $\sqeq$
    with least upper bounds $t$ and $t'$ such that $\llbracket t_i \rrbracket
    \leq \llbracket t_i' \rrbracket$ for every $i$.

    \begin{align*}
        \left\llbracket \bigsqcup_i t_i \right\rrbracket &= \bigsqcup_i \llbracket t_i
        \rrbracket & \text{ scott-continuity } \\
                   &\leq \bigsqcup_i \llbracket t_i' \rrbracket & 
                    \text{ hypothesis } \\
                   &= \left\llbracket \bigsqcup_i t_i' \right\rrbracket &
                    \text{ scott-continuity } \\
    \end{align*}
\qed\end{proof}



\subsection{Compositionality}


A natural assumption for compositionality is asking for 
the interpretation to be a homomorphism and for $D$ 
to be a $\Sigma$-continuous algebra. Indeed, we are 
trying to interpret effects and it seems obvious 
that it is a good way to do so: in Plotkin and Power's work
algebraic effects are the one that are preserved 
under some homomorphisms [TODO cite].

\begin{lemma}[First half of compositionality]
    Assume $D$ is a $\Sigma$-continuous algebra
    and $\llbracket \cdot \rrbracket$ is a homomorphism,
    then for every tree $t$ and pair of substitutions 
    $(\sigma_1,\sigma_2)$ such that $\sigma_1(n) \sqeq_b \sigma_2 (n)$
    for every $n$, we have:

    \begin{equation*}
        t \sigma_1 \sqeq_b t \sigma_2
    \end{equation*}
\end{lemma}


\begin{proof}
    For $i \in \{ 1, 2\}$ we have the following commutation diagram 
    obtained using the universal property of $\Tree_\Nat$ on 
    $\sigma_i$ and $\llbracket \cdot \rrbracket \circ \sigma_i$:

    \begin{center}
        \begin{tikzcd}
            \Nat \arrow[r, "\sigma_i"] \arrow[d, hook, "i"] & 
            \Tree_\Nat \arrow[r, "\llbracket \cdot \rrbracket"] & D \\
            \Tree_\Nat \arrow[ru, "\hat{\sigma_i}"]
                       \arrow[rru, bend right, "\widehat{\llbracket \cdot
                               \rrbracket \circ \sigma_i}"] 
        \end{tikzcd}
    \end{center}

    By definition, $t \sigma_i = \hat{\sigma_i} (t)$, 
    and therefore $\llbracket t \sigma_i \rrbracket = \llbracket \hat{\sigma_i}
    (t) \rrbracket$.

    Using the unicity of the homomorphism in the universal property of
    $\Tree_\Nat$ it is easy to show that $\llbracket \cdot \rrbracket 
    \circ \hat{\sigma_i} = \widehat{ \llbracket \cdot \rrbracket \circ \sigma_i
    }$

    But the hypothesis specifically states that $\llbracket \cdot \rrbracket 
    \circ \sigma_1$ is a function pointwise lower than $\llbracket \cdot
    \rrbracket \circ \sigma_2$. Therefore using the lemma about 
    lifting functions (Lemma \ref{lem:orderpreservinglift})
    we know that the inequality is still true for the lifting 
    of both functions. Therefore:

    \begin{align*}
        \llbracket t \sigma_1 \rrbracket &= 
        (\llbracket \cdot \rrbracket \circ \hat{\sigma_1}) (t) \\
        &= \widehat{ \llbracket \cdot \rrbracket \circ \sigma_1} (t) \\
        &\leq \widehat{ \llbracket \cdot \rrbracket \circ \sigma_2} (t) \\
        &= 
        (\llbracket \cdot \rrbracket \circ \hat{\sigma_2}) (t) \\
        &= \llbracket t \sigma_2 \rrbracket
    \end{align*}
    
    And we have the conclusion: $t \sigma_1 \sqeq_b t \sigma_2$.

\qed\end{proof}


But it is not enough to  guarantee compositionality of the preorder,
as it can be seen in the example \ref{ex:supercounterexample}.

\begin{example}[Simple counter example]
    \label{ex:supercounterexample}

    Let $\Sigma = \{ (+,2), (\times,2) \}$ be a signature.
    Let $D$ be $\bar{\mathbb{N}}$
    with the usual ordering and the usual $\Sigma$-algebra
    structure. It is a continuous $\Sigma$-algebra 
    and the homomorphism that arises from identity 
    on natural numbers is just the regular interpretation 
    of tree of operations.

    But if $t_1 = 1 + 2$ and $t_2 = 1 \times 3$, we can
    take $\sigma$ defined as follows to break compositionality:

    \begin{equation*}
        \sigma(k) = \begin{cases}
            0 & \text{ if } k = 1 \\
            k & \text{ otherwise } 
        \end{cases}
    \end{equation*}

    We have $t_1 \sqeq_b t_2$ but 
    not $t_1 \sigma \sqeq_b t_2$ because $2 \not \leq 0$.
\end{example}

It is therefore necessary to ask for the interpretation 
to respect effects in a deeper way, related to 
the definition of \emph{algebraic} effects.

\begin{definition}[Factorisation of homomorphism]
    The interpretation $\llbracket \cdot \rrbracket$
    factors homomorphisms when for every function 
    $\sigma : \Nat \to D$ there exists a 
    homomorphism $h_\sigma : D \to D$ such that $\sigma = h_\sigma \circ \llbracket
    \cdot \rrbracket$.     
    \begin{center}
        \begin{tikzcd}
            \Nat \arrow[r, "\llbracket \cdot \rrbracket"] 
                 \arrow[rr, bend right, "\sigma"] &
            D \arrow[r, "h_\sigma", dashed] & 
            D  
        \end{tikzcd}
    \end{center}
\end{definition}

\begin{lemma}[Factorisation of tree homomorphisms]
    Assume that $\llbracket \cdot \rrbracket$ factors homomorphisms. 
    If $\tau : \Tree_\Nat \to D$ is a homomorphism 
    then there exists $h_\tau$ a homomorphism
    from $D$ to $D$ such that $\tau = h_\tau \circ \llbracket \cdot \rrbracket$.
\end{lemma}


\begin{proof}
    It suffices to use the universal property of $\Tree_\Nat$ 
    and the fact that $\tau$ can be restricted to a function 
    on $\Nat$. Note that it is crucial that $h_\tau$ is a 
    homomorphism of $\Sigma$-continuous algebras, 
    otherwise the universal property cannot be used.
\qed\end{proof}


\begin{lemma}[Second half of compositionality]
    If $\llbracket \cdot \rrbracket$ is a homomorphism 
    that factors functions from $\Nat$ to $D$ then 
    the preorder $\sqeq_b$ is compositional.
\end{lemma}


\begin{proof}
    The only thing to prove is that given two trees $t_1$ and 
    $t_2$ such that $\llbracket t_1 \rrbracket \leq \llbracket t_2 \rrbracket$
    and a substitution $\sigma : \Nat \to \Tree_\Nat$ 
    we have $\llbracket t_1 \sigma \rrbracket \leq \llbracket t_2 \sigma
    \rrbracket$. 
    Then using the first half of compositionality we can conclude.

    The idea is then to use the factorisation and 
    the monotonicity of $h_\sigma$ to prove that:

    \begin{equation*}
        \llbracket t_1 \sigma \rrbracket 
        = h_\sigma (\llbracket t_1 \rrbracket) 
        \leq h_\sigma (\llbracket t_2 \rrbracket)
        = \llbracket t_2 \sigma \rrbracket 
    \end{equation*}

    This result is obtained using a technique similar 
    to the first part of compositionality proof: using 
    the universal property several times and prove 
    equality of the different ways of writing 
    the same homomorphism.
\qed\end{proof}


\subsection{Examples}

We now list examples where the previous results can be used 
to reuse work done on denotational semantics.

The first example is when $D$ is a free algebra. That 
can happen in cases such as Powerdomains 
\cite{abramsky1994}, Powercones \cite{tix2009semantic}, 
or Powerkegelspitze \cite{KeimelP2016}.

\begin{example}[Free algebras]
    \label{lem:freealgebra}
    Assume that $D$ is a free algebra
    over $\Nat$ and it holds that it 
    is a $\Sigma$-continuous algebra.
    Then the preorder defined using the embedding 
    of $\Nat$ into the free structure gives 
    an admissible and compositional preorder.
\end{example}


    \begin{proof}
        Use the freeness property.
    \qed\end{proof}


The second possibility is to already have a denotational semantics 
for the language defined in a monadic way on the category of $\omega$CPPOs.

\begin{example}[Monadic interpretation]
    If the language is interpreted in a $\omega$CPPO with 
    interpretations for effect that are \emph{natural}
    with respect to the EM-morphisms then the
    preorder obtained is admissible and compositional.
\end{example}


\begin{proof}
    In a full interpretation with a monad $T$, for every effect $\sigma$
    there is a family of morphisms $(\sigma_X) : \mathbb{N} \times (T X)^\mathbb{N} \to T X$
    such that $\sigma_X$ is natural in $T X$ with respect to EM-morphisms, meaning that for 
    every $g : X \to T Y$ we have:

    \begin{center}
        \begin{tikzcd}
            \mathbb{N}_\bot \times (T X)^\mathbb{N} 
            \arrow[d, "\pi_0 \times (g^\dagger \circ \pi_i)_{i \geq 1}" left] 
            \arrow[r, "\sigma_X"] & T
            X \arrow[d, "g^\dagger"] \\
            \mathbb{N}_\bot \times (T Y)^\mathbb{N} \arrow[r, "\sigma_Y"] & T Y
        \end{tikzcd}
    \end{center}

    Therefore when there is a morphism 
    $f : \mathbb{N}_\bot \to T \mathbb{N}_\bot$
    we have the following diagram:

    \begin{center}
        \begin{tikzcd}
            \mathbb{N}_\bot \times (T \mathbb{N}_\bot )^\mathbb{N} 
            \arrow[d, "\pi_0 \times (f^\dagger \circ \pi_i)_{i \geq 1}" left] 
            \arrow[r, "\sigma_X"] & T \mathbb{N}_\bot\arrow[d, "f^\dagger"] \\
            \mathbb{N}_\bot \times (T \mathbb{N}_\bot)^\mathbb{N} 
            \arrow[r, "\sigma_Y"] & T \mathbb{N}_\bot
        \end{tikzcd}
    \end{center}

    Therefore, $f^\dagger$ is a homomorphism of $\Sigma$-continuous algebras,
    and we have 
    $f = f^\dagger \circ \eta_{\mathbb{N}_\bot}$.
\qed\end{proof}

This result can be used to show that given an interpretation,
the contextual preorder we can define is the same as 
the contextual preorder defined using first the tree
calculation and then the monadic interpretation. This 
result comes from \cite{plotkin2001adequacy}.

\begin{corollary}[Adequacy with the interpretation]
    \em
    Assume the interpretation $\llbracket \cdot \rrbracket$ 
    is obtained using a strong monad on $\omega$CPPO that 
    respects the order structure, then for closed ground 
    type terms:

    \begin{equation*}
        |M| \sqeq_b |M'| \iff \llbracket M \rrbracket \leq \llbracket M' \rrbracket
    \end{equation*}

    Therefore the contextual preorder defined using $\sqeq_b$
    is the same as the one defined using the interpretations, and 
    we can apply all theorems to the contextual preorder defined 
    using the interpretation.
\end{corollary}

\begin{proof}
    On closed ground type terms $|M| \sqeq_b |M'|$ is equivalent 
    to $\llbracket |M| \rrbracket \leq \llbracket |M'| \rrbracket$.
    Using the adequacy result from \cite{plotkin2001adequacy} 
    it is equivalent to $\llbracket M \rrbracket \leq \llbracket M' \rrbracket$.
\qed\end{proof}

An interesting lemma can be used to prove compositionality 
and admissibility of a preorder. The idea is to find a 
bigger $\omega$CPPO where the properties are clear, and 
an embedding-projection pair of $\Sigma$-continuous 
algebras between the two domains. 

\begin{lemma}[Embedding-Projection pair]
    Let $E$ and $D$ be $\omega$CPPOS with a $\Sigma$-continuous
    algebra structure. Assume that 
    the interpretation $\llbracket \cdot \rrbracket$ from $\Tree_\Nat$ to $E$
    gives an admissible and compositional preorder. 
    Assume moreover that there exists
    an embedding projection pair from $D$ to $E$,
    meaning two homomorphisims of $\Sigma$-continuous algebra 
    $e$ and $p$ such that $p \circ e = id$, $e \circ p \leq id$, 
    $e: D \to E$ and $p : E \to D$.

    If $\llbracket \cdot \rrbracket = e \circ [ \cdot ]$ on natural 
    numbers, then both preorders are equal and therefore share all 
    the required properties.
\end{lemma}

\begin{proof}
    If $\llbracket \cdot \rrbracket = e \circ [ \cdot ]$ then 
    by applying the projection we have $p \circ \llbracket \cdot \rrbracket
     = [\cdot]$ because $p \circ e = id$. Therefore 
     it is easy to see that $t$ is less than $t'$ for one preorders 
     implies that it is the same for the other one, using the 
     monotonicity properties of $p$ and $e$. The homomorphism 
     allows to lift the equalities on $\Nat$ to equalities on $\Tree_\Nat$.
\qed\end{proof}

Using free structures in the category 
of $\omega$CPPO it is possible to define 
a free preorder in an abstract way.

\begin{lemma}[Free interpretational preorder]
    \label{lem:freedomainpreorder}
    Given an inequational theory $\mathcal{T}$
    there exists a preorder $\sqeq_{\llbracket \mathcal{T} \rrbracket}$
    arising from an interpretation into a $\Sigma$-continuous algebra satisfying the 
    inequational theory $\mathcal{T}$, and containing 
    any preorder constructed using the same pattern.
\end{lemma}

\begin{proof}
    There is a free $\omega$CPPO for this inequational 
    theory \cite{abramsky1994}. Using Lemma \ref{lem:freealgebra}
    we have an associated preorder that is admissible and compositional.
    
    Let $D$ be another domain satisfying the formulas from $\mathcal{T}$,
    and an interpretation in this domain. Because of the 
    freeness property we know that this interpretation 
    can be factored out, which can be translated back into 
    an inclusion between the two preorders.
\qed\end{proof}



%
\section{Free preorders}

Given an inequational theory $\mathcal{T}$ one can always 
build some preorder $\sqeq_b$ corresponding to it as in Lemma 
\ref{lem:freedomainpreorder}.
This uses the previous part about denotational semantics 
and preorders built using them. However we saw 
that one could naturally build the smallest admissible,
compositional preorder extending $\sqeq$ without 
requiring such machinery by using Lemma \ref{lem:freepreo}. 
It is not known (yet) if the two constructions coincide,
however one inclusion is clear: the free preorder obtained 
using a denotational interpretation always contains 
the other one, defined as the intersection of all preorders 
satisfying admissibilty, compositionality and pointedness properties.
The difficulty is going to prove the equality between the free preorder 
(obtained via Lemma \ref{lem:freepreo}) and the other ones 
either obtained operationally or using the denotational free preorder.


\begin{lemma}[Angelic preorder]
    Let $\mathcal{T}_A$ be the following theory:
    \begin{equation*}
        \begin{array}{rl}
            a \orEff a &= a \\
            a \orEff b &= b \orEff a \\
            a \orEff (b \orEff c) &= (a \orEff b) \orEff c \\
            a \orEff b &\geq a
        \end{array}
    \end{equation*}


    The free preorder for the theory 
    $\mathcal{T}_A$ coincides 
    with the preorder $\sqeq_A$ defined 
    operationally by:

    \begin{equation*}
        t \sqeq_A t' 
        \iff
        \forall n \in \Nat, n \in t \implies n \in t'
    \end{equation*}

    And $\sqeq_A$ itself coincides with the denotational
    iterpretation in the Hoare powerdomain for $\Nat$
    (the free join-semilattice over $\Nat$).
\end{lemma}

\begin{proof}
    First, it is easy to see that indeed the denotational
    and the operational preorders are equal using 
    a characterisation of the Hoare powerdomain \cite{abramsky1994}.

    Thus, $\sqeq_A$ is admissible and compositional. Moreover,
    $\sqeq_A$ satisfies the inequational 
    theory $\mathcal{T}$ in a trivial way.

    Therefore, $(\sqeq_{\mathcal{T}_A}) \subseteq (\sqeq_A)$.

    The other inclusion is done in two steps, following the 
    idea that the free preorder on $\mathcal{T}$ is
    the "admissible extension" of the free preorder on finite 
    trees.
    
    Let $t$ and $t'$ be two finite trees, we can use 
    idempotence to prove that $t$ is equivalent to 
    a complete binary tree with depth $n$, and 
    the same for $t'$. Given two complete binary
    trees of equal depth, using first associativity 
    and commutativity one can reorder the leaves, and 
    then using only $\bot \sqeq_{\mathcal{T}_A} x$ for any
    $x$ and compositionality construct 
    a finite proof that $t \sqeq_{\mathcal{T}_A} t'$.
    Indeed, for every leaf in $t$, the corresponding 
    leaf in $t'$ is either the same one or $\bot$.

    
    If two infinite trees have their results included,
    then it suffices to notice that any finite approximation 
    of the first has it's results included in a finite 
    approximation of the second. With this and the result 
    on finite tree, admissibilty can be applied to 
    show that $t \sqeq_{\mathcal{T}_A} t'$. 
\qed\end{proof}


\begin{lemma}[Demonic preorder]
    \label{lem:demopreo}
    Let $\mathcal{T}_D$ be the following theory:
    \begin{equation*}
        \begin{array}{rl}
            a \orEff a &= a \\
            a \orEff b &= b \orEff a \\
            a \orEff (b \orEff c) &= (a \orEff b) \orEff c \\
            a \orEff b &\leq a
        \end{array}
    \end{equation*}

    The free preorder for the theory $\mathcal{T}_D$ coincides with 
    the preorder $\sqeq_D$ given by:

    \begin{equation*}
        t \sqeq_D t' \iff 
        \begin{cases}
            \bot \in t' \implies \bot \in t \\
            n    \in t' \implies n \in t    \\
            \bot \in t
        \end{cases}
    \end{equation*}

    And $\sqeq_D$ itself coincides with the denotational
    iterpretation in the Smyth powerdomain for $\Nat$
    (the free meet-semilattice over $\Nat$).
\end{lemma}

\begin{proof}
    The proof follows the exact same pattern as the previous one,
    except that infinite trees are all equivalent to $\bot$ 
    in this theory, and therefore the infinite case is even 
    simpler.
\qed\end{proof}


\begin{lemma}[Probabilistic preorder]
    \label{lem:probpreo}
    We use the notation $\oplus$ for the infix notation 
    of $\prEff$ to ease lecture.
    Let $\mathcal{T}_P$ be the following probabilistic theory 
    \cite{heckmann1994probabilistic}:

    \begin{equation*}
        \begin{array}{rl}
            a \oplus a &= a \\
            a \oplus b &= b \oplus a \\
            (a \oplus b) \oplus (c \oplus d) &= (a \oplus c) \oplus (b \oplus d) \\
            a \oplus b \leq b &\implies a \leq b
        \end{array}
    \end{equation*}

    The free preorder for the theory $\mathcal{T}_P$
    coincides with the preorder defined by:
    
    \begin{equation*}
        t \sqeq_P t' \iff 
        \forall n \in \Nat, \nu (t) \leq \nu (t')
    \end{equation*}

    Where $\nu(t)$ corresponds to the probability distribution 
    over $\Nat$ encoded by the tree $t$.

    And $\sqeq_P$ itself coincides with the denotational
    iterpretation in the probabilistic powerdomain for $\Nat$
    (the free full kegelspitze over $\Nat$ \cite{KeimelP2016}).

\end{lemma}

\begin{proof}
    As before, the first inclusion is not a problem, and on finite 
    trees transforming into a complete binary tree with 
    an ordering on leafs is 
    possible to prove the inequality using some normal form.

    Let $t$ and $t'$ be two possibly infinite trees 
    of natural numbers such that the leaves are bounded by
    a constant $C$ and assume that $\nu(t) \leq \nu(t')$.
    We can make a case distinction:

    \begin{enumerate}
        \item There exists a finite tree $t_i'$
            such that $\nu (t_i') = \nu (t')$ and $t_i' \sqeq t'$.

        \item For any finite tree $t_i'$ such that $t_i' \sqeq t'$, there exists 
            an $n$ such that $\nu (t_i')(n) < \nu (t')(n)$.
    \end{enumerate}

    Now let's take two approximating chains of finite trees 
    for $t$ and $t'$, and build the chain $t_i \oplus t_i'$.
    
    We know that $\nu (t_i \oplus t_i') \leq \nu (t')$ by 
    some simple calculation. Now, assume we are in the first 
    case, then we have a finite approximating tree $t_j'$ with $j > i$ such that 
    $\nu (t_i \oplus t_i') \leq \nu (t_j')$. In the second case,
    we know that there is an $n$ such that the inequality is strict: 
    but because $\nu$ is scott-continuous, there is a $j > i$ such that 
    $\nu (t_i \oplus t_i') (n) < \nu (t_j')$. Because the support 
    is finite, we know that we can take the maximum of such $j$'s 
    and have a finite tree $t_j'$ such that $\nu (t_i \oplus t_i') (n) \leq \nu
    (t_j')$.

    But all the trees in this last equation are finite, and therefore 
    they are true for $\sqeq_{\mathcal{T}_P}$.

    \begin{equation*}
        \forall i \in \mathbb{N}, \exists j > i, 
        t_i \oplus t_i' \sqeq_{\mathcal{T}_P} t_j'
    \end{equation*}

    Using admissibility, we can now conclude:

    \begin{equation*}
        t \oplus t' \sqeq_{\mathcal{T}_P} t'
    \end{equation*}

    But then we can deduce that $t \sqeq_{\mathcal{T}_P} t'$ using 
    the last axiom of $\mathcal{T}$.


    To extend this result to infinite support, it suffices to 
    use the family of substitutions:

    \begin{equation*}
        \sigma_k (i) = \begin{cases}
            \bot & \text{ when } i > k \\
            i    & \text{ otherwise } 
        \end{cases}
    \end{equation*}

    If $\nu (t) \leq \nu (t')$ then we know that 
    for all $k$, $\nu (t\sigma_k) \leq \nu (t'\sigma_k)$
    and they have finite support, therefore 
    $t \sigma_k \sqeq_{\mathcal{T}_P} t' \sigma_k$.
    To conclude it suffices to see that $t = \sqcup_k t\sigma_k$
    and use admissibility.

\qed\end{proof}


%\section{Preorder for combined non determinism and probabilities}

In this section we are going to fix a specific signature $\Sigma$
containing two binary operators \texttt{pr} and \texttt{or}. The two
operators are used to model a language where both probabilistic choice 
and non-determinism coexist. Combining these specific effects has been 
the subject of numerous papers, and even when restricting ourselves to the 
denotational setting, the work of Regina Tix on powercones \cite{tix2009semantic} 
continued afterwards by Plotkin and Keimel \cite{KeimelP2016} on Kegelspitze
shows the interest of such combination.
A more functional version of theses domains can also be found in the work of Jean-Goubault Larrecq 
\cite{JGL-mscs16}.

Given the recent developments of the denotational interpretation, 
it would be perfectly fine to use an interpretation 
to define our basic preorder $\sqeq_b$. But as we are studying 
operational semantics, it is more consistent to use an operational 
definition of the said preorder.

In the case of combined (demonic) non-determinism and probabilities we can define 
the preorder $\sqeq_b$ on trees over natural numbers in a simple and
effective way. We consider a tree as a Markov Decision Process 
and given an cost function from $\Nat \to \overline{\mathbb{R}_+}$
we find a strategy for the $\orEff$ nodes that minimizes the average
cost of the tree. 

A tree $t$ is under a tree $t'$ for this preorder when for any 
cost function, the minimal expected cost for $t$ is under the 
minimal expected cost for $t'$.

In order to formalise this intuition, while not diving into the details of 
all the specifications one can 
define $\mathcal{S}$ to be the space representing 
the set of strategies. Given a strategy $s \in \mathcal{S}$ and 
a tree $t \in \Tree_\Nat$, one can build $t*s$ the application of 
the strategy to the tree, that builds a new \emph{probability} tree, that 
is \emph{without} $\orEff$ nodes. Given a probability tree $t$, and a 
cost function $h$ from $\Nat$ to $\overline{\mathbb{R}_+}$, one can define the expected cost $\mathbb{E} (h(t))$.
It is now possible to write the following definition for the operational 
preorder:

\begin{equation*}
    t \sqeq_b t' \iff 
    \forall h : \Nat \to \overline{\mathbb{R}_+}, 
    \inf_{s \in \mathcal{S}} \mathbb{E}(h(t*s)) \leq 
    \inf_{s \in \mathcal{S}} \mathbb{E}(h(t'*s))
\end{equation*}

Admissibility is going to rely on the \emph{Scott-continuity} 
of the following function:

\begin{equation*}
    t \mapsto \left(h \mapsto \inf_{s \in \mathcal{S}} \mathbb{E} (h
(t*s))\right)
\end{equation*}

Compositionality on the other hand is going to rely on an elementary 
decomposition result of the above function.

\subsection{Formalisation of strategies} 

The goal of this subsection is to formalise strategies and 
to build a \emph{continuous} function from $\Tree_\Nat \times \mathcal{S}$
to $\Tree_\Nat$ (without $\orEff$ nodes) corresponding to the 
application of a strategy to a tree.
The first step to define strategies is to define the space 
$\{ L; R\}$ of directions that can be taken in the tree (left and right).
Using this set, one can define the set of paths as $\{ L; R \}^*$. A 
strategy can be seen as a function that takes a path as input (position 
in a tree) and outputs the direction to choose.

\begin{definition}[Strategies]
     The set of strategies is 
     defined by $\mathcal{S} = \{ L;R\}^* \to \{ L; R\}$.
     This set is compact for the topology induced by the 
     following distance on strategies:

     \begin{equation*}
         d(s_1,s_2) = \inf_{n \geq 1} \left\{ \frac{1}{n} ~|~ \forall p \in \{ L;R\}^*,
                                         |p| \leq n, s_1(p) = s_2(p) \right\}
     \end{equation*}
\end{definition}

\begin{proof}
    The function $d$ is clearly a distance and a converging sequence 
    of strategies can be extracted from any other one by the usual argument 
    (looking at the output on the empty sequence, and then extracting
    infinitly many elements from the sequence outputting the same thing etc.).
\end{proof}

Now that we have theses functions, we can define the space 
of strategies evaluations that are \emph{continuous}
functions from the cantor space to trees. 

This construction may seem unnatural, but it is the simplest 
way to get continuity of strategy application to a tree.

\begin{definition}[Application space]
    The space $\mathcal{C}(S,\Tree_\Nat)$ is a $\Sigma$-continuous 
    algebra where:

    \begin{equation*}
        \orEff (e_1,e_2) (s) = 
        \begin{cases}
            e_1 (s \circ L) & \text{when } s(\varepsilon) = L \\
            e_1 (s \circ R) & \text{when } s(\varepsilon) = R 
        \end{cases}
    \end{equation*}

    And where:
    \begin{equation*}
        \prEff (e_1,e_2) (s) = \prEff (e_1 (s \circ L), e_2 (s \circ R)) 
    \end{equation*}

    With $L$ and $R$ being considered interpreted as the function 
    from paths to paths that appends the corresponding letter in front of 
    the path.
\end{definition}

Now that we have this space of function, we can define the 
homomorphism that maps a tree to a function from strategies 
to trees, that is the curryfication of the function we want to build.

\begin{definition}[Strategy Application]
    We define the function $*$ from $\Tree_\Nat$ to $\mathcal{C}(\mathcal{S},\Tree_\Nat)$
    as the unique homomorphism such that:

    \begin{enumerate}
        \item $n* = s \mapsto n$
        \item $\bot* = s \mapsto \bot$
    \end{enumerate}
\end{definition}

Note that during this definition, we actually defined very precisely 
how strategies apply, and it is straightforward to check that 
application works as expected.

\begin{lemma}[Continuity]
    Given a tree $t$ and a strategy $s$ we can 
    build a tree $(t*) (s)$. The function 
    $\operatorname{app}(t,s) = (t*)(s)$ written sometimes
    $(t*s)$ 
    is continuous 
    from $\Tree_\Nat \times \mathcal{S}$ to $\Tree_\Nat$.
\end{lemma}

\begin{proof}
    It is clear that $(t*s)$ is continuous in each of it's 
    arguments. Using the fact that $\Tree_\Nat$ is a 
    continuous domain we can use a well know fact 
    \cite{battenfeld2009two} (Lemma A.1) that allows us to conclude.
    \qed
\end{proof}


\subsection{Formalisation of probabilities}

Now that we can get a tree without \texttt{or} nodes,
we can define the probability space we are using, mainly 
infinite paths on binary trees, corresponding to real numbers.

\begin{definition}[Probability Space]
    We define the probability space $\Omega$
    to be $\{0,1\}^\mathbb{N}$, the 
    $\sigma$-algebra to be the Borel sets 
    and use the uniform probability measure on them.
\end{definition}

Any tree can then be turned into a random variable in 
a very natural way. 

\begin{definition}[Turning a tree into a random variable]
    Given a tree $t$ with only \texttt{or} nodes we 
    build a measurable function $va(t)$ from $\Omega$ to $\Tree_\Nat$:

    \begin{equation*}
        va(t)(p) = \begin{cases}
            n  & \text{ if there is a node } n \text{ on the path } p \\
            \bot & \text{ otherwise } 
        \end{cases}
    \end{equation*}
\end{definition}

%\begin{proof}
    %The random variable for a given tree is a measurable function 
    %in an obvious way.
%\end{proof}

\subsection{Construction of the preorder}

We now have all the constructions needed to define the preorder

\begin{definition}[Preorder]
    The preorder is defined by

    \begin{equation*}
        t \sqeq_b t' \iff \forall h : \Nat \to \overline{\mathbb{R}_+}, 
        \inf_{ s \in \mathcal{S}} F (t,s,h) \leq \inf_{s \in \mathcal{S}} F (t',s,h)
    \end{equation*}

    Where the function $F$ is defined as 

    \begin{equation*}
        F(t,s,h) = \mathbb{E}(h \circ va(t * s))
    \end{equation*}
\end{definition}

To prove the admissibility, it suffices to prove that the function 
is Scott-continuous, and we made sure that this fact is easy to prove.

\begin{lemma}[Scott-continuity]
    Given a cost function $h$, the function 
    $(t,s) \mapsto F(t,s,h)$ is continuous from $\Tree_\Nat \times
    \mathcal{S}$ with the product topology to $\overline{\mathbb{R}_+}$ with the 
    Scott topology.

    Moreover, given $t$ and $s$, the function $h \mapsto F(t,s,h)$ is monotone.
\end{lemma}

\begin{proof}
    We write $RV(X)$ for the space of random variables 
    that have values on $X$, that is measurable 
    functions from $\Omega$ to $X$. Moreover when 
    $X$ is $\mathbb{N}_\bot$ or $\overline{\mathbb{R}_+}$
    we can use the Scott topology on $X$ and notice that in both cases 
    measurable functions are closed under directed suprema 
    allowing us to consider $RV(\mathbb{N}_\bot)$ and $RV(\overline{\mathbb{R}_+})$ as 
    domains.

    \begin{center}
        \begin{tikzcd}
            \Tree_\Nat \arrow[r, "va"] & 
            RV( \mathbb{N}_\bot ) \arrow[r, "h \circ"] &
            RV(\overline{\mathbb{R}_+}) \arrow[r, "\mathbb{E}(\square)"] & 
            \overline{\mathbb{R}_+}
        \end{tikzcd}
    \end{center}

    This function is Scott-continuous as a composition of 
    Scott-continuous functions. Indeed $va$ is clearly Scott-continuous 
    by definition, left composition with a function is also
    Scott-continuous, and the expectancy is a monotone operator 
    that preserves directed suprema via the monotone convergence theorem. 

    We can then say that $F(t,s,h)$ is continuous at a fixed $h$
    because it is the composition of $(t*s)$ with a Scott-continuous 
    function.

    \qed
\end{proof}

Now the last bit is showing that taking the infimum gives 
a continuous function in $(s,t)$ that is still monotone on $h$.
This follows from a general result using the compactness of the 
set of strategies and the continuity of the functions 
\cite{AndreaShalk} (Theorem 7.31).

\begin{lemma}[Taking the infimum]
    \label{lem:mixedscottcontinuous}
    Fixing $h$ and $s$, 
    the function $t \mapsto \inf_s \mathbb{E} (h \circ va(t*s))$
    is Scott-continuous.
\end{lemma}

\begin{proof}
    It is possible to use a result from Andrea Shalk's thesis 
    \cite{AndreaShalk} (Theorem 7.31)
    to directly prove the result. Note that 
    the compactness of $S$ is a crucial hypothesis. 

    It is also possible to build a direct proof of Scott-continuity:
    first of all monotonicity is simply obtained because an infimum 
    of monotone functions is monotone. Then, the only result left 
    to prove is that given an ascending chain of trees $(t_i)_{i \in \mathbb{N}}$
    we have the following equation:

    \begin{equation*}
        \sup_{i \in \mathbb{N}} \inf_{s \in \mathcal{S}} F(t_i,s,h) = 
    \inf_{s \in \mathcal{S}} F (\sqcup_i t_i, s, h)
    \end{equation*}
    
    Because $F$ is Scott-continuous in $t$, we can rewrite the right hand 
    side of the equality, and obtain the following inequality: 

    \begin{equation*}
        \sup_{i \in \mathbb{N}} \inf_{s \in \mathcal{S}} F(t_i,s,h) \leq 
        \inf_{s \in \mathcal{S}} \sup_{i \in \mathbb{N}} F (t_i, s, h)
    \end{equation*}
    
    But given an index $i$ the infinum $\inf_{s \in \mathcal{S}} F(t_i,s,h)$ 
    can be obtained as the limit of $F(t_i,s_j,h)$ for a sequence $s_j$ of
    strategies.
    Because 
    $\mathcal{S}$ is compact, we can extract a converging sequence and 
    assume that $s_j$ converges to $s_\infty$ in $\mathcal{S}$, because 
    $F$ is continuous in $s$ for the Scott-topology on
    $\overline{\mathbb{R}_+}$, it is lower-semicontinuous in $s$ 
    for the usual topology on $\overline{\mathbb{R}_+}$ and therefore:

    \begin{equation*}
        \inf_s F(t_i, s, h) = \lim_i F(t_n, s_i, h) 
                            \geq F (t_n, s_\infty, h)
    \end{equation*}

    However, by definition this means that 
    $s_\infty$ realises the infimum. Therefore given any 
    number $i$ there exists a strategy $s_i$ that realises 
    the infimum $\inf_s F(t_i, s, h)$.

    Now it is possible to rewrite the right-hand part 
    of the equation we wanted to obtain:

    \begin{equation*}
        \sup_{i \in \mathbb{N}} \inf_{s \in \mathcal{S}} F(t_i,s,h) 
        = 
        \sup_{i \in \mathbb{N}}  F(t_i,s_i,h)
    \end{equation*}

    But using again the compactness of $\mathcal{S}$, we can 
    extract a sequence such that $(t_i,s_i) \longrightarrow (t,s_\infty)$
    and by using again the lower-semicontinuity of $F$ in $(t,s)$ we can 
    conclude:

    \begin{equation*}
        \sup_{i \in \mathbb{N}}  F(t_i,s_i,h) = \lim_i F(t_i, s_i, h) \geq
        F(t,s_\infty, h)
    \end{equation*}

    Now it is possible to obtain the desired inequality:

    \begin{equation*}
        \sup_{i \in \mathbb{N}} \inf_{s \in \mathcal{S}} F(t_i,s,h) 
        \geq 
        F(t,s_\infty, h)
        \geq 
        \inf_{s \in \mathcal{S}} F(t,s,h)
        = 
        \inf_{s \in \mathcal{S}} F (\sqcup_i t_i, s h)
    \end{equation*}
    \qed
\end{proof}

We can reuse the lemma \ref{lem:continuousadm} to deduce 
that the preorder is admissible. But about compositionality ?

\begin{lemma}[Decomposition]
    \label{lem:mixeddecomposition}
    Given a function $h$, a tree $t$ and a substitution $\sigma$,
    the following equality holds:
    \begin{equation*}
        \inf_s F(t\sigma ,s,h) = \inf_s F(t,s,h_\sigma)
    \end{equation*}
    Where
    \begin{equation*}
        h_\sigma (n) = \inf_s F(\sigma(n),s,h)
    \end{equation*}
\end{lemma}

\begin{proof}

    It is possible to enumerate the leaves of $t$, 
    and build a tree $t'$ where each leave is replaced 
    with its corresponding unique number. There is a substitution
    $\tau$ replacing the unique number by the corresponding value in $t$, 
    meaning that $t' \tau = t$.

    Now it is easy to see that $t \sigma = t' \tau \sigma$, where $\tau \sigma$
    is a substitution that specifically targets a unique leaf, and associates 
    a tree. By definition of $F$, it is possible to see that 
    $F (t' \tau \sigma, s, h)$ is the expected value obtained 
    when considering the tree $t' \tau \sigma * s$ with the cost function $h$. 
    However the strategy $s$ can be decomposed into a "head"
    acting only on $t'$ called $s_h$, and "tails" acting on the trees added by $\tau \sigma$. 
    Because we numbered uniquely all leaves of $t'$, we can enumerate the
    corresponding strategies: for all leaf $i$ of $t'$ there exists 
    a unique corresponding part in $s$ that such that concerns positions 
    under it which is called $s_i$.

    It is then easy to check that $t' \tau \sigma * s$ is in fact 
    the same tree as $(t' * s_h) \xi$ where the substitution $\xi$ is 
    defined as:

    \begin{equation*}
        \xi (i) = i \tau \sigma * s_i 
    \end{equation*}

    Indeed, one is applying the substitutions and then the strategy, 
    whereas the second one is applying the strategy on the different 
    parts of the tree, and then is doing a substitution.

    \begin{align*}
        F (t' \tau \sigma, s, h) &= 
        \mathbb{E} \left( h \circ va ((t' * s) \xi) \right)\\
        &= 
        \sum_{n \in \text{leaves}(t')} 
        \mathbb{P}( va(t' * s) = n ) \times \mathbb{E}\left( h \circ va (\xi
                                                        (n)) \right) \\
        &=
        \sum_{n \in \text{leaves}(t')} 
        \mathbb{P}( va(t' * s) = n ) \times \mathbb{E}\left( h \circ va (
        \sigma(\tau(n)) * s_n) \right) \\
        &\geq 
        \sum_{n \in \text{leaves}(t')} 
        \mathbb{P}( va(t' * s) = n ) \times \inf_{s \in \mathcal{S}} \mathbb{E}\left( h \circ va (
        \sigma(\tau(n)) * s) \right) \\
        &= 
        F (t' \tau, s, h_\sigma)
    \end{align*}

    Therefore we do have:

    \begin{equation*}
        \inf_{s \in \mathcal{S}} F(t\sigma, s, h) \geq \inf_{s \in \mathcal{S}} F (t,s,h_\sigma)
    \end{equation*}

    Using the fact that the infimum on $s$ are obtained 
    for a given strategy (compacteness) we can find $s_t$
    such that the infimum is obtained:
    
    \begin{equation*}
        \inf_s F (t,s,h_\sigma) = F(t,s_t, h_\sigma)
    \end{equation*}

    Using the same compactness property, we can find a strategy 
    $s_n$ to obtain the infimum for all  
    $h_\sigma (n)$. 

    \begin{align*}
        \inf_{s \in \mathcal{S}} F (t' \tau, s, h_\sigma) &= 
        \sum_{n \in \text{leaves}(t')} 
        \mathbb{P}( va(t' * s_t) = n ) \times \inf_{s \in \mathcal{S}} \mathbb{E}\left( h \circ va (
        \sigma(\tau(n)) * s) \right) \\
        &= 
        \sum_{n \in \text{leaves}(t')} 
        \mathbb{P}( va(t' * s_t) = n ) \times \mathbb{E}\left( h \circ va (
        \sigma(\tau(n)) * s_n) \right) \\
        &= 
        \mathbb{E} \left( h \circ va ((t' * s_t) \xi) \right)\\
        &= 
        F (t' \tau \sigma, \hat{s}, h)
    \end{align*}
    
    Where $\hat{s}$ is the startegy obtained by replacing in $s_t$ the subtree  
    determined by \emph{unique number} $n$
    in $t'$ by the strategy $s_{\tau(n)}$. 

    \qed
\end{proof}

\begin{lemma}[Compositionality]
    \label{lem:operiscomp}
    The preorder $\sqeq_b$ is compositional
\end{lemma}

    \begin{proof}
        Let $t \sqeq_b t'$ and $\sigma \sqeq_b \sigma'$ pointwise.
        We want to show that $t\sigma \sqeq_b t' \sigma'$. 

        Let $h : \Nat \to \overline{\mathbb{R}_+}$ be a function.
        We can see that $\sigma \sqeq_b \sigma'$ pointwise 
        implies that $h_\sigma \sqeq_b h_{\sigma'}$ pointwise 
        by definition of $h_\sigma$.


        \begin{align*}
            \inf_{s \in \mathcal{S}} F(t\sigma, s, h)  
            &= \inf_{s \in \mathcal{S}} F(t, s, h_\sigma) 
            &\text{ lemma \ref{lem:mixeddecomposition} } \\
            &\leq \inf_{s \in \mathcal{S}} F(t, s, h_{\sigma'})  & \text{ monotonicity in } h\\
            &\leq \inf_{s \in \mathcal{S}} F(t',s, h_{\sigma'}) & \text{ monotonicity in } t \\
            &= \inf_{s \in \mathcal{S}} F(t'\sigma', s,h )
            &\text{ lemma \ref{lem:mixeddecomposition} } \\
        \end{align*}
    \qed
    \end{proof}

\subsection{The angelic case}

Everything can be adapted to the angelic case by replacing 
$\inf$ with $\sup$ in the definition of the preorder. In fact,
admissibility even becomes easier because suprema commute, 
but the general proof can be almost copy-pasted.

\subsection{Link with interpretations}

All the work that has been done uses domain theory in a very 
simple and specific way, and it cannot be totally avoided 
because of the nature of the property that has to be proven.

But as we discussed before defining the operational preorder,
the full power of the denotational interpretation could have 
been used \emph{as is} using our results about denotational
interpretations and in fact would lead to the exact same preorder.
This gives another way to see things: starting from 
an interpretation that abstracts all the difficulties,
and then finding a direct way of expressing this 
abstract notion as a more "concrete" property of the tree.
Note that the proof that the two preorders coincide 
is almost exactly the same as the proof stating that 
the "handmade" one is well-behaved.


Expanding the definitions of the power Kegelspitze in our 
simple setting \cite{KeimelP2016} gives us the following domain: 
first by constructing the subprobability Kegelspitze
$\mathcal{V}_{\leq 1}(\mathbb{N}_\bot)$, and then by taking the domain 
formed using the \emph{convex}, \emph{Scott-compact}, \emph{upward-closed}, \emph{non-empty}
subsets of $\mathcal{V}_{\leq 1}(\mathbb{N}_\bot)$ ordered by reverse inclusion.
This domain is isomorphic to a functional one \cite{KeimelP2016} defined by 
the domain of strongly non-expansive superlinear functions taking 
Scott-continuous functions from $\mathbb{N}_\bot$ to $\overline{\mathbb{R}_+}$
as input and producing an element of $\overline{\mathbb{R}_+}$.

The link is clear when looking at the operationally defined preorder,
because the function $t \mapsto (h \mapsto \inf_{s \in \mathcal{S}} F(t,s,h))$
is actually mapping $\Tree_\Nat$ into the functional Kegelspitze for 
combined demonic non-determinism and probabilistic choice. The equality between 
the interpretation into this domain and the operational preorder 
can be proven on natural numbers easily, and then extended using the fact that 
the operational function we defined is a $\Sigma$-continuous algebra
homomorphism.
In fact, we are going to write $\llbracket \cdot \rrbracket$
to denote the map $t \mapsto h \mapsto \inf_{s \in \mathcal{S}} \mathbb{E}(h
(t*s))$.


\subsection{Link with the free preorder}

One can consider the preorder $\sqeq_\mathcal{T}$
freely generated (using lemma \ref{lem:freepreo}) 
by some horn-clause inequational theory $\mathcal{T}$.

The inequational theory for demonic non-determinism
is well known and we call it $\mathcal{D}$, 
a good choice of axiomatisation $\mathcal{P}$ for the probability 
can be found in \cite{heckmann1994probabilistic}.
This theory has the advantage of not explicitly referring  
to real numbers and is therefore perfectly suited to 
our setting.

Given the two theories, and following the laws from 
\cite{KeimelP2016} we can build the combined theory
of demonic non-determinism and probabilities by adding 
a distributivity axiom as seen 
in Figure \ref{fig:mixtheory}.

\begin{figure}[h]
    \begin{equation*}
        \begin{array}{lrl}
            \mathcal{P} & a \oplus a &= a \\
                        & a \oplus b &= b \oplus a \\
                        & (a \oplus b) \oplus (c \oplus d) &= (a \oplus c) \oplus (b \oplus d) \\
                        & a \oplus b \leq b &\implies a \leq b  \\
            %\hline
            \\
            \mathcal{D} & a \sqcap a &= a \\
                        & a \sqcap b &= b \sqcap a \\
                        & (a \sqcap b) \sqcap c &= a \sqcap (b \sqcap c) \\
                        & a \sqcap b &\leq a \\ 
            \\
            %\hline 
            \text{Distributivity}
            & (a \sqcap b) \oplus c &= (a \oplus c) \sqcap (b \oplus c)
        \end{array}
    \end{equation*}
    \caption{Inequational theory for mixed probability and demonic non
    determinism}
    \label{fig:mixtheory}
\end{figure}

We know that each part corresponds to the usual 
preorders for probability (resp. non determinism) using 
Lemma \ref{lem:probpreo} (resp. Lemma \ref{lem:demopreo}), 
and we are going to show the following theorem.

\begin{theorem}[Equality of preorders]
The 
free preorder of the joint theories as described in Figure \ref{fig:mixtheory}
is the one that was obtained 
operationally which is itself equal to the preorder 
obtained by the interpretation inside the free 
algebra for this theory in $\omega$CPPO.
\end{theorem}

\begin{proof}
    
    It is easily checked that the operational preorder satisfies the theory 
    $\mathcal{T}$ and we already have a proof stating its admissibility and 
    compositionality using Lemma \ref{lem:operiscomp}. Therefore, the free preorder 
    $\sqeq_\mathcal{T}$ is contained in the operational preorder $\sqeq_b$.

    We are going to prove the other inclusion in two steps. The first one 
    is restricting ourselves to trees with a \emph{finite} number of
    $\sqcap$-nodes. Indeed, they can be put into the following form 
    using only the distributivity laws from $\mathcal{T}$:

    \begin{center}
        \begin{tikzpicture}[scale=0.5]
            \path[fill,black] (0,0) -- (1, -3) -- (-1, -3) -- (0,0) ;
            \path[fill,gray ] (1, -3) -- (2, -5) -- (1, -5) ;
            \path[fill,gray ] (-1, -3) -- (-2, -5) -- (-1, -5) ;
            \path[fill,gray ] (0, -3) -- (-0.5, -5) -- (0.5, -5) ;

            \draw [decorate,decoration={brace},xshift=4pt,yshift=0pt]
                (2,0) -- (2,-3) node [black,midway,xshift=1cm] 
                {$\sqcap$ nodes};
            \draw [decorate,decoration={brace},xshift=4pt,yshift=0pt]
                (2,-3) -- (2,-5) node [black,midway,xshift=1cm] 
                {$\oplus$ nodes};
        \end{tikzpicture} 
    \end{center}

    Because we now that $\sqeq_b$ satisfies $\mathcal{T}$, we can 
    always transform an inequality $t \sqeq_b t'$ into an inequality 
    where both trees are of the previously defined shape: with a finite 
    heap of $\sqcap$-nodes and possibly infinite subtrees of $\oplus$-nodes.

    Assume that we are given two trees $t$ and $t'$ with a finite number 
    of $\sqcap$-nodes such that $t \sqeq_b t'$. It suffices to show that 
    the equivalent trees with a finite heap of $\sqcap$-nodes and subtrees 
    of $\oplus$-nodes are related for $\sqeq_\mathcal{T}$. But because 
    of the laws of the demonic-choice operator, it is enough to prove that 
    for all $\oplus$-nodes subtree of $t'$, there exists a $\oplus$-nodes
    subtree of $t$ that is under it for $\sqeq_\mathcal{T}$. This is illustrated 
    in the following figure, where arrows symbolises an inequality relation for
    $\sqeq_\mathcal{T}$:

    \begin{center}
        \begin{tikzpicture}[scale=0.5]
            \path[fill,black] (0,0) -- (1, -3) -- (-1, -3) -- (0,0) ;
            \path[fill,gray ] (1, -3) -- (2, -5) -- (1, -5) ;
            \path[fill,gray ] (-1, -3) -- (-2, -5) -- (-1, -5) ;
            \path[fill,gray ] (0, -3) -- (-0.5, -5) -- (0.5, -5) ;

            \draw (2.5,-2) node {$\sqeq_b$};

            \path[fill,black] (5,0) -- (6, -3) -- (4, -3) -- (5,0) ;
            \path[fill,gray ] (6, -3) -- (7, -5) -- (6, -5) ;
            \path[fill,gray ] (4, -3) -- (3, -5) -- (4, -5) ;
            \path[fill,gray ] (5, -3) -- (4.5, -5) -- (5.5, -5) ;


            \path[thick,->] (5,-5) edge [bend left=45] (1.5,-5) ;
            \path[thick,->] (6.5,-5) edge [bend left=45] (-1.5,-5) ;
            \path[thick,->] (3.5,-4) edge [bend right=45] (0,-4) ;
        \end{tikzpicture} 
    \end{center}
    
    However, it is not always possible to find a matching $\oplus$-subtree 
    in $t$ for all $\oplus$-subtrees in $t'$. The key to overcome this 
    issue is to notice  that $a \sqcap (a \oplus b)
    \sqcap b$ is provably equivalent to $a \sqcap b$ in the theory
    $\mathcal{T}$ \cite{mislove2004axioms}. 
    An $\oplus$-combination of a set of trees $t_1, \dots, t_n$
    is determined by a tree (possibly infinite) 
    with only $\oplus$-nodes and leaves numbered from $1$ to $n$,
    and the combination itself is obtained by substituting the leaf $i$
    in this tree by the tree $t_i$.
    The previous result with only two trees can be extended to prove that any
    $\oplus$-combination
    of $\oplus$-subtrees of $t$ can be added as a new
    $\oplus$-subtree while being provably equivalent in $\mathcal{T}$.



    Thus, it is enough to find for all $\oplus$-subtrees in $t'$ an 
    $\oplus$-combination of $\oplus$-subtrees in $t$ that is 
    under it for $\sqeq_\mathcal{T}$ as illustrated in the following 
    Figure:

    \begin{center}
        \begin{tikzpicture}[scale=0.5]
            \path[fill,black] (0,0) -- (1, -3) -- (-1, -3) -- (0,0) ;
            \path[fill,gray ] (1, -3) -- (2, -5) -- (1, -5) ;
            \path[fill,gray ] (-1, -3) -- (-2, -5) -- (-1, -5) ;
            \path[fill,gray ] (0, -3) -- (-0.5, -5) -- (0.5, -5) ;

            \draw (2.5,-2) node {$\sqeq_b$};

            \path[fill,black] (5,0) -- (6, -3) -- (4, -3) -- (5,0) ;
            \path[fill,gray ] (6, -3) -- (7, -5) -- (6, -5) ;
            \path[fill,gray ] (4, -3) -- (3, -5) -- (4, -5) ;
            \path[fill,gray ] (5, -3) -- (4.5, -5) -- (5.5, -5) ;


            \path[thick] (5,-5) edge [bend left=45] (3,-6) ;
            \path[thick,->] (3,-6) edge [bend left=45] (1.5,-5) ;
            \path[thick,->] (3,-6) edge [bend left=45] (-1.5,-5) ;
            \draw[fill,white] (3,-6) circle (0.5) ;
            \draw (3,-6) circle (0.5) ;
            \draw (3,-6) node {$\oplus$};
        \end{tikzpicture} 
    \end{center}

    In order to continue the proof, we use the fact that the operational 
    preorder on trees containing only $\oplus$-nodes can be captured 
    by translating the tree into a linear function from $(\mathbb{N}_\bot \to
    \overline{\mathbb{R}_+})$ to $\overline{\mathbb{R}_+}$ because 
    there is no $\sqcap$-node and therefore changing strategy does not 
    change the outcome, which allows us to remove the infimum. Using this 
    correspondence, we are going to prove two separate results:

    \begin{enumerate}[(i)]
        \item If $t \sqeq_b t'$ both with a finite number of $\sqcap$-nodes,
            then the $\oplus$-subtrees extracted from them 
            can be translated into linear functions from $(\mathbb{N}_\bot \to
            \overline{\mathbb{R}_+}) \to \overline{\mathbb{R}_+}$. 
            For all $\oplus$-subtree 
            of $t'$, the corresponding linear function is above 
            a \emph{convex combination} of linear functions corresponding 
            to $\oplus$-subtrees of $t$.

        \item A linear combination of functions corresponding to 
            $\oplus$-subtrees can be obtained as the linear 
            function corresponding to an $\oplus$-combination 
            of the said subtrees.

        \item If a probabilistic tree $t$ is under a probabilistic tree $t'$ 
            for $\sqeq_b$ then it is provable that they are also related for
            $\sqeq_\mathcal{T}$.
    \end{enumerate}

    It is clear that if both steps are proven, given an $\oplus$-subtree of $t'$
    it suffices to use (i) 
    to find a convex combination of linear functions that is under its
    corresponding linear function, and then use (ii) to convert this 
    back into trees combined with $\oplus$, allowing us to conclude using 
    the last point (iii).

    The last point (iii) is referring to the completeness result already proven 
    using only the probabilistic choice in Lemma \ref{lem:probpreo}, because 
    the theory $\mathcal{T}$ is an extension of the theory containing only 
    probabilistic choice, the proofs transported.

    Proving the point (ii) is simply proving that any distribution of probability 
    over a finite subset $U$ of $\mathbb{N}$
    can be obtained using an infinite binary tree of $\oplus$-nodes and 
    leaves in $U \cup \{ \bot \}$.

    The proof of point (i) is more complex, and requires several technical 
    tools \cite{JGL-mscs16}. We fix an arbitrary $\oplus$-subtree of $t'$ and
    write $L$ for the associated linear function, we also write 
    $L_1, \dots, L_n$ the linear functions associated to $\oplus$-subtrees of
    $t$. By definition of the operational preorder $\sqeq_b$ and because 
    of the shape of the tree we know that:

    \begin{equation*}
        \forall h : \mathbb{N} \to \overline{\mathbb{R}_+}, 
            \min (L_1 (h), \dots, L_n (h)) \leq L(h)
    \end{equation*}

    We want to prove that this implies that there exists a convex combination 
    of $L_1, \dots, L_n$ that is under $L$ for any $h$. We write 
    $A = \operatorname{Conv}\left( \{ M \text{ linear } ~|~ \exists i, M \geq
    L_i \} \right)$, and our goal is equivalent to proving that $L \in A$. Assuming it 
    is not the case, we will find a contraction when looking at the 
    set of linear functions under $L$: $B = \{ M \text{ linear } ~|~ M \leq L
    \}$.

    Indeed, because $L \not \in A$, it is easy to show that $B \cap A =
    \emptyset$. However, referring to \cite{JGL-mscs16} and \cite{KeimelP2016} 
    both are \emph{convex} and \emph{closed} set of a locally-convex topological 
    cone: the cone of linear functions from $(\mathbb{N}_\bot \to
    \overline{\mathbb{R}_+})$ to $\overline{\mathbb{R}_+}$. It is therefore 
    possible to use a convex separation theorem \cite{JGL-mscs16} and build 
    a linear functional $\Lambda$ from the cone of linear functions to
    $\overline{\mathbb{R}_+}$ such that there exists a real number $r$
    and:
    \begin{align*}
        \forall M \in B, \Lambda (M) < 1 < r \\
        \forall M \in A, \Lambda (M) > r > 1
    \end{align*}

    Now using the Schröder-Simpson representation theorem \cite{SchroderS06}, 
    there exists a test function $h$ such that for all linear functions $M$
    in our cone:
    \begin{equation*}
        \Lambda (M) = M(h)
    \end{equation*}

    But we can therefore deduce that for all $i$, $\Lambda (L_i) > r > 1 > \Lambda (L)$
    meaning that $L_i (h) > L(h)$ for all $i$. This is a contradiction with 
    the inequality derived from the operational preorder, and therefore $L \in
    A$, allowing us to conclude.

    \vspace{2em}

    It is now proven that if $t$ and $t'$ have a finite number of $\sqcap$-nodes 
    then $t \sqeq_b t'$ implies $t \sqeq_\mathcal{T} t'$. To extend this result 
    to trees with an infinite number of $\sqcap$-nodes, we are going to use 
    the admissibility rule of $\sqeq_\mathcal{T}$. Assume $t$ and $t'$ are 
    two trees such that $t \sqeq_b t'$, we can build two ascending chains 
    of \emph{finite} trees $(t_i)$ and $(t_i')$ having as least upper bounds 
    respectively $t$ and $t'$. The goal is to prove that $t \sqeq_\mathcal{T}
    t'$, however, it is not possible to directly prove that $t_i
    \sqeq_\mathcal{T} t_i'$ and conclude using admissibility.
    
    We are instead going to use several approximation steps to obtain
    the desired result. First of all, we are going to define 
    $a^n$ where $a$ is a tree and $n$ a natural number as follows:

    \begin{equation*}
        \begin{cases}
            a^0 &= \bot \\
            a^{n+1} &= a \oplus a^{n}
        \end{cases}
    \end{equation*}

    This operator is useful to obtain \emph{strict} inequalities at any test function $h$, 
    indeed it is easy to see that:

    \begin{equation*}
        \inf_{s \in \mathcal{S}} F(a^n,s,h) = 
        \inf_{s \in \mathcal{S}} \frac{2^n - 1}{2} F(a,s,h) 
        < 
        \inf_{s \in \mathcal{S}} F(a,s,h)
    \end{equation*}

    Moreover we showed that the function
    $\llbracket \cdot \rrbracket : t \mapsto \inf_{s \in S} F (t,s,h)$
    is Scott-continuous in $t$ in Lemma \ref{lem:mixedscottcontinuous}.
    
    It is now possible to look at the ascending chains, and 
    notice that:

    \begin{align*}
        \llbracket t_i \oplus (t_i')^n \rrbracket(h) &= \frac{\llbracket t_i
        \rrbracket(h) + \llbracket (t_i')^n \rrbracket (h)}{2} \\
        &< 
        \frac{\llbracket t_i \rrbracket(h) + \llbracket t_i' \rrbracket (h)}{2} \\
        &\leq \llbracket t' \rrbracket(h)
    \end{align*}
    
    Using the Scott-continuity of $\llbracket \cdot \rrbracket$ allows us to 
    rewrite it in the following way:

    \begin{equation*}
        \llbracket t_i \oplus (t_i')^n \rrbracket < \sup_i \llbracket t_i' \rrbracket
    \end{equation*}

    Therefore for all $i$ there exists a $j>i$ such that:

    \begin{equation*}
        \llbracket t_i \oplus (t_i')^n \rrbracket < \llbracket t_j' \rrbracket
    \end{equation*}

    Because all trees in this equation are finite, and the inequality is
    equivalent to being related for $\sqeq_b$, we can use the previous 
    result on trees with finite number of $\sqcap$-nodes to conclude:

    \begin{equation*}
        t_i \oplus (t_i')^n \sqeq_\mathcal{T} t_j'
    \end{equation*}

    Now using the admissibility property while fixing $n$, we can deduce that:

    \begin{equation*}
        t \oplus (t')^n \sqeq_\mathcal{T} t'
    \end{equation*}

    Using again the admissibility property, this time on the sequence $(t')^n$ 
    we can obtain:

    \begin{equation*}
        t \oplus (t') \sqeq_\mathcal{T} t'
    \end{equation*}

    Now using the least-fixed-point rule from the probability theory we can 
    conclude:

    \begin{equation*}
        t \sqeq_\mathcal{T} t'
    \end{equation*}
    \qed
\end{proof}

\subsection{Counterexample for the simpler preorder}

From a domain perspective, it was natural to consider 
the functional domain using arbitrary test functions 
$h$ from $\mathbb{N}_\bot$ to $\overline{\mathbb{R}_+}$
because they arise from a bidual construction \cite{JGL-mscs16}. 
From a Markov Decision Process point of view 
however, we can ask ourselves if restricting 
to \emph{characteristic functions} is enough. 
Indeed, this leads us with a better understanding 
of the process, where the objective is simply
to \emph{avoid} one set with the highest probability 
possible. The preorder would then become:

\begin{equation*}
    t \sqeq_b t' \iff \forall U \subseteq \mathbb{N}, 
    \inf_{ s \in \mathcal{S}} F (t,s, \chi_U ) \leq \inf_{s \in \mathcal{S}}
    F (t',s, \chi_U )
\end{equation*}

This idea, while appealing is not possible because 
the following inequation cannot be true in general
(see \cite{Mislove2000} and \cite{mislove2004axioms}) without 
implying that probabilistic choice behaves like demonic choice: 

\begin{equation*}
    (a +_p b) \sqcup c \neq 
    (a \sqcup c) +_p (b \sqcup c)
\end{equation*}

However, for our preorder using only sets, the 
two trees in Figure \ref{fig:counterexampletree} that 
are obtained by replacing $a,b,c$ 
by $1,2,3$ are behaving in the same way for any 
characteristic function: the lowest probability to 
be inside a given subset of $\{1,2,3\}$ is the same 
for both trees.

\begin{figure}[h]

    \begin{center}
        \begin{tikzpicture}
            \node [circle,draw] (z){$\orEff$}
                child { 
                    node [circle,draw] (a) {$\prEff$}
                    child { node[circle,draw] (b) {$1$} } 
                    child { node[circle,draw] (c) {$2$} }
                }
                child {
                    node [circle,draw] (d) {$3$}    
                };
        \end{tikzpicture}
        \hspace{2em}
        \begin{tikzpicture}[level 1/.style={sibling distance=3cm},
                            level 2/.style={sibling distance=1.5cm}]
            \node [circle,draw] (z){$\prEff$}
                child { 
                    node [circle,draw] (h) {$\orEff$}
                    child { node[circle,draw] (b) {$1$} } 
                    child { node[circle,draw] (c) {$3$} }
                }
                child {
                    node [circle,draw] (g) {$\orEff$}
                    child { node[circle,draw] (e) {$2$} } 
                    child { node[circle,draw] (f) {$3$} }
                };
        \end{tikzpicture}
    \end{center}

    \begin{equation*}
        \orEff (\prEff (1,2), 3) \quad \quad \prEff (\orEff (1,3), \orEff (2,3))
    \end{equation*}
    \caption{The two trees from the counterexample}
    \label{fig:counterexampletree}
\end{figure}

We can see that the two trees in Figure \ref{fig:counterexampletree} 
have the same interpretation in terms of characteristic functions,
and the result from \cite{Mislove2000} shows us that a reasonable 
preorder shouldn't equate them. But we have an explicit 
cost function $h$ that shows that the interpretation in terms of 
expectancies is different for the two trees:

\begin{equation*}
    h (i) = \begin{cases}
        1 & \text{ when } i = 1 \\
        8 & \text{ when } i = 2 \\
        2 & \text{ when } i = 3 \\
        0 & \text{ otherwise } 
    \end{cases}
\end{equation*}

This function can distinguish between the two trees 
because the least expected cost for the first one is $1/4$
where the least expected cost for the second one is $3/16$.

The choice of costs for $h$ is not random: we can divide 
everything by $8$ to get costs of the form $1/2^i$ 
and build an actual substitution $\sigma$ where $\sigma(i)$ 
is the tree where the probability to get $1$ is $h(i)/8$.


%\section{Conclusion and future work}

The first part of this paper was about extending the results 
of Alex Simpson, Patricia Johann and Janis Voigtländer \cite{gom}
to a call-by-value setting. By doing so, some properties of 
the basic preorder $\sqeq_b$ were developed and a direct link
to denotational semantics has been made. This is a first step 
in a better understanding of basic preorders and the generality 
of the method itself. 

Some generic theorems and sanity checks have been proven 
abstractly for the logical relation and the contextual preorder
arising from $\sqeq_b$, allowing to decline them with any 
effect signature $\Sigma$. 

The ability to automatically build free preorders for an equational 
theory $\mathcal{T}$ was studied, and compared to the operational 
and denotational method in the case of probability, non-determinism
and the combination of both, showing how robust this general setting 
is.

\vspace{1em}

Obviously, the study of the requirements for a basic preorder 
are not enough, and it would be interesting to see how far we 
can weaken the admissibility property and still get results.
For instance, countable non-determinism [CIT] does \emph{not} have 
this admissibility property, but is believed to still fit in 
our setting. On the other side of the requirements, compositionality 
could be better understood using sets of observations as done in 
\cite{gom} or looking at the continuity properties of the 
monadic multiplication on trees.

It would then be interesting to try and generalise the class of 
effects, be it by adding more effects known to be algebraic, 
or blatantly non algebraic effects such as exception handling 
that would require changing the language itself, but still 
be captured by the same general method.

This work can be extended to a richer type system in the obvious 
way, and even recursive types are not an issue using step-indexing 
techniques [CITE] or by defining the projections in the language [CITE].

It could be interesting to generalise the notion of $\top\top$-closure 
to metric relations 
and talk about the distance of terms, see the work of Ugo and Raphaëlle
(and more)

Finally, the study of mixed non-determinism and probability is 
done very briefly in this paper, and there would be a lot more 
to talk about. For instance, how does the functional representation 
evolves when combining angelic,demonic and probability operators ?
Can our result be safely transported into a realisability world 
such as done in the work by Niels ?


%\section{Migrating material}

It is worth noticing that the lifting operation $f \mapsto \hat{f}$ is
a continuous functional. A consequence of this is the following monotonicity 
property of the lifting.

\begin{lemma}[Order preserving lift]
    \label{lem:orderpreservinglift}
    If $\sigma_1 : \Nat \to A$ and $\sigma_2 : \Nat \to A$ 
    are two functions such that $\sigma_1 \leq \sigma_2$ pointwise,
    then $\hat{\sigma_1} \leq \hat{\sigma_2}$ pointwise.
\end{lemma}

We can already see the usefulness of this definition 
when looking at the construction of substitution:
given abstractly, in full generality and 
guaranteed to be continuous without any hassle. 

\begin{definition}[Substitution]
    Let $t \in \textnormal{Tree}_X$ 
    and $\sigma : X \to \textnormal{Tree}_Y$.
    Using the universal property, there exists a
    unique lift $\hat{\sigma} : \textnormal{Tree}_X \to \textnormal{Tree}_Y$,
    which is the substitution. We write $t\sigma$ as a shorthand for 
    $\hat{\sigma}t$.
\end{definition}


Using this new construction it is now possible  
to continue our definition of the operational semantics without 
interpreting effects by building an infinite tree
labeled by the effects encountered during evaluation.



% 

We view the fact that the use of quantitative properties is required to obtain a compositional preorder as being
a mathematical analogue of the situation found in the work of Kozen [[REF]] and McIver and Morgan [[REF]], where 
logics of quantitative properties are seen to be necessary to obtain compositional reasoning principles for 
probabilistic programs.

and given an cost function from $\Nat \to \overline{\mathbb{R}_+}$
we find a strategy for the $\orEff$ nodes that minimizes the average
cost of the tree. 

A tree $t$ is under a tree $t'$ for this preorder when for any 
cost function, the minimal expected cost for $t$ is under the 
minimal expected cost for $t'$.

In order to formalise this intuition, while not diving into the details of 
all the specifications one can 
define $\mathcal{S}$ to be the space representing 
the set of strategies. Given a strategy $s \in \mathcal{S}$ and 
a tree $t \in \Tree_\Nat$, one can build $t*s$ the application of 
the strategy to the tree, that builds a new \emph{probability} tree, that 
is \emph{without} $\orEff$ nodes. Given a probability tree $t$, and a 
cost function $h$ from $\Nat$ to $\overline{\mathbb{R}_+}$, one can define the expected cost $\mathbb{E} (h(t))$.
It is now possible to write the following definition for the operational 
preorder:

\begin{equation*}
    t \sqeq_b t' \iff 
    \forall h : \Nat \to \overline{\mathbb{R}_+}, 
    \inf_{s \in \mathcal{S}} \mathbb{E}(h(t*s)) \leq 
    \inf_{s \in \mathcal{S}} \mathbb{E}(h(t'*s))
\end{equation*}

Admissibility is going to rely on the \emph{Scott-continuity} 
of the following function:

\begin{equation*}
    t \mapsto \left(h \mapsto \inf_{s \in \mathcal{S}} \mathbb{E} (h
(t*s))\right)
\end{equation*}

Compositionality on the other hand is going to rely on an elementary 
decomposition result of the above function.

\subsection{Formalisation of strategies} 

The goal of this subsection is to formalise strategies and 
to build a \emph{continuous} function from $\Tree_\Nat \times \mathcal{S}$
to $\Tree_\Nat$ (without $\orEff$ nodes) corresponding to the 
application of a strategy to a tree.
The first step to define strategies is to define the space 
$\{ L; R\}$ of directions that can be taken in the tree (left and right).
Using this set, one can define the set of paths as $\{ L; R \}^*$. A 
strategy can be seen as a function that takes a path as input (position 
in a tree) and outputs the direction to choose.

\begin{definition}[Strategies]
     The set of strategies is 
     defined by $\mathcal{S} = \{ L;R\}^* \to \{ L; R\}$.
     This set is compact for the topology induced by the 
     following distance on strategies:

     \begin{equation*}
         d(s_1,s_2) = \inf_{n \geq 1} \left\{ \frac{1}{n} ~|~ \forall p \in \{ L;R\}^*,
                                         |p| \leq n, s_1(p) = s_2(p) \right\}
     \end{equation*}
\end{definition}

\begin{proof}
    The function $d$ is clearly a distance and a converging sequence 
    of strategies can be extracted from any other one by the usual argument 
    (looking at the output on the empty sequence, and then extracting
    infinitly many elements from the sequence outputting the same thing etc.).
\end{proof}

Now that we have theses functions, we can define the space 
of strategies evaluations that are \emph{continuous}
functions from the cantor space to trees. 

This construction may seem unnatural, but it is the simplest 
way to get continuity of strategy application to a tree.

\begin{definition}[Application space]
    The space $\mathcal{C}(S,\Tree_\Nat)$ is a $\Sigma$-continuous 
    algebra where:

    \begin{equation*}
        \orEff (e_1,e_2) (s) = 
        \begin{cases}
            e_1 (s \circ L) & \text{when } s(\varepsilon) = L \\
            e_1 (s \circ R) & \text{when } s(\varepsilon) = R 
        \end{cases}
    \end{equation*}

    And where:
    \begin{equation*}
        \prEff (e_1,e_2) (s) = \prEff (e_1 (s \circ L), e_2 (s \circ R)) 
    \end{equation*}

    With $L$ and $R$ being considered interpreted as the function 
    from paths to paths that appends the corresponding letter in front of 
    the path.
\end{definition}

Now that we have this space of function, we can define the 
homomorphism that maps a tree to a function from strategies 
to trees, that is the curryfication of the function we want to build.

\begin{definition}[Strategy Application]
    We define the function $*$ from $\Tree_\Nat$ to $\mathcal{C}(\mathcal{S},\Tree_\Nat)$
    as the unique homomorphism such that:

    \begin{enumerate}
        \item $n* = s \mapsto n$
        \item $\bot* = s \mapsto \bot$
    \end{enumerate}
\end{definition}

Note that during this definition, we actually defined very precisely 
how strategies apply, and it is straightforward to check that 
application works as expected.

\begin{lemma}[Continuity]
    Given a tree $t$ and a strategy $s$ we can 
    build a tree $(t*) (s)$. The function 
    $\operatorname{app}(t,s) = (t*)(s)$ written sometimes
    $(t*s)$ 
    is continuous 
    from $\Tree_\Nat \times \mathcal{S}$ to $\Tree_\Nat$.
\end{lemma}

\begin{proof}
    It is clear that $(t*s)$ is continuous in each of it's 
    arguments. Using the fact that $\Tree_\Nat$ is a 
    continuous domain we can use a well know fact 
    \cite{battenfeld2009two} (Lemma A.1) that allows us to conclude.
    \qed
\end{proof}


\subsection{Formalisation of probabilities}

Now that we can get a tree without \texttt{or} nodes,
we can define the probability space we are using, mainly 
infinite paths on binary trees, corresponding to real numbers.

\begin{definition}[Probability Space]
    We define the probability space $\Omega$
    to be $\{0,1\}^\mathbb{N}$, the 
    $\sigma$-algebra to be the Borel sets 
    and use the uniform probability measure on them.
\end{definition}

Any tree can then be turned into a random variable in 
a very natural way. 

\begin{definition}[Turning a tree into a random variable]
    Given a tree $t$ with only \texttt{or} nodes we 
    build a measurable function $va(t)$ from $\Omega$ to $\Tree_\Nat$:

    \begin{equation*}
        va(t)(p) = \begin{cases}
            n  & \text{ if there is a node } n \text{ on the path } p \\
            \bot & \text{ otherwise } 
        \end{cases}
    \end{equation*}
\end{definition}

%\begin{proof}
    %The random variable for a given tree is a measurable function 
    %in an obvious way.
%\end{proof}

\subsection{Construction of the preorder}

We now have all the constructions needed to define the preorder

\begin{definition}[Preorder]
    The preorder is defined by

    \begin{equation*}
        t \sqeq_b t' \iff \forall h : \Nat \to \overline{\mathbb{R}_+}, 
        \inf_{ s \in \mathcal{S}} F (t,s,h) \leq \inf_{s \in \mathcal{S}} F (t',s,h)
    \end{equation*}

    Where the function $F$ is defined as 

    \begin{equation*}
        F(t,s,h) = \mathbb{E}(h \circ va(t * s))
    \end{equation*}
\end{definition}

To prove the admissibility, it suffices to prove that the function 
is Scott-continuous, and we made sure that this fact is easy to prove.

\begin{lemma}[Scott-continuity]
    Given a cost function $h$, the function 
    $(t,s) \mapsto F(t,s,h)$ is continuous from $\Tree_\Nat \times
    \mathcal{S}$ with the product topology to $\overline{\mathbb{R}_+}$ with the 
    Scott topology.

    Moreover, given $t$ and $s$, the function $h \mapsto F(t,s,h)$ is monotone.
\end{lemma}

\begin{proof}
    We write $RV(X)$ for the space of random variables 
    that have values on $X$, that is measurable 
    functions from $\Omega$ to $X$. Moreover when 
    $X$ is $\mathbb{N}_\bot$ or $\overline{\mathbb{R}_+}$
    we can use the Scott topology on $X$ and notice that in both cases 
    measurable functions are closed under directed suprema 
    allowing us to consider $RV(\mathbb{N}_\bot)$ and $RV(\overline{\mathbb{R}_+})$ as 
    domains.

    \begin{center}
        \begin{tikzcd}
            \Tree_\Nat \arrow[r, "va"] & 
            RV( \mathbb{N}_\bot ) \arrow[r, "h \circ"] &
            RV(\overline{\mathbb{R}_+}) \arrow[r, "\mathbb{E}(\square)"] & 
            \overline{\mathbb{R}_+}
        \end{tikzcd}
    \end{center}

    This function is Scott-continuous as a composition of 
    Scott-continuous functions. Indeed $va$ is clearly Scott-continuous 
    by definition, left composition with a function is also
    Scott-continuous, and the expectancy is a monotone operator 
    that preserves directed suprema via the monotone convergence theorem. 

    We can then say that $F(t,s,h)$ is continuous at a fixed $h$
    because it is the composition of $(t*s)$ with a Scott-continuous 
    function.

    \qed
\end{proof}

Now the last bit is showing that taking the infimum gives 
a continuous function in $(s,t)$ that is still monotone on $h$.
This follows from a general result using the compactness of the 
set of strategies and the continuity of the functions 
\cite{AndreaShalk} (Theorem 7.31).

\begin{lemma}[Taking the infimum]
    \label{lem:mixedscottcontinuous}
    Fixing $h$ and $s$, 
    the function $t \mapsto \inf_s \mathbb{E} (h \circ va(t*s))$
    is Scott-continuous.
\end{lemma}

\begin{proof}
    It is possible to use a result from Andrea Shalk's thesis 
    \cite{AndreaShalk} (Theorem 7.31)
    to directly prove the result. Note that 
    the compactness of $S$ is a crucial hypothesis. 

    It is also possible to build a direct proof of Scott-continuity:
    first of all monotonicity is simply obtained because an infimum 
    of monotone functions is monotone. Then, the only result left 
    to prove is that given an ascending chain of trees $(t_i)_{i \in \mathbb{N}}$
    we have the following equation:

    \begin{equation*}
        \sup_{i \in \mathbb{N}} \inf_{s \in \mathcal{S}} F(t_i,s,h) = 
    \inf_{s \in \mathcal{S}} F (\sqcup_i t_i, s, h)
    \end{equation*}
    
    Because $F$ is Scott-continuous in $t$, we can rewrite the right hand 
    side of the equality, and obtain the following inequality: 

    \begin{equation*}
        \sup_{i \in \mathbb{N}} \inf_{s \in \mathcal{S}} F(t_i,s,h) \leq 
        \inf_{s \in \mathcal{S}} \sup_{i \in \mathbb{N}} F (t_i, s, h)
    \end{equation*}
    
    But given an index $i$ the infinum $\inf_{s \in \mathcal{S}} F(t_i,s,h)$ 
    can be obtained as the limit of $F(t_i,s_j,h)$ for a sequence $s_j$ of
    strategies.
    Because 
    $\mathcal{S}$ is compact, we can extract a converging sequence and 
    assume that $s_j$ converges to $s_\infty$ in $\mathcal{S}$, because 
    $F$ is continuous in $s$ for the Scott-topology on
    $\overline{\mathbb{R}_+}$, it is lower-semicontinuous in $s$ 
    for the usual topology on $\overline{\mathbb{R}_+}$ and therefore:

    \begin{equation*}
        \inf_s F(t_i, s, h) = \lim_i F(t_n, s_i, h) 
                            \geq F (t_n, s_\infty, h)
    \end{equation*}

    However, by definition this means that 
    $s_\infty$ realises the infimum. Therefore given any 
    number $i$ there exists a strategy $s_i$ that realises 
    the infimum $\inf_s F(t_i, s, h)$.

    Now it is possible to rewrite the right-hand part 
    of the equation we wanted to obtain:

    \begin{equation*}
        \sup_{i \in \mathbb{N}} \inf_{s \in \mathcal{S}} F(t_i,s,h) 
        = 
        \sup_{i \in \mathbb{N}}  F(t_i,s_i,h)
    \end{equation*}

    But using again the compactness of $\mathcal{S}$, we can 
    extract a sequence such that $(t_i,s_i) \longrightarrow (t,s_\infty)$
    and by using again the lower-semicontinuity of $F$ in $(t,s)$ we can 
    conclude:

    \begin{equation*}
        \sup_{i \in \mathbb{N}}  F(t_i,s_i,h) = \lim_i F(t_i, s_i, h) \geq
        F(t,s_\infty, h)
    \end{equation*}

    Now it is possible to obtain the desired inequality:

    \begin{equation*}
        \sup_{i \in \mathbb{N}} \inf_{s \in \mathcal{S}} F(t_i,s,h) 
        \geq 
        F(t,s_\infty, h)
        \geq 
        \inf_{s \in \mathcal{S}} F(t,s,h)
        = 
        \inf_{s \in \mathcal{S}} F (\sqcup_i t_i, s h)
    \end{equation*}
    \qed
\end{proof}

We can reuse the lemma \ref{lem:continuousadm} to deduce 
that the preorder is admissible. But about compositionality ?

\begin{lemma}[Decomposition]
    \label{lem:mixeddecomposition}
    Given a function $h$, a tree $t$ and a substitution $\sigma$,
    the following equality holds:
    \begin{equation*}
        \inf_s F(t\sigma ,s,h) = \inf_s F(t,s,h_\sigma)
    \end{equation*}
    Where
    \begin{equation*}
        h_\sigma (n) = \inf_s F(\sigma(n),s,h)
    \end{equation*}
\end{lemma}

\begin{proof}

    It is possible to enumerate the leaves of $t$, 
    and build a tree $t'$ where each leave is replaced 
    with its corresponding unique number. There is a substitution
    $\tau$ replacing the unique number by the corresponding value in $t$, 
    meaning that $t' \tau = t$.

    Now it is easy to see that $t \sigma = t' \tau \sigma$, where $\tau \sigma$
    is a substitution that specifically targets a unique leaf, and associates 
    a tree. By definition of $F$, it is possible to see that 
    $F (t' \tau \sigma, s, h)$ is the expected value obtained 
    when considering the tree $t' \tau \sigma * s$ with the cost function $h$. 
    However the strategy $s$ can be decomposed into a "head"
    acting only on $t'$ called $s_h$, and "tails" acting on the trees added by $\tau \sigma$. 
    Because we numbered uniquely all leaves of $t'$, we can enumerate the
    corresponding strategies: for all leaf $i$ of $t'$ there exists 
    a unique corresponding part in $s$ that such that concerns positions 
    under it which is called $s_i$.

    It is then easy to check that $t' \tau \sigma * s$ is in fact 
    the same tree as $(t' * s_h) \xi$ where the substitution $\xi$ is 
    defined as:

    \begin{equation*}
        \xi (i) = i \tau \sigma * s_i 
    \end{equation*}

    Indeed, one is applying the substitutions and then the strategy, 
    whereas the second one is applying the strategy on the different 
    parts of the tree, and then is doing a substitution.

    \begin{align*}
        F (t' \tau \sigma, s, h) &= 
        \mathbb{E} \left( h \circ va ((t' * s) \xi) \right)\\
        &= 
        \sum_{n \in \text{leaves}(t')} 
        \mathbb{P}( va(t' * s) = n ) \times \mathbb{E}\left( h \circ va (\xi
                                                        (n)) \right) \\
        &=
        \sum_{n \in \text{leaves}(t')} 
        \mathbb{P}( va(t' * s) = n ) \times \mathbb{E}\left( h \circ va (
        \sigma(\tau(n)) * s_n) \right) \\
        &\geq 
        \sum_{n \in \text{leaves}(t')} 
        \mathbb{P}( va(t' * s) = n ) \times \inf_{s \in \mathcal{S}} \mathbb{E}\left( h \circ va (
        \sigma(\tau(n)) * s) \right) \\
        &= 
        F (t' \tau, s, h_\sigma)
    \end{align*}

    Therefore we do have:

    \begin{equation*}
        \inf_{s \in \mathcal{S}} F(t\sigma, s, h) \geq \inf_{s \in \mathcal{S}} F (t,s,h_\sigma)
    \end{equation*}

    Using the fact that the infimum on $s$ are obtained 
    for a given strategy (compacteness) we can find $s_t$
    such that the infimum is obtained:
    
    \begin{equation*}
        \inf_s F (t,s,h_\sigma) = F(t,s_t, h_\sigma)
    \end{equation*}

    Using the same compactness property, we can find a strategy 
    $s_n$ to obtain the infimum for all  
    $h_\sigma (n)$. 

    \begin{align*}
        \inf_{s \in \mathcal{S}} F (t' \tau, s, h_\sigma) &= 
        \sum_{n \in \text{leaves}(t')} 
        \mathbb{P}( va(t' * s_t) = n ) \times \inf_{s \in \mathcal{S}} \mathbb{E}\left( h \circ va (
        \sigma(\tau(n)) * s) \right) \\
        &= 
        \sum_{n \in \text{leaves}(t')} 
        \mathbb{P}( va(t' * s_t) = n ) \times \mathbb{E}\left( h \circ va (
        \sigma(\tau(n)) * s_n) \right) \\
        &= 
        \mathbb{E} \left( h \circ va ((t' * s_t) \xi) \right)\\
        &= 
        F (t' \tau \sigma, \hat{s}, h)
    \end{align*}
    
    Where $\hat{s}$ is the startegy obtained by replacing in $s_t$ the subtree  
    determined by \emph{unique number} $n$
    in $t'$ by the strategy $s_{\tau(n)}$. 

    \qed
\end{proof}

\begin{lemma}[Compositionality]
    \label{lem:operiscomp}
    The preorder $\sqeq_b$ is compositional
\end{lemma}

    \begin{proof}
        Let $t \sqeq_b t'$ and $\sigma \sqeq_b \sigma'$ pointwise.
        We want to show that $t\sigma \sqeq_b t' \sigma'$. 

        Let $h : \Nat \to \overline{\mathbb{R}_+}$ be a function.
        We can see that $\sigma \sqeq_b \sigma'$ pointwise 
        implies that $h_\sigma \sqeq_b h_{\sigma'}$ pointwise 
        by definition of $h_\sigma$.


        \begin{align*}
            \inf_{s \in \mathcal{S}} F(t\sigma, s, h)  
            &= \inf_{s \in \mathcal{S}} F(t, s, h_\sigma) 
            &\text{ lemma \ref{lem:mixeddecomposition} } \\
            &\leq \inf_{s \in \mathcal{S}} F(t, s, h_{\sigma'})  & \text{ monotonicity in } h\\
            &\leq \inf_{s \in \mathcal{S}} F(t',s, h_{\sigma'}) & \text{ monotonicity in } t \\
            &= \inf_{s \in \mathcal{S}} F(t'\sigma', s,h )
            &\text{ lemma \ref{lem:mixeddecomposition} } \\
        \end{align*}
    \qed
    \end{proof}

\subsection{The angelic case}

Everything can be adapted to the angelic case by replacing 
$\inf$ with $\sup$ in the definition of the preorder. In fact,
admissibility even becomes easier because suprema commute, 
but the general proof can be almost copy-pasted.

\subsection{Link with interpretations}

All the work that has been done uses domain theory in a very 
simple and specific way, and it cannot be totally avoided 
because of the nature of the property that has to be proven.

But as we discussed before defining the operational preorder,
the full power of the denotational interpretation could have 
been used \emph{as is} using our results about denotational
interpretations and in fact would lead to the exact same preorder.
This gives another way to see things: starting from 
an interpretation that abstracts all the difficulties,
and then finding a direct way of expressing this 
abstract notion as a more "concrete" property of the tree.
Note that the proof that the two preorders coincide 
is almost exactly the same as the proof stating that 
the "handmade" one is well-behaved.


Expanding the definitions of the power Kegelspitze in our 
simple setting \cite{KeimelP2016} gives us the following domain: 
first by constructing the subprobability Kegelspitze
$\mathcal{V}_{\leq 1}(\mathbb{N}_\bot)$, and then by taking the domain 
formed using the \emph{convex}, \emph{Scott-compact}, \emph{upward-closed}, \emph{non-empty}
subsets of $\mathcal{V}_{\leq 1}(\mathbb{N}_\bot)$ ordered by reverse inclusion.
This domain is isomorphic to a functional one \cite{KeimelP2016} defined by 
the domain of strongly non-expansive superlinear functions taking 
Scott-continuous functions from $\mathbb{N}_\bot$ to $\overline{\mathbb{R}_+}$
as input and producing an element of $\overline{\mathbb{R}_+}$.

The link is clear when looking at the operationally defined preorder,
because the function $t \mapsto (h \mapsto \inf_{s \in \mathcal{S}} F(t,s,h))$
is actually mapping $\Tree_\Nat$ into the functional Kegelspitze for 
combined demonic non-determinism and probabilistic choice. The equality between 
the interpretation into this domain and the operational preorder 
can be proven on natural numbers easily, and then extended using the fact that 
the operational function we defined is a $\Sigma$-continuous algebra
homomorphism.
In fact, we are going to write $\llbracket \cdot \rrbracket$
to denote the map $t \mapsto h \mapsto \inf_{s \in \mathcal{S}} \mathbb{E}(h
(t*s))$.


\subsection{Link with the free preorder}

One can consider the preorder $\sqeq_\mathcal{T}$
freely generated (using lemma \ref{lem:freepreo}) 
by some horn-clause inequational theory $\mathcal{T}$.

The inequational theory for demonic non-determinism
is well known and we call it $\mathcal{D}$, 
a good choice of axiomatisation $\mathcal{P}$ for the probability 
can be found in \cite{heckmann1994probabilistic}.
This theory has the advantage of not explicitly referring  
to real numbers and is therefore perfectly suited to 
our setting.

Given the two theories, and following the laws from 
\cite{KeimelP2016} we can build the combined theory
of demonic non-determinism and probabilities by adding 
a distributivity axiom as seen 
in Figure \ref{fig:mixtheory}.

\begin{figure}[h]
    \begin{equation*}
        \begin{array}{lrl}
            \mathcal{P} & a \oplus a &= a \\
                        & a \oplus b &= b \oplus a \\
                        & (a \oplus b) \oplus (c \oplus d) &= (a \oplus c) \oplus (b \oplus d) \\
                        & a \oplus b \leq b &\implies a \leq b  \\
            %\hline
            \\
            \mathcal{D} & a \sqcap a &= a \\
                        & a \sqcap b &= b \sqcap a \\
                        & (a \sqcap b) \sqcap c &= a \sqcap (b \sqcap c) \\
                        & a \sqcap b &\leq a \\ 
            \\
            %\hline 
            \text{Distributivity}
            & (a \sqcap b) \oplus c &= (a \oplus c) \sqcap (b \oplus c)
        \end{array}
    \end{equation*}
    \caption{Inequational theory for mixed probability and demonic non
    determinism}
    \label{fig:mixtheory}
\end{figure}

We know that each part corresponds to the usual 
preorders for probability (resp. non determinism) using 
Lemma \ref{lem:probpreo} (resp. Lemma \ref{lem:demopreo}), 
and we are going to show the following theorem.

\begin{theorem}[Equality of preorders]
The 
free preorder of the joint theories as described in Figure \ref{fig:mixtheory}
is the one that was obtained 
operationally which is itself equal to the preorder 
obtained by the interpretation inside the free 
algebra for this theory in $\omega$CPPO.
\end{theorem}

\begin{proof}
    
    It is easily checked that the operational preorder satisfies the theory 
    $\mathcal{T}$ and we already have a proof stating its admissibility and 
    compositionality using Lemma \ref{lem:operiscomp}. Therefore, the free preorder 
    $\sqeq_\mathcal{T}$ is contained in the operational preorder $\sqeq_b$.

    We are going to prove the other inclusion in two steps. The first one 
    is restricting ourselves to trees with a \emph{finite} number of
    $\sqcap$-nodes. Indeed, they can be put into the following form 
    using only the distributivity laws from $\mathcal{T}$:

    \begin{center}
        \begin{tikzpicture}[scale=0.5]
            \path[fill,black] (0,0) -- (1, -3) -- (-1, -3) -- (0,0) ;
            \path[fill,gray ] (1, -3) -- (2, -5) -- (1, -5) ;
            \path[fill,gray ] (-1, -3) -- (-2, -5) -- (-1, -5) ;
            \path[fill,gray ] (0, -3) -- (-0.5, -5) -- (0.5, -5) ;

            \draw [decorate,decoration={brace},xshift=4pt,yshift=0pt]
                (2,0) -- (2,-3) node [black,midway,xshift=1cm] 
                {$\sqcap$ nodes};
            \draw [decorate,decoration={brace},xshift=4pt,yshift=0pt]
                (2,-3) -- (2,-5) node [black,midway,xshift=1cm] 
                {$\oplus$ nodes};
        \end{tikzpicture} 
    \end{center}

    Because we now that $\sqeq_b$ satisfies $\mathcal{T}$, we can 
    always transform an inequality $t \sqeq_b t'$ into an inequality 
    where both trees are of the previously defined shape: with a finite 
    heap of $\sqcap$-nodes and possibly infinite subtrees of $\oplus$-nodes.

    Assume that we are given two trees $t$ and $t'$ with a finite number 
    of $\sqcap$-nodes such that $t \sqeq_b t'$. It suffices to show that 
    the equivalent trees with a finite heap of $\sqcap$-nodes and subtrees 
    of $\oplus$-nodes are related for $\sqeq_\mathcal{T}$. But because 
    of the laws of the demonic-choice operator, it is enough to prove that 
    for all $\oplus$-nodes subtree of $t'$, there exists a $\oplus$-nodes
    subtree of $t$ that is under it for $\sqeq_\mathcal{T}$. This is illustrated 
    in the following figure, where arrows symbolises an inequality relation for
    $\sqeq_\mathcal{T}$:

    \begin{center}
        \begin{tikzpicture}[scale=0.5]
            \path[fill,black] (0,0) -- (1, -3) -- (-1, -3) -- (0,0) ;
            \path[fill,gray ] (1, -3) -- (2, -5) -- (1, -5) ;
            \path[fill,gray ] (-1, -3) -- (-2, -5) -- (-1, -5) ;
            \path[fill,gray ] (0, -3) -- (-0.5, -5) -- (0.5, -5) ;

            \draw (2.5,-2) node {$\sqeq_b$};

            \path[fill,black] (5,0) -- (6, -3) -- (4, -3) -- (5,0) ;
            \path[fill,gray ] (6, -3) -- (7, -5) -- (6, -5) ;
            \path[fill,gray ] (4, -3) -- (3, -5) -- (4, -5) ;
            \path[fill,gray ] (5, -3) -- (4.5, -5) -- (5.5, -5) ;


            \path[thick,->] (5,-5) edge [bend left=45] (1.5,-5) ;
            \path[thick,->] (6.5,-5) edge [bend left=45] (-1.5,-5) ;
            \path[thick,->] (3.5,-4) edge [bend right=45] (0,-4) ;
        \end{tikzpicture} 
    \end{center}
    
    However, it is not always possible to find a matching $\oplus$-subtree 
    in $t$ for all $\oplus$-subtrees in $t'$. The key to overcome this 
    issue is to notice  that $a \sqcap (a \oplus b)
    \sqcap b$ is provably equivalent to $a \sqcap b$ in the theory
    $\mathcal{T}$ \cite{mislove2004axioms}. 
    An $\oplus$-combination of a set of trees $t_1, \dots, t_n$
    is determined by a tree (possibly infinite) 
    with only $\oplus$-nodes and leaves numbered from $1$ to $n$,
    and the combination itself is obtained by substituting the leaf $i$
    in this tree by the tree $t_i$.
    The previous result with only two trees can be extended to prove that any
    $\oplus$-combination
    of $\oplus$-subtrees of $t$ can be added as a new
    $\oplus$-subtree while being provably equivalent in $\mathcal{T}$.



    Thus, it is enough to find for all $\oplus$-subtrees in $t'$ an 
    $\oplus$-combination of $\oplus$-subtrees in $t$ that is 
    under it for $\sqeq_\mathcal{T}$ as illustrated in the following 
    Figure:

    \begin{center}
        \begin{tikzpicture}[scale=0.5]
            \path[fill,black] (0,0) -- (1, -3) -- (-1, -3) -- (0,0) ;
            \path[fill,gray ] (1, -3) -- (2, -5) -- (1, -5) ;
            \path[fill,gray ] (-1, -3) -- (-2, -5) -- (-1, -5) ;
            \path[fill,gray ] (0, -3) -- (-0.5, -5) -- (0.5, -5) ;

            \draw (2.5,-2) node {$\sqeq_b$};

            \path[fill,black] (5,0) -- (6, -3) -- (4, -3) -- (5,0) ;
            \path[fill,gray ] (6, -3) -- (7, -5) -- (6, -5) ;
            \path[fill,gray ] (4, -3) -- (3, -5) -- (4, -5) ;
            \path[fill,gray ] (5, -3) -- (4.5, -5) -- (5.5, -5) ;


            \path[thick] (5,-5) edge [bend left=45] (3,-6) ;
            \path[thick,->] (3,-6) edge [bend left=45] (1.5,-5) ;
            \path[thick,->] (3,-6) edge [bend left=45] (-1.5,-5) ;
            \draw[fill,white] (3,-6) circle (0.5) ;
            \draw (3,-6) circle (0.5) ;
            \draw (3,-6) node {$\oplus$};
        \end{tikzpicture} 
    \end{center}

    In order to continue the proof, we use the fact that the operational 
    preorder on trees containing only $\oplus$-nodes can be captured 
    by translating the tree into a linear function from $(\mathbb{N}_\bot \to
    \overline{\mathbb{R}_+})$ to $\overline{\mathbb{R}_+}$ because 
    there is no $\sqcap$-node and therefore changing strategy does not 
    change the outcome, which allows us to remove the infimum. Using this 
    correspondence, we are going to prove two separate results:

    \begin{enumerate}[(i)]
        \item If $t \sqeq_b t'$ both with a finite number of $\sqcap$-nodes,
            then the $\oplus$-subtrees extracted from them 
            can be translated into linear functions from $(\mathbb{N}_\bot \to
            \overline{\mathbb{R}_+}) \to \overline{\mathbb{R}_+}$. 
            For all $\oplus$-subtree 
            of $t'$, the corresponding linear function is above 
            a \emph{convex combination} of linear functions corresponding 
            to $\oplus$-subtrees of $t$.

        \item A linear combination of functions corresponding to 
            $\oplus$-subtrees can be obtained as the linear 
            function corresponding to an $\oplus$-combination 
            of the said subtrees.

        \item If a probabilistic tree $t$ is under a probabilistic tree $t'$ 
            for $\sqeq_b$ then it is provable that they are also related for
            $\sqeq_\mathcal{T}$.
    \end{enumerate}

    It is clear that if both steps are proven, given an $\oplus$-subtree of $t'$
    it suffices to use (i) 
    to find a convex combination of linear functions that is under its
    corresponding linear function, and then use (ii) to convert this 
    back into trees combined with $\oplus$, allowing us to conclude using 
    the last point (iii).

    The last point (iii) is referring to the completeness result already proven 
    using only the probabilistic choice in Lemma \ref{lem:probpreo}, because 
    the theory $\mathcal{T}$ is an extension of the theory containing only 
    probabilistic choice, the proofs transported.

    Proving the point (ii) is simply proving that any distribution of probability 
    over a finite subset $U$ of $\mathbb{N}$
    can be obtained using an infinite binary tree of $\oplus$-nodes and 
    leaves in $U \cup \{ \bot \}$.

    The proof of point (i) is more complex, and requires several technical 
    tools \cite{JGL-mscs16}. We fix an arbitrary $\oplus$-subtree of $t'$ and
    write $L$ for the associated linear function, we also write 
    $L_1, \dots, L_n$ the linear functions associated to $\oplus$-subtrees of
    $t$. By definition of the operational preorder $\sqeq_b$ and because 
    of the shape of the tree we know that:

    \begin{equation*}
        \forall h : \mathbb{N} \to \overline{\mathbb{R}_+}, 
            \min (L_1 (h), \dots, L_n (h)) \leq L(h)
    \end{equation*}

    We want to prove that this implies that there exists a convex combination 
    of $L_1, \dots, L_n$ that is under $L$ for any $h$. We write 
    $A = \operatorname{Conv}\left( \{ M \text{ linear } ~|~ \exists i, M \geq
    L_i \} \right)$, and our goal is equivalent to proving that $L \in A$. Assuming it 
    is not the case, we will find a contraction when looking at the 
    set of linear functions under $L$: $B = \{ M \text{ linear } ~|~ M \leq L
    \}$.

    Indeed, because $L \not \in A$, it is easy to show that $B \cap A =
    \emptyset$. However, referring to \cite{JGL-mscs16} and \cite{KeimelP2016} 
    both are \emph{convex} and \emph{closed} set of a locally-convex topological 
    cone: the cone of linear functions from $(\mathbb{N}_\bot \to
    \overline{\mathbb{R}_+})$ to $\overline{\mathbb{R}_+}$. It is therefore 
    possible to use a convex separation theorem \cite{JGL-mscs16} and build 
    a linear functional $\Lambda$ from the cone of linear functions to
    $\overline{\mathbb{R}_+}$ such that there exists a real number $r$
    and:
    \begin{align*}
        \forall M \in B, \Lambda (M) < 1 < r \\
        \forall M \in A, \Lambda (M) > r > 1
    \end{align*}

    Now using the Schröder-Simpson representation theorem \cite{SchroderS06}, 
    there exists a test function $h$ such that for all linear functions $M$
    in our cone:
    \begin{equation*}
        \Lambda (M) = M(h)
    \end{equation*}

    But we can therefore deduce that for all $i$, $\Lambda (L_i) > r > 1 > \Lambda (L)$
    meaning that $L_i (h) > L(h)$ for all $i$. This is a contradiction with 
    the inequality derived from the operational preorder, and therefore $L \in
    A$, allowing us to conclude.

    \vspace{2em}

    It is now proven that if $t$ and $t'$ have a finite number of $\sqcap$-nodes 
    then $t \sqeq_b t'$ implies $t \sqeq_\mathcal{T} t'$. To extend this result 
    to trees with an infinite number of $\sqcap$-nodes, we are going to use 
    the admissibility rule of $\sqeq_\mathcal{T}$. Assume $t$ and $t'$ are 
    two trees such that $t \sqeq_b t'$, we can build two ascending chains 
    of \emph{finite} trees $(t_i)$ and $(t_i')$ having as least upper bounds 
    respectively $t$ and $t'$. The goal is to prove that $t \sqeq_\mathcal{T}
    t'$, however, it is not possible to directly prove that $t_i
    \sqeq_\mathcal{T} t_i'$ and conclude using admissibility.
    
    We are instead going to use several approximation steps to obtain
    the desired result. First of all, we are going to define 
    $a^n$ where $a$ is a tree and $n$ a natural number as follows:

    \begin{equation*}
        \begin{cases}
            a^0 &= \bot \\
            a^{n+1} &= a \oplus a^{n}
        \end{cases}
    \end{equation*}

    This operator is useful to obtain \emph{strict} inequalities at any test function $h$, 
    indeed it is easy to see that:

    \begin{equation*}
        \inf_{s \in \mathcal{S}} F(a^n,s,h) = 
        \inf_{s \in \mathcal{S}} \frac{2^n - 1}{2} F(a,s,h) 
        < 
        \inf_{s \in \mathcal{S}} F(a,s,h)
    \end{equation*}

    Moreover we showed that the function
    $\llbracket \cdot \rrbracket : t \mapsto \inf_{s \in S} F (t,s,h)$
    is Scott-continuous in $t$ in Lemma \ref{lem:mixedscottcontinuous}.
    
    It is now possible to look at the ascending chains, and 
    notice that:

    \begin{align*}
        \llbracket t_i \oplus (t_i')^n \rrbracket(h) &= \frac{\llbracket t_i
        \rrbracket(h) + \llbracket (t_i')^n \rrbracket (h)}{2} \\
        &< 
        \frac{\llbracket t_i \rrbracket(h) + \llbracket t_i' \rrbracket (h)}{2} \\
        &\leq \llbracket t' \rrbracket(h)
    \end{align*}
    
    Using the Scott-continuity of $\llbracket \cdot \rrbracket$ allows us to 
    rewrite it in the following way:

    \begin{equation*}
        \llbracket t_i \oplus (t_i')^n \rrbracket < \sup_i \llbracket t_i' \rrbracket
    \end{equation*}

    Therefore for all $i$ there exists a $j>i$ such that:

    \begin{equation*}
        \llbracket t_i \oplus (t_i')^n \rrbracket < \llbracket t_j' \rrbracket
    \end{equation*}

    Because all trees in this equation are finite, and the inequality is
    equivalent to being related for $\sqeq_b$, we can use the previous 
    result on trees with finite number of $\sqcap$-nodes to conclude:

    \begin{equation*}
        t_i \oplus (t_i')^n \sqeq_\mathcal{T} t_j'
    \end{equation*}

    Now using the admissibility property while fixing $n$, we can deduce that:

    \begin{equation*}
        t \oplus (t')^n \sqeq_\mathcal{T} t'
    \end{equation*}

    Using again the admissibility property, this time on the sequence $(t')^n$ 
    we can obtain:

    \begin{equation*}
        t \oplus (t') \sqeq_\mathcal{T} t'
    \end{equation*}

    Now using the least-fixed-point rule from the probability theory we can 
    conclude:

    \begin{equation*}
        t \sqeq_\mathcal{T} t'
    \end{equation*}
    \qed
\end{proof}

\subsection{Counterexample for the simpler preorder}

From a domain perspective, it was natural to consider 
the functional domain using arbitrary test functions 
$h$ from $\mathbb{N}_\bot$ to $\overline{\mathbb{R}_+}$
because they arise from a bidual construction \cite{JGL-mscs16}. 
From a Markov Decision Process point of view 
however, we can ask ourselves if restricting 
to \emph{characteristic functions} is enough. 
Indeed, this leads us with a better understanding 
of the process, where the objective is simply
to \emph{avoid} one set with the highest probability 
possible. The preorder would then become:

\begin{equation*}
    t \sqeq_b t' \iff \forall U \subseteq \mathbb{N}, 
    \inf_{ s \in \mathcal{S}} F (t,s, \chi_U ) \leq \inf_{s \in \mathcal{S}}
    F (t',s, \chi_U )
\end{equation*}

This idea, while appealing is not possible because 
the following inequation cannot be true in general
(see \cite{Mislove2000} and \cite{mislove2004axioms}) without 
implying that probabilistic choice behaves like demonic choice: 

\begin{equation*}
    (a +_p b) \sqcup c \neq 
    (a \sqcup c) +_p (b \sqcup c)
\end{equation*}

However, for our preorder using only sets, the 
two trees in Figure \ref{fig:counterexampletree} that 
are obtained by replacing $a,b,c$ 
by $1,2,3$ are behaving in the same way for any 
characteristic function: the lowest probability to 
be inside a given subset of $\{1,2,3\}$ is the same 
for both trees.

\begin{figure}[h]

    \begin{center}
        \begin{tikzpicture}
            \node [circle,draw] (z){$\orEff$}
                child { 
                    node [circle,draw] (a) {$\prEff$}
                    child { node[circle,draw] (b) {$1$} } 
                    child { node[circle,draw] (c) {$2$} }
                }
                child {
                    node [circle,draw] (d) {$3$}    
                };
        \end{tikzpicture}
        \hspace{2em}
        \begin{tikzpicture}[level 1/.style={sibling distance=3cm},
                            level 2/.style={sibling distance=1.5cm}]
            \node [circle,draw] (z){$\prEff$}
                child { 
                    node [circle,draw] (h) {$\orEff$}
                    child { node[circle,draw] (b) {$1$} } 
                    child { node[circle,draw] (c) {$3$} }
                }
                child {
                    node [circle,draw] (g) {$\orEff$}
                    child { node[circle,draw] (e) {$2$} } 
                    child { node[circle,draw] (f) {$3$} }
                };
        \end{tikzpicture}
    \end{center}

    \begin{equation*}
        \orEff (\prEff (1,2), 3) \quad \quad \prEff (\orEff (1,3), \orEff (2,3))
    \end{equation*}
    \caption{The two trees from the counterexample}
 %   \label{fig:counterexampletree}
\end{figure}

We can see that the two trees in Figure \ref{fig:counterexampletree} 
have the same interpretation in terms of characteristic functions,
and the result from \cite{Mislove2000} shows us that a reasonable 
preorder shouldn't equate them. But we have an explicit 
cost function $h$ that shows that the interpretation in terms of 
expectancies is different for the two trees:

\begin{equation*}
    h (i) = \begin{cases}
        1 & \text{ when } i = 1 \\
        8 & \text{ when } i = 2 \\
        2 & \text{ when } i = 3 \\
        0 & \text{ otherwise } 
    \end{cases}
\end{equation*}

This function can distinguish between the two trees 
because the least expected cost for the first one is $1/4$
where the least expected cost for the second one is $3/16$.

The choice of costs for $h$ is not random: we can divide 
everything by $8$ to get costs of the form $1/2^i$ 
and build an actual substitution $\sigma$ where $\sigma(i)$ 
is the tree where the probability to get $1$ is $h(i)/8$.




\end{document}
