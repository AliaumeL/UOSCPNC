
\documentclass[a4paper,UKenglish]{lipics-v2018}
%This is a template for producing LIPIcs articles. 
%See lipics-manual.pdf for further information.
%for A4 paper format use option "a4paper", for US-letter use option "letterpaper"
%for british hyphenation rules use option "UKenglish", for american hyphenation rules use option "USenglish"
% for section-numbered lemmas etc., use "numberwithinsect"

\usepackage{microtype}%if unwanted, comment out or use option "draft"
\usepackage{tikz}
\usetikzlibrary{arrows,positioning,decorations.pathreplacing,automata,shadows}
\usepackage{tikz-cd}
\usepackage{stmaryrd}
\usepackage{bussproofs}
\usepackage{todonotes}

%\graphicspath{{./graphics/}}%helpful if your graphic files are in another directory

\bibliographystyle{plainurl}% the recommnded bibstyle


% Syntax 
%\newcommand{\Nat}{\textnormal{Nat}}
\newcommand{\Nat}{\mathbb{N}}

\newcommand{\Succ}{\operatorname{S}}
\newcommand{\Zero}{\textnormal{Z}}
\newcommand{\Fix}{\operatorname{fix}}
\newcommand{\Unroll}{\operatorname{unroll}}
\newcommand{\Pos}{\operatorname{Pos}}
\newcommand{\In}{\operatorname{in}}
\newcommand{\Inr}{\operatorname{inl}}
\newcommand{\Inl}{\operatorname{inr}}
\newcommand{\Out}{\operatorname{out}}
\newcommand{\Ret}{\operatorname{return}}
\newcommand{\Bind}[3]{\operatorname{let} #2 \Leftarrow #1 \operatorname{in} #3}
\newcommand{\lcase}[3]{\operatorname{case} #1 \operatorname{of} \, \Zero \Rightarrow #2; \Succ(x) \Rightarrow #3}
\newcommand{\scase}[3]{\operatorname{case} #1 \operatorname{of} \, \Inl(x) \Rightarrow #2; \Inr (x) \Rightarrow #3}
\newcommand{\pcase}[2]{\operatorname{case} #1 \operatorname{of} \, (x,y) \Rightarrow #2}

% Trees

\newcommand{\Trees}{\textnormal{Trees}}
\newcommand{\Tree}{\textnormal{Trees}} %%%!!!

\newcommand{\Treeleq}{\sqsubseteq}

\newcommand{\Basicleq}{\preccurlyeq}

% Results
\newcommand{\res}{\operatorname{results}}

% Effects 
\newcommand{\prEff}{\operatorname{\textsf{pr}}}
\newcommand{\orEff}{\operatorname{\textsf{or}}}
\newcommand{\exEff}{\operatorname{\textsf{raise}}}

\newcommand{\prnd}{\text{pr/nd}}
\newcommand{\pr}{\text{pr}}
\newcommand{\nd}{\text{nd}}
\newcommand{\ang}{\text{ang}}
\newcommand{\dem}{\text{dem}}
\newcommand{\prang}{\text{pr/ang}}
\newcommand{\prdem}{\text{pr/dem}}

\newcommand{\Op}{\text{op}}
\newcommand{\Den}{\text{den}}
\newcommand{\Ax}{\text{ax}}

% Denotational

\newcommand{\Sem}[1]{\llbracket #1 \rrbracket}

\newcommand{\Set}{\mathbf{Set}}
\newcommand{\ContAlg}{\mathbf{ContAlg}}

\newcommand{\wCPO}{\mathbf{{\omega}CPO}}

\newcommand{\Conv}{\text{conv}}

% Axiomatic

\newcommand{\Vars}{\mathsf{Vars}}

\newcommand{\Iterx}[1]{(1\!-\!2^{- #1})x}
\newcommand{\Iter}[2]{(1\!-\!2^{- #2})#1}


% Reduction
\newcommand{\reduces}{\rightarrowtail}
\newcommand{\reductionPair}[2]{#1 \leadsto #2}
\newcommand{\sqeq}{\sqsubseteq}


% Mixed non-determinism
\newcommand{\drop}{\operatorname{drop}}
\newcommand{\odd}{\operatorname{odd}}
\newcommand{\even}{\operatorname{even}}

% Theorem environments
\theoremstyle{plain}
\newtheorem{proposition}[theorem]{Proposition}

\title{Basic operational preorders  for algebraic effects in general, and for
combined probability and nondeterminism in particular}

\titlerunning{Basic operational preorders for probability and nondeterminism}%optional, please use if title is longer than one line

\author{Aliaume Lopez}{\'Ecole Normale Sup\'erieure Paris-Saclay\\{Universit\'e Paris-Saclay, France}}{aliaume.lopez@ens-paris-saclay.fr}{}{}%mandatory, please use full name; only 1 author per \author macro; first two parameters are mandatory, other parameters can be empty.

\author{Alex Simpson}{Faculty of Mathematics and Physics\\{University of Ljubljana, Slovenia}}{Alex.Simpson@fmf.uni-lj.si}{}{}



\authorrunning{A. Lopez and A. Simpson}%mandatory. First: Use abbreviated first/middle names. Second (only in severe cases): Use first author plus 'et. al.'

\Copyright{Aliaume Lopez and Alex Simpson}%mandatory, please use full first names. LIPIcs license is "CC-BY";  http://creativecommons.org/licenses/by/3.0/

\subjclass{\ccsdesc[500]{Theory of computation~Operational semantics}; \ccsdesc[300]{Theory of computation~Denotational semantics}; \ccsdesc[300]{Theory of computation~Axiomatic semantics}}% mandatory: Please choose ACM 2012 classifications from https://www.acm.org/publications/class-2012 or https://dl.acm.org/ccs/ccs_flat.cfm . E.g., cite as "General and reference $\rightarrow$ General literature" or \ccsdesc[100]{General and reference~General literature}. 

\keywords{contextual equivalence, algebraic effects, operational semantics, domain theory, nondeterminism, probabilistic choice, Markov decision process}%mandatory

\category{}%optional, e.g. invited paper

\relatedversion{}%optional, e.g. full version hosted on arXiv, HAL, or other respository/website

\supplement{}%optional, e.g. related research data, source code, ... hosted on a repository like zenodo, figshare, GitHub, ...

\funding{}%optional, to capture a funding statement, which applies to all authors. Please enter author specific funding statements as fifth argument of the \author macro.

\acknowledgements{We thank Gordon Plotkin, Matija Pretnar  and Niels Voorneveld for helpful discussions.}%optional

%Editor-only macros:: begin (do not touch as author)%%%%%%%%%%%%%%%%%%%%%%%%%%%%%%%%%%
\EventEditors{Dan Ghica and Achim Jung}
\EventNoEds{2}
\EventLongTitle{27th EACSL Annual Conference on Computer Science Logic (CSL 201}
\EventShortTitle{CSL 2018}
\EventAcronym{CSL}
\EventYear{2018}
\EventDate{September 4--7, 2018}
\EventLocation{Birmingham, GB}
\EventLogo{}
\SeriesVolume{119}
%The article number for CSL 2018 proceedings
\ArticleNo{28}
%\nolinenumbers %uncomment to disable line numbering
%\hideLIPIcs  %uncomment to remove references to LIPIcs series (logo, DOI, ...), e.g. when preparing a pre-final version to be uploaded to arXiv or another public repository
%%%%%%%%%%%%%%%%%%%%%%%%%%%%%%%%%%%%%%%%%%%%%%%%%%%%%%

\begin{document}

\maketitle

\begin{abstract}
The ``generic operational metatheory'' of  Johann, Simpson and Voigtl\"{a}nder (LiCS 2010) defines
contextual equivalence, 
in the presence of algebraic effects, in terms of a
\emph{basic operational preorder} on ground-type effect trees. We propose three general approaches to 
specifying such preorders: (i)~operational (ii)~denotational, and (iii)~axiomatic; coinciding with the three major styles of program semantics. We illustrate these via a nontrivial case study: the combination of probabilistic choice with nondeterminism, for which we show that  natural instantiations of the three specification methods (operational in terms of Markov decision processes, denotational using  a powerdomain, and axiomatic) all determine the same canonical preorder. We do this in the case of both angelic and demonic nondeterminism. 
 \end{abstract}

\section{Introduction}
\label{section:intro}

%\todo[inline]{OBVIOUSLY INTRO NEEDS REVISING. BUT WE SHOULD WRITE THIS TOWARDS THE END}

%Contextual equivalence, in the style of Morris,
%has imposed itself as a very simple and powerful
%way to express what an equivalence on programs should be. 
%Two programs are said to be contextually equivalent if 
%they \emph{behave} equivalently under any context. 
%However this operational notion of equivalence is rarely usable 
%as-is because of the quantification over potentially complex 
%contexts, which can lead to unintuitive results \cite{pitts1997operationally}.

\emph{Contextual equivalence}, in the style of Morris,
is a powerful and general method for defining program equivalence, applicable to many 
programming languages. 
Two programs are said to be contextually equivalent if 
they `behave' equivalently when embedded in any suitable context that leads to `observable' behaviour. 
%However this operational notion of equivalence is rarely usable 
%as-is because of the quantification over potentially complex 
%contexts, which can lead to unintuitive results \cite{pitts1997operationally}.
More generally,\footnote{It is more general, since every equivalence relation is a preorder.} one can define \emph{contextual preorder} in the same manner. Let $P_1$ and $P_2$ be comparable programs (for example, in a typed language, $P_1$ and $P_2$ would  have the same type in order to be comparable). Suppose further  that we have some {basic preorder} $\Basicleq$, defined on `observable' computations, according to appropriate behavioural considerations. Then the contextual preorder is defined by
\begin{equation}
\label{equation:contextual-preorder}
P_1 \sqsubseteq_\text{ctxt} P_2 ~ \iff ~
\text{for all observation contexts $C[-]$, $~ C[P_1] \Basicleq C[P_2]$} \enspace . 
\end{equation}
This method of definition has important consequences. For example, the relation
$\sqsubseteq_\text{ctxt}$ is guaranteed to be a precongruence with respect 
to the constructors of the programming language.
However, the quantification over contexts makes the definition awkward to work with directly.
So various more manageable techniques for reasoning about contextual preorder relations have been developed, including:
%The choices of citations below all seem quite arbitrary, so I am removing them
(bi)simulations 
and their refinements (applicative/environmental bisimulations, 
bisimulations up-to), %\cite{koutavas2011applicative}, 
denotational interpretation in domains, % \cite{scott1982domains}, 
game semantics, %\cite{abramsky1999game}, 
program logics,
%higher order logic on  programs \cite{honda2005observationally} 
and logical relations. %\cite{Pitts2000}. [[EXPAND REFERENCES]].
These techniques are all reasonably general, in the sense that they adapt to different styles of programming languages, and combinations of programming features. Nonetheless, they are usually studied on a language-by-language basis.
%A major reason for this is that it is difficult to give mathematical definitions that apply uniformly across a range of languages and features. 

One direction for the systematisation of a range of programming features has been provided by Plotkin and Power through their work on
 \emph{algebraic effects}~\cite{plotkin2001adequacy,PlotkinPower2002}. Broadly speaking, effects are interactions between a  program and its environment (including the machine state), and include features such as
 error raising, global/local state, input/output, nondeterminism and probabilistic choice. 
 Plotkin and Power realised that the majority of effects (including all the aforementioned ones) are \emph{algebraic}, in the sense that the operations that trigger them %can be given in an algebraic signature, and also 
 satisfy a certain natural behavioural constraint.\footnote{In operational terms, the constraint  is that the behaviour of the operation does not depend on the content of the continuation at the time the operation is triggered.} 
 %THIS PARAGRAPH: \cite{plotkin2001adequacy} [OTHER REFS]

The algebraic effects  in a programming language can  be supplied via an algebraic signature $\Sigma$ of effect-triggering operations,
and the operational semantics of the language can then be defined parametrically in $\Sigma$. 
This is achieved by effectively splitting the semantics of 
the language into two steps. In the first step, operational rules specify how any program $P$ evaluates 
to an associated \emph{effect tree} $|P|$, 
which documents  all the effects that might potentially occur during execution. % and their interdependencies.
In an effect tree, the effects themselves are uninterpreted, in the sense that no specific execution behaviour is imposed upon them. 
As the second step, an interpretation is given to effect trees, by one means or another, from which a semantics for the whole language is extrapolated.
This methodology was first followed in \cite{plotkin2001adequacy}, where the operational reduction to effect trees (there called \emph{infinitary effect values}) is used as a method for proving the computational adequacy of denotational semantics. 
In \cite{gom}, effect trees (there called \emph{computation trees}) are used to give a uniform definition of 
contextual preorder, %for a language with algebraic effects, 
and to characterise it as a logical relation.
Effect trees also allow a general definition of applicative (bi)similarity for effects~\cite{SV2018} (see \cite{Ugo2017} for a related approach not based on trees).

%In~\cite{Ugo2017}, effect trees implicitly underpin the definition of relator used to give a general definition of applicative (bi)similarity for effects.
%And in~\cite{SV2018,Voor2018} effect trees are used to specify program logics that characterise applicative (bi)similarity.

In this paper, as in  \cite{gom}, our aim is to exploit the notion of effect tree for the purpose of 
giving a unified theory of contextual preorders for programming languages with algebraic effects.
In~\cite{gom}, this was carried out in the context of a specific polymorphically-typed call-by-name functional language with general recursion, to which algebraic effects were added. In this paper, we build on the technical work of~\cite{gom}, but an important departure is that we detach the development from any fixed choice of background programming language. This is based on the following general considerations. In order to define contextual preorder via~(\ref{equation:contextual-preorder}) above, one needs to specify what constitutes an observation context, and also the basic behavioural relation $\Basicleq$ on the computations such contexts induce. In the case of a language with algebraic effects, we can observe two things about a computation. Firstly, we can observe any discrete return value. 
In any sufficiently expressive language, discrete values should be convertible to natural numbers. So it is a not unreasonable  restriction to restrict observation contexts to \emph{ground contexts} whose return values (if any) are natural numbers.  
Secondly, we can also potentially observe aspects of effectful behaviour of such  computations, with exactly what is observable very much depending on the effects in question. 
%Since process of computation for such a context can be  modelled as an effect tree with natural-number-labelled leaves.
%We call such contexts \emph{ground contexts}.
One general approach to taking such effectful behaviour into account is to  specify a  \emph{basic operational preorder} 
$\Basicleq$ on the set
%$\Trees(\Nat)$ 
of effect trees with natural-number-labelled leaves, which implements a desired behavioural preorder on
effectful computations with return values in $\mathbb{N}$.
% Irrespective of how we actually interpret these effects, (\ref{equation:contextual-preorder}) requires that we specify a relation $\Basicleq$ between effectful computations with natural number return values.
 We are thus 
led to the following general formulation of contextual preorder. Given a chosen basic operational preorder
$\Basicleq$, we define the induced contextual preorder on programs by:
\begin{equation}
\label{equation:contextual-preorder-via-trees}
P_1 \sqsubseteq_\text{ctxt} P_2 ~ \iff ~
\text{for all ground contexts $C[-]$, $~ |C[P_1]| \Basicleq |C[P_2]|$} \enspace . 
\end{equation}

In~\cite{gom}, this general approach was developed in detail for 
a polymorphically typed call-by-name functional language with algebraic effects. 
The main result was that the resulting contextual preorder, defined by~(\ref{equation:contextual-preorder-via-trees}), is well behaved if the basic operational preorder satisfies two technical properties, \emph{admissibility} and \emph{compositionality}. In particular, it follows from these conditions that 
the contextual preorder is characterisable as a {logical relation} (and hence amenable to an important proof technique), and also that, on ground type programs $P_1,P_2$,
the contextual and basic operational preorders coincide (i.e., $P_1 \sqsubseteq_\text{ctxt} P_2$ if and only if
$|P_1| \Basicleq |P_2|$). 
%These results turn out not to be specific to the call-by-name language of~\cite{gom}.
Recently, we have carried out a similar programme for a call-by-value language,
similar to the language in~\cite{plotkin2001adequacy}, and obtained analogous results.%
% :  if the basic operational preorder is admissible and compositional then it is characterisable as a logical relation, 
% and coincides with the basic preorder at ground type.
\footnote{\label{footnote:unpublished}Unfortunately, there is no space to include these results, which were obtained while the first author was on an internship in Ljubljana in 2017, in this paper.}
It seems likely that similar results hold for other language variants.
% (call-by-push-value~\cite{LevyCBPV}, untyped or with additional type constructors, etc.). %are obtainable for other  %We had hoped to include it in this paper but 



The
notion of admissible and  compositional basic operational preorder thus provides a uniform and well-behaved definition of contextual preorder, for different languages with algebraic effects. Furthermore,
as is argued in~\cite[\S{V}]{gom}, it can also be given an intrinsic, more conceptually motivated justification in terms of an explicit  notion of \emph{observation}. 
Our general position is that the notion of admissible and  compositional basic operational preorder is a fundamental one. 
\emph{For any given combination of algebraic effects, one need only define a corresponding admissible and compositional basic operational preorder.} Once this has been done,  one obtains, via (\ref{equation:contextual-preorder-via-trees}), a definition of contextual preorder that can be applied to many programming languages containing those effects, and which will enjoy good properties.

In this paper, we describe three different approaches to defining basic operational preorders. 
The first is an \emph{operational} approach. One explicitly models the execution of the effects in question, and uses this model to determine the preorder. This is the approach that was followed 
in~\cite{gom}. Under this approach, admissibility and compositionality do not hold automatically, and so need to be explicitly verified. The second is a  \emph{denotational} approach. One builds a suitable domain-based model of the relevant effect operations. This induces a basic operational preorder on  effect trees that is automatically admissible and compositional. 
The third is \emph{axiomatic}. One finds a set of (possibly infinitary) Horn-clause axioms asserting desired properties of the intended preorder. The basic operational preorder is then taken to be the smallest admissible preorder satisfying the axioms. In addition to being admissible by definition, the resulting preorder is automatically compositional. 

It will not have escaped the readers attention that our three approaches to defining preorders  parallel the three main styles of program semantics: \emph{operational}, \emph{denotational} and \emph{axiomatic}. Nonetheless, irrespective of how they are defined, we view basic operational preorders themselves as a part of operational semantics, for their purpose is to define the operational notion of contextual preorder. 
%Thus our triptych of approaches shows that operational semantics can itself come in
% \emph{operational}, \emph{denotational} and \emph{axiomatic} flavours. 

The general identification of these three approaches is the first main contribution of the paper.
%which is a contribution of a methodological nature. 
Our second contribution is more technical. We illustrate the three approaches
with a nontrivial case study: the combination of (finitary) nondeterminism with probabilistic choice, which is a combination of effects that enjoys a certain notoriety 
for some of the technical complications it incurs~\cite{Mislove2000,mislove2004axioms,VW06,tix2009semantic,JGL15,JGL-mscs16,KeimelP2016}.
%We treat this combination of effects from each of the three viewpoints. 
On the operational side, we consider effect trees as Markov decision processes (MDPs), and we define a basic operational preorder based on the comparison of values of MDPs. On the denotational side, we make use of recently developed domain-theoretic models of combined nondeterministic and probabilistic choice~\cite{tix2009semantic,JGL-mscs16,KeimelP2016}.
On the axiomatic side, we give a simple axiomatisation, similar to axiomatisations in~\cite{mislove2004axioms,KeimelP2016}. 
Our main result is that the
operationally, denotationally and axiomatically-defined basic operational preorders all coincide with each other.
In fact, we give this result in two different versions.
The first is for an \emph{angelic} interpretation of nondeterminism, in which nondeterministic choices are resolved by a cooperative scheduler. The second is for \emph{demonic} nondeterminism, where an antagonistic scheduler is assumed. 
In each case, our coincidence theorem suggests the canonicity of the  preorder we obtain for the form of nondeterminism in question, with each of the three methods of definition providing a distinct perspective on it. 

%The paper is structured as follows. 
In Sections~\ref{section:trees} and~\ref{section:basic}, we review the 
definition of effect trees and basic operational preorders, largely following~\cite{gom}. Our main contribution starts in Sections~\ref{section:operational}, \ref{section:denotational} and~\ref{section:axiomatic}, which discuss the 
operational, denotational and axiomatic approaches to defining basic operational preorders. The discussion is illustrated using the example of combined nondeterminism and probabilistic choice. 
The main coincidence theorem, for this example, is then proved in Section~\ref{section:equivalence}. Finally, in Section~\ref{section:conclusions}, we briefly discuss related and further work.

\section{Effect trees}
\label{section:trees}

The general scenario this paper addresses is that of a programming language whose programs may perform effects as they compute. In this paper, we assume that the available effects are  specified 
by  an \emph{effect signature}: a set $\Sigma$ of operation symbols, each with an associated finite arity. We call the operations in $\Sigma$ \emph{effect operations}. This setting is explicitly that of \cite{plotkin2001adequacy}.
More general effect signatures appear in the literature, e.g., allowing parameterised operations and infinite arities
\cite{gom,StatonInstances}. The technical development in this paper can be generalised to such
more general signatures. Since, however, the main running example considered in this paper has only binary operations, we restrict ourselves to finite arity operations 
for the sake of presentational convenience. %We now present this example.
\begin{example}[Signature for combined probabilistic and non-deterministic choice]
\label{example:prnd}
    Consider a programming language that can perform two effects: probabilistic and nondeterministic choice.
    An appropriate signature for such a language is 
    $\Sigma_{\prnd} = \{ (\prEff,2), (\orEff,2) \}$ containing two binary operations:
    nondeterministic choice $\orEff$, 
    and fair probabilistic choice $\prEff$. (As is well known, in programming languages with general recursion, all computable discrete probability distributions can be 
     simulated using fair probabilistic choice.)
 \end{example}

During the execution of a program with effects, three different situations can arise. Firstly, the computation process may
trigger an effect, represented by some $o \in \Sigma$. The execution will then continue along one of the $n$ possible continuation processes given as arguments to the operation $o$. Secondly, the execution may terminate, 
in which case it may produce a resulting value. 
Thirdly, the execution may continue forever without terminating and without invoking any effects. We call this last situation
 \emph{silent nontermination} to distinguish it from \emph{noisy nontermination}, which occurs
 when the computation process computes for ever while performing an infinite sequence of effects along the way.

The global behaviour of such a program is captured by the notion of an \emph{effect tree}: a  finitely branching tree, whose 
internal nodes represent effect operations, and whose leaves represent either termination with a result, or silent nontermination. The branches of the tree represent potential execution sequences of the program. 
Trees are allowed to be infinitely deep, with their infinite branches representing noisy nontermination.
Such trees were introduced as \emph{infinitary effect values} in  \cite{plotkin2001adequacy}, and used extensively in \cite{gom}, where they are called
\emph{computation trees}. Two example trees, for computations that return natural number values, are drawn in
Figure~\ref{fig:exampletrees} below. The left-hand tree $\orEff (\prEff (1,2), 3)$ represents a program that first makes a nondeterministic choice and then a potential probabilistic choice, with the choices determining the resulting number. In the second tree $\prEff (\orEff (1,3), \orEff (2,3))$, the probabilistic choice is made first, followed by the relevant nondeterministic choice.

\begin{figure}[h]
\small
    \begin{center}
        \begin{tikzpicture}
            \node [circle,draw] (z){$\orEff$}
                child { 
                    node [circle,draw] (a) {$\prEff$}
                    child { node[circle,draw] (b) {$1$} } 
                    child { node[circle,draw] (c) {$2$} }
                }
                child {
                    node [circle,draw] (d) {$3$}    
                };
        \end{tikzpicture}
        \hspace*{8ex}
        \begin{tikzpicture}[level 1/.style={sibling distance=3cm},
                            level 2/.style={sibling distance=1.5cm}]
            \node [circle,draw] (z){$\prEff$}
                child { 
                    node [circle,draw] (h) {$\orEff$}
                    child { node[circle,draw] (b) {$1$} } 
                    child { node[circle,draw] (c) {$3$} }
                }
                child {
                    node [circle,draw] (g) {$\orEff$}
                    child { node[circle,draw] (e) {$2$} } 
                    child { node[circle,draw] (f) {$3$} }
                };
        \end{tikzpicture}
    \end{center}
%  \begin{equation*}
%        \orEff (\prEff (1,2), 3) \quad \quad \prEff (\orEff (1,3), \orEff (2,3))
 %   \end{equation*}
    \caption{Two effect trees}
    \label{fig:exampletrees}
\end{figure}






\begin{definition}
The set $\Trees(X)$ of \emph{effect trees} with values from  the set $X$ is coinductively defined so that
every tree has one of the following  forms.
\begin{itemize}
\item The root of the tree is labelled with an operation $o \in \Sigma$, and the tree has the form
         $o(t_1, \dots, t_n)$ where $n$ is the arity of $o$ and $t_1, \dots, t_n \in \Trees(X)$; or
\item the tree is a leaf labelled with a value $x \in X$; or
\item the tree is a leaf labelled with $\bot$.
\end{itemize}
\end{definition}
As this is a coinductive definition, $\Trees(X)$ contains trees of both finite and infinite depth.

We define a partial order on  $\Trees(X)$ by
$t_1 \Treeleq t_2$ if and only if $t_2$ can be obtained from $t_1$ by replacing (possibly infinitely many)
$\bot$-leaves appearing in $t_1$ with arbitrary replacement trees (rooted where the leaves were located). With this ordering,  $\Trees(X)$  is an $\omega$-complete 
partial order ($\omega$CPO) with least element $\bot$. Furthermore, by considering it as
a tree constructor,
every operation $o \in \Sigma$  defines a continuous (i.e., $\omega$-continuous) function $o : \Trees(X)^n \to \Trees(X)$, where $n$ is the arity of $o$.
(For notational convenience, we use $o$ for both operation symbol and function. The ambiguity can be resolved from the context.) 

The properties described above state that $\Trees(X)$ is a continuous $\Sigma$-algebra. In general,
a \emph{continuous $\Sigma$-algebra} is a pointed (i.e., with least element) $\omega$CPO $A$ with associated
continuous functions $o_A : A^n \to A$ for every $o \in \Sigma$ of arity $n$. 
As morphisms between continuous $\Sigma$-algebras
$A$ and $B$, we consider functions $h: A \to B$  that are strict (i.e., preserve least element) continuous homomorphisms with respect to the $\Sigma$-algebra structure. 
We refer to such  functions $h: A \to B$ as \emph{continuous homomorphisms}, leaving the strictness property implicit.
We write $\ContAlg_\Sigma$ for the category of continuous $\Sigma$-algebras and continuous homomorphisms.
The characterisation of $\Trees(X)$ below is standard.
%\todo[inline]{[[REFERENCES]]}
\begin{proposition}
\label{proposition:free}
$\Trees(X)$ is the free
    continuous $\Sigma$-algebra over the set $X$.
   \begin{center}
        \begin{tikzcd}
            X
            \arrow[r, "f"] 
            \arrow[d, hook, "i"]
            & A \\
            \Trees(X) \arrow[ru, dashrightarrow, "\hat{f}" below]
        \end{tikzcd}
    \end{center}
    That is, for every function $f : X \to A$, where 
    $A$ is a continuous $\Sigma$-algebra,
    there exists a unique continuous homomorphism $$\hat{f} : \Trees(X) \to A$$
    such that $
        f = \hat{f} \circ i $, where $i : X \to \Trees(X)$ is the function mapping every $x \in X$ to the 
        leaf-tree labelled $x$.
 \end{proposition}
 
 We use the above proposition to define a substitution operation on trees. For any tree $t \in \Trees(X)$, 
 every function $f \colon X \to \Trees(Y)$ determines a tree $t[f]$ in $\Trees(Y)$ defined by substitution, \emph{viz}:
 $t[f] ~ := ~ \hat{f}(t)\,$.
 
\section{Basic operational preorders}
\label{section:basic}

As discussed in Section~\ref{section:intro}, our interest in effect trees is that they provide a 
uniform template for defining 
 \emph{contextual preorders} for programming languages with algebraic effect operations
specified by signature $\Sigma$. 
As in~\cite{gom}, the crucial data is provided by a 
preorder  $\Basicleq$  on  $\Trees(\Nat)$, called the \emph{basic operational preorder}.  
In order for the resulting contextual preorder to be well behaved, we ask for the 
the basic operational preorder satisfy two properties: \emph{admissibility} and \emph{compositionality}.
In this section, we review the definitions of these and related notions. 


\begin{definition}[Admissibility]
    A binary relation $R$ on $\Trees(X)$ is \emph{admissible} if,
    for every ascending chain $(t_i)_{i \geq 0}$ and 
    $(t_i')_{i\geq 0}$, we have:
    \[ \text{($\,t_i  R \, t_i'$ for all $i\,$)} ~ \implies~
        \left(\bigsqcup_{i \geq 0} t_i\right) \, R \, \left(\bigsqcup_{i \geq 0} t_i'\right) \enspace .
    \]
\end{definition}

\begin{definition}[Compatibility]
    A binary relation $R$ on $\Trees(X)$ is  \emph{compatible} if,
    for every $o \in \Sigma$ of arity $n$, and for all trees 
     $t_1,\dots, t_n$ and $t'_1, \dots, t'_n$, we have:
    \[ \text{($\,t_i  R \, t'_i$ for all $i = 1, \dots, n\,$)} ~ \implies ~ 
        o(t_1, \dots, t_n) \, R \; o(t'_1, \dots, t'_n) \enspace .
    \]
\end{definition}
If a compatible relation is a preorder then it is called a \emph{precongruence}. If it is an equivalence relation it is called a \emph{congruence}.


The next two definitions make use of the substitution operation on trees defined at the end of
Section~\ref{section:trees}.
\begin{definition}[Substitutivity]
    A binary relation $R$ on $\Trees(X)$ is  \emph{substitutive} if,
    for all trees $t$, $t'$ and $\{t_x\}_{x \in X}$, we have:
    \[ \text{$\,t\, R \, t'$} ~ \implies ~ 
       t[ x \mapsto t_x] \, R \, t'[ x \mapsto t_x] \enspace .
    \]
\end{definition}



\begin{definition}[Compositionality]
    A binary relation $R$ on $\Trees(X)$ is \emph{compositional} if, for all 
    trees $t$, $t'$,  $\{t_x\}_{x \in X}$,  and $\{t'_x\}_{x \in X}$, we have:
        \[ \text{($\,t \, R \, t'$ and $t_x \, R \, t'_x$ for all $x \in X\,$)} ~ \implies ~ 
        t[ x \mapsto t_x] \, R \, t'[ x \mapsto t'_x] \enspace .
    \]
\end{definition}



\begin{proposition} 
\label{proposition:substitutive}
Let $\Basicleq$ be a preorder  on $\Trees(\mathbb{N})$.
\begin{enumerate} 
\item If  $\Basicleq$ is compositional then it is a substitutive precongruence.
\item If $\Basicleq$ is an admissible substitutive precongruence then it is compositional.
\end{enumerate}
\end{proposition}

%\noindent
%\todo[inline]{Only give the ideas of the proof to gain space}
\begin{proof}
We prove statement 2.
%    \begin{enumerate}
%        \item 
%             Suppose $\Basicleq$ is compositional. 
%            %It is substitutive because if $t \Basicleq t'$ and $\{ t_n \}_{n
%            %\in \mathbb{N}}$ is a family of trees then, since 
%            %$t_n \Basicleq t_n$,  we have 
%            $t[ n \mapsto t_n] \Basicleq t'[n \mapsto t_n]$ by compositionality.
%            For compatibility, if $o$ is an operation of arity $n$, and $t_i \Basicleq t'_i$ for $i = 1 \dots n$, then since $o(1,\dots, n) \Basicleq o(1,\dots, n)$,
%             we indeed have $o(t_1, \dots, t_n) \, \Basicleq \; o(t'_1, \dots, t'_n)$, by compositionality.
%             Substitutivity is by a similar argument, also using reflexivity.
%
%        \item 
            Suppose $\Basicleq$ is admissible, substitutive and compatible. 
            Suppose also that $t \Basicleq t'$ and $t_n \Basicleq t_n'$, for all $n \in \mathbb{N}$.
            By substitutivity, we have $t[n \mapsto t_n] \Basicleq t'[n \mapsto t_n]$.
             We would like to use compatibility to derive 
            that also $t'[n \mapsto t_n] \Basicleq t'[n \mapsto
            t_n']$, however this is only possible if $t'$ is finite. 
            The solution is to use finite approximations $(s_i')$ of $t'$
            satisfying $\bigsqcup_i s_i' = t'$. For each finite tree $s_i'$
            we have that $s_i'[n \mapsto t_n] \Basicleq s_i'[n \mapsto t_n']$, by compatibility.
            Hence, by admissibility, $t'[n \mapsto t_n] \Basicleq t'[n \mapsto
            t_n']$, whence $t[n \mapsto t_n] \Basicleq t'[n \mapsto t_n']$ by transitivity.
 %       \end{enumerate}
\end{proof}

\section{Operationally-defined preorders}
\label{section:operational}

In this section, we consider our first approach to defining an admissible and compositional basic operational
preorder $\Basicleq$ on $\Trees(\mathbb{N})$. We call this method \emph{operational}. Its characteristic is that the preorder
 $\Basicleq$ is directly defined using a mathematical model of the way that an effect tree in $\Trees(\mathbb{\Nat})$ will be executed.
There is not much to say in general about this approach, since such execution models vary enormously from one effect to another.
The main point to emphasise is that there is no general reason for  admissibility and compositionality to hold for such operationally defined preorders. Accordingly, these properties need to be established on a case-by-case basis.

The operational approach to defining basic preorders is illustrated for several examples of effects in~\cite{gom}. 
The main goal of the section is to demonstrate the approach using a different example, the signature
$\Sigma_\prnd = \{\prEff,\orEff\}$ from Example~\ref{example:prnd}, which is of interest because of the 
interplay between probabilistic and nondeterministic effects. 
In this case, trees in $\Trees(\mathbb{N})$ have both probabilistic and nondeterministic branching nodes,
as in Figure~\ref{fig:exampletrees}.


It is natural to consider such trees as (countable state) Markov decision processes, with the leaves representing nodes which either carry an observable value from $\mathbb{N}$, or which represent nontermination $\bot$.
%The behaviour of probabilistic nodes is clear. 
Nondeterministic choices may be thought of as being resolved by an external agent, the scheduler. We model the actions of the scheduler by a 
function $s: \{l,r\}^* \to \{l,r\}$. The idea is that a word $w \in \{l,r\}^*$ represents a finite path of left/right choices from the root of a 
tree $t \in \Trees(\mathbb{N})$. If the computation reaches a nondeterministic choice at the node indexed by 
$w$ then it takes the left/right branch according to the value of $s(w)$. This way of representing choices has some redundancy
(in every tree that is not a complete infinite binary tree, there will be words $w$ that do not index nodes in the tree; if $s(\varepsilon) = l$ then the value of $s$ on words beginning with $r$ is immaterial; the value of $s(w)$ on words $w$ that index 
probabilistic nodes in $t$ is irrelevant, etc.), but it is simple and convenient for future purposes. 
For any given $t \in \Trees(\mathbb{N})$, such a function
$s: \{l,r\}^* \to \{l,r\}$ can be thought of as a (deterministic) \emph{strategy} for the scheduler, in which the choice of direction at a nondeterministic node  
can respond to  the outcomes of probabilistic nodes higher up the tree.

A strategy $s$ and a tree $t$ in combination determine a subtree $t\restriction s$, defined by 
removing, at every nondeterministic node in $t$ with index $w$, the child tree that is not selected by $s(w)$. So $t\restriction s$ is a tree that has binary branching at probabilistic nodes, and unary branching at nondeterministic nodes. It is thus, in effect, a purely probabilistic tree, with leaves in $\mathbb{N}\cup\{\bot\}$, and so may be viewed as a Markov chain, in which the branching nodes are fair binary choices, determining  a subprobability distribution over $\mathbb{N}$. Specifically, each $n \in \mathbb{N}$ is assigned the probability that a run of the Markov chain will end at a leaf labelled with $n$. This is a subprobability distribution in general because there can  be a positive probability of nontermination (either at a $\bot$ leaf, or along an infinite branch).

%Given such a scheduling function $s$, we write $t@s$ for the result of the computation as scheduled by $s$. This is defined by:
%\[
%t@s ~ = ~ \begin{cases} 
% n & \text{if there exists $w \in \{l,r\}^*$ indexing an $n$ node in $t$} \\
%    & ~~~\text{such that, for every $i < |w|$, $~w_{i+1} = s(w\!\restriction_i)\,$;} \\
%  \bot & \text{otherwise.}
% \end{cases}
%\]
%Here we write $|w|$ for the length of a word, $w_i$ for the $i$-th symbol in a word, and $w \!\restriction_i$ for the prefix of $w$ that has length $i$.

The  \emph{angelic} interpretation of nondeterminism takes into account the possibility of a nondeterministic computation achieving a specified goal, given a cooperative scheduler.  The  \emph{demonic} interpretation, 
models the {certainty} with which a goal can be achieved, however adversarial the scheduler. 
This suggests the  two  basic operational preorders below. 
In each case, we consider functions $h \colon \mathbb{N} \to [0,\infty]$ assigning desirability weightings to possible results of a run of the computation. We then define 
%$H \subseteq \mathbb{N}$ to be a desirable set of outcomes, and we define 
$t \Basicleq t'$ if, for any $h$, the `expected' desirability weighting of $t'$ exceeds that of $t$. Here, `expected' is in inverted commas, because we have to take into account the actions of the scheduler, so this is not just a probabilistic expectation. In the case of 
angelic nondeterminism, the scheduler will help us, whereas, under demonic nondeterminism, it will impede us.
Technically this is taken account of by considering suprema of probabilistic expectations in the angelic case, and infima in the demonic case.
\begin{align*}
t \Basicleq^\Op_\prang t' ~ \Leftrightarrow ~ ~& \forall h \colon \mathbb{N} \to [0,\infty]~~ \sup_s  \mathbf{E}_{t\restriction\!s} (h)~ \leq~ \sup_s \mathbf{E}_{t'\restriction s} (h)
\\
t \Basicleq^\Op_\prdem t' ~ \Leftrightarrow ~ ~& \forall h \colon \mathbb{N} \to [0,\infty]~~ \inf_s  \mathbf{E}_{t\restriction s} (h)~ \leq~ \inf_s \mathbf{E}_{t'\restriction s} (h)
\end{align*}
Here $\mathbf{E}_{t\restriction\!s} (h)$ means the expectation of the function $h$ under the subprobability distribution on $\mathbb{N}$ induced by the Markov chain $t  \restriction\! s$.
In Markov-decision-process terminology, each preorder says that the \emph{value} of the MDP $t$, for any weighting $h$, is below the value of of $t'$ for $h$. In the angelic case the value maximises the expectation of $h$, in the demonic case it minimises it. 
\begin{proposition}
The preorders $\Basicleq^\Op_\prang$ and $\Basicleq^\Op_\prdem$ are admissible and compositional.
\end{proposition}

We outline the proof of this proposition in the case of $\Basicleq^\Op_\prdem$. The proof for  $\Basicleq^\Op_\prang$ is easier, largely because the analogue of the lemma below is trivial in the case of angelic nondeterminism.
\begin{lemma} 
\label{lemma:F-continuous}
Consider $\Tree(\mathbb{N})$ and $[0,+\infty]$ as $\omega$CPOs. Then,
for any $h \colon \mathbb{N} \to [0,\infty]$, 
the value-finding function $F_h$ is continuous:
\[F_h : t \mapsto \inf_s  \mathbf{E}_{t\restriction s} (h) : \Tree(\mathbb{N}) \to [0,+\infty]\]
\end{lemma}
\begin{proof}
The set $S = \{l,r\}^{\{l,r\}^*}$ of strategies is a countably-based compact Hausdorff space under the product topology. (It is Cantor space.)
It is easy to see that  the function 
\[G_h \colon (s,t) \mapsto \mathbf{E}_{t\restriction s} (h) : S \times \Tree(\mathbb{N})  \to [0,+\infty] \]
is continuous.
Essentially, it follows that $F_h$ is continuous because it is defined from $G_h$ by taking an infimum over a compact set. 
This can be made precise using, e.g.,  the general machinery in Section 7.3 of~\cite{AndreaShalk}. For completeness, we give a self-contained argument. 

Suppose $(t_i)$ is an ascending chain of trees.
 %Clearly $\inf_s \sup_i G_h(s,t_i) \geq \sup_i \inf_s G_h(s,t_i)$.
Because $S$ is compact,  there is $s_i \in S$ with $\inf_s G_h(s,t_i) = G_h(s_i, t_i)$,
and we can then extract a convergent 
subsequence  $(s_{a_i})$ of $(s_i)$ such that $s_{a_i} \rightarrow s_\infty$ in $S$. Then:
%it is clear that the following inequalities hold:
\begin{equation*}
                \sup_i \inf_s G_h(s,t_i)
                \, =\, 
                \sup_i G_h(s_i, t_i)
                \, = \,
                \sup_i G_h(s_{a_i}, t_{a_i})
                \, = \,
                G_h(s_\infty, \, \bigsqcup_i t_i)
                \, \geq \,
                \inf_s  G_h(s, \, \bigsqcup_i t_i)\, ,
            \end{equation*}
where the second equality holds because $G_h(s_i, t_i)$ is an ascending sequence, and the third by the continuity of  $G_h$.
We have shown that $ \sup_i  \inf_s G_h(s,t_i) \geq \inf_s  G_h(s, \bigsqcup_i t_i)$, i.e., 
$ \sup_i  F_h(t_i) \geq F_h(\bigsqcup_i t_i)$. 
Therefore $F_h$ is continuous (since it is obviously monotone).
\end{proof}

\noindent
The admissibility of $\Basicleq^\Op_\prdem$ follows easily from the lemma.
Suppose $t_i \Basicleq^\Op_\prdem t_i'$, for ascending chains $(t_i)$ and $(t'_i)$.
Then $F_h(t_i) \leq F_h (t_i')$, for all $i$ and $h$. By the lemma, 
 $F_h (\bigsqcup_i t_i) \leq F_h (\bigsqcup_i t_i')$, for all $h$.
So indeed  $\bigsqcup_i t_i \Basicleq^\Op_\prdem \bigsqcup_i t_i'$.

For compositionality, by Proposition~\ref{proposition:substitutive}, it suffices to show
that $\Basicleq^\Op_\prdem$ is a substitutive precongruence. The compatibility properties 
of a precongruence are easily shown. Substitutivity follows from the lemma below.
%
\begin{lemma}
Suppose $t$ and 
$\{ t_n \}_{n \in \mathbb{N}}$ are trees in 
            $\Tree(\Nat)$ then, for any weighting $h$,
             \begin{equation*}
                \inf_s \mathbf{E}_{t [ n \mapsto t_n ] \restriction  s } (h)
                = 
                \inf_s \mathbf{E}_{t \restriction  s} (\hat{h})
                \quad \text{ where } \quad \hat{h} (n) = \inf_s \mathbf{E}_{t_n \restriction  s} (h) \enspace .
            \end{equation*}
\end{lemma}
This lemma is proved first for finite trees, by induction on their height. It is then extended to infinite trees
by expressing them as suprema of finite trees, and applying Lemma~\ref{lemma:F-continuous}.
%
%            \todo[inline]{The proof of this lemma is long and not interesting, 
%            should we mention it ? Do we need the rest of the proof ?}
%            
%            Using this key lemma the proof can go smoothly because (by
%            definition)
%            if $\sigma \Basicleq^\Op_\prang \sigma'$ then 
%            $h_\sigma \leq h_\sigma'$.
%
%            \begin{align*}
%                \inf_s \mathbf{E}_{t\sigma \restriction  s} (h)
%                 & = \inf_s \mathbf{E}_{t \restriction  s} (h_\sigma)       & \text{ key lemma }  \\
%                 & \leq \inf_s \mathbf{E}_{t \restriction  s} (h_{\sigma'}) & \text{ monotonicity wrt the weighting function } \\
%                 & \leq \inf_s \mathbf{E}_{t'\restriction  s} (h_{\sigma'}) & \text{ monotonicity wrt the tree } \\
%                 & = \inf_s \mathbf{E}_{t'\sigma' \restriction  s} (h )     & \text{ key lemma } \\
%            \end{align*}
%
%            Therefore $ t \Basicleq^\Op_\prang t'$ and $\sigma
%            \Basicleq^\Op_\prang \sigma'$ implies $t\sigma \Basicleq^\Op_\prang
%            t'\sigma'$.
% 

We end this section by observing that a natural attempt to simplify  the definitions of 
 $\Basicleq^\Op_\prang$ and $\Basicleq^\Op_\prdem$ does not work. Instead of considering arbitrary weightings
 $h \colon \mathbb{N} \to [0,\infty]$, one might restrict to  
 functions $h \colon \mathbb{N} \to \{0,1\}$, which can be viewed as specifying goal subsets $H \subseteq \mathbb{N}$.
 Proceeding analogously to above, we compare suprema of probabilities of landing in $H$ in the angelic case, and infima in the demonic case. For both the angelic and demonic versions, the desired compositionality property fails.
\begin{proposition}
Neither of the formulas below defines a compositional relation $t \Basicleq t'$. %$\Basicleq^\bullet_\prang$ nor $\Basicleq^\bullet_\prang$ is compositional.
 \begin{align*}
%t \Basicleq^\bullet_\prang t' ~ \Leftrightarrow ~ ~
 & \forall H \subseteq \mathbb{N}  ~~ \sup_s  \mathbf{P}_{t\restriction s} (H)~ \leq~ \sup_s \mathbf{P}_{t'\restriction s} (H)
\\
%t \Basicleq^\bullet_\prdem t' ~ \Leftrightarrow ~ ~
 & \forall H \subseteq \mathbb{N}  ~~ \inf_s  \mathbf{P}_{t\restriction s} (H)~ \leq~ \inf_s \mathbf{P}_{t'\restriction s} (H)
\end{align*}
\end{proposition}


 
 




%Although natural, the above definitions do not work. 

\begin{proof}
    We use the two trees in Figure~\ref{fig:exampletrees},
    representing the expressions $A = 3 \orEff (1 \prEff 2)$
    and $B = (3 \orEff 1) \prEff (3 \orEff 2)$. It is easily checked that, for every subset $H \subseteq \{ 1, 2, 3 \}$,
    it holds that $\sup_s  \mathbf{P}_{A\restriction s} (H) =  \sup_s \mathbf{P}_{B\restriction s} (H)$ and
    $\inf_s  \mathbf{P}_{A\restriction s} (H) =  \inf_s \mathbf{P}_{B\restriction s} (H)$.
     Thus $A$ is equivalent 
    to $B$ under  both preorders.

    However, one can build a family $\{ t_1, t_2, t_3\}$ such that 
    $A[ i \mapsto t_i] = t_3 \orEff (t_1 \prEff t_2) = C$ is not equivalent to 
    $B[ i\mapsto t_i] = (t_3 \orEff t_1) \prEff (t_3 \orEff t_2) = D$,
    which contradicts substitutivity.
    Let $t_1 = 0 \prEff (0 \prEff (0 \prEff (0 \prEff 1)))$,
    $t_2 = 1$ and $t_3 = 0 \prEff (0 \prEff (0 \prEff 1))$. The distinguishing 
    factor will be the probability associated with the subset $\{ 1 \}$.
    %We dress a table of the minimal/maximal said probability on the different  trees.

    A simple calculation shows that $\sup_s  \mathbf{P}_{C\restriction s} (\{1\}) = 9/16
    \neq 5/8 = \sup_s \mathbf{P}_{D\restriction s} (\{ 1\})$. Similarly
    $\inf_s  \mathbf{P}_{C\restriction s} (\{1\}) = 1/4 \neq 3/16 =
     \inf_s \mathbf{P}_{D \restriction s } (\{1\})$.
     This contradicts the substitutivity and hence also the compositionality of both preorders.
        %\begin{equation*}
        %\begin{array}{c|c|c}
                %& \textbf{Min} & \textbf{Max} \\ \hline
            %t_1 & 1/8 & - \\
            %t_2 & 1   & -   \\
            %t_3 & 1/4 & - \\
            %t_1 \prEff t_2 & 9 / 16 & - \\
            %t_3 \orEff t_1 & 1/8    & 1/4 \\
            %t_3 \orEff t_2 & 1/4    & 1   \\
            %t_3 \orEff (t_1 \prEff t_3) & 1/4 & 9/16 \\
            %(t_3 \orEff t_1) \prEff (t_3 \orEff t_2) & 3 / 16 & 5/8 \\
        %\end{array}
    %\end{equation*}
\end{proof}

The necessity of using quantitative properties to obtain a compositional preorder is consistent with a general need for quantitative concepts that can be found
in the literature on
probabilistic computation. For example, in~\cite{KozenPDL,MciverMorgan}, quantitative logics  are required to obtain compositional reasoning methods.
% for programs with probabilistic (combined with nondeterministic in~\cite{MciverMorgan}) behaviour. 
Similarly, in~\cite{MioUpper}, quantitative observations are needed to distinguish non-bisimilar processes combining probabilistic and nondeterministic choice. 

%(combined with nondeterministic in~\cite{MciverMorgan}) behaviour.
%We view the necessity of using quantitative properties to obtain a compositional preorder as%an analogue of the situation found in, for example, , in, quantitative observations are 

%The above argument is also curiously  reminiscent of the proof in [[VARACCA PHD]]
% that there is no distributivity law between the finite powerset and distribution monads.

\section{Denotationally-defined preorders}
\label{section:denotational}

Our second approach to defining an admissible and compositional basic
denotational 
preorder $\Basicleq$ on $\Trees(\mathbb{N})$ is to make use of established constructions from domain theory.
Under this approach, admissibility and compositionality of the defined preorder $\Basicleq$ hold
for general reasons. Since this approach essentially amounts to giving a denotational semantics to effect trees, we call it the \emph{denotational} method of defining a basic operational preorder.



In order to define a basic operational preorder using the denotational method, one needs to merely provide
a continuous $\Sigma$-algebra $D$ (see Section~\ref{section:trees}),  together with a function
% with a distinguished function 
$j\colon \mathbb{N} \to D$. 
Define   $\llbracket \cdot \rrbracket \colon \Trees(\mathbb{N}) \to D$ to be the unique continuous homomorphism that makes the diagram below commute.
   \begin{center}
        \begin{tikzcd}
            \mathbb{N}
            \arrow[r, "j"] 
            \arrow[d, hook, "i"]
            & D \\
            \Trees(\mathbb{N}) \arrow[ru, dashrightarrow, "\llbracket \cdot \rrbracket" below]
        \end{tikzcd}
    \end{center}
\noindent
The map $\llbracket \cdot \rrbracket \colon \Trees(\mathbb{N}) \to D$ is used to induce
the basic operational preorder $\Basicleq_D$ from the partial order relation on the $\omega$CPO $D$.
\[
t \Basicleq_D t' ~~ \Leftrightarrow ~~ \Sem{t} \sqsubseteq \Sem{t' } \enspace .
\]
\begin{proposition}
The relation $\Basicleq_D$ is admissible pregongruence.
\end{proposition}
%
The proof is immediate: admissibility follows from the continuity of 
$\llbracket \cdot \rrbracket$, and compatibility because  $\llbracket \cdot \rrbracket$ is a homomorphism.

In order to obtain substitutivity, hence compositionality, a further property is required.

\begin{definition}[Factorisation property]
    The map $j\colon \mathbb{N} \to D$ is said to have  the \emph{factorisation property} if,
    for every function $f \colon \mathbb{N} \to D$, there exists a 
    continuous homomorphism $h_{\!f} : D \to D$ such that $f = h_{\!f} \circ j$.
    \begin{center}
        \begin{tikzcd}
            \mathbb{N} \arrow[r, "j"] 
                 \arrow[rr, bend right, "f"] &
            D \arrow[r, "h_{\!f}", dashed] & 
            D  
        \end{tikzcd}
    \end{center}
\end{definition}
\begin{proposition}
If $j\colon \mathbb{N} \to D$ has the factorisation property then 
the relation $\Basicleq_D$ is substitutive, hence it is an admissible compositional precongruence.
\end{proposition}

\begin{proof}
Suppose $\sigma: \Nat \to \Tree(\Nat)$ is any  substitution.
Let $\hat{\sigma} : \Tree(\Nat) \to \Tree(\Nat)$ be the continuous homomorphism
such that $\hat{\sigma} \circ i = \sigma$. Consider the map $g := \llbracket \cdot \rrbracket \circ \hat{\sigma} \circ i : \Nat \to D$. By the factorisation property, there exists $h_g : D \to D$ such that
$g = h_g \circ j$. Expanding this, and using the definition of $\Sem{\cdot}$, we have:
\[
 \llbracket \cdot \rrbracket \circ \hat{\sigma} \circ i ~ = ~ h_g \circ j ~ = ~  h_g \circ  \llbracket \cdot \rrbracket \circ i \enspace .
\]
It then follows from  the uniqueness property of Proposition~\ref{proposition:free} that
\begin{equation}
\label{equation:before-pizza}
\llbracket \cdot \rrbracket \circ \hat{\sigma} ~ = ~ h_g \circ  \llbracket \cdot \rrbracket \enspace ,
\end{equation}
because both maps are continuous homomorphisms.

Now, for substitutivity, suppose  that $t \Basicleq_D t'$, i.e., $\Sem{t} \leq \Sem{t'}$. Then 
$h_g (\Sem{t})  \leq h_g(\Sem{t'})$ by monotonicity. That is
$\Sem{ \hat{\sigma}(t)} \leq \Sem{ \hat{\sigma}(t')}$, by~(\ref{equation:before-pizza}). 
This says that $\Sem{ t[\sigma]} \leq \Sem{t'[\sigma]}$. That is
$t[\sigma] \Basicleq_D t'[\sigma]$, as required.
\end{proof}

In practice, it is usually not necessary to prove the factorisation property directly. Instead  it holds as a consequence of the continuous algebra $D$ and map $j: \Nat \to D$ being derived from a suitable monad. The next result establishes general conditions under which this holds.
\begin{proposition}
\label{proposition:monad}
Let $\mathbf{S}$ be a category with a faithful functor $U : \mathbf{S} \to \Set$. Suppose also that 
$\mathbf{S}$ has an object $N$ such that $UN = \mathbb{N}$, and every hom set $\mathbf{S}(N,X)$
is mapped bijectively by $U$ to $\Set(\Nat,UX)$. Suppose also that $(T,\eta,\mu)$ is a monad on $\mathbf{S}$
with the following properties: there is a continuous $\Sigma$-algebra structure on $UTN$; and, for 
every map $f \colon N \to TN$, the induced function $Uf^*$, where 
$f^* \colon TN \to TN$ is the Kleisli lifting, is a continuous  homomorphism.
Then defining $D$ to be the continuous $\Sigma$-algebra on $U T N$, and
$j$ to be $U\eta \colon \mathbb{N} \to UTN$, it follows that $j$ has the factorisation property.
\end{proposition}

%\begin{proof}
%Consider any function $f \colon \mathbb{N} \to UTN$. This is the $U$ image of a unique morphism 
%$g : N \to TN$ in $\mathbf{S}$. Let $g^*: TN \to TN$ be the Kleisli lifting of $g$, which satisfies $g^* \circ \eta = g$. Defining $h_f = Ug^*$, we indeed have  that $h_f$ is a continuous algebra homomorphism, and
%$h_f \circ j = f$.
%\end{proof}


\noindent
We omit the easy proof. 
Although the statement of the proposition is verbose, the result is relatively easy to apply in practice, as the examples we consider next will illustrate.

%
%\begin{lemma}
%Let $\mathcal{A}$ be any category given together with a functor $U' \colon \mathcal{A} \to \ContAlg_\Sigma$.
%Suppose it holds that that the
%composite functor $UU' : \mathcal{A} \to \Set$ has a left adjoint $F$, where $U: \ContAlg_\Sigma \to \Set$ is the forgetful functor.
%Then, defining $D$ to be the continuous $\Sigma$-algebra $U' F \, \mathbb{N}$, the unit of the adjunction defines a function
%$j \colon \mathbb{N} \to D$ which has the factorisation property.
%\end{lemma}
%
%\begin{lemma}
%Let $T$ be a monad on the category $\wCPO$.  Suppose that for every $\omega$CPO $X$, we have a 
%continuous $\Sigma$-algebra structure on TX. Suppose also that all Kleisli liftings of 
%$f^* \colon TX \to TY$, of maps $f \colon X \to TY$, are  continuous  homomorphisms. Then,
%defining $D$ to be the continuous $\Sigma$-algebra $T \mathbb{N}$, the unit of the adjunction defines a function
%$j \colon \mathbb{N} \to D$ which has the factorisation property.
%\end{lemma}

In the remainder of the section, we return to our main example, and again define basic operational preorders for the combination of probabilistic choice and nondeterminism (both angelic and demonic), but this time we use the denotational method. Accordingly, we call the defined preorders
$\Basicleq^\Den_\prang$ and $\Basicleq^\Den_\prdem$

We use the powerdomains combining probabilistic choice and nondeterminism
defined in~\cite[\S3.4]{KeimelP2016}, although our setting is simpler because we only need to apply them to sets.
The basic idea of these constructions is that a computation with probabilistic and nondeterministic choice is modelled as a set of subprobability distributions, where the set collects the possible nondeterministic outcomes, each of which is probabilistic in nature. As is standard, 
the sets relevant to angelic nondeterminism are the closed sets in the Scott topology, whereas those relevant to demonic nondeterminism are the compact upper-closed sets, see~\cite{smyth}.
Due to the combination with probabilistic choice,  sets are further required to be convex; see, for example, the discussion in~\cite{KeimelP2016}.
% suggested by Smyth, the sets relevant to demonic nondeterminism are the compact upper-closed sets, 



Let $\mathcal{V}_{\leq 1} \,X$ be the $\omega$CPO of (discrete) subprobability distributions on a set $X$.
We write $\mathcal{H}\mathcal{V}_{\leq 1} \,X$ for the $\omega$CPO of nonempty Scott-closed convex subsets
of  $\mathcal{V}_{\leq 1} \,X$  ordered by subset inclusion $\subseteq$. 
We write $\mathcal{S}\mathcal{V}_{\leq 1} \,X$ for the $\omega$CPO of nonempty Scott-compact convex upper-closed subsets
of  $\mathcal{V}_{\leq 1} \,X$  ordered by reverse inclusion $\supseteq$.
The $\omega$CPOs $\mathcal{H}\mathcal{V}_{\leq 1} \,X$ and $\mathcal{S}\mathcal{V}_{\leq 1} \,X$ are both continuous algebras for $\Sigma_\prnd$. In both cases, the operations are defined by:
\begin{align*}
\orEff(A,B) ~ = ~ & \Conv(A \cup B)  
% & & \text{(definition for $\mathcal{H}\mathcal{V}_{\leq 1} \,X$)}
% \\ \orEff(A,B) ~ = ~ & & & \text{(definition for $\mathcal{S}\mathcal{V}_{\leq 1} \,X$)}
& 
\prEff(A,B) ~ = ~ & \{\frac{1}{2}a + \frac{1}{2}b \mid a \in A, b \in B\} \enspace ,
% & & \text{(definition for both cases)}
\end{align*}
where $\Conv$ is the convex closure operation. 
We remark that these straightforward uniform definitions are possible because of the simple structure of the 
domains  $\mathcal{H}\mathcal{V}_{\leq 1} \,X$ and $\mathcal{S}\mathcal{V}_{\leq 1} \,X$, over a set $X$. For the more general 
lower and upper `Kegelspitze' considered in \cite{KeimelP2016}, additional order-theoretic and topological closure operations need to be applied.

To apply the above in the angelic case, we use the fact that $\mathcal{H}\mathcal{V}_{\leq 1} \,X$  is the free 
Kegelspitze join semilattice over a set $X$  \cite[Corollary 3.15]{KeimelP2016} (where the result is proved more generally for domains). It follows that $\mathcal{H}\mathcal{V}_{\leq 1} $ is
a monad on  $\Set$ itself  satisfying the conditions
of Proposition~\ref{proposition:monad}. Thus defining 
$D_\prang = \mathcal{H}\mathcal{V}_{\leq 1} \,\mathbb{N}$, and
$j(n) = \downarrow\!\delta(n)$ (where $\delta(n)$ is the Dirac probability distribution that assigns probability 1 to $n$ and $0$ to all other numbers, and $\downarrow\!x$ is the down-closure $\{y \mid y \leq x\}$), the induced $\Sem{\cdot}_\prang : \Tree(\Nat) \to  D_\prang$ defines an admissible and compositional preorder
\[
t \Basicleq^\Den_\prang t' ~~ \Leftrightarrow ~~ \Sem{t}_\prang  \leq \Sem{t'}_\prang \enspace .
\]

Similarly, in the demonic case, we use \cite[Corollary 3.16]{KeimelP2016}, which characterises $\mathcal{S}\mathcal{V}_{\leq 1} \,X$ as the free Kegelspitze meet semilattice over $X$. Again $\mathcal{S}\mathcal{V}_{\leq 1}$ is a monad
on  $\Set$ to which Proposition~\ref{proposition:monad} applies. In this case, we define 
$D_\prdem = \mathcal{S}\mathcal{V}_{\leq 1} \,\mathbb{N}$, and
$j(n) = \{\delta(n)\}$. Then the induced $\Sem{\cdot}_\prdem : \Tree(\Nat) \to  D_\prdem$ defines an admissible and compositional preorder
\[
t \Basicleq^\Den_\prdem t' ~~ \Leftrightarrow ~~ \Sem{t}_\prdem  \leq \Sem{t'}_\prdem \enspace .
\]


\section{Axiomatically-defined preorders}
\label{section:axiomatic}

In this section, we look at the definition of basic operational preorders by axiomatising
properties of the operations in the effect signature $\Sigma$.
Since we are defining a preorder, it is appropriate for the axiomatisation to involve inequalities
specifying desired properties of the operational preorder. As the technical framework for this, we allow
axiomatisations involving infinitary Horn-clauses of inequalities between infinitary terms.  This provides a flexible general setting for
axiomatising admissible and compositional preorders on 
$\Trees(\mathbb{N})$. 

Let $\Vars$ be a set of countably many distinct variables. By an  \emph{expression}, we mean a
tree $e \in \Trees(\Vars)$. The use of trees incorporates infinitary non-well-founded terms alongside the usual finite
algebraic terms. By an \emph{inequality} we mean a statement $e_1 \leq e_2$, where $e_1, e_2$ are expressions.
By an \emph{(infinitary) Horn clause} we mean an implication of the form:
\begin{equation}
\label{equation:horn-clause}
\left( \bigwedge_{i \in I} e_i \leq e'_i \right)~ \implies ~ e \leq e' \enspace ,
\end{equation}
An \emph{effect theory} $T$  is a set of Horn clauses.

A precongruence $\Basicleq$ on $\Trees(X)$ is said to \emph{satisfy} a Horn clause (\ref{equation:horn-clause}) if,
for every environment $\rho \colon \Vars \to \Trees(X)$, the implication below holds (recall the notation for tree substitution from Section~\ref{section:trees}).
\[
\left( \bigwedge_{i \in I} e_i[\rho] \Basicleq e'_i[\rho] \right) ~ \implies ~  e[\rho] \Basicleq e'[\rho] \enspace .
\]
We say that a precongruence $\Basicleq$ is a \emph{model} of a Horn clause theory $T$ if it satisfies every Horn clause in $T$.
We consider models as subsets of $\Trees(X) \times \Trees(X)$, partially ordered by inclusion. Note that models are precongruences by assumption.
\begin{proposition}
\label{proposition:axiomatic}
Every Horn clause theory $T$ has a smallest admissible model  
\[{\Basicleq_T} ~\subseteq~ \Trees(X) \times \Trees(X) \enspace ,\]
for any set $X$. The model ${\Basicleq_T}$
is substitutive. In the case that $X = \mathbb{N}$, the smallest admissible model is thus an
admissible compositional preorder.
\end{proposition}
\begin{proof}
%The total relation $X \times X$ gives one admissible model. 
It is easily seen that the intersection of any set of admissible models is itself an admissible model.
Thus the
intersection of the set of all admissible models is the required smallest admissible model $\Basicleq_T\,$. For substitutivity, 
define 
\begin{equation}
\label{equation:substitutive-T}
t \Basicleq_S t' ~ \Leftrightarrow ~ \forall \sigma : X \to \Trees(X). ~ t[\sigma] \Basicleq_T t'[\sigma] \enspace .
\end{equation}
Using the substitution $\sigma(x) = x$, we see that ${\Basicleq_S}\, \subseteq\, {\Basicleq_T}$.
Conversely, it is easily shown that the relation $\Basicleq_S$ is itself an admissible model of $T$. 
Thus ${\Basicleq_T}\, \subseteq\, {\Basicleq_S}$. Since ${\Basicleq_T}$ and ${\Basicleq_S}$ coincide,
(\ref{equation:substitutive-T}) asserts the substitutivity of ${\Basicleq_T}$.
The statement about compositionality now follows from Proposition~\ref{proposition:substitutive}.
\end{proof}


Given the proposition, we can use any effect theory to define an admissible and compositional basic operational preorder, namely the smallest admissible model over $\mathbb{N}$. We now apply this method to our running example of combined nondeterminism and probabilistic choice. The  axioms are given in
Figure~\ref{fig:axiomsmixed}.
\begin{figure}[t]
\[
\begin{array}{ll}
\text{Bot:} & \bot \leq x
\\
\text{Prob:} & x \prEff x = x , ~ x \prEff y = y \prEff x , ~ 
                                (x \prEff y) \prEff\, (z \prEff w) = (x \prEff z) \prEff\, (y \prEff w)
\\
\text{Appr:} & x \prEff y \leq y \implies  x \leq y
 \\
\text{Nondet:} & x \orEff x = x, ~ x \orEff y = y \orEff x, ~ ~ x \orEff\, (y \orEff z) = (x \orEff y) \orEff z
\\
\text{Ang:} & x \orEff y \geq x \\ 
\text{Dem:} & x \orEff y \leq x \\ 
\text{Dist:} & x \prEff\, (y \orEff z) = (x \prEff y) \orEff\, (x \prEff z)
\end{array}
\]
    \caption{Horn theory for mixed probability and non determinism}
    \label{fig:axiomsmixed}
\end{figure}

The axioms include a special axiom for $\bot$, which is legitimate since $\bot$ is a tree, hence an expression.
The axioms  for probability include three standard equalities (each of which is given officially as two inequalities), and one Horn approximation axiom, $\text{Appr}$, which  is separated out for the sake of Proposition~\ref{proposition:horn} below.
The axioms for nondeterminism are split into a neutral list,  followed by further axioms  for angelic and demonic nondeterminism respectively. Finally, there is a distributivity axiom that relates
probabilstic and nondeterministic choice. 
Our two effect theories of interest are: 
\[
T_\prang \!=\!  \text{Bot},  \text{Prob},  \text{Appr}, \text{Nondet}, \text{Ang},\text{Dist} \quad
T_\prdem\! = \!\text{Bot},  \text{Prob},  \text{Appr},  \text{Nondet},  \text{Dem},\text{Dist}\,.
\]
We then define  $\Basicleq^\Ax_\prang$ as the smallest admissible model
of $T_\prang$ over $\mathbb{N}$, and 
$\Basicleq^\Ax_\prdem$ as the smallest admissible model
of $T_\prdem$. By Proposition~\ref{proposition:axiomatic}, both these basic operational preorders are admissible and compositional.

To end the section, we observe that the Horn-clause axiom in Figure~\ref{fig:axiomsmixed} can be replaced with an equational axiom, albeit one involving an infinitary expression.



\begin{definition}
    \label{def:probaApproxConstruct}
    Let $t$ be a  tree. For each $n \in \mathbb{N} \cup \{\infty\}$, we define a tree $\Iter{t}{n}$ inductively by
    $\Iter{t}{0} = \bot$ and $\Iter{t}{(n+1)}  = t\,  \prEff \,\Iter{t}{n}$. The
    tree $\Iter{t}{\infty}$ is defined as $\bigsqcup_n \Iter{t}{n}$.
\end{definition}

\begin{proposition}%[Removing Horn-Clauses]
\label{proposition:horn}
    For any effect theory containing the Bot and Prob axioms, an admissible model satisfies the $\text{Appr}$
     axiom if and only if it satisfies the equation 
    $\Iterx{\infty}= x$.
\end{proposition}

\begin{proof} We first derive $\Iterx{\infty}= x$, from the axioms with $\text{Appr}$.
    It is clear that $\Iterx{n} \leq  x$ for every $n < \infty$,
    and therefore $\Iterx{\infty} \leq  x$ by admissibility.
    We also have $x \prEff \Iterx{n} \leq  \Iterx{(n+1)}$,
    and so, again by admissibility, $x \prEff \Iterx{\infty} \leq \Iterx{\infty}$.
    Whence, by the Horn axiom, $x  \leq \Iterx{\infty}$. 
     We have thus derived $\Iterx{\infty}= x$.

    For the converse, we assume $\Iterx{\infty}= x$ and derive  $\text{Appr}$.
    Suppose $x \prEff y \leq y$.  Then also  $x \prEff \, (x \prEff y) \leq y$, and $x \prEff \, ( x \prEff \, (x \prEff y)) \leq y$, etc. So also $x \prEff \bot \leq y$, and $x \prEff \, (x \prEff \bot) \leq y$, and $x \prEff \, ( x \prEff \, (x \prEff \bot)) \leq y$,
 etc. That is, $\Iterx{n} \leq y$, for every $n < \infty$. By admissibility, $\Iterx{\infty} \leq y$. Whence, by the assumed axiom, $x \leq y$ as required.
 \end{proof}


\section{The coincidence theorem}
\label{section:equivalence}

Our main theorem is that our operational, denotational and axiomatic preorders for combined probability and nondeterminism all coincide, in both the angelic and demonic cases.
\begin{theorem}[Coincidence theorem] \leavevmode
\begin{enumerate} 
\item The three preorders $\Basicleq^\Op_\prang$, $\Basicleq^\Den_\prang$ and $\Basicleq^\Ax_\prang$, for mixed probability and angelic nondeterminism, coincide.

\item Similarly, the preorders $\Basicleq^\Op_\prdem$, $\Basicleq^\Den_\prdem$ and $\Basicleq^\Ax_\prdem$,
for mixed probability and demonic nondeterminism, coincide.
\end{enumerate}
\end{theorem}

\noindent
We outline the proof of the theorem in the demonic case, which we split into three lemmas. The proof for the angelic case is similar. 
\begin{lemma}
${\Basicleq^\Ax_\prdem} \, \subseteq \, \, \Basicleq^\Op_\prdem\,$.
\end{lemma}
\begin{proof}
 It is easily checked that $\Basicleq^\Op_\prdem$ satsfies the axioms
 of $T_\prdem$.  Since $\Basicleq^\Op_\prdem$ is admissible and 
$\Basicleq^\Ax$ is the smallest admissible model, ${\Basicleq^\Ax_\prdem} \, \subseteq \, \Basicleq^\Op_\prdem\,$.
\end{proof}
%
\noindent
We remark on the following aspect of the above result.
The distributivity axiom Dist of Figure~\ref{fig:axiomsmixed} is sometimes discussed as expressing that  nondeterministic choices  are  resolved before probabilistic ones; see, e.g., \cite{mislove2004axioms,KeimelP2016}. Such statements need careful interpretation. 
The definition of $\Basicleq^\Op_\prdem$, which is based on implementing nondeterministic schedulers as strategies for MDPs,
explicitly allows the  scheduler's choices to take account of the outcomes of probabilistic choices that precede it. Nevertheless, the distributivity axiom is sound.

\begin{lemma}
${\Basicleq^\Op_\prdem} \, = \,\, \Basicleq^\Den_\prdem\,$.
\end{lemma}
\begin{proof}
We make use of the functional representation of $\mathcal{S}\mathcal{V}_{\leq 1}\, \mathbb{N}$ from \cite{KeimelP2016} (see also~\cite{JGL-mscs16}). 
For any topological space $X$, we write $\mathcal{L}(X)$ for the space of all
\emph{lower semicontinuous} functions from $X$ to $[0,\infty]$ (i.e., functions that are continuous with respect to the Scott topology on $[0,\infty]$), and we endow $\mathcal{L}(X)$ itself with the Scott topology. The space
$D' = \mathcal{L}(\mathcal{L}(\mathbb{N}))$ carries a continuous $\Sigma_\prnd$-algebra structure
\[
(F \orEff G)(f) ~ = ~ \min(F(f), G(f)) \qquad 
(F \prEff G)(f) ~ = ~ \frac{1}{2}F(f) + \frac{1}{2}G(f) \enspace .
\]
(There is another $\Sigma_\prnd$-algebra structure, relevant to angelic nondeterminism, in which  $\min$ is replaced with $\max$.) Define $j' : \mathbb{N} \to D'$ by $j'(n)(f) = f(n)$. This induces $\Sem{\cdot}'_\prdem : \Tree(\mathbb{N}) \to D'$
satisfying  $\Sem{\cdot}'_\prdem \circ i = j'$, as in Section~\ref{section:denotational}.
We show that $\Sem{t}'_\prdem (h) = F_h (t)$, where $F_h$ is defined as in
Lemma~\ref{lemma:F-continuous}. For this, the function $t \mapsto (h \mapsto F_h (t))$ is easily shown to be a $\Sigma_\prnd$-algebra homomorphism
satisfying $F_h(i(n)) = j'(n)$. Moreover, it is continuous by 
Lemma~\ref{lemma:F-continuous}. Thus it indeed coincides with 
$\Sem{\cdot}'_\prdem$. By the definition of $F_h$, if follows that that
$t \Basicleq^\Op_\prdem t'$ if and only if 
$\Sem{t}'_\prdem \leq \Sem{t'}'_\prdem \,$.

Corollary 4.7 of \cite{KeimelP2016} provides a functional representation of 
$\mathcal{S}\mathcal{V}_{\leq 1}\, X$ inside $\mathcal{L}(\mathcal{L}(X))$. In the case $X = \mathbb{N}$, consider the function
\[
\Lambda : A \mapsto \left(f \mapsto \inf_{p \in A} 
  \mathbf{E}_p\, f 
% \left( \sum_{n \in \mathbb{N}} f(n)\, . \, p(n) \right)
\right) 
 : \; \mathcal{S}\mathcal{V}_{\leq 1}\, \mathbb{N} \;  \to \; D' \enspace .
\]
It is shown in  \cite{KeimelP2016} that $\Lambda$  is a continuous $\Sigma_\prnd$-algebra homomorphism, and also an order embedding (i.e., $\Lambda(A) \leq \Lambda(B)$ implies $A \supseteq B$). 
%(Furthermore, the image of $\Lambda$ can be characterised as exactly the \emph{superlinear strongly non-expansive} functionals.) 
By the uniqueness property of Proposition~\ref{proposition:free}, it thus holds that $\Lambda \circ \Sem{\cdot}_\prdem = \Sem{\cdot}'_\prdem$. We therefore have
\[
t \Basicleq^\Op_\prdem t' ~~ \Leftrightarrow ~ ~
  \Sem{t}'_\prdem \leq \Sem{t'}'_\prdem 
   ~~ \Leftrightarrow ~ ~
   \Sem{t}_\prdem \leq \Sem{t'}_\prdem 
      ~ ~\Leftrightarrow ~ ~
  t \Basicleq^\Den_\prdem t' \enspace ,
   \]
where the middle equivalence holds because $\Lambda$ is an order embedding.
\end{proof}

\begin{lemma}
\label{lemma:completeness}
${\Basicleq^\Den_\prdem} \, \subseteq \, \, \Basicleq^\Ax_\prdem\,$.
\end{lemma}
\begin{proof}
The proof proceeds in three steps.
    \begin{enumerate}
        \item Prove that both preorders coincide 
            on \emph{probability trees} (i.e., trees without $\orEff$ nodes).
        \item Prove the inclusion of preorders for
             trees with a \emph{finite} number 
            of $\orEff$ nodes.
        \item Use finite approximations and admissibility
            to conclude the general case.
    \end{enumerate}

We omit discussion of the first step, which  is comparatively straightforward, cf.~\cite{heckmann}.

For step 2, suppose $t \Basicleq^\Den_\prdem t'$ where $t,t'$ are 
trees with finitely many $\orEff$ nodes. For each of $t, t'$, we use the distributivity axiom to rewrite the tree
          as an $\orEff$-combination of finitely many (possibly infinite) {probability trees}. We then establish the following.
    \begin{enumerate}[(a)]
         \item 
            If for every 
            probability
            tree $t_i'$ in $t'$ there exists 
            a corresponding tree $t_i$ in $t$ 
            such that $t_i \Basicleq^\Den_\prdem t_i'$,
            then we have that $t \Basicleq^\Ax_\prdem t'$, using the $\texttt{Dem}$ axiom, and  step 1 above.

        \item 
            %If $t_i$ and $t_j$ are two  of the probability trees of $t$,
            The tree $t$ is equivalent  in both preorders  to 
            $t \orEff k$, where $k = \lambda_1t_1+ \dots + \lambda_n t_n$
            is any tree representing a convex combination of the probability trees of $t$.
            The tree $k$ is defined  using infinite combinations of $\prEff$ nodes to assign the correct weight to each $t_i$.
    
        \item Making direct use of the definition of $\mathcal{S}\mathcal{V}_{\leq 1}\, \mathbb{N}$, it follows from $t \Basicleq^\Den_\prdem t'$ that, for every
            probability tree   $t_i'$ of $t'$,
            there is a convex combination
            $k_i := \lambda_1t_1+ \dots + \lambda_n t_n$
            of probability trees of $t$,
                        such that $k_i \Basicleq^\Den_\prdem t_i'$.


  
      \end{enumerate}
To complete the argument for step 2, the  tree $t'$ has the form $t'_1 \orEff \dots \orEff t'_m$. By (c), there exist corresponding  $k_1, \dots, k_m$. By (b), $t$ is equivalent to
$t \orEff k_1 \orEff \dots \orEff k_m$. It now follows from (a) that $t \Basicleq^\Ax_\prdem t'$,
by the property of the $k_j$ given by (c).


For step 3, suppose 
$t \Basicleq^\Den_\prdem t'$, where $t,t'$ are arbitrary.
Take approximating sequences  $t = \bigsqcup_i t_i$ and $t' = \bigsqcup_i t_i'$,
    where both ascending sequences are composed of {finite} trees.

We use Definition~\ref{def:probaApproxConstruct} to further restrict the approximations of $t$.
Using the finiteness of $t_i$,
we have $\Sem{\Iter{t_i}{n}}_\prdem \ll \Sem{t_i}_\prdem$ in the way-below relation 
on $\mathcal{S}\mathcal{V}_{\leq 1} \mathbb{N}$, via the explicit characterisation of this relation in~\cite{KeimelP2016}. Also, $(\Iter{t_i}{i})$ is an ascending sequence 
of finite trees with $\bigsqcup_i \Iter{t_i}{i} = \Iter{t}{\infty}$
% $ \simeq_\prdem t$.

%So $(\Sem{\Iter{t_i}{i}}_\prdem)$ is an ascending sequence in 
%$\mathcal{S}\mathcal{V}_{\leq 1} \mathbb{N}$ with 
For every $i$, we have 
$\Sem{\Iter{t_i}{i}}_\prdem \ll  \Sem{t_i}_\prdem \leq \Sem{t}_\prdem \leq \Sem{t'}_\prdem$. That is $\Sem{\Iter{t_i}{i}}_\prdem \ll  \Sem{t'}_\prdem$. 
Since $\Sem{t'}_\prdem = \bigsqcup \Sem{t_i'}_\prdem$, it follows from the way-below property that, for every $i$, 
$\Sem{\Iter{t_i}{i}}_\prdem \leq \Sem{t'_{j_i}}_\prdem$ for some $j_i$, where the sequence
 $(j_i)$ can be assumed strictly ascending.
So, by step 2 above, 
$\Iter{t_i}{i} \Basicleq^\Ax_\prdem t'_{j_i}$, for every $i$.
Whence by admissibility, 
$\bigsqcup_i \Iter{t_i}{i} \Basicleq^\Ax_\prdem  \bigsqcup_i t'_{j_i}$; i.e.,
$\Iter{t}{\infty}  \Basicleq^\Ax_\prdem  t'$. 
Thus $t \Basicleq^\Ax_\prdem  t'$, by Proposition~\ref{proposition:horn}.
 \end{proof}



\section{Related and future work}
\label{section:conclusions}

The results in this paper concern three methods of defining \emph{basic operational preorders} on effect trees, which we claim to be a useful abstraction for defining contextual preorder for programming languages with algebraic effects. This has been verified for simple call-by-name~\cite{gom} and call-by-value\footnotemark[3] languages, but needs further substantiation.

The axiomatic approach to defining basic operational preorders in Section~\ref{section:axiomatic} is close in spirit to the algebraic axiomatisation of effects of Plotkin and Power~\cite{PlotkinPower2002}, but with a different focus. 
In~\cite{PlotkinPower2002}, (in)equational axiomatisations are required in order to determine a
free-algebra monad modelling denotational equality of programs. Such axiomatisations have also been used to combine effects~\cite{hyland2006combining},
 and to induce a logic of effects~\cite{PlotkinPretnarLogic}; but they have not 
 hitherto been explicated as a method for defining  contextual preorder/equivalence. 
 In this paper, we have used  infinitary Horn clause axioms between infinitary terms for this purpose, with the notion of admissible model playing an important role.
 
 The main coincidence theorem in Section~\ref{section:equivalence} has some precursors in the literature. 
 The characterisations of  $\mathcal{H}\mathcal{V}_{\leq 1} D$
 and  $\mathcal{S}\mathcal{V}_{\leq 1} D$ as free Kegelspitze in~\cite{KeimelP2016} can be viewed as completeness theorems for inequational axiomatisations with respect to
 \emph{domains} $D$. In the special case $D = \mathbb{N}$, this is implied by our results, for it can be derived from Lemma~\ref{lemma:completeness} that the partial-order quotients of 
 $\Tree(\mathbb{N})$ by $\Basicleq^\Ax_\prang$ and $\Basicleq^\Ax_\prdem$ are isomorphic to
  $\mathcal{H}\mathcal{V}_{\leq 1} \mathbb{N}$
 and  $\mathcal{S}\mathcal{V}_{\leq 1} \mathbb{N}\,$.  Another related completeness result is given in~\cite{mislove2004axioms},  where inequational axioms for 
 a simple process algebra with nondeterministic and probabilistic choice are proved
 complete with respect to a domain-theoretic semantics. Translated into our setting, this process algebra corresponds to  \emph{regular trees} in a signature that combines
 the operations $\orEff$, $\prEff$ with an additional prefix operation and zero constant.
In~\cite{mislove2004axioms}, the semantics uses the convex powerdomain, rather than the upper $\mathcal{S}$ and lower $\mathcal{H}$ that we consider. In  the present paper, we have 
not considered  convex powerdomains and the associated \emph{neutral} (as opposed to angelic or demonic) nondeterminism. However, it would be a natural extension to do so.

The main limitation we see of the present paper is the restriction throughout to \emph{admissible} basic operational preorders. The admissibility condition plays a fundamental role in almost everything we do. It is, however, violated by some natural operational preorders; for example,  for countable demonic nondeterminism. 
It is an open question how to incorporate such more general preorders into our theory.








%%
%% Bibliography
%%

%% Please use bibtex, 

\bibliography{bibliographie}


\end{document}
