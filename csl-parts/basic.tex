\section{Basic operational preorders}
\label{section:basic}

As discussed in Section~\ref{section:intro}, out  interest in effect trees is that they provide a 
uniform template for defining 
 \emph{contextual preorders} for programming languages with algebraic effect operations
specified by signature $\Sigma$. 
As in~\cite{gom}, the crucial data is provided by a 
preorder  $\Basicleq$  on  $\Trees(\Nat)$, called the \emph{basic operational preorder}.  
In order for the resulting contextual preorder to be well behaved, we ask for the 
the basic operational preorder satisfy two properties: \emph{admissibility} and \emph{compositionality}.
In this section, we review the definitions of these and related notions. 

%We consider how to define a notion of \emph{contextual preorder} between programs in some unspecified programming language with algebraic effects from signature $\Sigma$. (Notions of \emph{contextual equivalence}
%are included as a special cases, as any equivalence relation is a preorder.) The general idea of a contextual preorder, is to relate comparable programs $P_1$ and $P_2$ (in a typed language, programs of different type would not be considered comparable), by considering how they behave in any program context $C[-]$ that produces `observable' behaviour. One thing that can certainly be observed is the return value (if any) of a computation, as long as this return value is discrete. In any sufficiently expressive language, such discrete values should be convertible to natural numbers. So it is a not unreasonable  restriction to consider contexts whose return values (if any) are natural numbers.  We call such contexts \emph{ground contexts}.
%
%We thus compare the behaviour of $P_1$ and $P_2$ by wrapping them in ground contexts $C[-]$. Since the programming language contains effects, the `observable'  behaviour of  $C[P_1]$ and $C[P_2]$ will, in general, involve not only return values, but also  the effects performed by the two computations. 
%%Moreover, exactly what counts as  `observable' and how it affects the comparison of ground type computations
%% will be highly dependent on the specific effects present. 
%Independently of the specific effects present, the execution behaviour of any ground type computation $M$ can be captured by an effect tree $|M|$ that represents all the effect operations that $M$ might potentially perform, and their dependencies. This suggests the following uniform approach to defining contextual equivalence.  As long as we have a specified \emph{basic operational preorder}  $\Basicleq$  on ground  computation trees, that is, on the set $\Trees(\Nat)$, we can define 
%contextual preorder on comparable programs $P_1,P_2$, by:
%\[ P_1 \sqsubseteq_\text{ctxt} P_2 ~ \iff ~
%\text{for all ground contexts $C[-]$, $~ |C[P_1]| \Basicleq |C[P_2]|$} \enspace . \]
%
%As in~\cite{gom}, we view the crucial data is provided by a 
%preorder  $\Basicleq$  on  $\Trees(\Nat)$, called the \emph{basic operational preorder}.  
%Any effect-specific behaviour resulting from  the particular effect operations present is captured in the relation $\Basicleq$ itself. Given this, the remaining definition of the contextual preorder $\sqsubseteq_\text{ctxt}$ is generic, relying only on a general operational semantics that transforms any computation $M$ to its effect tree $|M|$.
%
%In~\cite{gom}, this general approach was developed in detail for 
%a polymorphically typed call-by-name functional language with algebraic effects. 
%The main result was that the resulting contextual preorder is well-behaved if the basic operational preorder satisfies two properties, called \emph{admissibility} and \emph{compositionality}. Such well-behaved properties include that it the contextual preorder is definable as a {logical relation} (and hence amenable to an important proof technique), and also that, on ground type programs $P_1,P_2$,
%the contextual and basic operational preorders coincide (i.e., $P_1 \sqsubseteq_\text{ctxt} P_2$ iff 
%$|P_1| \Basicleq |P_2|$).
%
%It is our belief that the conditions of admissibility and compositionality are fundamental ones that  are applicable
%across a range of programming  languages. Indeed, we believe that it holds in broad generality that any
%admissible, compositional basic operational preorder  give rise  to a well-behaved notion of contextual preorder. In addition to the 
%call-by-name language of~\cite{gom}, we have verified this explicitly in the case of a call-by-value functional language with effects,
%similar to the language in~\cite{plotkin2001adequacy}. More generally, it seems  that the crucial language-independent property required, for this to hold, is that all effect operations are \emph{algebraic} in the sense of Plotkin and Power ~\cite{plotkin2001adequacy} [[CHECK THIS REFERENCE CONTAINS THE RIGHT USE OF ALGEBRAIC]].
%
%\todo[inline]{PERHAPS PUT MOST OF ABOVE IN INTRO}
%
%In summary, defining a \emph{basic operational preorder} on  $\Trees(\Nat)$ provides a uniform language-independent template for defining contextual preorder. Moreover, if the basic operational is \emph{admissible} and \emph{compositional} then the 
%contextual preorder is well behaved. The rest of this section reviews the technical definitions of these notions. 



\begin{definition}[Admissibility]
    A binary relation $R$ on $\Trees(X)$ is \emph{admissible} if,
    for every ascending chain $(t_i)_{i \geq 0}$ and 
    $(t_i')_{i\geq 0}$, we have:
    \[ \text{($\,t_i  R \, t_i'$ for all $i\,$)} ~ \implies~
        \left(\bigsqcup_{i \geq 0} t_i\right) \, R \, \left(\bigsqcup_{i \geq 0} t_i'\right) \enspace .
    \]
\end{definition}

\begin{definition}[Compatibility]
    A binary relation $R$ on $\Trees(X)$ is  \emph{compatible} if,
    for every $o \in \Sigma$ of arity $n$, and for all trees 
     $t_1,\dots, t_n$ and $t'_1, \dots, t'_n$, we have:
    \[ \text{($\,t_i  R \, t'_i$ for all $i = 1, \dots, n\,$)} ~ \implies ~ 
        o(t_1, \dots, t_n) \, R \; o(t'_1, \dots, t'_n) \enspace .
    \]
\end{definition}
If a compatible relation is a preorder then it is called a \emph{precongruence}. If it is an equivalence relation it is called a \emph{congruence}.


The next two definitions make use of the substitution operation on trees defined at the end of
Section~\ref{section:trees}.
\begin{definition}[Substitutivity]
    A binary relation $R$ on $\Trees(X)$ is  \emph{substitutive} if,
    for all trees $t$, $t'$ and $\{t_x\}_{x \in X}$, we have:
    \[ \text{$\,t\, R \, t'$} ~ \implies ~ 
       t[ x \mapsto t_x] \, R \, t'[ x \mapsto t_x] \enspace .
    \]
\end{definition}



\begin{definition}[Compositionality]
    A binary relation $R$ on $\Trees(X)$ is \emph{compositional} if, for all 
    trees $t$, $t'$,  $\{t_x\}_{x \in X}$,  and $\{t'_x\}_{x \in X}$, we have:
        \[ \text{($\,t \, R \, t'$ and $t_x \, R \, t'_x$ for all $x \in X\,$)} ~ \implies ~ 
        t[ x \mapsto t_x] \, R \, t'[ x \mapsto t'_x] \enspace .
    \]
\end{definition}



\begin{proposition} 
\label{proposition:substitutive}
Let $\Basicleq$ be a preorder  on $\Trees(\mathbb{N})$.
\begin{enumerate} 
\item If  $\Basicleq$ is compositional then it is a substitutive precongruence.
\item If $\Basicleq$ is an admissible substitutive precongruence then it is compositional.
\end{enumerate}
\end{proposition}

%\noindent
%\todo[inline]{Only give the ideas of the proof to gain space}
\begin{proof}
    \begin{enumerate}
        \item 
             Suppose $\Basicleq$ is compositional. 
            %It is substitutive because if $t \Basicleq t'$ and $\{ t_n \}_{n
            %\in \mathbb{N}}$ is a family of trees then, since 
            %$t_n \Basicleq t_n$,  we have 
            $t[ n \mapsto t_n] \Basicleq t'[n \mapsto t_n]$ by compositionality.
            For compatibility, if $o$ is an operation of arity $n$, and $t_i \Basicleq t'_i$ for $i = 1 \dots n$, then since $o(1,\dots, n) \Basicleq o(1,\dots, n)$,
             we indeed have $o(t_1, \dots, t_n) \, \Basicleq \; o(t'_1, \dots, t'_n)$, by compositionality.
             Substitutivity is by a similar argument, also using reflexivity.

        \item 
            Suppose $\Basicleq$ is admissible, substitutive and compatible. 
            Suppose also that $t \Basicleq t'$ and $t_n \Basicleq t_n'$, for all $n \in \mathbb{N}$.
            By substitutivity, we have $t[n \mapsto t_n] \Basicleq t'[n \mapsto t_n]$.
             We would like to use compatibility to derive 
            that also $t'[n \mapsto t_n] \Basicleq t'[n \mapsto
            t_n']$, however this is only possible if $t'$ is finite. 
            The solution is to use finite approximations $(s_i')$ of $t'$
            satisfying $\sqcup_i s_i' = t'$. For each finite tree $s_i'$
            we have that $s_i'[n \mapsto t_n] \Basicleq s_i'[n \mapsto t_n']$, by compatibility.
            Hence, by admissibility, $t'[n \mapsto t_n] \Basicleq t'[n \mapsto
            t_n']$, whence $t[n \mapsto t_n] \Basicleq t'[n \mapsto t_n']$ by transitivity.
        \end{enumerate}
\end{proof}


%
%\todo[inline]{[[MOTIVATE]]}
%Following~\cite{gom}, we consider the basic ingredient for specifying  a notion of
%contextual equivalence for a programming language to be a preorder $\Basicleq$ on $\Trees(\mathbb{N})$.
%As long as the preorder is both admissible and compositional (equivalently an
%admissible substitutive precongruence), the mathematical tools of~\cite{gom} and
%Sections \textbf{[[WHICH]]} of the present paper are applicable. These allow fundamental properties of contextual equivalence to be proved.
%
%We observe that admissible compositional preorders are closed under arbitrary intersection. That is, if
%$\mathcal{R} \subseteq \mathcal{P}(\mathbb{N} \times \mathbb{N})$ is a family of admissible compositional preorders
%then so is $\bigcap \mathcal{R}$.
%
%Every family $\mathcal{O} \subseteq \mathcal{P}(\Trees(\mathbb{N}))$ determines a preorder $\Basicleq_\mathcal{O}$ on 
%$\Trees(\mathbb{N})$ by
%\begin{equation}
%\label{equation:observational-preorder}
%t \Basicleq_\mathcal{O} t' ~ \Leftrightarrow ~ \forall U \in \mathcal{O}~ (t \in U ~ \Rightarrow ~ t' \in U) \enspace .
%\end{equation}
%
%\begin{proposition} 
%The following are equivalent for
%any admissible  preorder  $\Basicleq$ on $\Trees(\mathbb{N})$.
%\begin{enumerate} 
%\item $\bot \Basicleq t$,  for every $t \in \Trees(\mathbb{N})$.
%\item $t \Treeleq t'$ implies $t \Basicleq t'$ for all $t,t' \in \Trees(\mathbb{N})$.
%\end{enumerate}
%For an arbitrary preorder $\Basicleq$, the following are equivalent.
%\begin{enumerate}
%\setcounter{enumi}{2}
%\item $\Basicleq$ is admissible and satisfies 1 (equivalently 2) above.
%\item $\Basicleq$ arises as $\Basicleq_\mathcal{O}$ for some family $\mathcal{O}$ of Scott-open subsets
%of $\Trees(\mathbb{N})$.
%\end{enumerate}
%\end{proposition}
%
%It is possible to characterise the compositionality property for relations of the form $\Basicleq_\mathcal{O}$, using a notion of \emph{decomposability} of $\mathcal{O}$, see~\cite{gom}. 
%\todo[inline]{[[IT WOULD POSSIBLY BE NICE TO REPEAT THIS AND REFINE IT, E.G., TO
%REFLECT THE SUBSTITUTIVE PRECONGRUENCE DEFINITION OF COMPOSITIONALITY.]]}
%


