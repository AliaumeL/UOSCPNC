\section{Operationally-defined preorders}
\label{section:operational}

In this section, we consider our first approach to defining an admissible and compositional basic operational
preorder $\Basicleq$ on $\Trees(\mathbb{N})$. We call this method \emph{operational}. Its characteristic is that the preorder
 $\Basicleq$ is directly defined using a mathematical model of the way that an effect tree in $\Trees(\mathbb{\Nat})$ will be executed.
There is not much to say in general about this approach, since such execution models vary enormously from one effect to another.
The main point to emphasise is that there is no general reason for  admissibility and compositionality to hold for such operationally defined preorders. Accordingly, these properties need to be established on a case-by-case basis.

The operatuional approach to defining basic preorders is illustrated for several examples of effects in~\cite{gom}. 
The main goal of the section is to demonstrate the approach using a different example, the signature
$\Sigma_\prnd = \{\prEff,\orEff\}$ from Example~\ref{example:prnd}, which is of interest because of the 
interplay between probabilistic and nondeterministic effects. 
In this case, trees in $\Trees(\mathbb{N})$ have both probabilistic and nondeterministic branching nodes,
as in Figure~\ref{fig:exampletrees}.


It is natural to consider such trees as (countable state) Markov decision processes, with the leaves representing nodes which either carry an observable value from $\mathbb{N}$, or which represent nontermination $\bot$.
%The behaviour of probabilistic nodes is clear. 
Nondeterministic choices may be thought of as being resolved by an external agent, the scheduler. We model the actions of the scheduler by a 
function $s: \{l,r\}^* \to \{l,r\}$. The idea is that a word $w \in \{l,r\}^*$ represents a finite path of left/right choices from the root of a 
tree $t \in \Trees(\mathbb{N})$. If the computation reaches a nondeterministic choice at the node indexed by 
$w$ then it takes the left/right branch according to the value of $s(w)$. This way of representing choices has some redundancy
(in every tree that is not a complete infinite binary tree, there will be words $w$ that do not index nodes in the tree; if $s(\varepsilon) = l$ then the value of $s$ on words beginning with $r$ is immaterial; the value of $s(w)$ on words $w$ that index 
probabilistic nodes in $t$ is irrelevant, etc.), but it is simple and convenient for future purposes. 
For any given $t \in \Trees(\mathbb{N})$, such a function
$s: \{l,r\}^* \to \{l,r\}$ can be thought of as a (deterministic) \emph{strategy} for the scheduler, in which the choice of direction at a nondeterministic node  
can respond to  the outcomes of probabilistic nodes higher up the tree.

A strategy $s$ and a tree $t$ in combination determine a subtree $t\restriction s$, defined by 
removing, at every nondeterministic node in $t$ with index $w$, the child tree that is not selected by $s(w)$. So $t\restriction s$ is a tree that has binary branching at probabilistic nodes, and unary branching at nondeterministic nodes. It is thus, in effect, a purely probabilistic tree, with leaves in $\mathbb{N}\cup\{\bot\}$, and so may be viewed as a Markov chain, in which the branching nodes are fair binary choices, determining  a subprobability distribution over $\mathbb{N}$. Specifically, each $n \in \mathbb{N}$ is assigned the probability that a run of the Markov chain will end at a leaf labelled with $n$. This is a subprobability distribution in general because there can  be a positive probability of nontermination (either at a $\bot$ leaf, or along an infinite branch).

%Given such a scheduling function $s$, we write $t@s$ for the result of the computation as scheduled by $s$. This is defined by:
%\[
%t@s ~ = ~ \begin{cases} 
% n & \text{if there exists $w \in \{l,r\}^*$ indexing an $n$ node in $t$} \\
%    & ~~~\text{such that, for every $i < |w|$, $~w_{i+1} = s(w\!\restriction_i)\,$;} \\
%  \bot & \text{otherwise.}
% \end{cases}
%\]
%Here we write $|w|$ for the length of a word, $w_i$ for the $i$-th symbol in a word, and $w \!\restriction_i$ for the prefix of $w$ that has length $i$.

The  \emph{angelic} interpretation of nondeterminism takes into account the possibility of a nondeterministic computation achieving a specified goal, given a cooperative scheduler.  The  \emph{demonic} interpretation, 
models the {certainty} with which a goal can be achieved, however adversarial the scheduler. 
This suggests the  two  basic operational preorders below. 
In each case, we consider functions $h \colon \mathbb{N} \to [0,\infty]$ assigning desirability weightings to possible results of a run of the computation. We then define 
%$H \subseteq \mathbb{N}$ to be a desirable set of outcomes, and we define 
$t \Basicleq t'$ if, for any $h$, the `expected' desirability weighting of $t'$ exceeds that of $t$. Here, `expected' is in inverted commas, because we have to take into account the actions of the scheduler, so this is not just a probabilistic expectation. In the case of 
angelic nondeterminism, the scheduler will help us, whereas, under demonic nondeterminism, it will impede us.
Technically this is taken account of by considering suprema of probabilistic expectations in the angelic case, and infima in the demonic case.
\begin{align*}
t \Basicleq^\Op_\prang t' ~ \Leftrightarrow ~ ~& \forall h \colon \mathbb{N} \to [0,\infty]~~ \sup_s  \mathbf{E}_{t\restriction\!s} (h)~ \leq~ \sup_s \mathbf{E}_{t'\restriction s} (h)
\\
t \Basicleq^\Op_\prdem t' ~ \Leftrightarrow ~ ~& \forall h \colon \mathbb{N} \to [0,\infty]~~ \inf_s  \mathbf{E}_{t\restriction s} (h)~ \leq~ \inf_s \mathbf{E}_{t'\restriction s} (h)
\end{align*}
In Markov-decision-process terminology, each preorder says that the \emph{value} of the MDP $t$, for any weighting $h$, is below the value of of $t'$ for $h$. In the angelic case the value maximises the expectation of $h$, in the demonic case it minimises it. 
\begin{proposition}
The preorders $\Basicleq^\Op_\prang$ and $\Basicleq^\Op_\prdem$ are admissible and compositional.
\end{proposition}

We outline the proof of this proposition in the case of $\Basicleq^\Op_\prdem$. The proof for  $\Basicleq^\Op_\prang$ is easier, largely because the analogue of the lemma below is trivial in the case of angelic nondeterminism.
\begin{lemma} 
\label{lemma:F-continuous}
Consider $\Tree(\mathbb{N})$ and $[0,+\infty]$ as $\omega$CPOs. Then,
for any $h \colon \mathbb{N} \to [0,\infty]$, 
the value-finding function \[F_h : t \mapsto \inf_s  \mathbf{E}_{t\restriction s} (h) : \Tree(\mathbb{N}) \to [0,+\infty]\]
is continuous.
\end{lemma}
\begin{proof}
The set $S = \{l,r\}^{\{l,r\}^*}$ of strategies is a countably-based compact Hausdorff space under the product topology. (It is Cantor space.)
It is easy to see that  the function 
\[G_h \colon (s,t) \mapsto \mathbf{E}_{t\restriction s} (h) : S \times \Tree(\mathbb{N})  \to [0,+\infty] \]
is continuous.
Then $F_h$ is continuous because it is defined from $G_h$ by taking an infimum over a compact set. 
This follows, for example, from Section 7.3 of~\cite{AndreaShalk}. For completeness, we give a self-contained argument. 

Suppose $(t_i)$ is an ascending chain of trees.
 %Clearly $\inf_s \sup_i G_h(s,t_i) \geq \sup_i \inf_s G_h(s,t_i)$.
Because $S$ is compact,  there is $s_i \in S$ with $\inf_s G_h(s,t_i) = G_h(s_i, t_i)$,
and we can then extract a convergent 
subsequence  $(s_{a_i})$ of $(s_i)$ such that $s_{a_i} \rightarrow s_\infty$ in $S$. Then:
%it is clear that the following inequalities hold:
\begin{equation*}
                \sup_i \inf_s G_h(s,t_i)
                \, =\, 
                \sup_i G_h(s_i, t_i)
                \, = \,
                \sup_i G_h(s_{a_i}, t_{a_i})
                \, = \,
                G_h(s_\infty, \, \sqcup_i t_i)
                \, \geq \,
                \inf_s  G_h(s, \, \sqcup_i t_i)\, ,
            \end{equation*}
where the second equality holds because $G_h(s_i, t_i)$ is an ascending sequence, and the third by the continuity of  $G_h$.
We have shown that $ \sup_i  \inf_s G_h(s,t_i) \geq \inf_s  G_h(s, \sqcup_i t_i)$, i.e., 
$ \sup_i  F_h(t_i) \geq F_h(\sqcup_i t_i)$. 
Therefore $F_h$ is continuous (since it is obviously monotone).
\end{proof}

\noindent
The admissibility of $\Basicleq^\Op_\prdem$ follows easily from the lemma.
Suppose $t_i \Basicleq^\Op_\prdem t_i'$, for ascending chains $(t_i)$ and $(t'_i)$.
Then $F_h(t_i) \leq F_h (t_i')$, for all $i$ and $h$. By the lemma, 
 $F_h (\sqcup_i t_i) \leq F_h (\sqcup_i t_i')$, for all $h$.
So indeed  $\sqcup_i t_i \Basicleq^\Op_\prdem \sqcup_i t_i'$.

For compositionality, by Proposition~\ref{proposition:substitutive}, it suffices to show
that $\Basicleq^\Op_\prdem$ is a substitutive precongruence. The compatibility properties 
of a precongruence are easily shown. Substitutivity follows from the lemma below.
%
\begin{lemma}
Suppose $t$ and 
$\{ t_n \}_{n \in \mathbb{N}}$ are trees in 
            $\Tree(\Nat)$ then, for any weighting $h$,
             \begin{equation*}
                \inf_s \mathbf{E}_{t [ n \mapsto t_n ] \restriction  s } (h)
                = 
                \inf_s \mathbf{E}_{t \restriction  s} (\hat{h})
                \quad \text{ where } \quad \hat{h} (n) = \inf_s \mathbf{E}_{t_n \restriction  s} (h) \enspace .
            \end{equation*}
\end{lemma}
This lemma is proved first for finite trees, by induction on the height of the tree. It is then extended to infinite trees
by expressing them as suprema of finite trees, and applying Lemma~\ref{lemma:F-continuous}.
%
%            \todo[inline]{The proof of this lemma is long and not interesting, 
%            should we mention it ? Do we need the rest of the proof ?}
%            
%            Using this key lemma the proof can go smoothly because (by
%            definition)
%            if $\sigma \Basicleq^\Op_\prang \sigma'$ then 
%            $h_\sigma \leq h_\sigma'$.
%
%            \begin{align*}
%                \inf_s \mathbf{E}_{t\sigma \restriction  s} (h)
%                 & = \inf_s \mathbf{E}_{t \restriction  s} (h_\sigma)       & \text{ key lemma }  \\
%                 & \leq \inf_s \mathbf{E}_{t \restriction  s} (h_{\sigma'}) & \text{ monotonicity wrt the weighting function } \\
%                 & \leq \inf_s \mathbf{E}_{t'\restriction  s} (h_{\sigma'}) & \text{ monotonicity wrt the tree } \\
%                 & = \inf_s \mathbf{E}_{t'\sigma' \restriction  s} (h )     & \text{ key lemma } \\
%            \end{align*}
%
%            Therefore $ t \Basicleq^\Op_\prang t'$ and $\sigma
%            \Basicleq^\Op_\prang \sigma'$ implies $t\sigma \Basicleq^\Op_\prang
%            t'\sigma'$.
% 

We end this section by observing that a natural attempt to simplify  the definitions of 
 $\Basicleq^\Op_\prang$ and $\Basicleq^\Op_\prdem$ does not work. Instead of considering arbitrary weightings
 $h \colon \mathbb{N} \to [0,\infty]$, one might restrict to  
 functions $h \colon \mathbb{N} \to \{0,1\}$, which can be viewed as specifying goal subsets $H \subseteq \mathbb{N}$.
 Proceeding analogously to above, we compare suprema of probabilities of landing in $H$ in the angelic case, and infima in the demonic case. For both the angelic and demonic versions, the desired compositionality property fails.
\begin{proposition}
Neither of the formulas below defines a compositional relation $t \Basicleq t'$. %$\Basicleq^\bullet_\prang$ nor $\Basicleq^\bullet_\prang$ is compositional.
 \begin{align*}
%t \Basicleq^\bullet_\prang t' ~ \Leftrightarrow ~ ~
 & \forall H \subseteq \mathbb{N}  ~~ \sup_s  \mathbf{P}_{t\restriction s} (H)~ \leq~ \sup_s \mathbf{P}_{t'\restriction s} (H)
\\
%t \Basicleq^\bullet_\prdem t' ~ \Leftrightarrow ~ ~
 & \forall H \subseteq \mathbb{N}  ~~ \inf_s  \mathbf{P}_{t\restriction s} (H)~ \leq~ \inf_s \mathbf{P}_{t'\restriction s} (H)
\end{align*}
\end{proposition}


 
 




%Although natural, the above definitions do not work. 

\begin{proof}
    We use the two trees in Figure~\ref{fig:exampletrees},
    representing the expressions $A = 3 \orEff (1 \prEff 2)$
    and $B = (3 \orEff 1) \prEff (3 \orEff 2)$. It is easily checked that, for every subset $H \subseteq \{ 1, 2, 3 \}$,
    it holds that $\sup_s  \mathbf{P}_{A\restriction s} (H) =  \sup_s \mathbf{P}_{B\restriction s} (H)$ and
    $\inf_s  \mathbf{P}_{A\restriction s} (H) =  \inf_s \mathbf{P}_{B\restriction s} (H)$.
     Thus $A$ is equivalent 
    to $B$ under  both preorders.

    However, one can build a family $\{ t_1, t_2, t_3\}$ such that 
    $A[ i \mapsto t_i] = t_3 \orEff (t_1 \prEff t_3) = C$ is not equivalent to 
    $B[ i\mapsto t_i] = (t_3 \orEff t_1) \prEff (t_3 \orEff t_2) = D$,
    which contradicts substitutivity.
    Let $t_1 = 0 \prEff (0 \prEff (0 \prEff (0 \prEff 1)))$,
    $t_2 = 1$ and $t_3 = 0 \prEff (0 \prEff (0 \prEff 1))$. The distinguishing 
    factor will be the probability associated with the subset $\{ 1 \}$.
    %We dress a table of the minimal/maximal said probability on the different  trees.

    A simple calculation shows that $\sup_s  \mathbf{P}_{C\restriction s} (\{1\}) = 9/16
    \neq 5/8 = \sup_s \mathbf{P}_{D\restriction s} (\{ 1\})$. Similarly
    $\inf_s  \mathbf{P}_{C\restriction s} (\{1\}) = 1/4 \neq 3/16 =
     \inf_s \mathbf{P}_{D \restriction s } (\{1\})$.
     This contradicts the substitutivity and hence also the compositionality of both preorders.
        %\begin{equation*}
        %\begin{array}{c|c|c}
                %& \textbf{Min} & \textbf{Max} \\ \hline
            %t_1 & 1/8 & - \\
            %t_2 & 1   & -   \\
            %t_3 & 1/4 & - \\
            %t_1 \prEff t_2 & 9 / 16 & - \\
            %t_3 \orEff t_1 & 1/8    & 1/4 \\
            %t_3 \orEff t_2 & 1/4    & 1   \\
            %t_3 \orEff (t_1 \prEff t_3) & 1/4 & 9/16 \\
            %(t_3 \orEff t_1) \prEff (t_3 \orEff t_2) & 3 / 16 & 5/8 \\
        %\end{array}
    %\end{equation*}
\end{proof}



We view the fact that the use of quantitative properties is required to obtain a compositional preorder as being
a mathematical analogue of the situation found in, for example,~\cite{KozenPDL,MciverMorgan}, where it is argued that the use of quantitative rather than Boolean program logics is necessary for obtaining compositional reasoning methods for programs with probabilistic (combined with nondeterministic in~\cite{MciverMorgan}) behaviour.

%The above argument is also curiously  reminiscent of the proof in [[VARACCA PHD]]
% that there is no distributivity law between the finite powerset and distribution monads.


%The situation is rescued by generalising the boolean-valued test properties $H$ to 
%quantitative properties $h \colon \mathbb{N} \to [0,\infty]$. 
%Accordingly, we define:

%\todo[inline]{I THINK FOR SPACE REASONS WE WILL NEED TO JUMP IN STRAIGHT AWAY
%    WITH THE MIXED CASE. SO THE FOLLOWING DISCUSSION NEEDS TO BE CUT OUT WITH
%SOME REQUIRED PARTS MOVED TO THE MIXED CASE}
%
%However, we first
%briefly review the simpler cases, $\Sigma_\pr = \{\prEff\}$ and $\Sigma_\nd = \{\orEff\}$, of purely probabilistic and 
%purely nondeterministic computation. These two cases are already covered in~\cite{gom}, so our treatment will be brief. 
%
%First, we  consider the case $\Sigma_\pr = \{\prEff\}$. Every internal node in a  tree $t \in \Trees(\mathbb{N})$  is a binary branching node labelled with $\prEff$ representing a probabilistic choice with each branch having probability $\frac{1}{2}$. The tree can thus be interpreted as a (countable state) Markov-chain, that determines a discrete subprobability distribution over the set $\mathbb{N}$.
%(It is a subprobability distribution because there can be a positive probability of nontermination, both silent and noisy.) We write
%$\mathbf{P}_t (H)$ for the probability assigned to a set $H \subseteq \mathbb{N}$
%by the tree $t$, and $\mathbf{E}_t (h)$ for the (possibly infinite) expectation of a function $h\colon \mathbb{N} \to [0,\infty]$ under the subprobability distribution on $\mathbb{N}$ determined by $t$.
%The basic operational preorder can be defined using any of the three equivalent properties below.
%\begin{align*}
%t \Basicleq_\pr t' ~~ \Leftrightarrow ~~ &  \forall n ~ ~ \mathbf{P}_t (\{n\}) \leq \mathbf{P}_{t'}(\{n\})  \\
% ~ \Leftrightarrow ~~ & \forall H \subseteq \mathbb{N} ~~ \mathbf{P}_t (H) \leq \mathbf{P}_{t'}(H) \\
% ~ \Leftrightarrow ~~ & \forall h\colon \mathbb{N} \to [0,\infty] ~~ \mathbf{E}_t (h) \leq \mathbf{E}_{t'}(h) \enspace .
%\end{align*}
%Here one can view $H$ as a test set of desirable results, and $h$ as a function that assigns (possibly infinite) desirability weights 
%to result values in $\mathbb{N}$. The first alternative above has the advantage of simplicity. The second strengthens the definition by testing the probability of arbitrary properties. The third further strengthens it by using quantitative properties.
%\begin{proposition}[\cite{gom}]
%The preorder $\Basicleq_\pr$ is admissible and compositional. 
%\end{proposition}
%%\noindent
%%It is also possible to define $\Basicleq_\pr$ cas $\Basicleq_{\mathcal{O}_\pr}$ for the a family of Scott-open subsets of $\Trees(\mathbb{N})$, \emph{viz}:
%%\[
%%\mathcal{O}_\pr ~ = ~ \{\,  \{t \mid \mathbf{P}_t (n) > q\} \mid n \in \mathbb{N}, \,q \in \mathbb{Q}\, \}  \enspace .
%%\]
%
%Next, we consider the case $\Sigma_\nd = \{\orEff\}\,$, in which every internal node in a tree is a binary nondeterministic choice.
%Compared with~\cite{gom}, we give a more directly operational treatment in terms of 
% \emph{schedulers}, namely external agents that resolve nondeterministic choices.
%We model the scheduler as a 
%function $s: \{l,r\}^* \to \{l,r\}$. The idea is that a word $w \in \{l,r\}^*$ represents a finite path of left/right choices from the root of a 
%tree $t \in \Trees(\mathbb{N})$. If the computation reaches a nondeterministic choice at the node indexed by 
%$w$ then it takes the left/right branch according to the value of $s(w)$. This way of representing choices has some redundancy
%(in every tree other than the complete infinite binary tree, there will be words $w$ that do not index nodes in the tree; if $s(\varepsilon) = l$ then the value of $s$ on words beginning with $r$ is immaterial; etc.), but it is simple and convenient for future purposes. 
%Given such a scheduling function $s$, we write $t@s$ for the result of the computation as scheduled by $s$. This is defined by:
%\[
%t@s ~ = ~ \begin{cases} 
% n & \text{if there exists $w \in \{l,r\}^*$ indexing an $n$ node in $t$} \\
%    & ~~~\text{such that, for every $i < |w|$, $~w_{i+1} = s(w\!\restriction_i)\,$;} \\
%  \bot & \text{otherwise.}
% \end{cases}
%\]
%Here we write $|w|$ for the length of a word, $w_i$ for the $i$-th symbol in a word, and $w \!\restriction_i$ for the prefix of $w$ that has length $i$.
%
%For angelic nondeterminism, one again has three possible definitions of the basic operational preorder
%$\Basicleq_\ang$, paralleling the three alternatives above. In the definitions below, $s$ ranges over schedulers, the statement $t@s \in H$, where $H \subseteq \mathbb{N}$, implies in particular that $t@s \neq\bot$, and we write $h_\bot$ for the unique strict function from $\mathbb{N}_\bot$ to
%$[0,\infty]$ extending $h$.
%\begin{align*}
%t \Basicleq_\ang t' ~ \Leftrightarrow ~ ~& \forall n ~~ (\exists s ~\; t@s =n)~ \Rightarrow~ (\exists s ~ \; t'@s =n) 
%\\
%~ \Leftrightarrow ~ ~& \forall H \subseteq \mathbb{N}~~ (\exists s ~\; t@s \in H)~ \Rightarrow~ (\exists s ~ \; t'@s \in H) 
%\\
%~ \Leftrightarrow ~ ~& \forall h\colon \mathbb{N}\to [0,\infty]~~ \sup_s h_\bot(t@s)~ \leq ~ \sup_s h_\bot(t'@s)
%\end{align*}
%For demonic nondeterminism, the first pattern of definition is not available, because, although  the relation defined by the formula below is admissible, it is  \emph{not} compositional.
%\[
%\forall n ~~ (\forall s ~\; t@s =n)~ \Rightarrow~ (\forall s ~ \; t'@s =n) \enspace .
%\]
%To define the basic preorder $\Basicleq_\dem$ for demonic nondeterminism one can use either of the other two patterns of definition.
%\begin{align*}
%t \Basicleq_\dem t' ~ \Leftrightarrow ~ ~& \forall H \subseteq \mathbb{N}~~ (\forall s ~ \; t@s \in H)~ \Rightarrow~ (\forall s ~ \; t'@s \in H) 
%\\
%~ \Leftrightarrow ~ ~& \forall h\colon \mathbb{N}\to [0,\infty]~~ \inf_s  h_\bot(t@s)~ \leq ~ \inf_s h_\bot(t'@s)
%\end{align*}
%\begin{proposition}[\cite{gom}]
%The preorders $\Basicleq_\ang$  and $\Basicleq_\dem$ are admissible and compositional. 
%\end{proposition}
%\noindent
%Because admissible compositional preorders are closed under intersection, 
%the relation ${\Basicleq_\ang} \cap {\Basicleq_\dem}$ is an admissible compositional preorder too. This gives a basic operational preorder corresponding to a neutral view of nondeterminism that takes account of both possibility and necessity.
%

