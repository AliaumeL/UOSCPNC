\section{Preliminaries}


\subsection{University of Ljubljana (FMF)}

My internship took place in Ljubljana the capital city of Slovenia.
The Faculty of Mathematics and Physics (FMF) of the University of Ljubljana is 
the main university in Slovenia for mathematical studies. Located 
in the southern part of the town, it is accessible by bus and bike,
and is at a walking distance from the center of Ljubljana.

\subsection{Research environment}

My research supervisor was Alex Simpson
who is currently a 
"Professor of Computer Science" at the Faculty of Mathematics and Physics (FMF)
of Ljubljana.
Even if the FMF is not a faculty of computer science,
Alex Simpson is well known in the theoretical area of 
this discipline and presented two talks during the 
CALCO/MFPS conference in Ljubljana on June 2017. 
This year he was 
supervising Niels Vorneeveld, a PhD student, working 
on effects inside programming languages.

The university was able to give me a desk from the first 
day, and when Théo Winterhalter — another 
ENS Paris-Saclay student — left Ljubljana at the end 
of his internship, I was able to move to the office of 
the PhD students. 

The group of PhD students, containing both theoretical 
computer science and complex analysis research,
was very friendly. Once or twice a week, they would 
go to the climbing gym, and on a regular basis having beers 
in the town center. Every working day we had meals 
with Andrej Bauer and Alex Simpson, two teachers and 
supervisors of some PhD students of the group at the 
local restaurant.

I attended the almost regular 
Tuesday seminar on "Foundations of mathematics and theoretical computer science"
with the rest of the group.
I gave one seminar myself in a joint seminar with Niels, where 
we presented the recent work of Dal Lago, Gavazzo and Levy \cite{Ugo2017} on 
abstract bisimulations for a parametrized class of programming languages.

I also attended with Niels Voorneveld and Philipp Haselwarter 
a reading course given by Dr. Mtija Pretnar for PhD students on operational semantics. The 
objective of this course was to understand a paper by Andrew Pitts 
\cite{Pitts2000}, which was tightly related to the technical tools I was 
using in my research. 

I would also like to mention the presence of 
Pierre-Marie Pédrot, a former student from ENS Lyon 
who was also part of this cohesive group, while 
not being a PhD student anymore (post-doc).

On top of this ideal setting for research, 
Ljubljana was hosting the seventh Conference on Algebra and
Coalgebra in Computer Science (CALCO) and 
Mathematical Foundations of Programming Semantics (MFPS)
the same week in the building of the FMF. This was
a wonderful week because it was the occasion 
to see the ongoing work in the area I was working on,
meet and befriend researchers in this domain 
of computer science. The complete list of participants 
and the abstracts of the talks are
available online at \url{http://coalg.org/mfps-calco2017/}.

\subsection{Research subject}

Even before it started, the internship 
was tied to an article published in 2010 
by Alex Simpson and two other authors \cite{gom}.
The paper itself was a step towards having a 
"General Operational Metatheory" for calculi 
with effects, a project that can be linked 
to the theory of algebraic effects of Plotkin 
and Power, and from which some technicalities 
were borrowed, for instance from 
Adequacy for Algebraic Effects \cite{plotkin2001adequacy}.

The goal of the internship was very simple: extend 
as much as possible the results from the last paper. 
The first imperative improvement was to adapt the theory 
from a call-by-name to a call-by-value setting:
it is indeed very unnatural to consider effects in 
a programming language where the evaluation order of said 
effects cannot be determined easily. For instance, consider 
the two programs:

\begin{enumerate}
    \item A function that returns its input with probability 
        one half, and returns one with probability one half: 
        \texttt{function x -> pr (x,1)}
    \item A function that is the identity with probability 
        one half, and the constant function equal to one 
        with probability one half: 
        \texttt{pr (function x -> x, function x -> 1)}
\end{enumerate}

They represent 
two functions that give the same result $\prEff (x, 1)$ when 
applied to the same input $x$. There exists a metatheorem 
for the call-by-name evaluation strategy, stating that this is 
enough to conclude that the two terms are contextually equivalent
\cite{gom}. Contextual equivalence has to be understood
informally as the fact that 
the two programs cannot be distinguished when placed into 
different contexts. However, it is clear that the two programs are fundamentally 
different in call-by-value: one is going to "evaluate" the probability 
only once, and the other one every time the function is called. An example 
of context that can distinguish this behaviour is the following one:

\begin{verbatim}
        let f = *hole* in
        if f 1 == f 1 then 
            return "It is not the first function"
        else
            return "It is the first function"
\end{verbatim}

If the first function is passed, then the result is \texttt{"It is the first
function"} with probability
$3/4$, and if the second function is passed, then the result is \texttt{"It is
not the first function"} with probability one. The context is actually
here to distinguish between a function that is constant and a function 
that is not.

A second very 
natural goal was to tie the results to the denotational 
setting, in order to make a clear connection between 
the two approaches to contextual equivalence.
After this several ways to improve 
the understanding of the method and technical but insightful 
constructions were developed. 

Another crucial point was the study of a non-trivial and 
well-studied combination of two effects: non-determinism and probability.
This combination is known to be challenging \cite{Mislove2000} and
being able to consider it inside our restricted setting is a concrete 
proof of its expressive power. 


This internship has been the occasion to prepare a paper that is going to be 
submitted to FOSSACS in 2018. For the moment, this is still a draft, but most 
of the content is present, and the draft can be found in appendix of this
report. 
