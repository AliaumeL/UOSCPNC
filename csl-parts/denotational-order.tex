\section{Denotationally-defined preorders}
\label{section:denotational}

Our second approach to defining an admissible and compositional basic operational
preorder $\Basicleq$ on $\Trees(\mathbb{N})$ is to make use of established constructions from domain theory.
Under this approach, admissibility and compositionality of the defined preorder $\Basicleq$ hold automatically,
for general reasons. Since this approach essentially amounts to giving a denotational semantics to effect trees, we call it the \emph{denotational} method of defining a basic operational preorder.



In order to define a basic operational preorder using the denotational method, one needs to merely to provide
a continuous $\Sigma$-algebra $D$ (see Section~\ref{section:trees}),  together with a function
% with a distinguished function 
$j\colon \mathbb{N} \to D$. 
Then let   $\llbracket \cdot \rrbracket \colon \Trees(\mathbb{N}) \to D$ be the unique continuous homomorphism that makes the diagram below commute.
   \begin{center}
        \begin{tikzcd}
            \mathbb{N}
            \arrow[r, "j"] 
            \arrow[d, hook, "i"]
            & D \\
            \Trees(\mathbb{N}) \arrow[ru, dashrightarrow, "\llbracket \cdot \rrbracket" below]
        \end{tikzcd}
    \end{center}
\noindent
The map $\llbracket \cdot \rrbracket \colon \Trees(\mathbb{N}) \to D$ is used to induce
the basic operational preorder $\Basicleq_D$ from the partial order relation on the $\omega$CPO $D$.
\[
t \Basicleq_D t' ~~ \Leftrightarrow ~~ \Sem{t} \sqsubseteq \Sem{t' } \enspace .
\]
\begin{proposition}
The relation $\Basicleq_D$ is admissible pregongruence.
\end{proposition}
%
The proof is immediate: admissibility follows from the continuity of 
$\llbracket \cdot \rrbracket$, and compatibility from it being a homomorphism.

In order to obtain substitutivity, hence compositionality, a further property is required.

\begin{definition}[Factorisation property]
    The map $j\colon \mathbb{N} \to D$ is said to have  the \emph{factorisation property} if,
    for every function $f \colon \mathbb{N} \to D$, there exists a 
    continuous homomorphism $h_{\!f} : D \to D$ such that $f = h_{\!f} \circ j$.
    \begin{center}
        \begin{tikzcd}
            \mathbb{N} \arrow[r, "j"] 
                 \arrow[rr, bend right, "f"] &
            D \arrow[r, "h_{\!f}", dashed] & 
            D  
        \end{tikzcd}
    \end{center}
\end{definition}
\begin{proposition}
If $j\colon \mathbb{N} \to D$ has the factorisation property then 
the relation $\Basicleq_D$ is substitutive, hence it is an admissible compositional precongruence.
\end{proposition}

\begin{proof}
Suppose $\sigma: \Nat \to \Tree(\Nat)$ is any  substitution.
Let $\hat{\sigma} : \Tree(\Nat) \to \Tree(\Nat)$ be the continuous homomorphism
such that $\hat{\sigma} \circ i = \sigma$. Consider the map $g := \llbracket \cdot \rrbracket \circ \hat{\sigma} \circ i : \Nat \to D$. By the factorisation property, there exists $h_g : D \to D$ such that
$g = h_g \circ j$. Expanding this, and using the definition of $\Sem{\cdot}$, we have:
\[
 \llbracket \cdot \rrbracket \circ \hat{\sigma} \circ i ~ = ~ h_g \circ j ~ = ~  h_g \circ  \llbracket \cdot \rrbracket \circ i \enspace .
\]
It then follows from  the uniqueness property of Proposition~\ref{proposition:free} that
\begin{equation}
\label{equation:before-pizza}
\llbracket \cdot \rrbracket \circ \hat{\sigma} ~ = ~ h_g \circ  \llbracket \cdot \rrbracket \enspace ,
\end{equation}
because both maps are continuous homomorphisms.

Now, for substitutivity, suppose  that $t \Basicleq_D t'$, i.e., $\Sem{t} \leq \Sem{t'}$. Then 
$h_g (\Sem{t})  \leq h_g(\Sem{t'})$ by monotonicity. That is
$\Sem{ \hat{\sigma}(t)} \leq \Sem{ \hat{\sigma}(t')}$, by~(\ref{equation:before-pizza}). 
This says that $\Sem{ t[\sigma]} \leq \Sem{t'[\sigma]}$. That is
$t[\sigma] \Basicleq_D t'[\sigma]$, as required.
\end{proof}


\begin{lemma}
Let $\mathcal{A}$ be any category given together with a functor $U' \colon \mathcal{A} \to \ContAlg_\Sigma$.
Suppose it holds that that the
composite functor $UU' : \mathcal{A} \to \Set$ has a left adjoint $F$, where $U: \ContAlg_\Sigma \to \Set$ is the forgetful functor.
Then, defining $D$ to be the continuous $\Sigma$-algebra $U' F \, \mathbb{N}$, the unit of the adjunction defines a function
$j \colon \mathbb{N} \to D$ which has the factorisation property.
\end{lemma}

[[THE STATEMENT CAN PERHAPS BE IMPROVED. ALSO DOUBLE CHECK WHETHER $U'$ PERHAPS NEEDS TO BE FAITHFUL.]]


\begin{lemma}
Let $T$ be a monad on the category $\wCPO$.  Suppose that for every $\omega$CPO $X$, we have a 
continuous $\Sigma$-algebra structure on TX. Suppose also that all Kleisli liftings of 
$f^* \colon TX \to TY$, of maps $f \colon X \to TY$, are  continuous  homomorphisms. Then,
defining $D$ to be the continuous $\Sigma$-algebra $T \mathbb{N}$, the unit of the adjunction defines a function
$j \colon \mathbb{N} \to D$ which has the factorisation property.
\end{lemma}

MOTIVATE THE DEFINITIONS BELOW, REFER TO KEIMEL/PLOTKIN AND JEAN G-L

Let $\mathcal{V}_{\leq 1} \,X$ be the $\omega$CPO of (discrete) subprobability distributions on a set $X$.
We write $\mathcal{H}\mathcal{V}_{\leq 1} \,X$ for the $\omega$CPO of nonempty Scott-closed convex subsets
of  $\mathcal{V}_{\leq 1} \,X$  ordered by subset inclusion $\subseteq$. 
We write $\mathcal{S}\mathcal{V}_{\leq 1} \,X$ for the $\omega$CPO of nonempty Scott-compact convex upper-closed subsets
of  $\mathcal{V}_{\leq 1} \,X$  ordered by reverse inclusion $\supseteq$.

DEFINE THE OPERATIONS ON THE ABOVE

USE THE ABOVE TO DEFINE $\Basicleq^\Den_\prang$ and $\Basicleq^\Den_\prdem$

FOLLOWS FROM THE GENERAL MATERIAL ABOVE THAT THESE RELATIONS ADMISSIBLE AND COMPOSITIONAL


