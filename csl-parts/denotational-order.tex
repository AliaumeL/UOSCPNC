\section{Denotationally-defined preorders}

Our second approach to defining an admissible and compositional basic operational
preorder $\Basicleq$ on $\Trees(\mathbb{N})$ is to make use of established constructions from domain theory.
Under this approach, admissibility and compositionality of the defined preorder $\Basicleq$ hold automatically,
for general reasons. Since this approach essentially amounts to giving a denotational semantics to effect trees, we call it the 
\emph{denotational} method of defining a basic operational preorder.

Let $D$ be a continuous $\Sigma$-algebra (see Section~\ref{section:trees}) with a distinguished function 
$j\colon \mathbb{N} \to D$. Let   $\llbracket \cdot \rrbracket \colon \Trees(\mathbb{N}) \to D$ be the unique continuous homomorphism
making the diagram below commute.
   \begin{center}
        \begin{tikzcd}
            \mathbb{N}
            \arrow[r, "j"] 
            \arrow[d, hook, "i"]
            & D \\
            \Trees(\mathbb{N}) \arrow[ru, dashrightarrow, "\llbracket \cdot \rrbracket" below]
        \end{tikzcd}
    \end{center}
\noindent
We use the map $\llbracket \cdot \rrbracket \colon \Trees(\mathbb{N}) \to D$ to define 
the basic operational preorder $\Basicleq_D$ using the partial order relation on the $\omega$CPO $D$.
\[
t \Basicleq_D t' ~~ \Leftrightarrow ~~ \Sem{t} \sqsubseteq \Sem{t' } \enspace .
\]
\begin{proposition}
The relation $\Basicleq_D$ is admissible pregongruence.
\end{proposition}

\begin{proof}
    \todo[inline]{This proof is trivial ?!}
    Both admissibility and compatibility follows directly from the 
    fact that $\llbracket \cdot \rrbracket$ is a continuous 
    homomorphism of $\Sigma$-algebra.
\end{proof}

\begin{definition}[Factorisation property]
    The map $j\colon \mathbb{N} \to D$ is said to have  the \emph{factorisation property} if,
    for every function $f \colon \mathbb{N} \to D$, there exists a 
    continuous homomorphism $h_f : D \to D$ such that $f = h_f \circ j$.
    \begin{center}
        \begin{tikzcd}
            \mathbb{N} \arrow[r, "j"] 
                 \arrow[rr, bend right, "f"] &
            D \arrow[r, "h_f", dashed] & 
            D  
        \end{tikzcd}
    \end{center}
\end{definition}
\begin{proposition}
If $j\colon \mathbb{N} \to D$ has the factorisation property then 
the relation $\Basicleq_D$ is substitutive, hence it is an admissible compositional precongruence.
\end{proposition}

\begin{proof}
    \begin{itemize}
        \item Let $f$ be a function from $\mathbb{N}$ to $D$
            and $\hat{f}$ the unique continuous homomorphism of 
            $\Sigma$-algebra from $\Tree(\mathbb{N})$ to $D$ associated to $f$.
            There exists a unique continuous homomorphism of $\Sigma$-algebra 
            $h_f$ such that $\hat{f} = h_f \circ \llbracket \cdot \rrbracket$.

            Indeed, it suffices to use the analog property on $\mathbb{N}$
            and the universal property of $\Tree(\mathbb{N})$.
        
        \item Let $t,t'$ two trees and $\{ t_n \}_{n \in \mathbb{N}}$
            a substition noted $\sigma$ as a function from $\mathbb{N}$ to
            $\Tree(\mathbb{N})$, and $\hat{\sigma}$ the corresponding 
            lift to trees.

            Using the factorisation property on $\Tree(\mathbb{N})$
            with $f = \llbracket \cdot \rrbracket \circ \hat{\sigma}$,
            there exists a unique continuous homomorphism of 
            $\Sigma$-algebra $h_{\hat{\sigma}}$ such that $\llbracket \cdot
            \rrbracket \circ \hat{\sigma} = h_{\hat{\sigma}} \circ \llbracket
            \cdot \rrbracket$.

            It follows that $\llbracket t[n \mapsto t_n] \rrbracket 
            = \llbracket \hat{\sigma}(t) \rrbracket = h_{\hat{\sigma}} (\llbracket
            t \rrbracket )$, is monotone in $\llbracket t \rrbracket$.

            This proves the substitutivity.
            
    \end{itemize}
\end{proof}


\begin{lemma}
Let $\mathcal{A}$ be any category given together with a functor $U' \colon \mathcal{A} \to \ContAlg_\Sigma$.
Suppose it holds that that the
composite functor $UU' : \mathcal{A} \to \Set$ has a left adjoint $F$, where $U: \ContAlg_\Sigma \to \Set$ is the forgetful functor.
Then, defining $D$ to be the continuous $\Sigma$-algebra $U' F \, \mathbb{N}$, the unit of the adjunction defines a function
$j \colon \mathbb{N} \to D$ which has the factorisation property.
\end{lemma}

[[THE STATEMENT CAN PERHAPS BE IMPROVED. ALSO DOUBLE CHECK WHETHER $U'$ PERHAPS NEEDS TO BE FAITHFUL.]]


\begin{lemma}
Let $T$ be a monad on the category $\wCPO$.  Suppose that for every $\omega$CPO $X$, we have a 
continuous $\Sigma$-algebra structure on TX. Suppose also that all Kleisli liftings of 
$f^* \colon TX \to TY$, of maps $f \colon X \to TY$, are  continuous  homomorphisms. Then,
defining $D$ to be the continuous $\Sigma$-algebra $T \mathbb{N}$, the unit of the adjunction defines a function
$j \colon \mathbb{N} \to D$ which has the factorisation property.
\end{lemma}

MOTIVATE THE DEFINITIONS BELOW, REFER TO KEIMEL/PLOTKIN AND JEAN G-L

Let $\mathcal{V}_{\leq 1} \,X$ be the $\omega$CPO of (discrete) subprobability distributions on a set $X$.
We write $\mathcal{H}\mathcal{V}_{\leq 1} \,X$ for the $\omega$CPO of nonempty Scott-closed convex subsets
of  $\mathcal{V}_{\leq 1} \,X$  ordered by subset inclusion $\subseteq$. 
We write $\mathcal{S}\mathcal{V}_{\leq 1} \,X$ for the $\omega$CPO of nonempty Scott-compact convex upper-closed subsets
of  $\mathcal{V}_{\leq 1} \,X$  ordered by reverse inclusion $\supseteq$.

DEFINE THE OPERATIONS ON THE ABOVE

USE THE ABOVE TO DEFINE $\Basicleq^\Den_\prang$ and $\Basicleq^\Den_\prdem$

FOLLOWS FROM THE GENERAL MATERIAL ABOVE THAT THESE RELATIONS ADMISSIBLE AND COMPOSITIONAL


