\section{Denotationally-defined preorders}
\label{section:denotational}

Our second approach to defining an admissible and compositional basic
denotational 
preorder $\Basicleq$ on $\Trees(\mathbb{N})$ is to make use of established constructions from domain theory.
Under this approach, admissibility and compositionality of the defined preorder $\Basicleq$ hold
for general reasons. Since this approach essentially amounts to giving a denotational semantics to effect trees, we call it the \emph{denotational} method of defining a basic operational preorder.



In order to define a basic operational preorder using the denotational method, one needs to merely to provide
a continuous $\Sigma$-algebra $D$ (see Section~\ref{section:trees}),  together with a function
% with a distinguished function 
$j\colon \mathbb{N} \to D$. 
Define   $\llbracket \cdot \rrbracket \colon \Trees(\mathbb{N}) \to D$ to be the unique continuous homomorphism that makes the diagram below commute.
   \begin{center}
        \begin{tikzcd}
            \mathbb{N}
            \arrow[r, "j"] 
            \arrow[d, hook, "i"]
            & D \\
            \Trees(\mathbb{N}) \arrow[ru, dashrightarrow, "\llbracket \cdot \rrbracket" below]
        \end{tikzcd}
    \end{center}
\noindent
The map $\llbracket \cdot \rrbracket \colon \Trees(\mathbb{N}) \to D$ is used to induce
the basic operational preorder $\Basicleq_D$ from the partial order relation on the $\omega$CPO $D$.
\[
t \Basicleq_D t' ~~ \Leftrightarrow ~~ \Sem{t} \sqsubseteq \Sem{t' } \enspace .
\]
\begin{proposition}
The relation $\Basicleq_D$ is admissible pregongruence.
\end{proposition}
%
The proof is immediate: admissibility follows from the continuity of 
$\llbracket \cdot \rrbracket$, and compatibility because  $\llbracket \cdot \rrbracket$ is a homomorphism.

In order to obtain substitutivity, hence compositionality, a further property is required.

\begin{definition}[Factorisation property]
    The map $j\colon \mathbb{N} \to D$ is said to have  the \emph{factorisation property} if,
    for every function $f \colon \mathbb{N} \to D$, there exists a 
    continuous homomorphism $h_{\!f} : D \to D$ such that $f = h_{\!f} \circ j$.
    \begin{center}
        \begin{tikzcd}
            \mathbb{N} \arrow[r, "j"] 
                 \arrow[rr, bend right, "f"] &
            D \arrow[r, "h_{\!f}", dashed] & 
            D  
        \end{tikzcd}
    \end{center}
\end{definition}
\begin{proposition}
If $j\colon \mathbb{N} \to D$ has the factorisation property then 
the relation $\Basicleq_D$ is substitutive, hence it is an admissible compositional precongruence.
\end{proposition}

\begin{proof}
Suppose $\sigma: \Nat \to \Tree(\Nat)$ is any  substitution.
Let $\hat{\sigma} : \Tree(\Nat) \to \Tree(\Nat)$ be the continuous homomorphism
such that $\hat{\sigma} \circ i = \sigma$. Consider the map $g := \llbracket \cdot \rrbracket \circ \hat{\sigma} \circ i : \Nat \to D$. By the factorisation property, there exists $h_g : D \to D$ such that
$g = h_g \circ j$. Expanding this, and using the definition of $\Sem{\cdot}$, we have:
\[
 \llbracket \cdot \rrbracket \circ \hat{\sigma} \circ i ~ = ~ h_g \circ j ~ = ~  h_g \circ  \llbracket \cdot \rrbracket \circ i \enspace .
\]
It then follows from  the uniqueness property of Proposition~\ref{proposition:free} that
\begin{equation}
\label{equation:before-pizza}
\llbracket \cdot \rrbracket \circ \hat{\sigma} ~ = ~ h_g \circ  \llbracket \cdot \rrbracket \enspace ,
\end{equation}
because both maps are continuous homomorphisms.

Now, for substitutivity, suppose  that $t \Basicleq_D t'$, i.e., $\Sem{t} \leq \Sem{t'}$. Then 
$h_g (\Sem{t})  \leq h_g(\Sem{t'})$ by monotonicity. That is
$\Sem{ \hat{\sigma}(t)} \leq \Sem{ \hat{\sigma}(t')}$, by~(\ref{equation:before-pizza}). 
This says that $\Sem{ t[\sigma]} \leq \Sem{t'[\sigma]}$. That is
$t[\sigma] \Basicleq_D t'[\sigma]$, as required.
\end{proof}

In practice, it is usually not necessary to prove the factorisation property directly. Instead  it holds as a consequence of the continuous algebra $D$ and map $j: \Nat \to D$ being derived from a suitable monad. The next result establishes general conditions under which this holds.
\begin{proposition}
\label{proposition:monad}
Let $\mathbf{S}$ be a category with a faithful functor $U : \mathbf{S} \to \Set$. Suppose also that 
$\mathbf{S}$ has an object $N$ such that $UN = \mathbb{N}$, and every hom set $\mathbf{S}(N,X)$
is mapped bijectively by $U$ to $\Set(\Nat,UX)$. Suppose also that $(T,\eta,\mu)$ is a monad on $\mathbf{S}$
with the following properties: there is a continuous $\Sigma$-algebra structure on $UTN$; and, for 
every map $f \colon N \to TN$, the induced function $Uf^*$, where 
$f^* \colon TN \to TN$ is the Kleisli lifting, is a continuous  homomorphism.
Then defining $D$ to be the continuous $\Sigma$-algebra on $U T N$, and
$j$ to be $U\eta \colon \mathbb{N} \to UTN$, it follows that $j$ has the factorisation property.
\end{proposition}

%\begin{proof}
%Consider any function $f \colon \mathbb{N} \to UTN$. This is the $U$ image of a unique morphism 
%$g : N \to TN$ in $\mathbf{S}$. Let $g^*: TN \to TN$ be the Kleisli lifting of $g$, which satisfies $g^* \circ \eta = g$. Defining $h_f = Ug^*$, we indeed have  that $h_f$ is a continuous algebra homomorphism, and
%$h_f \circ j = f$.
%\end{proof}


\noindent
We omit the easy proof. 
Although the statement of the proposition is verbose, the result is relatively easy to apply in practice, as the examples we consider next will illustrate.

%
%\begin{lemma}
%Let $\mathcal{A}$ be any category given together with a functor $U' \colon \mathcal{A} \to \ContAlg_\Sigma$.
%Suppose it holds that that the
%composite functor $UU' : \mathcal{A} \to \Set$ has a left adjoint $F$, where $U: \ContAlg_\Sigma \to \Set$ is the forgetful functor.
%Then, defining $D$ to be the continuous $\Sigma$-algebra $U' F \, \mathbb{N}$, the unit of the adjunction defines a function
%$j \colon \mathbb{N} \to D$ which has the factorisation property.
%\end{lemma}
%
%\begin{lemma}
%Let $T$ be a monad on the category $\wCPO$.  Suppose that for every $\omega$CPO $X$, we have a 
%continuous $\Sigma$-algebra structure on TX. Suppose also that all Kleisli liftings of 
%$f^* \colon TX \to TY$, of maps $f \colon X \to TY$, are  continuous  homomorphisms. Then,
%defining $D$ to be the continuous $\Sigma$-algebra $T \mathbb{N}$, the unit of the adjunction defines a function
%$j \colon \mathbb{N} \to D$ which has the factorisation property.
%\end{lemma}

In the remainder of the section, we return to our main example, and again define basic operational preorders for the combination of probabilistic choice and nondeterminism (both angelic and demonic), but this time we use the denotational method. Accordingly, we call the defined preorders
$\Basicleq^\Den_\prang$ and $\Basicleq^\Den_\prdem$

We use the powerdomains combining probabilistic choice and nondeterminism
defined in~\cite[\S3.4]{KeimelP2016}, although our setting is simpler because we only need to apply them to sets.
The basic idea of these constructions is that a computation with probabilistic and nondeterministic choice is modelled as a set of subprobability distributions, where the set collects the possible nondeterministic outcomes, each of which is probabilistic in nature. As is standard, 
the sets relevant to angelic nondeterminism are the closed sets in the Scott topology, whereas those relevant to demonic nondeterminism are the compact upper-closed sets, see~\cite{smyth}.
Due to the combination with probabilistic choice,  sets are further required to be convex; see, for example, the discussion in~\cite{KeimelP2016}.
% suggested by Smyth, the sets relevant to demonic nondeterminism are the compact upper-closed sets, 



Let $\mathcal{V}_{\leq 1} \,X$ be the $\omega$CPO of (discrete) subprobability distributions on a set $X$.
We write $\mathcal{H}\mathcal{V}_{\leq 1} \,X$ for the $\omega$CPO of nonempty Scott-closed convex subsets
of  $\mathcal{V}_{\leq 1} \,X$  ordered by subset inclusion $\subseteq$. 
We write $\mathcal{S}\mathcal{V}_{\leq 1} \,X$ for the $\omega$CPO of nonempty Scott-compact convex upper-closed subsets
of  $\mathcal{V}_{\leq 1} \,X$  ordered by reverse inclusion $\supseteq$.
The $\omega$CPOs $\mathcal{H}\mathcal{V}_{\leq 1} \,X$ and $\mathcal{S}\mathcal{V}_{\leq 1} \,X$ are both continuous algebras for $\Sigma_\prnd$. In both cases, the operations are defined by:
\begin{align*}
\orEff(A,B) ~ = ~ & \Conv(A \cup B)  
% & & \text{(definition for $\mathcal{H}\mathcal{V}_{\leq 1} \,X$)}
% \\ \orEff(A,B) ~ = ~ & & & \text{(definition for $\mathcal{S}\mathcal{V}_{\leq 1} \,X$)}
& 
\prEff(A,B) ~ = ~ & \{\frac{1}{2}a + \frac{1}{2}b \mid a \in A, b \in B\} \enspace ,
% & & \text{(definition for both cases)}
\end{align*}
where $\Conv$ is the convex closure operation. 
We remark that these straightforward uniform definitions are possible because of the simple structure of the 
domains  $\mathcal{H}\mathcal{V}_{\leq 1} \,X$ and $\mathcal{S}\mathcal{V}_{\leq 1} \,X$, over a set $X$. For the more general 
lower and upper `Kegelspitze' considered in \cite{KeimelP2016}, additional order-theoretic and topological closure operations need to be applied.

To apply the above in the angelic case, we use the fact that $\mathcal{H}\mathcal{V}_{\leq 1} \,X$  is the free 
Kegelspitze join semilattice over a set $X$  \cite[Corollary 3.15]{KeimelP2016} (where the result is proved more generally for domains). It follows that $\mathcal{H}\mathcal{V}_{\leq 1} $ is
a monad on  $\Set$ itself  satisfying the conditions
of Proposition~\ref{proposition:monad}. Thus defining 
$D_\prang = \mathcal{H}\mathcal{V}_{\leq 1} \,\mathbb{N}$, and
$j(n) = \downarrow\!\delta(n)$ (where $\delta(n)$ is the Dirac probability distribution that assigns probability 1 to $n$ and $0$ to all other numbers, and $\downarrow\!x$ is the down-closure $\{y \mid y \leq x\}$), the induced $\Sem{\cdot}_\prang : \Tree(\Nat) \to  D_\prang$ defines an admissible and compositional preorder
\[
t \Basicleq^\Den_\prang t' ~~ \Leftrightarrow ~~ \Sem{t}_\prang  \leq \Sem{t'}_\prang \enspace .
\]

Similarly, in the demonic case, we use \cite[Corollary 3.16]{KeimelP2016}, which characterises $\mathcal{S}\mathcal{V}_{\leq 1} \,X$ as the free Kegelspitze meet semilattice over $X$. Again $\mathcal{S}\mathcal{V}_{\leq 1}$ is a monad
on  $\Set$ to which Proposition~\ref{proposition:monad} applies. In this case, we define 
$D_\prdem = \mathcal{S}\mathcal{V}_{\leq 1} \,\mathbb{N}$, and
$j(n) = \{\delta(n)\}$. Then the induced $\Sem{\cdot}_\prdem : \Tree(\Nat) \to  D_\prdem$ defines an admissible and compositional preorder
\[
t \Basicleq^\Den_\prdem t' ~~ \Leftrightarrow ~~ \Sem{t}_\prdem  \leq \Sem{t'}_\prdem \enspace .
\]



