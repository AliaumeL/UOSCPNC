\section{A functional language with algebraic effects}

We use a  simply-typed  functional language, with a base type of natural numbers, general recursion, and a call-by-value evaluation strategy. In addition, 
the language is 
parameterised by a collection $\Sigma$ of \emph{effect} operations, resulting in a 
language similar to that considered in.~\cite{plotkin2001adequacy}.
Fine-grained call-by-value [[REFERENCES]]  is used, instead of regular call-by-value,
because it simplifies the technical development by devolving all sequencing of effects to a single
language construct. 
This is a purely stylistic choice, as there are easy translations between fine-grained and regular call-by-value
[[REFERENCES]].

The syntax of types and terms is given in in Figure \ref{fig:language}. 
Note that there is a syntactic distinction between \emph{values} $V$ and \emph{computations} $M$.
The latter depend on the 
parameter set $\Sigma$, which  is 
an arbitrary set of operation symbols each with an associated finite arity.



\begin{figure}[h!]
    \begin{align*}
        \tau :=& ~\Nat ~|~ \tau \to \tau \\
        V :=& ~x ~|~ \lambda x:\tau. M ~|~ \Zero ~|~ \Succ V \\
        M :=& ~\Ret V ~|~ V V ~|~ \Fix V \\
                    &|~ \lcase{V}{M}{M} \\
                    &|~ \Bind{M}{x:\tau}{M} \\
                    &|~ \sigma(\underbrace{M, \dots, M}_n ) \quad \text{ where } (\sigma,n)  \in \Sigma
    \end{align*}
    \caption{Refined Call-By-Value PCF with effects}
    \label{fig:language}
\end{figure}
%
%\begin{figure}[h!]
%    \begin{equation*}
%        \begin{array}{rl}
%            (\lambda x:\tau. N)M &\leftarrow \Bind{M}{x:\tau}{N} \\
%            M N &\rightarrow \Bind{M}{f:\sigma \to \tau}{\Bind{N}{x:\sigma}{ f x }} 
%        \end{array}
%    \end{equation*}
%    \caption{Translation between refined and regular call-by-value}
%    \label{fig:refinedNormal}
%\end{figure}
%

\begin{figure}[h]
    \begin{center}
        % Identity on variables 
            \AxiomC{}
            \UnaryInfC{$\Gamma, x:\tau \vdash_V x : \tau$}
            \DisplayProof 
        \hskip 1.5em
        % Return a value 
            \AxiomC{$\Gamma \vdash_V V : \tau$}
            \UnaryInfC{$\Gamma \vdash_C \Ret V : \tau$}
            \DisplayProof 
        \hskip 1.5em
        % Lambda abstraction 
            \AxiomC{$\Gamma, x : \tau \vdash_C M : \tau'$}
            \UnaryInfC{$\Gamma \vdash_V \lambda x:\tau. M : \tau \to \tau'$}
            \DisplayProof 
        \hskip 1.5em
        % Zero 
            \AxiomC{}
            \UnaryInfC{$\Gamma \vdash_V \Zero : \Nat$}
            \DisplayProof
        \hskip 1.5em
        \vskip 1em
        % Succ 
            \AxiomC{$\Gamma \vdash_V V : \Nat$}
            \UnaryInfC{$\Gamma \vdash_V \Succ V : \Nat$}
            \DisplayProof
        \hskip 1.5em
        % Fixed point 
            \AxiomC{$\Gamma \vdash_V V : (\tau \to \tau') \to \tau \to \tau'$}
            \UnaryInfC{$\Gamma \vdash_C \Fix V : \tau \to \tau'$}
            \DisplayProof
        \hskip 1.5em
        % Application  
            \AxiomC{$\Gamma \vdash_V V : \tau \to \tau'$} 
            \AxiomC{$\Gamma \vdash_V W : \tau$}
            \BinaryInfC{$\Gamma \vdash_C V W : \tau'$}
            \DisplayProof
        \hskip 1.5em
        \vskip 1em
        % Case   
            \AxiomC{$\Gamma \vdash_V V : \Nat$} 
            \AxiomC{$\Gamma \vdash_C M_1 : \tau$}
            \AxiomC{$\Gamma, x:\Nat \vdash_C M_2 : \tau$}
            \TrinaryInfC{$\Gamma \vdash_C \lcase{V}{M_1}{M_2} : \tau$}
            \DisplayProof
        \hskip 1.5em
        % Bind 
            \AxiomC{$\Gamma \vdash_C M : \tau$} 
            \AxiomC{$\Gamma, x:\tau \vdash_C N : \tau'$}
            \BinaryInfC{$\Gamma \vdash_C \Bind{M}{x:\tau}{N} : \tau'$}
            \DisplayProof
        \hskip 1.5em
        \vskip 1em
        % Effect 
            \AxiomC{$(\sigma,n) \in \Sigma$}
            \AxiomC{$\forall 1 \leq i \leq n, \quad \Gamma \vdash_C M_i : \tau$}
            \BinaryInfC{$\Gamma \vdash_C \sigma(M_1, \dots, M_n) : \tau$}
            \DisplayProof
    \end{center}
    \caption{Inference rules for typing}
    \label{fig:inference:typing}
\end{figure}


The type inference rules are given  in Figure \ref{fig:inference:typing}.
It is easily shown that the set of values of type $\Nat$ is isomorphic to $\mathbb{N}$,
with $\Zero$ as $0$ and $\Succ$ as successor.
We use $\underline{n}$ to denote 
the closed value corresponding to $n$.
