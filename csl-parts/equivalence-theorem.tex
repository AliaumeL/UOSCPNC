\section{The equivalence theorem}

Our main theorem is that our operational, denotational and axiomatic preorders for combined probability and nondeterminism all coincide, in the case of both angelic and demonic nondeterminism.
\begin{theorem}[Equivalence theorem] \leavevmode
\begin{enumerate} 
\item The preorders $\Basicleq^\Op_\prang$, $\Basicleq^\Den_\prang$ and $\Basicleq^\Ax_\prang$, for mixed probability and angelic nondeterminism, coincide.

\item Similarly, the preorders $\Basicleq^\Op_\prdem$, $\Basicleq^\Den_\prdem$ and $\Basicleq^\Ax_\prdem$,
for mixed probability and demonic nondeterminism, coincide.
\end{enumerate}
\end{theorem}

\begin{proof}
    We give an outline of the proof in the case of demonic non-determinism.

    \begin{itemize}
        \item 
    It is easily checked that $\Basicleq^\Op_\prdem$ satisfies the Horn-clause
    theory of $\Basicleq^\Ax_\prdem$, and we know that $\Basicleq^\Op_\prdem$ is 
    both admissible and compositional. This gives us the inclusion 
    ${\Basicleq^\Ax_\prdem} \subseteq {\Basicleq^\Op_\prdem}\,$.

        \item 
    Now, we want to prove that $\Basicleq^\Op_\prdem \subseteq
    \Basicleq^\Den_\prang$. It is easier to notice that both preorders coincide.
    There is an isomorphism $\Lambda$ between $\mathcal{V}_{\leq 1} \,\mathbb{N}_\bot$
    and $\mathcal{L}_{sup}^{sne} (\mathcal{L} (\mathbb{N}_\bot),
    \overline{\mathbb{R}_+})$ \cite{KeimelP2016}. Given a tree $t$, the function  
    $F_t : h \mapsto \inf_s \mathbf{E}_{t \# s} (h)$ is superlinear and 
    strongly non-expansive. It can then be shown by induction on $t$ that 
    $\Lambda (\llbracket t \rrbracket) (h) = F_t (h)$.
    \todo[inline]{The function is strongly non-expansive if and only if 
        $F(h + 1) \leq F(h) + 1$ which is clear here \cite{KeimelP2016} page 58}
    \todo[inline]{$\Lambda (X)(h) = \inf_{\mu \in X} \int h d\mu$, allowing
        direct translation for the base case of the induction}

    We do conclude that $\Basicleq^\Op_\prang = \Basicleq^\Den_\prang$.

    \item 
        For the inclusion $\Basicleq^\Den_\prdem \subseteq
        \Basicleq^\Ax_\prdem$, the proof sketch is the following one:

    \begin{enumerate}
        \item Prove that both preorders coincide 
            on trees without $\orEff$ nodes
        \item Prove that both preorders coincide 
            on trees with a \emph{finite} number 
            of $\orEff$ nodes.
        \item Use finite approximations and admissibility
            of both preorders to conclude in the general case.
    \end{enumerate}

    \begin{description}
        \item[Trees with no disjunction]

    \todo[inline]{Remove this part of the proof, it is only here to show that it
    works for probabilities only ... Which is going to be assumed. However
    it is clearly the same proof in some sense when extending from finite 
    trees to infinite trees. It could be a more general lemma}

    For such trees, $\inf_s \mathbf{E}_{ t \# s } (h)
    = \mathbf{E}_{ t } (h)$.
    If both trees are finite, they can be put into 
    a normal form using the axioms concerning the 
    probability constructor: two complete binary trees 
    of same height,
    with leafs ordered by increasing number, and 
    such that a given leaf is $\bot$ the same 
    in both trees.

    It is easy to prove that two trees with 
    only $\prEff$ nodes are ordered for $\Basicleq^\Op_\prdem$
    if and only if their normal forms are ordered,
    and it only takes a simple induction to 
    deduce that they are ordered for $\Basicleq^\Ax_\prdem$.

    Now, if $t$ and $t'$ are infinite, 
    we approximate them with finite trees $(t_i)$ and 
    $(t_i')$.
    
    Given a tree $T$ we build $T^n$ as $\bot$ if $n = 0$, and $T \prEff T^{n-1}$
    otherwise. 

    It is clear that $T^n \Basicleq^\Op_\prdem T$ in a \emph{strict fashion}:
    the inequality is strict for all test function $h$.

    Therefore $t_i \prEff (t_i')^n \Basicleq^\Op_\prdem t_i$
    with a \emph{strict inequality for all tests functions h}.
    
    Let us fix $i \in \mathbb{N}$, and a $n \in \mathbb{N}$: 

    \begin{equation*}
        t_i \prEff (t_i')^n \Basicleq^\Op_\prdem t_i \prEff t_i'
        \Basicleq^\Op_\prdem t \prEff t' \Basicleq^\Op_\prdem t'
    \end{equation*}

    For any given test function $h$ we therefore have a strict 
    inequality between $t_i \prEff (t_i')^n$ and $t'$, however 
    $t'$ is the supremum of the family $(t_i')$ and the function 
    $ t \mapsto (h \mapsto \mathbf{E}_{ t } (h))$ is scott-continuous.

    Therefore, there exists \emph{a finite $j$} such that 
    $t_i \prEff (t_i')^n \Basicleq^\Op_\prdem t_j'$.
    All of the above inequalities are between finite trees, and 
    therefore are also true using $\Basicleq^\Ax_\prdem$.
    Using admissibility with a supremum over $n$ 
    one deduces $t_i \prEff t_i' \Basicleq^\Ax_\prdem t_j'$
    and using admissibility with a supremum over $i$ we can finally conclude
    $t \prEff t' \Basicleq^\Ax_\prdem t'$.
    
      

        \item[Trees with finite disjunction]

    Let $t \Basicleq^\Op_\prdem t'$ 
    where $t$ and $t'$ have only a finite number 
    of disjunction nodes.
    The proof is done using the following 
    remarks
    \begin{enumerate}
        \item We can first use the axioms of the theory 
            to put both trees in the form of 
            a \emph{finite disjunction} over 
            \emph{possibly infinite probability trees}.

        \item 
            If for all $t_i'$ 
            a probability
            tree of $t'$ there exists 
            a corresponding tree $t_i$ in $t$ 
            such that $t_i \Basicleq^\Den_\prdem t_i'$
            then it is clear that $t \Basicleq^\Ax_\prdem t'$.

        \item 
            If $t_i$ and $t_j$ are two 
            probability trees of $t$ 
            then $t$ is equivalent to 
            $t \orEff (t_i \prEff t_j)$ 
            in both preorders.
            This can be extended to any infinite 
            $\prEff$ combination 
            of probability trees of $t$.

        \item 
            \todo[inline]{Precise how it is done.
                It directly derives from the definition 
                as a compact-convex-upward-closed subset,
                which will contain all such combination 
                of probability distributions of the $t_i'$}
            If $t_i'$ is a probability tree of $t'$,
            then there exists an infinite $\prEff$
            combination of probability trees of $t$
            noted $\hat{t_i'}$
            such that $\hat{t_i'} \Basicleq^\Den_\prdem t_i'$.


        \item Because there is only a finite number 
            of probability trees in $t'$, we can 
            conclude by induction on their number that 
            $t \Basicleq^\Ax_\prdem t'$.
    \end{enumerate}


        \item[Extension]
            \todo[inline]{Use the definition of the 
                way below relation in the semantics 
                instead of "showing" how to do it}
    Let $t \Basicleq^\Den_\prdem t'$,
    such that $t = \sqcup_i t_i$ and $t' = \sqcup_i t_i'$ 
    where both families are composed of \emph{finite} trees.

    Given a tree $T$ we build $T^n$ as $\bot$ if $n = 0$, and $T \prEff T^{n-1}$
    otherwise. 

    It is clear \textbf{(Maybe explain this)} that $\llbracket T^n \rrbracket <\!\!< \llbracket T \rrbracket$ 
    in the domain $\mathcal{S}\mathcal{V}_{\leq 1} \mathbb{N}_\bot$.

    \todo[inline]{Use the following lemma: way-below is preserved by sum}
    Therefore at any given $i,n \in \mathbb{N}^2$, 
    $\llbracket t_i \prEff (t_i')^n \rrbracket <\!\!< \llbracket t' \rrbracket$

    We can conclude that for all $(i,n) \in \mathbb{N}$ there exists a 
    $j \geq i$ such that $\llbracket t_i \prEff (t_i')^n \rrbracket \leq
    \llbracket t_j' \rrbracket$ (way below relation).

    Now, this translates into inequalities between \emph{finite trees}
    and therefore between trees with finite disjunction, and thus 
    $ t_i \prEff (t_i')^n \Basicleq^\Ax_\prdem t_j'$.

    Using admissibility with a supremum over $n$ 
    one deduces $t_i \prEff t_i' \Basicleq^\Ax_\prdem t_j'$
    and using admissibility with a supremum over $i$ we can finally conclude
    $t \prEff t' \Basicleq^\Ax_\prdem t'$.
    \end{description}
    \end{itemize}
    

\end{proof}

