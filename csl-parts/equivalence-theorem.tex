\section{The equivalence theorem}
\label{section:equivalence}

Our main theorem is that our operational, denotational and axiomatic preorders for combined probability and nondeterminism all coincide, in both the angelic and demonic cases.
\begin{theorem}[Equivalence theorem] \leavevmode
\begin{enumerate} 
\item The three preorders $\Basicleq^\Op_\prang$, $\Basicleq^\Den_\prang$ and $\Basicleq^\Ax_\prang$, for mixed probability and angelic nondeterminism, coincide.

\item Similarly, the preorders $\Basicleq^\Op_\prdem$, $\Basicleq^\Den_\prdem$ and $\Basicleq^\Ax_\prdem$,
for mixed probability and demonic nondeterminism, coincide.
\end{enumerate}
\end{theorem}

\noindent
We outline the proof of the theorem in the demonic case, which we split into three lemmas. The proof for the angelic case is similar. 
\begin{lemma}
${\Basicleq^\Ax_\prdem} \, \subseteq \, \, \Basicleq^\Op_\prdem\,$.
\end{lemma}
\begin{proof}
 It is easily checked that $\Basicleq^\Op_\prdem$ satsfies the axioms
 of $T_\prdem$.  Since $\Basicleq^\Op_\prdem$ is admissible and 
$\Basicleq^\Ax$ is the smallest admissible model, ${\Basicleq^\Ax_\prdem} \, \subseteq \, \Basicleq^\Op_\prdem\,$.
\end{proof}
%
\noindent
We remark on the following aspect of the above result.
The distributivity axiom Dist of Figure~\ref{fig:axiomsmixed} is sometimes discussed as expressing that  nondeterministic choices  are  resolved before probabilistic ones; see, e.g., \cite{KeimelP2016}. Such statements need careful interpretation. 
The definition of $\Basicleq^\Op_\prdem$, which is based on implementing nondeterministic schedulers as strategies for MDPs,
explicitly allows the  scheduler's choices to take account of the outcomes of probabilistic choices that precede it. Nevertheless, the distributivity axiom is sound.

\begin{lemma}
${\Basicleq^\Op_\prdem} \, = \,\, \Basicleq^\Den_\prdem\,$.
\end{lemma}
\begin{proof}
We make use of the functional representation of $\mathcal{S}\mathcal{V}_{\leq 1}\, \mathbb{N}$ from \cite{KeimelP2016} (see also~\cite{JGL-mscs16}). 
For any topological space $X$, we write $\mathcal{L}(X)$ for the space of all
\emph{lower semicontinuous} functions from $X$ to $[0,\infty]$ (i.e., functions that are continuous with respect to the Scott topology on $[0,\infty]$), and we endow $\mathcal{L}(X)$ itself with the Scott topology. The space
$D' = \mathcal{L}(\mathcal{L}(\mathbb{N}))$ carries a continuous $\Sigma_\prnd$-algebra structure
\[
(F \orEff G)(f) ~ = ~ \min(F(f), G(f)) \qquad 
(F \prEff G)(f) ~ = ~ \frac{1}{2}F(f) + \frac{1}{2}G(f) \enspace .
\]
(There is another $\Sigma_\prnd$-algebra structure, relevant to angelic nondeterminism, in which  $\min$ is replaced with $\max$.) Define $j' : \mathbb{N} \to D'$ by $j'(n)(f) = f(n)$. This induces $\Sem{\cdot}'_\prdem : \Tree(\mathbb{N}) \to D'$
satisfying  $\Sem{\cdot}'_\prdem \circ i = j'$, as in Section~\ref{section:denotational}.
We show that $\Sem{t}'_\prdem (h) = F_h (t)$, where $F_h$ is defined as in
Lemma~\ref{lemma:F-continuous}. For this, the function $t \mapsto (h \mapsto F_h (t)$ is easily shown to be a $\Sigma_\prnd$-algebra homomorphism
satisfying $F_h(i(n)) = j'(n)$. Moreover, it is continuous by 
Lemma~\ref{lemma:F-continuous}. Thus it indeed coincides with 
$\Sem{\cdot}'_\prdem$. By the definition of $F_h$, if follows that that
$t \Basicleq^\Op_\prdem t'$ if and only if 
$\Sem{t}'_\prdem \leq \Sem{t'}'_\prdem \,$.

Theorem~[[WHAT]] of \cite{KeimelP2016} provides a functional representation of 
$\mathcal{S}\mathcal{V}_{\leq 1}\, X$ inside $\mathcal{L}(\mathcal{L}(X))$. In the case $X = \mathbb{N}$, consider the function
\[
\Lambda : A \mapsto \left(f \mapsto \inf_{p \in A} 
  \mathbf{E}_p\, f 
% \left( \sum_{n \in \mathbb{N}} f(n)\, . \, p(n) \right)
\right) 
 : \; \mathcal{S}\mathcal{V}_{\leq 1}\, \mathbb{N} \;  \to \; D' \enspace .
\]
It is shown in  \cite{KeimelP2016} that $\Lambda$  is a continuous $\Sigma_\prnd$-algebra homomorphism, and also an order embedding (i.e., $\Lambda(A) \leq \Lambda(B)$ implies $A \supseteq B$). 
%(Furthermore, the image of $\Lambda$ can be characterised as exactly the \emph{superlinear strongly non-expansive} functionals.) 
By the uniqueness property of Proposition~\ref{proposition:free}, it thus holds that $\Lambda \circ \Sem{\cdot}_\prdem = \Sem{\cdot}'_\prdem$. We therefore have
\[
t \Basicleq^\Op_\prdem t' ~~ \Leftrightarrow ~ ~
  \Sem{t}'_\prdem \leq \Sem{t'}'_\prdem 
   ~~ \Leftrightarrow ~ ~
   \Sem{t}_\prdem \leq \Sem{t'}_\prdem 
      ~ ~\Leftrightarrow ~ ~
  t \Basicleq^\Den_\prdem t' \enspace ,
   \]
where the middle equivalence holds because $\Lambda$ is an order embedding.
\end{proof}

\begin{lemma}
${\Basicleq^\Den_\prdem} \, \subseteq \, \, \Basicleq^\Ax_\prdem\,$.
\end{lemma}
\begin{proof}
The proof proceeds in three steps.
    \begin{enumerate}
        \item Prove that both preorders coincide 
            on \emph{probability trees} (i.e., trees without $\orEff$ nodes).
        \item Prove that both preorders coincide 
            on trees with a \emph{finite} number 
            of $\orEff$ nodes.
        \item Use finite approximations and admissibility
            to conclude the general case.
    \end{enumerate}

We omit discussion of the first step, which  is comparatively straightforward.

For step 2, suppose $t \Basicleq^\Den_\prdem t'$ where $t,t'$ are 
trees with finitely many $\orEff$ nodes. We proceed as follows.
    \begin{enumerate}[(a)]
        \item For each of $t, t'$, we use the distributivity axiom to rewrite the tree
          as an $\orEff$-combination of finitely many (possibly infinite) {probability trees}.

        \item 
            If for every 
            probability
            tree $t_i'$ in $t'$ there exists 
            a corresponding tree $t_i$ in $t$ 
            such that $t_i \Basicleq^\Den_\prdem t_i'$,
            then we have that $t \Basicleq^\Ax_\prdem t'$, using the $\texttt{Dem}$ axiom, and  step 1 above.

        \item 
            If $t_i$ and $t_j$ are two 
            probability trees of $t$ 
            then $t$ is equivalent to 
            $t \orEff (t_i \prEff t_j)$ 
            in both preorders.
            This can be extended to any infinite 
            $\prEff$ combination 
            of probability trees of $t$.

        \item 
            \todo[inline]{Precise how it is done.
                It directly derives from the definition 
                as a compact-convex-upward-closed subset,
                which will contain all such combination 
                of probability distributions of the $t_i'$}
            If $t_i'$ is a probability tree of $t'$,
            then there exists an infinite $\prEff$
            combination of probability trees of $t$
            noted $\hat{t_i'}$
            such that $\hat{t_i'} \Basicleq^\Den_\prdem t_i'$.


        \item Because there is only a finite number 
            of probability trees in $t'$, we can 
            conclude by induction on their number that 
            $t \Basicleq^\Ax_\prdem t'$.
    \end{enumerate}

  
\begin{description}
        \item[Trees with finite disjunction]

    Let $t \Basicleq^\Op_\prdem t'$ 
    where $t$ and $t'$ have only a finite number 
    of disjunction nodes.
    The proof is done using the following 
    remarks


        \item[Extension]
            \todo[inline]{Use the definition of the 
                way below relation in the semantics 
                instead of "showing" how to do it}
    Let $t \Basicleq^\Den_\prdem t'$,
    such that $t = \sqcup_i t_i$ and $t' = \sqcup_i t_i'$ 
    where both families are composed of \emph{finite} trees.

    It is clear that for every tree $t$
    \textbf{(Maybe explain this)} $\llbracket t^n \rrbracket <\!\!< \llbracket t \rrbracket$ 
    in the domain $\mathcal{S}\mathcal{V}_{\leq 1} \mathbb{N}_\bot$, 
    where $t^n$ is obtain using definition~\ref{def:probaApproxConstruct}.

    \todo[inline]{Use the following lemma: way-below is preserved by sum}
    Therefore at any given $i,n \in \mathbb{N}^2$, 
    $\llbracket t_i \prEff (t_i')^n \rrbracket <\!\!< \llbracket t' \rrbracket$

    We can conclude that for all $(i,n) \in \mathbb{N}$ there exists a 
    $j \geq i$ such that $\llbracket t_i \prEff (t_i')^n \rrbracket \leq
    \llbracket t_j' \rrbracket$ (way below relation).

    Now, this translates into inequalities between \emph{finite trees}
    and therefore between trees with finite disjunction, and thus 
    $ t_i \prEff (t_i')^n \Basicleq^\Ax_\prdem t_j'$.

    Using admissibility with a supremum over $n$ 
    one deduces $t_i \prEff t_i' \Basicleq^\Ax_\prdem t_j'$
    and using admissibility with a supremum over $i$ we can finally conclude
    $t \prEff t' \Basicleq^\Ax_\prdem t'$.
    \end{description}
 \end{proof}

