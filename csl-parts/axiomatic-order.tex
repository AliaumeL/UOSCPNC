\section{Axiomatically-defined preorders}
\label{section:axiomatic}

In this section, we look at the definition of basic operational preorders by axiomatising
properties of the operations in the effect signature $\Sigma$.
Since we are defining a preorder, it is appropriate for the axiomatisation to involve inequalities
specifying desired properties of the operational preorder. As the technical framework for this, we allow
axiomatisations involving infinitary Horn-clauses of inequalities between infinitary terms.  This provides a flexible general setting for
axiomatising admissible and compositional preorders on 
$\Trees(\mathbb{N})$. 

Let $\Vars$ be a set of countably many distinct variables. By an  \emph{expression}, we mean a
tree $e \in \Trees(\Vars)$. The use of trees incorporates infinitary non-well-founded terms alongside the usual finite
algebraic terms. By an \emph{inequality} we mean a statement $e_1 \leq e_2$, where $e_1, e_2$ are expressions.
By an \emph{(infinitary) Horn clause} we mean an implication of the form:
\begin{equation}
\label{equation:horn-clause}
\left( \bigwedge_{i \in I} e_i \leq e'_i \right)~ \implies ~ e \leq e' \enspace ,
\end{equation}
An \emph{effect theory} $T$  is a set of Horn clauses.

A precongruence $\Basicleq$ on $\Trees(X)$ is said to \emph{satisfy} a Horn clause (\ref{equation:horn-clause}) if,
for every environment $\rho \colon \Vars \to \Trees(X)$, the implication below holds (recall the notation for tree substitution from Section~\ref{section:trees}).
\[
\left( \bigwedge_{i \in I} e_i[\rho] \Basicleq e'_i[\rho] \right) ~ \implies ~  e[\rho] \Basicleq e'[\rho] \enspace .
\]
We say that a precongruence $\Basicleq$ is a \emph{model} of a Horn clause theory $T$ if it satisfies every Horn clause in $T$.
We consider models as subsets of $\Trees(X) \times \Trees(X)$, partially ordered by inclusion. Note that models are precongruences by assumption.
\begin{proposition}
\label{proposition:axiomatic}
Every Horn clause theory $T$ has a smallest admissible model  
\[{\Basicleq_T} ~\subseteq~ \Trees(X) \times \Trees(X) \enspace ,\]
for any set $X$. The model ${\Basicleq_T}$
is substitutive. In the case that $X = \mathbb{N}$, the smallest admissible model is thus an
admissible compositional preorder.
\end{proposition}
\begin{proof}
%The total relation $X \times X$ gives one admissible model. 
It is easily seen that the intersection of any set of admissible models is itself an admissible model.
Thus the
intersection of the set of all admissible models is the required smallest admissible model $\Basicleq_T\,$. For substitutivity, 
define 
\begin{equation}
\label{equation:substitutive-T}
t \Basicleq_S t' ~ \Leftrightarrow ~ \forall \sigma : X \to \Trees(X). ~ t[\sigma] \Basicleq_T t'[\sigma] \enspace .
\end{equation}
Using the substitution $\sigma(x) = x$, we see that ${\Basicleq_S}\, \subseteq\, {\Basicleq_T}$.
Conversely, it is easily shown that the relation $\Basicleq_S$ is itself an admissible model of $T$. 
Thus ${\Basicleq_T}\, \subseteq\, {\Basicleq_S}$. Since ${\Basicleq_T}$ and ${\Basicleq_S}$ coincide,
(\ref{equation:substitutive-T}) asserts the substitutivity of ${\Basicleq_T}$.
The statement about compositionality now follows from Proposition~\ref{proposition:substitutive}.
\end{proof}


Given the proposition, we can use any effect theory to define an admissible and compositional basic operational preorder, namely the smallest admissible model over $\mathbb{N}$. We now apply this method to our running example of combined nondeterminism and probabilistic choice. The  axioms are given in
Figure~\ref{fig:axiomsmixed}.
\begin{figure}[t]
\[
\begin{array}{ll}
\text{Bot:} & \bot \leq x
\\
\text{Prob:} & x \prEff x = x , ~ x \prEff y = y \prEff x , ~ 
                                (x \prEff y) \prEff\, (z \prEff w) = (x \prEff z) \prEff\, (y \prEff w)
\\
\text{Appr:} & x \prEff y \leq y \implies  x \leq y
 \\
\text{Nondet:} & x \orEff x = x, ~ x \orEff y = y \orEff x, ~ ~ x \orEff\, (y \orEff z) = (x \orEff y) \orEff z
\\
\text{Ang:} & x \orEff y \leq x \\ 
\text{Dem:} & x \orEff y \geq x \\ 
\text{Dist:} & x \prEff\, (y \orEff z) = (x \prEff y) \orEff\, (x \prEff z)
\end{array}
\]
    \caption{Horn theory for mixed probability and non determinism}
    \label{fig:axiomsmixed}
\end{figure}

The axioms include a special axiom for $\bot$, which is legitimate since $\bot$ is a tree, hence an expression.
The axioms  for probability include three standard equalities (each of which is given officially as two inequalities), and one Horn approximation axiom, $\text{Appr}$, which  is separated out for the sake of Proposition~\ref{proposition:horn} below.
The axioms for nondeterminism are split into a neutral list,  followed by further axioms  for angelic and demonic nondeterminism respectively. Finally, there is a distributivity axiom that relates
probabilstic and nondeterministic choice. 
Our two effect theories of interest are: 
\[
T_\prang \!=\!  \text{Bot},  \text{Prob},  \text{Appr}, \text{Nondet}, \text{Ang},\text{Dist} \quad
T_\prdem\! = \!\text{Bot},  \text{Prob},  \text{Appr},  \text{Nondet},  \text{Dem},\text{Dist}\,.
\]
We then define  $\Basicleq^\Ax_\prang$ as the smallest admissible model
of $T_\prang$ over $\mathbb{N}$, and 
$\Basicleq^\Ax_\prdem$ as the smallest admissible model
of $T_\prdem$. By Proposition~\ref{proposition:axiomatic}, both these basic operational preorders are admissible and compositional.

To end the section, we observe that the Horn-clause axiom in Figure~\ref{fig:axiomsmixed} can be replaced with an equational axiom, albeit one involving an infinitary expression.



\begin{definition}
    \label{def:probaApproxConstruct}
    Let $t$ be a  tree. For each $n \in \mathbb{N} \cup \{\infty\}$, we define a tree $\Iter{t}{n}$ inductively by
    $\Iter{t}{0} = \bot$ and $\Iter{t}{(n+1)}  = t\,  \prEff \,\Iter{t}{n}$. The
    tree $\Iter{t}{\infty}$ is defined as $\bigsqcup_n \Iter{t}{n}$.
\end{definition}

\begin{proposition}%[Removing Horn-Clauses]
\label{proposition:horn}
    For any effect theory containing the Bot and Prob axioms, an admissible model satisfies the $\text{Appr}$
     axiom if and only if it satisfies the equation 
    $\Iterx{\infty}= x$.
\end{proposition}

\begin{proof} We first derive $\Iterx{\infty}= x$, from the axioms with $\text{Appr}$.
    It is clear that $\Iterx{n} \leq  x$ for every $n < \infty$,
    and therefore $\Iterx{\infty} \leq  x$ by admissibility.
    We also have $x \prEff \Iterx{n} \leq  \Iterx{(n+1)}$,
    and so, again by admissibility, $x \prEff \Iterx{\infty} \leq \Iterx{\infty}$.
    Whence, by the Horn axiom, $x  \leq \Iterx{\infty}$. 
     We have thus derived $\Iterx{\infty}= x$.

    For the converse, we assume $\Iterx{\infty}= x$ and derive  $\text{Appr}$.
    Suppose $x \prEff y \leq y$.  Then also  $x \prEff \, (x \prEff y) \leq y$, and $x \prEff \, ( x \prEff \, (x \prEff y)) \leq y$, etc. So also $x \prEff \bot \leq y$, and $x \prEff \, (x \prEff \bot) \leq y$, and $x \prEff \, ( x \prEff \, (x \prEff \bot)) \leq y$,
 etc. That is, $\Iterx{n} \leq y$, for every $n < \infty$. By admissibility, $\Iterx{\infty} \leq y$. Whence, by the assumed axiom, $x \leq y$ as required.
 \end{proof}
