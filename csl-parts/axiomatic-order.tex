\section{Axiomatically-defined preorders}

In this section, we consider a third way of defining a basic operational preorder, by axiomatising
properties of the operations in the effect signature $\Sigma$.
Since we are defining a preorder, it is appropriate for the axiomatisation to involve inequalities
specifying desired properties of the operational preorder. As the technical framework for this, we allow
Horn clause axiomatisations of inequalities between infinitary terms.  This provides a flexible general setting for
axiomatising admissible and compositional preorders on 
$\Trees(\mathbb{N})$. 

Let $\Vars$ be a set of countably many distinct variables. By an  \emph{expression}, we mean a
tree $e \in \Trees(\Vars)$. The use of trees incorporates infinitary non-well-founded terms alongside the usual finite
algebraic terms. By an \emph{inequality} we mean a statement $e_1 \leq e_2$, where $e_1, e_2$ are expressions.
By an \emph{(infinitary) Horn clause} we mean an implication between inequalities:
\begin{equation}
\label{equation:horn-clause}
\bigwedge_{i \in I} e_i \leq e'_i ~ \rightarrow ~ e \leq e' \enspace ,
\end{equation}
An \emph{effect theory} $T$  is a set of Horn clauses.

A precongruence $\Basicleq$ on $\Trees(X)$ is said to \emph{satisfy} a Horn clause (\ref{equation:horn-clause}) if,
for every environment $\rho \colon \Vars \to \Trees(X)$, the implication below holds (recall the notation for tree substitution from Section~\ref{section:trees}).
\[
\bigwedge_{i \in I} e_i[\rho] \Basicleq e'_i[\rho]~~~ \Rightarrow ~~~ e[\rho] \Basicleq e'[\rho] 
\]
We say that a precongruence $\Basicleq$ is a \emph{model} of a Horn clause theory $T$ if it satisfies every Horn clause in $T$.
We consider models as subsets of $\Trees(X) \times \Trees(X)$, partially ordered by inclusion. Note that models are precongruences by assumption.
\begin{proposition}
Every Horn clause theory $T$ defines a smallest admissible model  ${\Basicleq_T} \subseteq \Trees(X) \times \Trees(X)$. The smallest model
is substitutive. In the case that $X = \mathbb{N}$, the smallest admissible model is thus an
admissible compositional preorder.
\end{proposition}

