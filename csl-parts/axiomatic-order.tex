\section{Axiomatically-defined preorders}

In this section, we consider a third way of defining a basic operational preorder, by axiomatising
properties of the operations in the effect signature $\Sigma$.
Since we are defining a preorder, it is appropriate for the axiomatisation to involve inequalities
specifying desired properties of the operational preorder. As the technical framework for this, we allow
Horn clause axiomatisations of inequalities between infinitary terms.  This provides a flexible general setting for
axiomatising admissible and compositional preorders on 
$\Trees(\mathbb{N})$. 

Let $\Vars$ be a set of countably many distinct variables. By an  \emph{expression}, we mean a
tree $e \in \Trees(\Vars)$. The use of trees incorporates infinitary non-well-founded terms alongside the usual finite
algebraic terms. By an \emph{inequality} we mean a statement $e_1 \leq e_2$, where $e_1, e_2$ are expressions.
By an \emph{(infinitary) Horn clause} we mean an implication of the form:
\begin{equation}
\label{equation:horn-clause}
\left( \bigwedge_{i \in I} e_i \leq e'_i \right)~ \implies ~ e \leq e' \enspace ,
\end{equation}
An \emph{effect theory} $T$  is a set of Horn clauses.

A precongruence $\Basicleq$ on $\Trees(X)$ is said to \emph{satisfy} a Horn clause (\ref{equation:horn-clause}) if,
for every environment $\rho \colon \Vars \to \Trees(X)$, the implication below holds (recall the notation for tree substitution from Section~\ref{section:trees}).
\[
\left( \bigwedge_{i \in I} e_i[\rho] \Basicleq e'_i[\rho] \right) ~ \implies ~  e[\rho] \Basicleq e'[\rho] 
\]
We say that a precongruence $\Basicleq$ is a \emph{model} of a Horn clause theory $T$ if it satisfies every Horn clause in $T$.
We consider models as subsets of $\Trees(X) \times \Trees(X)$, partially ordered by inclusion. Note that models are precongruences by assumption.
\begin{proposition}
Every Horn clause theory $T$ defines a smallest admissible model  ${\Basicleq_T} \subseteq \Trees(X) \times \Trees(X)$. The smallest model
is substitutive. In the case that $X = \mathbb{N}$, the smallest admissible model is thus an
admissible compositional preorder.
\end{proposition}


\begin{figure}[h!]
    \begin{equation*}
        \begin{array}{lrl}
            \text{Probability} & a \prEff a &= a \\
                        & a \prEff b &= b \prEff a \\
                        & (a \prEff b) \prEff (c \prEff d) &= (a \prEff c) \prEff (b \prEff d) \\
                        & a \prEff b \leq b &\implies a \leq b  \\
            %\hline
            \\
        \end{array}
        \begin{array}{lrl}
            \text{Non-Determinism} & a \orEff a &= a \\
                        & a \orEff b &= b \orEff a \\
                        & (a \orEff b) \orEff c &= a \orEff (b \orEff c) \\
            \\
        \end{array}
    \end{equation*}
    \begin{equation*}
        \begin{array}{lrl}

            \text{Angelic} & a \orEff b &\leq a \\ 
            \text{Demonic} & a \orEff b &\geq a \\ 
            %\hline 
            \text{Distributivity}
            & (a \orEff b) \prEff c &= (a \prEff c) \orEff (b \prEff c)
        \end{array}
    \end{equation*}
    \caption{Inequational theory for mixed probability and non
    determinism}
    \label{fig:axiomsmixed}
\end{figure}

\todo[inline]{AXIOMS FOR PROBABILITY PLUS TWO THEORIES OF NONDETERMINISM}

\todo[inline]{USE THE ABOVE TO DEFINE $\Basicleq^\Ax_\prang$ and
$\Basicleq^\Ax_\prdem$}

\todo[inline]{NOTE EXPLICITLY THAT ADMISSIBLE AND COMPOSITIONAL}

\todo[inline]{OBSERVE EQUIVALENCE OF INFINITARY EQUATIONAL AXIOM WITH HORN
CLAUSE AXIOM?}

\begin{definition}
    Let $t$ be a tree, $t^n$ is inductively defined as 
    $t^0 = \bot$ and $t^{n+1} = t \prEff t^n$. The
    tree $t^\infty$ is defined as $\sqcup_n t^n$.
\end{definition}

\begin{proposition}[Removing Horn-Clauses]
    The preorder $\Basicleq^{\Ax'}_\prdem$ obtained 
    by replacing the Horn-clause $a \prEff b \leq b \implies a \leq b$ by 
    the infinitary axiom $t^\infty = t$.
\end{proposition}

\begin{proof}
    \todo[inline]{Explain a bit more}
    It is clear that $t^n \Basicleq^\Ax_\prdem t$ forall $n$,
    and therefore $t^\infty \Basicleq^\Ax_\prdem t$. 
    We have $t \prEff t^n \Basicleq^\Ax_\prdem t^{n+1}$,
    and by admissibility we then deduce $t \prEff t^\infty \Basicleq^\Ax_\prdem
    t^\infty$, which using the Horn-clause axiom allows us to derive $t
    \Basicleq^\Ax_\prdem t^\infty$.


    For the converse inclusion assume $t \prEff t' \leq t'$,
    then $t \prEff (t \prEff t') \leq t \prEff t' \leq t'$.
    By induction and admissibility one deduces $t^\infty \leq t'$,
    and then $t = t^\infty$.
\end{proof}
