
\section{Domain theoretic preorders}

The goal of this section is to link the domain theoretic semantics 
with our setting. The first remark is that if 
$\llbracket \cdot \rrbracket$ is a semantic map from $\Tree_\Nat$ to 
a domain $D$, one can define:

\begin{equation*}
    t \sqeq_b t' \iff \llbracket t \rrbracket \leq \llbracket t' \rrbracket 
\end{equation*}

This preorder is not necessarily admissible or compositional, therefore 
we can ask ourselves what conditions on the semantic map $\llbracket \cdot
\rrbracket$ are sufficient to obtain the desired properties.

\begin{ensps}
\subsection{Admissibility}
\end{ensps}

One natural requirement for a semantic map is to be scott-continuous, and 
it turns out that it automatically gives an admissible preorder.

\begin{alemma}[Admissibility]
    \label{lem:continuousadm}
    If the function is scott-continuous, then the relation defined 
    is admissible.
\end{alemma}

\begin{ensps}
\begin{proof}
    Let $(t_i)_i$ and $(t_i')_i$ be two ascending chains for $\sqeq$
    with least upper bounds $t$ and $t'$ such that $\llbracket t_i \rrbracket
    \leq \llbracket t_i' \rrbracket$ for every $i$.

    \begin{align*}
        \left\llbracket \bigsqcup_i t_i \right\rrbracket &= \bigsqcup_i \llbracket t_i
        \rrbracket & \text{ scott-continuity } \\
                   &\leq \bigsqcup_i \llbracket t_i' \rrbracket & 
                    \text{ hypothesis } \\
                   &= \left\llbracket \bigsqcup_i t_i' \right\rrbracket &
                    \text{ scott-continuity } \\
    \end{align*}

\end{proof}
\end{ensps}

\begin{ensps}
\subsection{Compositionality}
\end{ensps}

A natural assumption for compositionality is asking for 
the interpretation to be a homomorphism and for $D$ 
to be a $\Sigma$-continuous algebra. Indeed, we are 
trying to interpret effects and it seems obvious 
that it is a good way to do so: in Plotkin and Power's work
algebraic effects are the one that are preserved 
under some homomorphisms [TODO cite].

\begin{alemma}[First half of compositionality]
    Assume $D$ is a $\Sigma$-continuous algebra
    and $\llbracket \cdot \rrbracket$ is a homomorphism,
    then for every tree $t$ and pair of substitutions 
    $(\sigma_1,\sigma_2)$ such that $\sigma_1(n) \sqeq_b \sigma_2 (n)$
    for every $n$, we have:

    \begin{equation*}
        t \sigma_1 \sqeq_b t \sigma_2
    \end{equation*}
\end{alemma}

\begin{ensps}
\begin{proof}
    For $i \in \{ 1, 2\}$ we have the following commutation diagram 
    obtained using the universal property of $\Tree_\Nat$ on 
    $\sigma_i$ and $\llbracket \cdot \rrbracket \circ \sigma_i$:

    \begin{center}
        \begin{tikzcd}
            \Nat \arrow[r, "\sigma_i"] \arrow[d, hook, "i"] & 
            \Tree_\Nat \arrow[r, "\llbracket \cdot \rrbracket"] & D \\
            \Tree_\Nat \arrow[ru, "\hat{\sigma_i}"]
                       \arrow[rru, bend right, "\widehat{\llbracket \cdot
                               \rrbracket \circ \sigma_i}"] 
        \end{tikzcd}
    \end{center}

    By definition, $t \sigma_1 = \hat{\sigma_1} (t)$, 
    and therefore $\llbracket t \sigma_1 \rrbracket = \llbracket \hat{\sigma_1}
    (t) \rrbracket$.

    Using the unicity of the homomorphism in the universal property of
    $\Tree_\Nat$ it is easy to show that $\llbracket \cdot \rrbracket 
    \circ \hat{\sigma_i} = \widehat{ \llbracket \cdot \rrbracket \circ \sigma_i
    }$

    But the hypothesis specifically states that $\llbracket \cdot \rrbracket 
    \circ \sigma_1$ is a function pointwise lower than $\llbracket \cdot
    \rrbracket \circ \sigma_2$. Therefore using the lemma about 
    lifting functions (lemma \ref{lem:orderpreservinglift})
    we know that the inequality is still true for the lifting 
    of both functions. Therefore:

    \begin{align*}
        \llbracket t \sigma_1 \rrbracket &= 
        (\llbracket \cdot \rrbracket \circ \hat{\sigma_1}) (t) \\
        &= \widehat{ \llbracket \cdot \rrbracket \circ \sigma_1} (t) \\
        &\leq \widehat{ \llbracket \cdot \rrbracket \circ \sigma_2} (t) \\
        &= 
        (\llbracket \cdot \rrbracket \circ \hat{\sigma_2}) (t) \\
        &= \llbracket t \sigma_2 \rrbracket
    \end{align*}
    
    And we have the conclusion: $t \sigma_1 \sqeq_b t \sigma_2$.
\end{proof}
\end{ensps}

But it is not enough to  guarantee compositionality of the preorder,
as it can be seen in the example \ref{ex:supercounterexample}.

\begin{example}[Simple counter example]
    \label{ex:supercounterexample}

    Let $\Sigma = \{ (+,2), (\times,2) \}$ be a signature.
    Let $D$ be extended natural numbers 
    with the usual ordering and the usual $\Sigma$-algebra
    structure. It is a continuous $\Sigma$-algebra 
    and the homomorphism that arises from identity 
    on natural numbers is just the regular interpretation 
    of tree of operations.

    But if $t_1 = 1 + 2$ and $t_2 = 1 \times 3$, we can 
    take $\sigma$ defined as follows to break compositionality:

    \begin{equation*}
        \sigma(k) = \begin{cases}
            0 & \text{ if } k = 1 \\
            k & \text{ otherwise } 
        \end{cases}
    \end{equation*}

    We have $t_1 \sqeq_b t_2$ but 
    not $t_1 \sigma \sqeq_b t_2$ because $2 \not \leq 0$.
\end{example}

It is therefore natural to ask for the interpretation 
to respect effects in a deeper way, related to 
the definition of \emph{algebraic} effects.

\begin{adefinition}[Factorisation of homomorphism]
    The interpretation $\llbracket \cdot \rrbracket$
    factors homomorphisms when for every function 
    $\sigma : \Nat \to D$ there exists a 
    homomorphism $h_\sigma : D \to D$ such that $\sigma = h_\sigma \circ \llbracket
    \cdot \rrbracket$. 
    
    Equivalently, when the following diagram commutes:

    \begin{center}
        \begin{tikzcd}
            \Nat \arrow[r, "\llbracket \cdot \rrbracket"] 
                 \arrow[rr, bend right, "\sigma"] &
            D \arrow[r, "h_\sigma", dashed] & 
            D  
        \end{tikzcd}
    \end{center}
\end{adefinition}

\begin{acorollary}[Factorisation of tree homomorphisms]
    If $\tau : \Tree_\Nat \to D$ is a homomorphism 
    then there exists $h_\tau$ a homomorphism
    from $D$ to $D$ such that $\tau = h_\tau \circ \llbracket \cdot \rrbracket$.
\end{acorollary}

\begin{ensps}
\begin{proof}
    It suffices to use the universal property of $\Tree_\Nat$ 
    and the fact that $\tau$ can be restricted to a function 
    on $\Nat$. Note that it is crucial that $h_\tau$ is a 
    homomorphism of $\Sigma$-continuous algebras, 
    otherwise the universal property cannot be used.
\end{proof}
\end{ensps}

\begin{alemma}[Second half of compositionality]
    If $\llbracket \cdot \rrbracket$ is a homomorphism 
    that factors functions from $\Nat$ to $D$ then 
    the preorder is compositional.
\end{alemma}

\begin{ensps}
\begin{proof}
    The only thing to prove is that given two trees $t_1$ and 
    $t_2$ such that $\llbracket t_1 \rrbracket \leq \llbracket t_2 \rrbracket$
    and a substitution $\sigma : \Nat \to \Tree_\Nat$ 
    we have $\llbracket t_1 \sigma \rrbracket \leq \llbracket t_2 \sigma
    \rrbracket$. 
    Then using the first half of compositionality we can conclude.

    The idea is then to use the factorisation and 
    the monotonicity of $h_\sigma$ to prove that:

    \begin{equation*}
        \llbracket t_1 \sigma \rrbracket 
        = h_\sigma (\llbracket t_1 \rrbracket) 
        \leq h_\sigma (\llbracket t_2 \rrbracket)
        = \llbracket t_2 \sigma \rrbracket 
    \end{equation*}

    This result is obtained using a technique similar 
    to the first part of compositionality proof: using 
    the universal property several times and prove 
    equality of the different ways of writing 
    the same homomorphism.

\end{proof}
\end{ensps}

\subsection{Examples}

We now list examples where the previous results can be used 
to reuse work done on denotational semantics.

The first example is when $D$ is a free structure. That 
can happen in cases such as Powerdomains 
\cite{abramsky1994}, Powercones \cite{tix2009semantic}, 
or Powerkegelspitze \cite{KeimelP2016}.

\begin{example}[Free structures]
    Assume that $D$ is a free structure 
    over $\Nat$ such that the structure 
    is a $\Sigma$-continuous algebra.
    Then the preorder defined using the embedding 
    of $\Nat$ into the free structure gives 
    an admissible and compositional preorder.
\end{example}

\begin{ensps}
    \begin{proof}
        Use the freeness property.
    \end{proof}
\end{ensps}

The second possibility is to already have a denotational semantics 
for the language defined in a monadic way on the category of $\omega$CPPOs.

\begin{example}[Monadic interpretation]
    If the language is interpreted in a $\omega$CPPO with 
    interpretations for effect that are \emph{natural}
    with respect to the EM-morphisms then the
    preorder obtained is admissible and compositional.
\end{example}

\begin{ensps}
\begin{proof}
    In a full interpretation with a monad $T$, for every effect $\sigma$
    there is a family of morphisms $(\sigma_X) : \mathbb{N} \times (T X)^\mathbb{N} \to T X$
    such that $\sigma_X$ is natural in $T X$ with respect to EM-morphisms, meaning that for 
    every $g : X \to T Y$ we have:

    \begin{center}
        \begin{tikzcd}
            \mathbb{N} \times (T X)^\mathbb{N} 
            \arrow[d, "\pi_0 \times (g^\dagger \circ \pi_i)_{i \geq 1}" left] 
            \arrow[r, "\sigma_X"] & T
            X \arrow[d, "g^\dagger"] \\
            \mathbb{N} \times (T Y)^\mathbb{N} \arrow[r, "\sigma_Y"] & T Y
        \end{tikzcd}
    \end{center}

    Therefore when there is a morphism 
    $f : \mathbb{N}_\bot \to T \mathbb{N}_\bot$
    we have the following diagram:

    \begin{center}
        \begin{tikzcd}
            \mathbb{N} \times (T \mathbb{N}_\bot )^\mathbb{N} 
            \arrow[d, "\pi_0 \times (f^\dagger \circ \pi_i)_{i \geq 1}" left] 
            \arrow[r, "\sigma_X"] & T \mathbb{N}_\bot\arrow[d, "f^\dagger"] \\
            \mathbb{N} \times (T \mathbb{N}_\bot)^\mathbb{N} 
            \arrow[r, "\sigma_Y"] & T \mathbb{N}_\bot
        \end{tikzcd}
    \end{center}

    Therefore, $f^\dagger$ is a homomorphism of $\Sigma$-continuous algebras,
    and we have 
    $f = f^\dagger \circ \eta_{\mathbb{N}_\bot}$.
\end{proof}
\end{ensps}

This result can be used to show that given an interpretation,
the contextual preorder we can define is the same as 
the contextual preorder defined using first the tree
calculation and then the monadic interpretation. This 
result comes from \cite{plotkin2001adequacy}.

\begin{acorollary}[Adequacy with the interpretation]
    \em
    Assume the interpretation $\llbracket \cdot \rrbracket$ 
    is obtained using a strong monad on $\omega$CPPO that 
    respects the order structure, then for closed ground 
    type terms:

    \begin{equation*}
        |M| \sqeq_b |M'| \iff \llbracket M \rrbracket \leq \llbracket M' \rrbracket
    \end{equation*}

    Therefore the contextual preorder defined using $\sqeq_b$
    is the same as the one defined using the interpretations, and 
    we can apply all theorems to the contextual preorder defined 
    using the interpretation.
\end{acorollary}

\begin{ensps}
\begin{proof}
    On closed ground type terms $|M| \sqeq_b |M'|$ is equivalent 
    to $\llbracket |M| \rrbracket \leq \llbracket |M'| \rrbracket$.
    Using the adequacy result from \cite{plotkin2001adequacy} 
    it is equivalent to $\llbracket M \rrbracket \leq \llbracket M' \rrbracket$.
\end{proof}
\end{ensps}

An interesting lemma can be used to prove compositionality 
and admissibility of a preorder. The idea is to find a 
bigger domain where the properties are clear, and 
an embedding-projection pair of $\Sigma$-continuous 
algebras between the two domains. 

\begin{alemma}[Embedding-Projection pair]
    If $f : \Nat \to E$ gives an admissible and 
    compositional preorder,
    $e : D \to E$  and $p : E \to D$ forms 
    an embedding-projection pair of homomorphisms,
    then $g = p \circ e$ gives an admissible 
    and compositional preorder.
\end{alemma}

\begin{proof}
    They define the same preorder.
\end{proof}

Using free structures in the category 
of $\omega$CPPO it is possible to define 
a free preorder in an abstract way.

\begin{alemma}[Free interpretational preorder]
    \label{lem:freedomainpreorder}
    Given an inequational theory $\mathcal{T}$
    there exists a preorder $\sqeq_{\llbracket \mathcal{T} \rrbracket}$
    that is obtnained using a denotational 
    intepretation and contains any preorder 
    given by such interpretation.
\end{alemma}

\begin{ensps}
\begin{proof}
    There is a free $\omega$CPPO for this inequational 
    theory. As seen before, this gives an interpretation
    and a preorder that is admissible and compositional.
    
    Let $D$ be another domain satisfying the formulas from $\mathcal{T}$,
    and an interpretaiton in this domain. Because of the 
    freeness property we know that this interpretation 
    can be factored out, and therefore we have 
    inclusion of preorders.
\end{proof}
\end{ensps}
