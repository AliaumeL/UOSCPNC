\section{Contextual Preorder}

Because we are going to study the contextual \emph{preorder},
the ground relation $\sqeq_b$ is going to be a \emph{preorder}: a reflexive 
and transitive relation. By building computation trees and using this basic 
preorder, we are going to define an abstract contextual preorder 
as the largest relation satisfying some axioms \cite{gom} \cite{Ugo2017}.
The definition is borrowed from the work of Andrew Pitts \cite{Pitts2000}, but can be
found in several other papers.

\newcommand{\CE}{\operatorname{\mathcal{E}}}

\begin{adefinition}[Formalisation of relations respecting type]
    Let $(\CE_V,\CE_C)$ be a pair of relations, the first 
    one on values and the second one on computations.
    If $\bullet$ is $V$ or $C$ then the relation 
    $\CE_\bullet$ is a set of tuples of the form $(\Gamma_\bullet, M,M', \tau)$
    and for every such tuple inside $\CE$ we have 
    $\Gamma \vdash_\bullet M : \tau$ and $\Gamma \vdash_\bullet M' : \tau$. 

    \begin{enumerate}[(i)]
        \item We say that $\CE$ is \emph{compatible} when
            $\CE$ is closed under rules in Figure
            \ref{fig:ax:compatibility}

        \item We say that $\CE$ is $\sqeq_b$-adequate 
            when for every pair of closed computation 
            terms $M$ and $M'$
            of ground type $\Nat$ we have $M \CE_C M'$ 
            implies $|M| \sqeq_b |M'|$
    \end{enumerate}

    We write $\Gamma \vdash M \CE M' : \tau$ 
    instead of $(\Gamma, M, M', \tau) \in \CE$ 
    to simplify reading.
\end{adefinition}

\begin{figure}[h]
    \begin{center}
        % Identity on variables 
            \AxiomC{}
            \UnaryInfC{$\Gamma, x:\tau \vdash x \CE_V x : \tau$}
            \DisplayProof 
        \hskip 1.5em
        % Return a value 
            \AxiomC{$\Gamma \vdash V \CE_V V': \tau$}
            \UnaryInfC{$\Gamma \vdash \Ret V \CE_C \Ret V' : \tau$}
            \DisplayProof 
        \hskip 1.5em
        \vskip 1em
        % Lambda abstraction 
            \AxiomC{$\Gamma, x : \tau \vdash M \CE_C M' : \tau'$}
            \UnaryInfC{$\Gamma \vdash (\lambda x:\tau. M) \CE_V
                (\lambda x:\tau. M'): \tau \to \tau'$}
            \DisplayProof 
        \hskip 1.5em
        % Zero 
            \AxiomC{}
            \UnaryInfC{$\Gamma \vdash \Zero \CE_V \Zero : \Nat$}
            \DisplayProof
        \hskip 1.5em
        % Succ 
            \AxiomC{$\Gamma \vdash V \CE_V V' : \Nat$}
            \UnaryInfC{$\Gamma \vdash \Succ V \CE_V \Succ V' : \Nat$}
            \DisplayProof
        \hskip 1.5em
        \vskip 1em
        % Fixed point 
            \AxiomC{$\Gamma \vdash V \CE_V V' :
                (\tau \to \tau') \to \tau \to \tau'$}
            \UnaryInfC{$\Gamma \vdash 
                \Fix V \CE_C \Fix V' : \tau \to \tau'$}
            \DisplayProof
        \hskip 1.5em
        % Application  
            \AxiomC{$\Gamma \vdash V \CE_V V': \tau \to \tau'$} 
            \AxiomC{$\Gamma \vdash W \CE_V W' : \tau$}
            \BinaryInfC{$\Gamma \vdash V W \CE_C V'W': \tau'$}
            \DisplayProof
        \hskip 1.5em
        \vskip 1em
        % Case   
            \AxiomC{$\Gamma \vdash V \CE_V V' : \Nat$} 
            \AxiomC{$\Gamma \vdash M_1 \CE_C M_1': \tau$}
            \AxiomC{$\Gamma, x:\Nat \vdash M_2 \CE_C M_2' : \tau$}
            \TrinaryInfC{$\Gamma \vdash 
                \left(\lcase{V}{M_1}{M_2}\right) 
                \CE_C
                \left(\lcase{V'}{M_1'}{M_2'}\right) : \tau$}
            \DisplayProof
        \hskip 1.5em
        \vskip 1em
        % Bind 
            \AxiomC{$\Gamma \vdash M \CE_C M' : \tau$} 
            \AxiomC{$\Gamma, x:\tau \vdash N \CE_C N' : \tau'$}
            \BinaryInfC{$\Gamma \vdash
                (\Bind{M}{x:\tau}{N})
                \CE_C
                (\Bind{M'}{x:\tau}{N'})
                : \tau'$}
            \DisplayProof
        \hskip 1.5em
        \vskip 1em
        % Effect 
            \AxiomC{$(\sigma,n) \in \Sigma$} 
            \AxiomC{$\forall 1 \leq i \leq n, 
            \quad \Gamma \vdash M_i \CE_C M_i' : \tau$}
            \BinaryInfC{$\Gamma \vdash
            \sigma(M_1, \dots, M_n)
            \CE_C
            \sigma(M_1', \dots, M_n')
            : \tau$}
            \DisplayProof
    \end{center}
    \caption{Compatibility rules}
    \label{fig:ax:compatibility}
\end{figure}


\begin{adefinition}[Contextual Preorder]
    There exists a largest compatible and $\sqeq_b$-adequate 
    relation called $\sqeq_{ctx}$
\end{adefinition}

In order to derive some of the theorems, we need to have more information 
about the $\sqeq_b$ preorder. One of the first properties is that 
it should behave nicely with approximation of trees (admissibility) and 
that it should behave nicely with composition of trees (compositionality).

\vspace{1em}

\begin{adefinition}[Admissibility]
    A relation $R$ on $\Tree_\Nat$ is admissible for $\sqsubseteq$ when 
    for every ascending chain $(t_i)_{i \geq 0}$ and 
    $(t_i')_{i\geq 0}$ such that $t_i \, R \, t_i'$:

    \[
        \left(\bigsqcup_{i \geq 0} t_i\right) \, R \, \left(\bigsqcup_{i \geq 0} t_i'\right)
    \]
\end{adefinition}

\begin{adefinition}[Compositionality]
    A relation $R$ on $\Tree_\Nat$ is compositional when 
    $t \, R \, t'$ and $\forall n, t_n \, R \, t_n'$ 
    implies that:
    
    \[
        t[ \bar{n} := t_n] \, R \, t'[ \bar{n} := t_n']
    \]

    Where $t[ \bar{n} := t_n]$ is defined as the lifted 
    $h_\sigma (t)$
    where $\sigma (n) = t_n$.
\end{adefinition}

The compositionality and admissibility requirements are 
natural ones, and they automatically have good properties 
on natural numbers (lemma \ref{lem:admcompnat}),
are nicely related to $\sqeq$ (lemma \ref{lem:coarserpreorder})
and can be constructed as the smallest preorder 
satisfying some inequational theory (lemma \ref{lem:freepreo}).

\begin{alemma}[Behaviour on natural numbers]
    \label{lem:admcompnat}
    If the preorder $\sqeq_b$ is  
    compositional, then on natural 
    numbers we have either that two distinct 
    natural numbers are not comparable, 
    or that every pair of tree is equated  
    by $\sqeq_b$.
\end{alemma}


\begin{ensps}

\begin{proof}
    Assume that there exists two distinct numbers $\underline{m}$
    and $\underline{n}$ such that $\underline{n} \sqeq_b \underline{m}$.
    Let $t$ and $t'$ be 
    two arbitrary trees and define:

    \begin{equation*}
        \sigma(k) = \begin{cases}
            t  & \text{ if } k = n \\
            t' & \text{ if } k = m \\
            \bot  & \text{ otherwise } 
        \end{cases}
    \end{equation*}

    We have $\sigma \sqeq_b \sigma$ because $\sqeq_b$
    is reflexive, and therefore using compositionality 
    of $\sqeq_b$ we have:

    \begin{equation*}
        \underline{n}\sigma \sqeq_b \underline{m}\sigma
    \end{equation*}

    Hence for every trees $t$ and $t'$ we have 
    $t \sqeq_b t'$.
\qed\end{proof}
\end{ensps}

\begin{alemma}[The preorder is coarser than $\sqeq$]
    \label{lem:coarserpreorder}
    Assume that $\sqeq_b$ is admissible and compositional,
    then $(\sqeq) \subseteq (\sqeq_b)$ if and only if
    $\bot$ is a least element for $\sqeq_b$.
\end{alemma}

\begin{ensps}
    
\begin{example}[Simple counter example]
    There exists an admissible and compositional 
    preorder that does not extend $\sqeq$.
\end{example}

\begin{proof}
    Define $t \sqeq_b t' \iff \bot \in t \implies \bot \in t'$ where
    the $\bot \in t$ means that there exists a leaf of $t$ 
    which is $\bot$ \emph{or} that there exists an infinite branch 
    in $t$.

    Compositionality and admissibility are simple, and we note 
    that $\bot \neq \sqeq_b \underline{n}$, which proves 
    the claim.
\qed\end{proof}
\end{ensps}


We can build free preorders respecting $\sqeq$
given an inequational theory $\mathcal{T}$ as shown
in the following lemma \ref{lem:freepreo}.

\begin{adefinition}[Horn Clause Inequational Theory]
    A theory $\mathcal{T}$ is a horn-clause inequational 
    theory over $\Tree_\Nat$ if and only if it consists 
    in a list of formulas obtained with the following grammar:

    \begin{align*}
        t    &:= x ~|~ \sigma (\overbrace{t,\dots,t}^n) \quad (\sigma,n) \in
        \Sigma\\
        \phi &:= t \leq t \\
        \psi &:= \phi \vee \dots \vee \phi \vee \neg \phi
    \end{align*}
\end{adefinition}

\begin{example}[Angelic non-determinism]
    It can be shown that the inequational theory 
    characterising the powerdomain for angelic 
    non-determinism is the following one:

    \begin{align*}
        x \leq \orEff (x,y) & & \orEff (x,x) \leq x & & \orEff (x,\orEff (y,z))
        \leq \orEff(\orEff (x,y),z) \\
        \orEff (x,y) \leq \orEff(y,x) & & & & \orEff(\orEff (x,y),z) \leq \orEff
        (x,\orEff (y,z))
    \end{align*}
\end{example}

\begin{alemma}[Free preorder construction]
    \label{lem:freepreo}
    Given an inequational theory with horn-clauses $\mathcal{T}$
    there exists a smallest admissible 
    and compositional preorder $\sqeq_\mathcal{T}$ 
    on $\Tree_\Nat$ satisfying the inequational theory 
    such that for any tree $t$, $\bot \sqeq_\mathcal{T} t$.
\end{alemma}

\begin{proof}
    It is clear that satisfying some inequational 
    theory is stable by arbitrary intersection. Because 
    admissibility, compositionality and being 
    a preorder are also 
    stable properties by such intersection, one 
    can take the intersection of all such preorders.
\qed\end{proof}
