\subsection{Uniform Small-Step Semantics}

The first step to define contextual equivalence 
is to define an operational semantics.
Because we are considering a \emph{class} of languages 
we are going to have a two steps approach: first 
interpret the core language independently of the parameter $\Sigma$
and only then \emph{refine} this semantics to give one to the effects.

A small-step semantics is given using 
stack and frames \cite{gom}.
The link between this small-step semantics 
and its big-step counterpart is 
made both in \cite{plotkin2001adequacy} 
and in \cite{Pitts2000}. This method
can be traced back to \cite{Amadio2008} (1998) page 184 and 
\cite{Felleisen1992}.

In a refined call-by-value calculus stacks and frames 
are very simple as it can be seen in Figure \ref{fig:stacks:frames}, 
but few cases does not imply little expressive power, and 
in fact any term with one free variable can be turned into 
a stack. A type system for stacks is also defined Figure \ref{fig:stacks:types}.

\begin{figure}[h]
    \begin{center}
        \begin{align*}
            E &:= \Bind{\square}{x:\tau}{M} \\
            S &:= Id ~|~ S \circ E
        \end{align*}
    \end{center}
    \caption{Stacks and Frames}
    \label{fig:stacks:frames}
\end{figure}


\begin{figure}[h]
    \begin{center}
        \AxiomC{$x:\tau \vdash_C M : \tau'$}
        \UnaryInfC{$\vdash \Bind{\square}{x:\tau}{M} : 
            \tau \multimap \tau'$}
        \DisplayProof
        \hskip 1.5em
        \AxiomC{$\vdash E : \tau \multimap \tau'$}
        \AxiomC{$\vdash S : \tau' \multimap \tau''$}
        \BinaryInfC{$\vdash S \circ E : \tau \multimap \tau''$}
        \DisplayProof
        \vskip 1.5em
        \AxiomC{}
        \UnaryInfC{$\vdash Id : \tau \multimap \tau$}
        \DisplayProof
    \end{center}
    \caption{Typing of Stacks and Frames}
    \label{fig:stacks:types}
\end{figure}


Because stacks are just a \texttt{let}-binding with one hole, we can define 
application of a stack to a \emph{computation} which returns a computation
obtained by substitution as defined in Figure \ref{fig:stackapplication}.

\begin{figure}[h]
    \begin{align*}
        Id\{ M \} &= M \\
        (S \circ E) \{ M \} &= S \{ E \{ M \} \} \\
        (\Bind{\square}{x : \tau}{N} \{ M \}) &= \Bind{M}{x : \tau}{N}
    \end{align*}
    \caption{Stack application}
    \label{fig:stackapplication}
\end{figure}

%\begin{adefinition}[Stack application]
    %Given a stack $S$ and a computation term $M$
    %we can define $S\{M\}$ to be the application 
    %of the stack to the term by induction on $S$
    %following the rules in Figure \ref{fig:stackapplication}.
%\end{adefinition}

We say that a pair $(S,M)$ of a stack and a term is well 
typed when $S : \sigma \multimap \tau$ and $M : \sigma$, 
this will be consistent with the following safety theorem.

\begin{alemma}[Safety]
    Typing of stacks and frame is coherent
    with the typing of terms and stack application.

    \begin{equation*}
        \vdash S : \tau \multimap \tau' \, \wedge \, 
        \vdash_C M : \tau
        \implies
        \vdash_C S\{M\} : \tau'
    \end{equation*}
\end{alemma}

Now that evaluation stacks are defined, we can use them to define 
the operational semantics of our language. First of all, on 
simple computations, we can define a term reduction 
as in Figure \ref{fig:termred}, then given this basic 
reduction, we can use stacks to generalise the evaluation 
process to more terms as seen in Figure \ref{fig:evalstep}.


\begin{figure}[h]
    \begin{center}
        \begin{equation*}
            \begin{array}{lcl}
                (\lambda x:\tau. M) V & \leadsto & M[x := V] \\
                \lcase{\Zero}{M_1}{M_2} & \leadsto & M_1 \\
                \lcase{\Succ(V)}{M_1}{M_2} & \leadsto & M_2[x := V] \\
                \Bind{\Ret V}{x : \tau}{M} & \leadsto & M[x := V] \\
                \Fix V & \leadsto & V (\lambda x. \Bind{(\Fix V)}{g}{g x})
            \end{array}
        \end{equation*}
    \end{center}
    \caption{Term reduction}
    \label{fig:termred}
\end{figure}


\begin{figure}[h]
    \begin{center}
        \begin{equation*}
            \begin{array}{lclr}
                (S,E\{M\}) & \reduces & (S \circ E, M) \\
                (S \circ E, \Ret V) & \reduces & (S, E \{ \Ret V \}) \\
                (S,M) & \reduces & (S,M') & \text{ when } M \leadsto M'
            \end{array}
        \end{equation*}
    \end{center}
    \caption{Evaluation steps}
    \label{fig:evalstep}
\end{figure}

\begin{alemma}[Safety]
    If a pair $(S,M)$ is well-typed then 
    evaluation preserves this property 
    and the co-domain of the stack never changes.
\end{alemma}

\begin{example}[Reduction for combination of non-determinism and probabilities]
    We can consider the following term in our language with signature $\Sigma =
    \{ \prEff, \orEff \}$:

    \begin{equation*}
        M = (\lambda x:\Nat. \prEff (x,\underline{1})) \underline{0}
    \end{equation*}
    
    This term corresponds to applying \underline{0} to a function 
    that tosses a coin and return either the input given to the 
    function or \underline{1}.
    It can be shown that $(Id,M)$ reduces to $(Id, \prEff (\underline{0},
    \underline{1}))$
    as one could expect. However there is no more reduction possible afterwards: the 
    evaluation is stuck, because there is no rule to evaluate 
    the effect of a coin toss.
\end{example}


\subsection{Computation Trees}

We now have a definition of an operational semantics 
for the parametrized language, but it lacks one thing: 
any evaluation that encounters an effect is stuck, because 
there is no rule to deal with them. The technical tool used to keep the 
semantics uniform across the class of languages 
is the notion of computation tree \cite{plotkin2001adequacy} \cite{gom}.

The idea is that any effect has a finite amount of arguments defined 
by its arity in $\Sigma$, because we do not know what the effect is going 
to do with the different arguments, we are going to compute all of them, 
and build a tree whose nodes are uninterpreted effects, and leaves are 
syntactical values of our calculus. One problem arises because of 
the fixed point operator: some trees can be infinite, and some computation 
are not terminating. 

To deal with both issues, trees of type $\tau$ are going to have an $\omega$CPPO 
structure, meaning that there is a preorder $\sqeq$ on trees,
that least-upper bounds of increasing sequences of trees 
can be taken, and that there is a bottom tree $\bot$
under any other tree. We are in fact building the free $\Sigma$-continuous algebra
over the set of values of type $\tau$ \cite{abramsky1994}.

\begin{adefinition}[Trees over a set]
    A computation tree over a set $X$ is the
    free continuous $\Sigma$-algebra over $X$.  
    A continuous $\Sigma$-algebra is an $\omega$CPPO 
    equipped with continuous functions of appropriate arity 
    for each operation symbol in $\Sigma$; 
    a morphism of such algebras is a strict continuous
    function preserving the operations; the free 
    continuous $\Sigma$-algebra functor
    is the left adjoint to the forgetful 
    functor from the category of continuous $\Sigma$-algebras to that of
    sets.
\end{adefinition}


\begin{adefinition}[Computation Tree]
    Given $\tau$ a type, a computation tree of type $\tau$
    is an element of $\Tree_{\tau}$ which 
    is a shorthand for the trees over the set of 
    values of type $\tau$.

    Because $\Tree_\tau$ is the free
    continuous $\Sigma$-algebra over values of type $\tau$
    we have the following universal property. 

    For every function $f : \textnormal{Values}_\tau \to A$ where 
    $A$ is a continuous $\Sigma$-algebra,
    there exists a unique morphism $\hat{f} : \Tree_\tau \to A$
    such that:

    \begin{equation*}
        f = \hat{f} \circ i
    \end{equation*}

    \begin{center}
        \begin{tikzcd}
            \textnormal{Values}_\tau
            \arrow[r, "f"] 
            \arrow[d, hook, "i"]
            & A \\
            \Tree_\tau \arrow[ru, dashrightarrow, "\hat{f}" below]
        \end{tikzcd}
    \end{center}
\end{adefinition}

It is worth noticing that the lifting operation $f \mapsto \hat{f}$ is
a continuous functional. A consequence of this is the following monotonicity 
property of the lifting.

\begin{alemma}[Order preserving lift]
    \label{lem:orderpreservinglift}
    If $\sigma_1 : \Nat \to A$ and $\sigma_2 : \Nat \to A$ 
    are two functions such that $\sigma_1 \leq \sigma_2$ pointwise,
    then $\hat{\sigma_1} \leq \hat{\sigma_2}$ pointwise.
\end{alemma}

We can already see the usefulness of this definition 
when looking at the construction of substitution:
given abstractly, in full generality and 
guaranteed to be continuous without any hassle. 

\begin{adefinition}[Substitution]
    Let $t \in \textnormal{Tree}_X$ 
    and $\sigma : X \to \textnormal{Tree}_Y$.
    Using the universal property, there exists a
    unique lift $\hat{\sigma} : \textnormal{Tree}_X \to \textnormal{Tree}_Y$,
    which is the substitution. We write $t\sigma$ as a shorthand for 
    $\hat{\sigma}t$.
\end{adefinition}

Using this new construction it is now possible  
to continue our definition of the operational semantics without 
interpreting effects by building an infinite tree
labeled by the effects encountered during evaluation.

\subsection{Interpretation}

Because of the injection of values of type $\tau$ into $\Tree_\tau$,
and to avoid unnecessary clutter in the equations, we are going to treat 
this injection as an inclusion and write $V$ instead of $i(V)$.

%The idea of this definition is to interpret terms "as most as we can"
%without having a semantics for the effects. When encountering an effect operation,
%the idea is to take advantage of the "continuation passing"
%and instead of having a specific rule for dealing with the effect, 
%create a branch for every possible input of the remaining computation 
%(see \cite{plotkin2001adequacy} and \cite{gom}). 

\begin{adefinition}[Computation tree]
    Given a well-typed pair $(S,M)$ and an integer $n$ we can 
    compute the tree $|S,M|_n$ by induction on $n$ 
    following the rules in Figure \ref{fig:treecalcul}.
    The sequence $|S,M|_n$ is ascending in $n$ given a fixed 
    pair $(S,M)$ and we write $|S,M|$ for its least upper bound.
\end{adefinition}

\begin{figure}[h!]
    \begin{center}
        \begin{equation*}
            \begin{array}{llr}
                |S,M|_{n+1} &= |S',M'|_n & (S,M) \reduces (S',M') \\
                |Id,\Ret V|_{n+1} &= i(V) & i \text{ is the injection from values into
                trees} \\
                |S,\sigma(M_1, \dots, M_l)|_{n+1} &= \sigma (|S, M_1|_n, \dots,|S,M_l|_n) \\
                |S,M|_0 &= \bot \\
            \end{array}
        \end{equation*}
    \end{center}
    \caption{Computation tree construction}
    \label{fig:treecalcul}
\end{figure}

This construction can be tested on a non terminating term 
that never uses effects:
because the computation tree is going to be $\bot$ at any 
step, the global computation tree is going to be $\bot$.


\begin{example}[Non termination]
    The following term does not terminate, and no effects occurs 
    during the evaluation of this term:
    
    \begin{equation*}
        \Omega_\tau = \Bind{\Fix (\lambda f : \Nat \to \tau. \Ret f
        )}{g : \Nat \to \tau}{g \underline{0}}
    \end{equation*}

    It is therefore possible to conclude that the computation tree
    associated to this term is always the least element of $\Tree_\tau$,
    independently of the stack $\vdash S : \tau \to \tau'$:

    \begin{equation*}
        \forall S : \sigma \multimap \tau, |S, \Omega_\sigma| = \bot
    \end{equation*}
\end{example}


One key result about this tree construction is the relationship between 
substitution on trees and application of stacks. To compute a computation 
tree for the pair $(S,M)$ it suffices to compute the computation tree 
for $(Id,M)$ and then substitute leaves with the computation tree obtained 
by $(S,\Ret V)$ where $V$ is the corresponding value of the leaf.

\begin{alemma}[Stack commutation]
    \label{lem:stackcom}
    Let $S : \tau \multimap \tau'$ and $M : \tau$, we always have:

    \begin{equation*}
        |S, M| = |Id,M| \sigma_S
    \end{equation*}

    Where $\sigma_S (V) = |S, \Ret V|$.
\end{alemma}

\begin{proof}
    We prove by induction on $n$ that for any well-typed pair $(S,M)$ we have 
    such that for every $m \geq n$ and $m' \geq n$ 
    we have $|S,M|_n \sqeq |M|_m \sigma^{m'}_S$, where 
    $\sigma^{m'}_S (V) = |S, \Ret V|_{m'}$.

    When $n=0$ the result is obvious because $\bot$ is under any other tree.

    \begin{itemize}
        \item If $|S,\sigma(M_1, \dots, M_k)|_{n+1} = \sigma (|S,M_1|_n, \dots,
            |S, M_k|_n)$ then we can use the induction hypothesis on all 
            subtrees and have $|S, M_i|_n \sqeq |M_i|_m \sigma^{m'}_S$ when 
            $m$ and $m'$ are above $n$. By continuity of $\sigma$ and 
            compatibility of substitution we have therefore when $m$ and $m'$
            are above $n$ we have:

            \begin{equation*}
                |S, \sigma (M_1, \dots, M_k)|_{n+1} \sqeq |\sigma(M_1, \dots,
                M_k)|_m \sigma^{m'}_S
            \end{equation*}

        \item If $|S, M|_{n+1} = |S, M'|_n$ because $M \leadsto M'$, and 
            therefore $|M|_{n+1} = |M'|_n$, and we can use this to trivially
            obtain the desired result.

        \item If $|S \circ E, M|_{n+1} = |S, E \{M\}|_n$ then 
            $M = \Ret V$ for some value $V$ and the result is obvious.

        \item If $|S, E\{M\}|_{n+1} = |S \circ E, M|_n$ then 
            we can use the induction hypothesis and the fact that 
            $\sigma^m_{S \circ E} \sqeq \sigma^m_S \circ \sigma^m_E$ pointwise
            to conclude.
    \end{itemize}
    
    Taking the limit we can conclude:

    \begin{equation*}
        |S,M| \sqeq |M| \sigma_S
    \end{equation*}
    
    The other inequality is obtained using a similar reasoning. 
\end{proof}


\subsection{Iteration Approximation}

The last thing to check with our construction is that approximation on trees 
is indeed capturing the construction of infinite trees obtained by fixed points 
in the language. This equivalence 
is precisely given by the Theorem \ref{thm:unrolling} stating that 
the least upper bound of the semantics of the approximations is 
exactly the semantics of the fixed point. An equivalent
result also exists in call-by-name \cite{gom}, note that 
because of the evaluation strategy, the theorem is stated 
on trees over values of type $\Nat$ to ensure full 
evaluation of the fixed-point.

Unrolling fixed points is a very common operation \cite{plotkin2001adequacy}
and this approximation result is a basic one in papers defining logical 
relations \cite{pitts1997operationally} \cite{Pitts2000}.

\begin{figure}[h]
    \begin{align*}
        \Unroll_0 V     &= \Omega_{\sigma \to \tau} \\
        \Unroll_{n+1} V &= 
        V (\lambda x : \sigma. \Bind{(\Unroll_n V)}{g : \sigma \to \tau}{gx})
    \end{align*}
    \caption{Unrolling fixed point}
    \label{fig:unrolling}
\end{figure}

\begin{atheorem}[Unrolling]
    \label{thm:unrolling}
    Let $\vdash S : (\sigma \to \tau) \multimap \Nat$ be a stack and 
    $\vdash_V V : (\sigma \to \tau) \to \sigma \to \tau$ be a 
    value term.

    \begin{equation*}
        \bigsqcup_{n \geq 0} |S, \Unroll_n V| = |S, \Fix V| 
    \end{equation*}
\end{atheorem}

\begin{ensps}
\begin{proof}
    \textbf{Alternative proof:}

    By induction on $n$ we prove that 

    \begin{equation*}
        \forall S,M,n,
        \exists m_0, \forall m \geq m_0, 
        |S[\Fix V], M[\Fix V]|_n = |S[\Unroll_m V], M[\Unroll_m V]|_n
    \end{equation*}


    If $n = 0$ then the result is obvious because all trees are $\bot$.
    When looking at $n+1$, we do a case analysis on the production rule 
    of the tree, and use the induction hypothesis. The only interresting 
    cases are: 

    \begin{enumerate}
        \item A fixed point evaluation, where it suffices to increase $m_0$ 
            by one
        \item An effect unfolding, where we take the maximal $m_0$ for each
            branch
    \end{enumerate}

    This proof is way simpler than what can be found in the work of Andrew Pitts
    but it does not scale to countable branches. For the sake of the argument 
    the outline of a more operational (and generic) proof is the following one.

    By induction on the derivation we can prove that 
    if 
    \begin{equation*}
        (S[x:=\Omega_\tau],M[x:=\Omega_\tau]) \reduces^* 
        (S'[x:=\Omega_\tau], M'[x:=\Omega_\tau])
    \end{equation*}
    then for any computation term $F$ of type $\tau$ 
    we have 
    \begin{equation*}
        (S[x:=F],M[x:=F]) \reduces^* 
        (S'[x:=F], M'[x:=F])
    \end{equation*}

    Using this result, we can prove by induction on $n$ 
    that if 
    \begin{equation*}
        (S[x:=\Unroll_n V],M[x:=\Unroll_n V]) \reduces^* 
        (S'[x:=\Unroll_n V], M'[x:=\Unroll_n V])
    \end{equation*}

    Then 
    \begin{equation*}
        (S[x:=\Unroll_{n+1} V],M[x:=\Unroll_{n+1} V]) \reduces^* 
        (S'[x:=\Unroll_{n+1} V], M'[x:=\Unroll_{n+1} V])
    \end{equation*}

    This allows us to prove by inudction on the derivation 
    that if 
    \begin{equation*}
        (S[x:=\Fix V],M[x:=\Fix V]) \reduces^* 
        (S'[x:=\Fix V], M'[x:=\Fix V])
    \end{equation*}
    Then there exists an $n_0$ such that for all $n \geq n_0$
    we have 
    \begin{equation*}
        (S[x:=\Unroll_n V],M[x:=\Unroll_n V]) \reduces^* 
        (S'[x:=\Unroll_n V], M'[x:=\Unroll_n V])
    \end{equation*}

    Using another induction we can prove the converse statement. 

    Now by induction on $k$ we can prove that for any depth 
    $k$ the truncation of $|S[x:= \Fix V, M[x := \Fix V]|$ 
    at depth $k$ is equal to the truncation of 
    the supremum of the chain $|S[x:= \Unroll_n V], M[x := \Unroll_n V]|$
    at depth $k$.

    This gives us the expected result because a computation tree 
    is the limit of it's finite approximations.

    \qed
\end{proof}
\end{ensps}


There is now an operational semantics for the language, respecting 
effects and capturing approximations. Up to this point, everything 
has been done uniformly over all signatures $\Sigma$. 
Given a computation term $M$ of type $\tau$, one can compute 
$|Id,M|$ (abbreviated $|M|$) which is a tree labelled with effects 
and has \emph{values} of type $\tau$ as leafs. 

The next goal is to define contextual equivalence for a specific signature 
$\Sigma$. This is usually done by fixing a relation on computation terms of type 
$\Nat$. However we can separate the
semantics of the effects contained in this relation from the semantics 
of the ground language: it suffices to give a relation on \emph{trees} of 
natural numbers to have one over computation terms of type $\Nat$.
The information required is therefore a simple relation $\sqeq_b$
over $\Tree_\Nat$.
