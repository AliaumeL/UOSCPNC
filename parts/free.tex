
\section{Free preorders}

Given an inequational theory $\mathcal{T}$ one can always 
build some preorder $\sqeq_b$ corresponding to it \ref{lem:freedomainpreorder}.
This uses the previous part about denotational semantics 
and preorders built using them. However we saw 
that one could naturally build the smallest admissible,
compositional preorder extending $\sqeq$ without 
requiring such machinery \ref{lem:freepreo}. 
It is not known (yet) if the two constructions coincide,
but for all the signature in this paper, they will.

\begin{alemma}[Angelic preorder]
    The free preorder for the theory 
    $\mathcal{T}_A$ coincides 
    with the preorder $\sqeq_A$ defined 
    using~:

    \begin{equation*}
        t \sqeq_A t' 
        \iff
        \forall n \in \Nat, n \in t \implies n \in t'
    \end{equation*}

    And $\sqeq_A$ itself coincides with the denotational
    iterpretation in the Hoare powerdomain for $\Nat$
    (the free join-semilattice over $\Nat$).

    \begin{equation*}
        \begin{array}{rl}
            a \orEff a &= a \\
            a \orEff b &= b \orEff a \\
            a \orEff (b \orEff c) &= (a \orEff b) \orEff c \\
            a \orEff b &\geq a
        \end{array}
    \end{equation*}
\end{alemma}

\begin{proof}
    First, it is easy to see that indeed the denotational
    and the operational part are giving the same preorders.

    Thus, $\sqeq_A$ is admissible and compositional. Moreover,
    $\sqeq_A$ satisfies the inequational 
    theory $\mathcal{T}$ in a trivial way.

    Therefore, $(\sqeq_{\mathcal{T}_A}) \subseteq (\sqeq_A)$.

    The other inclusion is done it two steps, following the 
    idea that the free preorder on $\mathcal{T}$ is
    the «~admissible extension~» of the free preorder on finite 
    trees.
    
    Let $t$ and $t'$ be two finite trees, we can use 
    idempotence to prove that $t$ is equivalent to 
    a complete binary tree with depth $n$, and 
    the same for $t'$. Given two complete binary
    trees of equal depth, using first associativity 
    and commutativity one can reorder the leaves, and 
    then using only $\bot \sqeq_{\mathcal{T}_A} x$ for any
    $x$ and compositionality construct 
    a finite proof that $t \sqeq_{\mathcal{T}_A} t'$.
    Indeed, for every leaf in $t$, the corresponding 
    leaf in $t'$ is either the same one or $\bot$.

    
    If two infinite trees have their results included,
    then it suffices to notice that any finite approximation 
    of the first has it's results included in a finite 
    approximation of the second. With this and the result 
    on finite tree, admissibilty can be applied to 
    show that $t \sqeq_{\mathcal{T}_A} t'$. 
\end{proof}


\begin{alemma}[Demonic preorder]
    The free preorder for the theory $\mathcal{T}_D$ coincides with 
    the preorder $\sqeq_D$ given by (TODO)~:

    \begin{equation*}
        t \sqeq_D t' \iff 
        \begin{cases}
            \bot \in t' \implies \bot \in t \\
            n    \in t' \implies n \in t    \\
            \bot \in t
        \end{cases}
    \end{equation*}

    And $\sqeq_D$ itself coincides with the denotational
    iterpretation in the Smyth powerdomain for $\Nat$
    (the free meet-semilattice over $\Nat$).

    \begin{equation*}
        \begin{array}{rl}
            a \orEff a &= a \\
            a \orEff b &= b \orEff a \\
            a \orEff (b \orEff c) &= (a \orEff b) \orEff c \\
            a \orEff b &\leq a
        \end{array}
    \end{equation*}
\end{alemma}

\begin{proof}
    The proof follows the exact same pattern as the previous one,
    except that infinite trees are all equivalent to $\bot$ 
    in this theory, and therefore the infinite case is even 
    simpler.
\end{proof}


\begin{alemma}[Probabilistic preorder]
    We use the notation $\oplus$ for the infix notation 
    of $\prEff$ to ease lecture.

    The free preorder for the theory $\mathcal{T}_P$
    coincides with the preorder defined by~:
    
    \begin{equation*}
        t \sqeq_P t' \iff 
        \forall n \in \Nat, \nu (t) \leq \nu (t')
    \end{equation*}

    And $\sqeq_P$ itself coincides with the denotational
    iterpretation in the probabilistic powerdomain for $\Nat$
    (the free full kegelspitze over $\Nat$).

    The axiomatisation can be found in \cite{heckmann1994probabilistic}.
    \begin{equation*}
        \begin{array}{rl}
            a \oplus a &= a \\
            a \oplus b &= b \oplus a \\
            (a \oplus b) \oplus (c \oplus d) &= (a \oplus c) \oplus (b \oplus d) \\
            a \oplus b \leq b &\implies a \leq b
        \end{array}
    \end{equation*}
\end{alemma}

\begin{proof}
    As before, the first inclusion is not a problem, and on finite 
    trees transforming into a complete binary tree with 
    an ordering on leafs is 
    possible to prove the inequality (normal form).

    Let $t$ and $t'$ be two possibly infinite trees 
    having only a \emph{finite} support. We can make
    a case distinction~:

    \begin{enumerate}
        \item There exists a finite approximation $t_i'$ of $t'$ 
            such that $\nu (t_i') = \nu (t')$.

        \item For any finite approximation $t_i'$, there exists 
            an $n$ such that $\nu (t_i')(n) < \nu (t')(n)$.
    \end{enumerate}

    Now let's take two approximating chains of finite trees 
    for $t$ and $t'$, and build the chain $t_i \oplus t_i'$.
    
    We know that $\nu (t_i \oplus t_i') \leq \nu (t')$ by 
    some simple calculation. Now, assume we are in the first 
    case, then we have a finite approximating tree $t_j'$ with $j > i$ such that 
    $\nu (t_i \oplus t_i') \leq \nu (t_j')$. In the second case,
    we know that there is an $n$ such that the inequality is strict~: 
    but because $\nu$ is scott-continuous, there is a $j > i$ such that 
    $\nu (t_i \oplus t_i') (n) < \nu (t_j')$. Because the support 
    is finite, we know that we can take the maximum of such $j$'s 
    and have a finite tree $t_j'$ such that $\nu (t_i \oplus t_i') (n) \leq \nu
    (t_j')$.

    But all the trees in this last equation are finite, and therefore 
    they are true for $\sqeq_{\mathcal{T}_P}$.

    \begin{equation*}
        \forall i \in \mathbb{N}, \exists j > i, 
        t_i \oplus t_i' \sqeq_{\mathcal{T}_P} t_j'
    \end{equation*}

    Using admissibility, we can now conclude~:

    \begin{equation*}
        t \oplus t' \sqeq_{\mathcal{T}_P} t'
    \end{equation*}

    But then we can deduce that $t \sqeq_{\mathcal{T}_P} t'$ using 
    the last axiom of $\mathcal{T}$.


    To extend this result to infinite support, it suffices to 
    use the family of substitutions~:

    \begin{equation*}
        \sigma_k (i) = \begin{cases}
            \bot & \text{ when } i > k \\
            i    & \text{ otherwise } 
        \end{cases}
    \end{equation*}

    If $\nu (t) \leq \nu (t')$ then we know that 
    for all $k$, $\nu (t\sigma_k) \leq \nu (t'\sigma_k)$
    and they have finite support, therefore 
    $t \sigma_k \sqeq_{\mathcal{T}_P} t' \sigma_k$.
    To conclude it suffices to see that $t = \sqcup_k t\sigma_k$
    and use admissibility.

\end{proof}
