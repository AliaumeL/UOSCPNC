\section{Conclusion and future work}




%% This should be in conclusion ! 
[TODO PLACE THIS SOMEWHERE] This setting is purposely restricted, and several improvements can 
be added without technical issue. For instance, more complex types 
such as sum, products and even type polymorphism can be studied 
in this context. In fact, logical relations excel in proving parametricity 
results \cite{wadler1989theorems}.
In the spirit of simplicity 
and to allow comparison with the work on bisimulations by
Ugo Dal Lago, Francesco Gavazzo and Paul Blain Levy
\cite{Ugo2017} we take the same kind of effect signature 
as they do. Technically, it means that compared to 
the paper from Johann et al. \cite{gom} it lacks 
three of the four effect constructions, but as noticed 
in the said paper, all of the constructions share the
same pattern of proof, so that they actually treated 
only one of the four cases in their proofs.

The first part of this paper was about extending the results 
of Alex Simpson, Patricia Johann and Janis Voigtländer \cite{gom}
to a call-by-value setting. By doing so, some properties of 
the basic preorder $\sqeq_b$ were developed and a direct link
to denotational semantics has been made. This is a first step 
in a better understanding of basic preorders and the generality 
of the method itself. 

Some generic theorems and sanity checks have been proven 
abstractly for the logical relation and the contextual preorder
arising from $\sqeq_b$, allowing to decline them with any 
effect signature $\Sigma$. 

The ability to automatically build free preorders for an equational 
theory $\mathcal{T}$ was studied, and compared to the operational 
and denotational method in the case of probability, non-determinism
and the combination of both, showing how robust this general setting 
is.

\vspace{1em}

Obviously, the study of the requirements for a basic preorder 
are not enough, and it would be interesting to see how far we 
can weaken the admissibility property and still get results.
For instance, countable non-determinism [CIT] does \emph{not} have 
this admissibility property, but is believed to still fit in 
our setting. On the other side of the requirements, compositionality 
could be better understood using sets of observations as done in 
\cite{gom} or looking at the continuity properties of the 
monadic multiplication on trees.

It would then be interesting to try and generalise the class of 
effects, be it by adding more effects known to be algebraic, 
or blatantly non algebraic effects such as exception handling 
that would require changing the language itself, but still 
be captured by the same general method.

This work can be extended to a richer type system in the obvious 
way, and even recursive types are not an issue using step-indexing 
techniques [CITE] or by defining the projections in the language [CITE].

It could be interesting to generalise the notion of $\top\top$-closure 
to metric relations 
and talk about the distance of terms, see the work of Ugo and Raphaëlle
(and more)

Finally, the study of mixed non-determinism and probability is 
done very briefly in this paper, and there would be a lot more 
to talk about. For instance, how does the functional representation 
evolves when combining angelic,demonic and probability operators ?
Can our result be safely transported into a realisability world 
such as done in the work by Niels ?
