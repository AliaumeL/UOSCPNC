\section{Migrating material}

It is worth noticing that the lifting operation $f \mapsto \hat{f}$ is
a continuous functional. A consequence of this is the following monotonicity 
property of the lifting.

\begin{alemma}[Order preserving lift]
    \label{lem:orderpreservinglift}
    If $\sigma_1 : \Nat \to A$ and $\sigma_2 : \Nat \to A$ 
    are two functions such that $\sigma_1 \leq \sigma_2$ pointwise,
    then $\hat{\sigma_1} \leq \hat{\sigma_2}$ pointwise.
\end{alemma}

We can already see the usefulness of this definition 
when looking at the construction of substitution:
given abstractly, in full generality and 
guaranteed to be continuous without any hassle. 

\begin{adefinition}[Substitution]
    Let $t \in \textnormal{Tree}_X$ 
    and $\sigma : X \to \textnormal{Tree}_Y$.
    Using the universal property, there exists a
    unique lift $\hat{\sigma} : \textnormal{Tree}_X \to \textnormal{Tree}_Y$,
    which is the substitution. We write $t\sigma$ as a shorthand for 
    $\hat{\sigma}t$.
\end{adefinition}


Using this new construction it is now possible  
to continue our definition of the operational semantics without 
interpreting effects by building an infinite tree
labeled by the effects encountered during evaluation.



% 