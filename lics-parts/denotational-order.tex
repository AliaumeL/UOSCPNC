\section{Denotationally-defined preorders}

Our second approach to defining an admissible and compositional basic operational
preorder $\Basicleq$ on $\Trees(\mathbb{N})$ is to make use of established constructions from domain theory.
Under this approach, admissibility and compositionality of the defined preorder $\Basicleq$ hold automatically,
for general reasons. Since this approach essentially amounts to giving a denotational semantics to effect trees, we call it the 
\emph{denotational} method of defining a basic operational preorder.

Let $D$ be a continuous $\Sigma$-algebra (see Section~\ref{section:trees}) with a distinguished function 
$j\colon \mathbb{N} \to D$. Let   $\llbracket \cdot \rrbracket \colon \Trees(\mathbb{N}) \to D$ be the unique continuous homomorphism
making the diagram below commute.
   \begin{center}
        \begin{tikzcd}
            \mathbb{N}
            \arrow[r, "j"] 
            \arrow[d, hook, "i"]
            & D \\
            \Trees(\mathbb{N}) \arrow[ru, dashrightarrow, "\llbracket \cdot \rrbracket" below]
        \end{tikzcd}
    \end{center}
\noindent
We use the map $\llbracket \cdot \rrbracket \colon \Trees(\mathbb{N}) \to D$ to define 
the basic operational preorder $\Basicleq_D$ using the partial order relation on the $\omega$CPO $D$.
\[
t \Basicleq_D t' ~~ \Leftrightarrow ~~ \Sem{t} \sqsubseteq \Sem{t' } \enspace .
\]
\begin{aproposition}
The relation $\Basicleq_D$ is always an admissible pregongruence.
\end{aproposition}

[[PROOF]]

\begin{adefinition}[Factorisation property]
    The map $j\colon \mathbb{N} \to D$ is said to have  the \emph{factorisation property} if,
    for every function $f \colon \mathbb{N} \to D$, there exists a 
    continuous homomorphism $h_f : D \to D$ such that $f = h_f \circ j$.
    \begin{center}
        \begin{tikzcd}
            \mathbb{N} \arrow[r, "j"] 
                 \arrow[rr, bend right, "h"] &
            D \arrow[r, "h_f", dashed] & 
            D  
        \end{tikzcd}
    \end{center}
\end{adefinition}
\begin{aproposition}
If $j\colon \mathbb{N} \to D$ has the factorisation property then 
the relation $\Basicleq_D$ is substitutive, hence it is an admissible compositional precongruence.
\end{aproposition}

[[PROOF]]

\begin{alemma}
Let $\mathcal{A}$ be any category given together with a functor $U' \colon \mathcal{A} \to \ContAlg_\Sigma$.
Suppose it holds that that the
composite functor $UU' : \mathcal{A} \to \Set$ has a left adjoint $F$, where $U: \ContAlg_\Sigma \to \Set$ is the forgetful functor.
Then, defining $D$ to be the continuous $\Sigma$-algebra $U' F \, \mathbb{N}$, the unit of the adjunction defines a function
$j \colon \mathbb{N} \to D$ which has the factorisation property.
\end{alemma}
\begin{alemma}
Let $T$ be a monad on the category $\wCPO$.  Suppose that for every $\omega$CPO $X$, we have a 
continuous $\Sigma$-algebra structure on TX. Suppose also that all Kleisli liftings of 
$f^* \colon TX \to TY$, of maps $f \colon X \to TY$, are  continuous  homomorphisms. Then,
defining $D$ to be the continuous $\Sigma$-algebra $T \mathbb{N}$, the unit of the adjunction defines a function
$j \colon \mathbb{N} \to D$ which has the factorisation property.
\end{alemma}

