\section{Denotationally-defined preorders}

Our second approach to defining an admissible and compositional basic operational
preorder $\Basicleq$ on $\Trees(\mathbb{N})$ is to make use of established constructions from domain theory.
Under this approach, admissibility and compositionality of the defined preorder $\Basicleq$ hold automatically,
for general reasons. Since this approach essentially amounts to giving a denotational semantics to effect trees, we call it the 
\emph{denotational} method of defining a basic operational preorder.


