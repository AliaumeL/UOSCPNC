\section{Logical Relation}

This section consists in defining a \emph{relational interpretation}
of types by induction on their structure, and proving that 
this relational interpretation characterises the contextual 
preorder $\sqeq_{ctx}$. 

This is a method that goes back to \cite{Reynolds83} 
(see \cite{wadler1989theorems}) [TODO CITE FROM PITTS] where 
instead of interpreting terms as elements of 
a set, and types as sets, terms are left 
uninterpreted, and types are interpreted as \emph{relations}.

To benefit from theses results, one need to parametrize 
over a class of \emph{well-behaved} relations \cite{Pitts2000} 
and it turns out that \emph{biorthogonality} 
is a generic construction that allows to construct them 
\cite{mellies2005recursive}.

\vspace{1em}

In the following section, we will fix a preorder $\sqeq_b$
and assume it admissible and compositional. This 
preorder can be used to define an antitone Galois Connection 
written $\top$ between relations on stacks and relations on closed 
computation terms, meaning 
that if $r$ is a relation on closed computation terms of type $\tau$ and $s$ is 
a relation on stacks of type $\tau \multimap \Nat$:

\begin{equation*}
    r^\top \subseteq s \iff s^\top \subseteq r
\end{equation*}

\begin{adefinition}[The $\top$ operation]
    Let $s$ be a relation on stacks of type $\tau \to \Nat$ 
    and $r$ be a relation on closed computation terms 
    of type $\tau$:

    \begin{equation*}
        \begin{array}{rl}
            (S,S') \in r^\top &\iff
            \forall (M,M') \in r, |S,M| \sqeq_b |S',M'|\\ 
            (M,M') \in s^\top &\iff
            \forall (S,S') \in s, |S,M| \sqeq_b |S',M'|
        \end{array}
    \end{equation*}
\end{adefinition}

The main interpretation is that it is possible to 
relate stacks 
when we apply them to closed computation terms we can relate to each other and 
we can relate closed computation terms when we apply them to stacks we can relate to
each other. 
It can be intuited that a relation $r$ satisfying $r = r^{\top\top}$ going 
to be a relation preserving observational equivalence 
in some way, which is exactly the kind of relation 
we are looking for.
In fact, the function $r \mapsto r^{\top\top}$ is a closure operator
and is going to be used in the definition 
of the logical relation to guarantee its adequacy and compatibility.


Because the $\top$ operation is an antitone Galois Connection, 
all the properties in Figure \ref{fig:galois} are automatically true.


\begin{figure}[h]
        \begin{equation*}
            r^{\top\top\top} = r^\top
        \end{equation*}
        \begin{equation*}
            r \subseteq s \implies s^\top \subseteq r^\top
        \end{equation*}
        \begin{equation*}
            r \subseteq r^{\top\top}
        \end{equation*}
        \begin{equation*}
            (r^{\top\top})^{\top\top} = r^{\top\top}
        \end{equation*}
        \begin{equation*}
            r \subseteq s \implies r^{\top\top} \subseteq s^{\top\top}
        \end{equation*}
    \caption{Properties of the $\top$ operation}
    \label{fig:galois}
\end{figure}


\begin{alemma}[Saturation for $\top\top$-closed relations and $\sqeq_{ctx}$]
    \label{lem:saturation}
    Let $r$ be a $\top\top$-closed relation on 
    computations, we always have:
    \begin{equation*}
        (\sqeq_{ctx} r \sqeq_{ctx}) \subseteq r 
    \end{equation*}
\end{alemma}

This closure operator is going to be used in the definition 
of the logical relation to guarantee it's adequacy and compatibility.

\subsection{Definition of the relation}

We combine ideas from \cite{gom} 
for the treatment of effects and 
\cite{pitts1998existential} for the 
adaptation to the call-by-value setting.
The evaluation strategy can also be 
seen in \cite{dagand2015normalization}
with an equivalent definition and 
interesting description of the
actual way things are going to compute.

As usual, we define an operation on relations on terms 
for each type constructor. Because there is only one 
constructor it suffices to define what is the \emph{arrow}
relation between two relations.

\begin{adefinition}[Arrow relation]
    Let $r_1$ be a relation on values 
    and $r_2$ be a relation on computations 
    of type respectively $\tau_1$ and $\tau_2$, we define 
    $r_1 \to r_2$ a relation on 
    values of type $\tau_1 \to \tau_2$ as:

    \begin{equation*}
        r_1 \to r_2 = 
        \left\{ 
            (V,V')  
            ~|~
            \forall (W,W') \in r_1, 
            (VW, V'W') \in r_2 
        \right\}
    \end{equation*}

    Note that values of type $\tau_1 \to \tau_2$
    are all of the form $\lambda x:\tau_1. M$
    where $x:\tau_1 \vdash_C M : \tau_2$.
\end{adefinition}


It is now possible to write in a very simple way the logical 
relation on closed terms, which is going to be automatically $\top\top$-closed 
on computation terms at any type.

\begin{adefinition}[Logical relation on closed terms]
    The logical relation on closed terms is defined in
    Figure \ref{fig:logicalrel}. For every type 
    $\tau$ it defines a relation $\| \tau \|_V$ 
    on values of type $\tau$ and $\| \tau \|_C$ 
    on computations of type $\tau$.
\end{adefinition}

\begin{figure}[h]
    \begin{align*}
        \| \Nat \|_V &= \  \sqeq_b \\
        \| \tau \to \tau' \|_V &= \| \tau \|_V \to \| \tau' \|_C \\
        \| \tau \|_C &= \left\{ (\Ret V, \Ret V') ~|~ (V,V') \in \|\tau\|_V
        \right\}^{\top\top}
    \end{align*}
    \caption{Logical relation}
    \label{fig:logicalrel}
\end{figure}

If one of the key results in 
a logical relation argument is the reflexivity of the relation, 
it is very common to have a similar reflexivity result on stacks \cite{Pitts2000}.
We are going to use several times the fact that $(Id,Id) \in \| \Nat \|_C^\top$
which is a consequence of the stack reflexivity at ground type $\Nat$.

\begin{alemma}[Stack reflexivity at ground type $\Nat$]
    \label{lem:stackrefl}
    For any stack $\vdash S : \Nat \multimap \Nat$ the pair $(S,S)$ is inside $\| \Nat \|_C^\top$
\end{alemma}

\begin{proof}
    Let $S$ be a stack such that $\vdash S : \Nat \multimap \Nat$. Let 
    $V$ and $V'$ be two values of type $\Nat$ such that $V \sqeq_b V'$.
    Using the stack commutation lemma \ref{lem:stackcom}, $|S, \Ret V| = |\Ret
    V| \sigma_S$. But we know that $|\Ret V| = V$, and now using
    compositionality and reflexivity of $\sqeq_b$ it can be shown 
    that $|S, \Ret V| \sqeq_b |S, \Ret V'|$. This proves that $(S,S) \in \| \Nat
    \|_C^\top$ because:

    \begin{equation*}
        \| \Nat \|_C^\top = \left\{ (\Ret V, \Ret V') ~|~ (V,V') \in \| \Nat \|_V
        \right\}^\top
    \end{equation*}
\qed\end{proof}

Now in order to compare the logical relation to the contextual preorder,
it is needed to be able to relate \emph{open} terms. The usual 
open extension of a relation is used, while being careful to only 
substitute \emph{value} terms for the free variables.

\begin{adefinition}[Generalisation to open terms]
    If $M$ and $M'$ are two open terms with variables 
    typed by $\Gamma$ then $\Gamma \vdash M \, \| \tau \|_C \, M'$ if 
    and only if for any pair of substitutions $\vec{V}$ and $\vec{V'}$
    for the free variables such that $\vdash V_x \| \tau_x \|_V V_x'$
    for any $x : \tau_x \in \Gamma$ we have:

    \begin{equation*}
        \vdash M[ x := V_x ] \, \| \tau \|_C M' [ x := V_x' ]
    \end{equation*}

    The definition can be adapted to value terms in the obvious way.
\end{adefinition}


One direct consequence of this defininition as an open extension is 
that two related computation terms with one free variable can be used 
to extend a stack.

\begin{alemma}[Stack extension]
    \label{lem:stackexten}
    Let $(S,S') \in \| \tau \|_C^\top$ be two related stacks,
    and $x: \sigma \vdash M \| \tau \|_C M'$. One can construct 
    new stacks $S \circ (\Bind{\square}{x:\sigma}{M})$
    and $S' \circ (\Bind{\square}{x:\sigma}{M'})$ that 
    have type $\sigma \multimap \Nat$. Theses two new 
    stacks are related for $\| \sigma \|_C^\top$.
\end{alemma}

\begin{proof}
    Let $(V,V') \in \| \sigma \|_V$, we can use the definition 
    of the computation tree to reduce the stack:

    \begin{equation*}
        | S \circ (\Bind{\square}{x:\sigma}{M}, \Ret V| 
            = 
        | S, M[x := V]|
    \end{equation*}

    This equation is also valid for the second computation 
    tree built using $S'$, $M'$ and $V'$.

    But we know that $\vdash M[x := V] \| \tau \|_C M'[x := V']$
    by definition of the open extension of the relation. This proves 
    that $|S, M[x:= V]| \sqeq_b |S', M'[x:= V']|$ and therefore 
    that the two stacks are indeed related for $\| \sigma \|_C^\top$.
\qed\end{proof}

\subsection{Inclusion in contextual preorder}

The first step is to prove a soundness result, mainly that 
if two terms are logically related then they are contextually 
related. This soundness is proven in two steps: first prove 
adequacy of the logical relation, then prove compatibility 
with the constructions of the language.

\begin{alemma}[Adequacy]
    The logical relation is adequate.
\end{alemma}

\begin{proof}
    Let $M$ and $M'$ be two computation terms of type $\Nat$
    such that $\emptyset \vdash M \| \Nat \|_C M'$. 
    We know that for every pair $(S,S') \in \| \Nat \|_C^\top$
    we have:

    \begin{equation*}
        | S, M | \sqeq_b |S',M'|
    \end{equation*}

    Therefore it suffices to show that $(Id,Id) \in \| \Nat \|_C^\top$ 
    to conclude, but this is a direct consequence of lemma \ref{lem:stackrefl}.
\qed\end{proof}

We are now going to prove compatibility of the logical relation with 
the different constructions of the language. To simplify the proofs, 
compatibility is proven in an empty typing context $\Gamma$, this is allowed 
because the logical relation is defined on open terms using closing 
substitutions.

\begin{alemma}[Compatibility for variables]
    If $\Gamma$ is a typing context and $x$ a variable name then:
    \begin{equation*}
        \Gamma, x:\tau \vdash x \| \tau \|_V x 
    \end{equation*}
\end{alemma}

\begin{proof}
    The compatibility property holds by definition of the open extension
\qed\end{proof}

\begin{alemma}[Compatibility for $\Ret$]
    If $\Gamma$ is a typing context and $\Gamma \vdash V \| \tau \|_V V'$ then:
    \begin{equation*}
        \Gamma \vdash \Ret V \| \tau \|_C \Ret V
    \end{equation*}
\end{alemma}

\begin{proof}
    The compatibility property holds because the $\top\top$ closure 
    is a monotone operator
\qed\end{proof}

\begin{alemma}[Compatibility for function application]
    If $\Gamma$ is a typing context,
    $\Gamma \vdash V \| \tau \to \tau' \|_V V'$ 
    and $\Gamma \vdash W \| \tau \|_V W'$:

    \begin{equation*}
        \Gamma \vdash VW \| \tau' \|_C V'W'
    \end{equation*}
\end{alemma}

\begin{proof}
    The compatibility property holds by defintion of the arrow relation.
\qed\end{proof}

\begin{alemma}[Compatibility for $\Zero$]
    If $\Gamma$ is a typing context then:
    \begin{equation*}
        \Gamma \vdash \Zero \| \Nat \|_V \Zero 
    \end{equation*}
\end{alemma}

\begin{proof}
The compatibility property holds 
because $\sqeq_b$ is reflexive. 
\qed\end{proof}

\begin{alemma}[Compatibility for $\Succ$]
    If $\Gamma$ is a typing context and $\Gamma \vdash V \| \Nat \|_V V'$. 
    then:
    \begin{equation*}
        \Gamma \vdash \Succ V \| \Nat \|_V \Succ V'
    \end{equation*}
\end{alemma}
\begin{proof}
The compatibility property
holds because of lemma \ref{lem:admcompnat}. Indeed, 
when two natural numbers are related, either they 
are equal, and then their successors are also related 
because $\sqeq_b$ is reflexive, or they are not 
equal and every number is equated by $\sqeq_b$ 
making the property trivial. 
\qed\end{proof}

\begin{alemma}[Compatibility for $\lambda$-abstraction]
    If $\Gamma$ is a tying context and 
    $\Gamma, x :\tau \vdash M \| \tau' \|_C M'$ then:
    \begin{equation*}
        \Gamma \vdash (\lambda x:\tau. M) \| \tau \to \tau' \|_V (\lambda
        x:\tau. M')
    \end{equation*}
\end{alemma}
\begin{proof}
Assume without loss of generality 
that the computation terms 
$M$ and $M'$ only have one free variable $x$,
and that $x : \sigma \vdash M \| \tau \|_C M'$.

We want to prove that $(\lambda x:\sigma .M, \lambda x:\sigma. M')$
is in $\| \sigma \to \tau\|_V$. Let $(V,V') \in \| \sigma \|_V$,
we have to prove that $((\lambda x:\sigma. M)V, (\lambda x:\sigma.
M')V')$ is in $\| \tau \|_C$.

But using the hypothesis and 
the definition of the open closure, 
we already know that 
$(M[x := V], M'[x := V']) \in \| \tau \|_C$.

Let $(S,S') \in \| \tau \|_C^\top$, we know that 
$|S, M[x := V]| \sqeq_b |S', M'[x := V']|$ by definition 
of the $\top\top$-closure. 

It suffices to see that $|S, M[x := V]| = |S, (\lambda x:\sigma. M)
V|$ by definition of $|-,-|$ to have the conclusion:

\begin{equation*}
    | S, (\lambda x:\sigma. M) V| \sqeq_b |S', (\lambda x:\sigma.
    M') V'| 
\end{equation*}

Therefore we do have $(\lambda x:\sigma. M, \lambda x:\sigma. M')
\in \| \sigma \to \tau \|_V$.
\qed\end{proof}

\begin{alemma}[Compatibility for computation binding]
    If $\Gamma$ is a typing context,
    $\Gamma \vdash M \| \tau \|_C M'$ and 
    $\Gamma, x : \tau \vdash N \| \tau' \|_C M'$
    then:

    \begin{equation*}
        \Gamma \vdash (\Bind{M}{x:\tau}{N}) \| \tau' \|_C 
        (\Bind{M'}{x:\tau}{N'})
    \end{equation*}
\end{alemma}
\begin{proof}
Let $M = \Bind{M_1}{x:\sigma}{M_2}$,
$M' = \Bind{M_1'}{x:\sigma}{M_2'}$ and assume 
that $\emptyset \vdash M_1 \| \sigma \|_C M_1'$
and $x : \sigma \vdash M_2 \| \tau \|_C M_2'$.

Let $(S,S') \in \| \tau \|_C^\top$, we know that

\begin{equation*}
    \begin{cases}
        | S, M | &=
        | S \circ \Bind{\square}{x : \sigma}{M_2}, M_1| \\
        | S', M' | &=
        | S' \circ \Bind{\square}{x : \sigma}{M_2'}, M_1'| \\
    \end{cases}
\end{equation*}

Therefore to prove the inequality for $\sqeq_b$ it suffices 
to show that

\begin{equation*}
    (S \circ \Bind{\square}{x : \sigma}{M_2},
     S' \circ \Bind{\square}{x : \sigma}{M_2'})
     \in \| \sigma \|_C^\top
\end{equation*}

But because of the properties of the 
$\top\top$-closure we know that 
$\| \sigma \|_C^\top = (\Ret \| \sigma \|_V)^\top$.

Let $(V,V') \in \| \sigma \|_V$, we can prove that 
applying the stacks to $(\Ret V, \Ret V')$ gives 
related trees, by using the fact that $M_2$ and $M_2'$
are related and that $(S,S') \in \| \tau \|_C^\top$.

\begin{align*}
    | S \circ \Bind{\square}{x : \sigma}{M_2}, \Ret V| 
    &= |S,  M_2[ x := V] | \\
    & \sqeq_b |S', M_2' [ x := V'] | \\
    &= 
    | S' \circ \Bind{\square}{x : \sigma}{M_2'}, \Ret V'| 
\end{align*}

Therefore we have the expected conclusion.
\qed\end{proof}

\begin{alemma}[Compatibility for the case distinction]
    If $\Gamma$ is a typing context, 
    $\Gamma \vdash V \| \Nat \|_V V'$,
    $\Gamma \vdash M_1 \| \tau \|_C M_1'$
    and $\Gamma, x:\Nat \vdash M_2 \| \tau \|_C M_2'$
    then:

    \begin{equation*}
        \Gamma \vdash (\lcase{V}{M_1}{M_2}) \| \tau \|_C
            (\lcase{V'}{M_1'}{M_2'})
    \end{equation*}
\end{alemma}
\begin{proof}
Let $M = \lcase{V_1}{M_2}{M_3}$ and $M' = \lcase{V_1'}{M_2'}{M_3'}$,
and assume without loss of generality that the terms are closed. 
If $\emptyset \vdash V_1 \| \Nat \|_V V_1'$, 
$\emptyset \vdash M_2 \| \tau \|_C M_2'$
and $ x: \Nat \vdash M_3 \| \tau \|_C M_3'$ 
then we are going to show that $\emptyset \vdash M \| \tau \|_C M'$.
Let $(S,S') \in \| \tau \|_C^\top$.

We can make a case distinction on $V_1$:
\begin{itemize}
    \item When $V_1 = \Zero$ then $|S,M| = |S,M_2|$,
        but because of lemma \ref{lem:admcompnat} 
        we can assume that $V_1 = V_2$ and therefore 
        $|S',M'| = |S', M_2'|$. But by hypothesis
        we know that $(M_2,M_2') \in \| \tau \|_C$ and therefore
        we have $| S, M| \sqeq_b |S',M'|$. When in the second 
        case of the lemma \ref{lem:admcompnat} 
        the result is obvious.

    \item When $V_1 = \Succ V'$, we have the same 
        disjunction as in the previous case, and 
        the same conclusion, except that we use 
        the open closure.
\end{itemize}
\qed\end{proof}



\begin{alemma}[Compatibility for the fixed-point construction]
    If $\Gamma$ is a typing context and $\Gamma \vdash V \| (\tau \to \tau') \to
\tau \to \tau' \|_V V'$ then:
    \begin{equation*}
        \Gamma \vdash \Fix V \| \tau \to \tau' \|_C \Fix V'
    \end{equation*}
\end{alemma}
\begin{proof}
Assume that $\emptyset \vdash V \| (\sigma \to \tau) \to \sigma \to
\tau \|_V V'$. Let $(S,S') \in \| \sigma \to \tau\|_C^\top$.
Using the unrolling theorem \ref{thm:unrolling} we know that:

\begin{equation*}
    \begin{cases}
        |S, \Fix V| &= \bigsqcup_i | S, \Unroll_i V |     \\
        |S', \Fix V'| &= \bigsqcup_i | S', \Unroll_i V' |
    \end{cases}
\end{equation*}

Using admissibility of $\sqeq_b$, we know that it suffices 
to show that $|S, \Unroll_i V| \sqeq_b |S', \Unroll_i V'|$
when $i \in \mathbb{N}$. This can be 
understood as a proof by fixed point induction, and we are 
going to prove that $(\Unroll_i V, \Unroll_i V') \in \| \sigma \to \tau \|_C$
which will give the expected conclusion.

\begin{itemize}
    \item We know that $\Unroll_0 V = \Unroll_0 V'$ and 
        that for every stack they give the tree $\bot$ 
        because they don't terminate. Therefore 
        we have obviously $(\Unroll_0 V, \Unroll_0 V') \in \|
        \sigma \to \tau \|_C$.

    \item Suppose that $(\Unroll_i V, \Unroll_i V') \in \| \sigma
        \to \tau \|_C$. By definition, we know that:

        \begin{equation*}
            \begin{cases}
                \Unroll_{i+1} V &= V (\lambda x:\sigma.
                        \Bind{(\Unroll_i V)}{g : \sigma \to \tau}{g
                    x}) \\
                \Unroll_{i+1} V' &= V' (\lambda x:\sigma.
                        \Bind{(\Unroll_i V')}{g : \sigma \to \tau}{g
                    x}) \\
            \end{cases}
        \end{equation*}

        But we can use the previous compatibility propeties 
        to simplify this proof. By application and variable 
        compatibility we know that 
        
        \begin{equation*}
            g : \sigma \to \tau, x : \sigma \vdash g x \| \tau \|_C g x
        \end{equation*}

        Then using a binding construction, because we know that
        $\Unroll_i V$ and $\Unroll_i V'$ are related we can derive:
        
        \begin{equation*}
            x : \sigma \vdash (\Bind{\Unroll_i V}{g : \sigma \to
                \tau}{g x}) \| \tau \|_C (\Bind{\Unroll_i V'}{g : \sigma
            \to \tau}{g x})
        \end{equation*}
        
        Then we use the lambda-abstraction compatibility to 
        relate the two functions terms.
        Afterwards using the fact that $V$ and $V'$ are related 
        and again function application we can derive:

        \begin{equation*}
            \emptyset \vdash (\Unroll_{i+1} V) \| \tau \|_C
            (\Unroll_{i+1} V')
        \end{equation*}

\end{itemize}
\qed\end{proof}

\begin{alemma}[Compatibility for the effect construction]
    If $\Gamma$ is a typing context, $(\sigma,n) \in \Sigma$
    and $\Gamma \vdash M_i \| \tau \|_C M_i'$ for $i \in \{ 1, \dots, n\}$
    then:
    \begin{equation*}
        \Gamma \vdash \sigma(M_1, \dots, M_n) \| \tau \|_C \sigma(M_1', \dots,
        M_n')
    \end{equation*}
\end{alemma}
\begin{proof}
        Let $M = \sigma (M_1, \dots, M_n)$ 
        and $M' = \sigma (M_1', \dots, M_n')$
        such that $(M_i,M_i') \in \| \tau \|_C$.

        Let $(S,S') \in \| \tau \|_C^\top$.

        \begin{equation*}
            \begin{cases}
                |S,M|   &= \sigma (|S,M_1|, \dots |S,M_n|) \\
                |S',M'| &= \sigma (|S',M_1'|, \dots |S',M_n'|) \\
            \end{cases}
        \end{equation*}

        But we know that $\sqeq_b$ is reflexive, and 
        therefore
        \begin{equation*}
            \sigma(\underline{1}, \dots, \underline{n})
            \sqeq_b \sigma(\underline{1}, \dots, \underline{n})
        \end{equation*}

        The hypothesis tells us that $|S,M_i| \sqeq_b |S',M_i'|$
        for every $1 \leq i \leq n$. We can then design
        two substitutions (the $\bullet$ represents either 
        nothing or a quote):

        \begin{equation*}
            \tau^\bullet (i) = \begin{cases}
                |S^\bullet, M_i^\bullet| & \text{ when } 1 \leq i \leq n \\
                \underline{i}            & \text{ otherwise } 
            \end{cases}
        \end{equation*}

        We know that $\tau \sqeq_b \tau'$ pointwise, and therefore
        by compositionality of $\sqeq_b$ we can deduce:

        \begin{equation*}
            \sigma (|S,M_1|, \dots, |S,M_n|)
            = \sigma(\underline{0}, \dots, \underline{n}) \tau
            \sqeq_b
             \sigma(\underline{0}, \dots, \underline{n}) \tau'
            =
            \sigma (|S',M_1'|, \dots, |S',M_n'|)
        \end{equation*}
\qed\end{proof}

We therefore have the compatibility with respect to the language constructions, 
allowing us to state the inclusion of the logical relation inside the contextual
preorder.

\begin{atheorem}[Inclusion of the preorders]
    \label{thm:inclusionpreorder}
    The logical relation is included in the contextual preorder.
\end{atheorem}

\begin{proof}
    The logical relation is adequate and compatible, and 
    therefore is included in the largest relation that 
    is adequate and compatible.
\qed\end{proof}

\subsection{Equality with contextual preorder}

Because the language is call-by-value, the logical 
relation satisfies a very strong property that 
can be found in \cite{pitts1998existential} in a slightly different form:
the relation can be recovered from its restriction to values.
Separating values from computation makes this result easier to 
prove and to grasp as seen in lemma \ref{lem:valuerelation}.
This result is necessary to prove completeness of the logical 
relation with respect to contextual equivalence, and can easily 
be extended when adding new types to the language like sum types 
or product types.

\begin{alemma}[Value relation]
    \label{lem:valuerelation}
    For any type $\tau$ we have 

    \begin{equation*}
        (\Ret V, \Ret V') \in \| \tau \|_C
        \iff 
        (V,V') \in \| \tau \|_V
    \end{equation*}
\end{alemma}

\begin{proof}
    The right-to-left implication is a consequence of the definition 
    and the monotonicity of the $\top\top$-closure.
    The other implication can be proven by case analysis on $\tau$.

    \begin{itemize}
        \item When $\tau = \Nat$. Let $(\Ret V, \Ret V') \in \| \Nat \|_C$,
            using lemma \ref{lem:stackrefl} we know that $(Id,Id) \in \| \Nat
            \|_C^\top$ and therefore that $|\Ret V| \sqeq_b |\Ret V'|$, thus 
            proving the result.
        
        \item When $\tau = \tau_1 \to \tau_2$.
            Assume that $(\Ret V, \Ret V') \in 
            \| \tau_1 \to \tau_2 \|_C$.

            Let $(W,W') \in \| \tau_1 \|_V$, $(S,S') \in \| \tau_2 \|_C^\top$.
            We want to prove $|S, VW| \sqeq_b |S',V'W'|$, because this would
            imply that $(VW,V'W')$ is an element of $\| \tau_2 \|_C^{\top\top}$
            and therefore that 
            $(V,V')$ is in $\| \tau_1 \to \tau_2 \|_V$.

            It suffices to notice that $x: \tau_1 \to \tau_2 \vdash xW \| \tau_2
            \|_C x W'$. Indeed, this can be proven using compatibility
            properties. Now using lemma \ref{lem:stackexten} we can extend the
            stacks $S$ and $S'$ into the stacks $S \circ
            (\Bind{\square}{x:\tau_1 \to \tau_2}{xW})$ and $S' \circ
            (\Bind{\square}{x:\tau_1 \to \tau_2}{xW'})$. The lemma 
            states that the two new stacks are related for $\| \tau_1 \to \tau_2
            \|_C^\top$, and therefore:

            \begin{align*}
                |S \circ \Bind{\square}{x:\tau_1 \to \tau_2}{xW}, \Ret V| &\sqeq_b \\
                |S' \circ \Bind{\square}{x:\tau_1 \to \tau_2}{xW'}, \Ret V'| 
            \end{align*}

            Which after reduction gives exactly the expected inequality.

    \end{itemize}
\qed\end{proof}


\begin{alemma}[Largest adequate compatible and substitutive relation]
    The logical relation is the largest adequate, compatible 
    and substitutive relation. Where being substitutive is 
    being compatible with the rules in Figure \ref{fig:substitutive}.
\end{alemma}

\begin{proof}
    First of all, it is clear that the logical relation is substitutive 
    by definition as an open extension.

    Let $\CE$ be an adequate compatible and substitutive relation.
    By definition of the contextual preorder, we know that $\CE$
    is included into the contextual preorder. 

    Let $\Gamma \vdash M \CE_C M' : \tau$, we are going 
    to prove that $\Gamma \vdash M \| \tau \|_C M'$. 
    
    Let $\vec{V}$ and $\vec{V'}$ be
    sequences of values related for the logical relation and
    types agreeing with $\Gamma$. 

    Because the logical relation is reflexive, we know that 
    we have $\Gamma \vdash M \| \tau \|_C M$, and using the
    fact that the logical relation is substitutive, we then 
    can deduce that

    \begin{equation*}
        \emptyset \vdash M[\vec{\Gamma} := \vec{V}] \| \tau \|_C 
                         M[\vec{\Gamma} := \vec{V'}]
    \end{equation*}
    
    On the other hand, we know that $\CE$ is reflexive 
    and therefore $\vec{V'}$ is pointwise related 
    to $\vec{V'}$ for this relation. We can therefore 
    conclude using substitutivity:

    \begin{equation*}
        \emptyset \vdash M[\vec{\Gamma} := \vec{V'}] 
        \CE_C
                         M'[\vec{\Gamma} := \vec{V'}]
    \end{equation*}

    But we know that the relation $\CE$ is included into 
    the contextual preorder, and therefore the 
    equation can be written:

    \begin{equation*}
        M[\vec{\Gamma} := \vec{V}] 
        \| \tau \|_C
        M[\vec{\Gamma} := \vec{V'}]
        \sqeq_{ctx}
        M'[\vec{\Gamma} := \vec{V'}]
    \end{equation*}

    Using the saturation lemma (lemma \ref{lem:saturation})
    we can then deduce that for every pair of
    logically related $(\vec{V},\vec{V'})$ we have:

    \begin{equation*}
        \emptyset \vdash M[\vec{\Gamma} := \vec{V}] \| \tau \|_C
        M'[\vec{\Gamma} := \vec{V'}] 
    \end{equation*}

    Which by definition means that $\Gamma \vdash M \| \tau \|_C M'$. 
    We now have proven that:


    \begin{equation*}
        \Gamma \vdash M \CE_C M' : \tau \implies \Gamma \vdash M
    \| \tau \|_C M'
    \end{equation*}
    
    Therefore to prove that $\CE$ is included in the logical 
    relation it suffices to prove the same for value terms.

    Let $\Gamma \vdash V \CE_V V' : \tau$, because the relation is compatible 
    we know that $\Gamma \vdash \Ret V \CE_C \Ret V' : \tau$, using the previous 
    result we know that $\Gamma \vdash \Ret V \| \tau \|_C \Ret V'$, but
    using lemma \ref{lem:valuerelation} we can deduce that $\Gamma \vdash V \|
    \tau \|_V V'$.

\qed\end{proof}

\begin{figure}[h]
    \begin{center}
        \AxiomC{$\Gamma, x : \tau \vdash W \CE_V W' : \tau'$}
        \AxiomC{$\Gamma \vdash V \CE_V V' : \tau$}
        \BinaryInfC{$\Gamma \vdash W[x := V] \CE_C W'[x := V'] : \tau'$}
        \DisplayProof
    \end{center}
    \caption{Substitutivity}
    \label{fig:substitutive}
\end{figure}

\begin{alemma}[Contextual preorder is substitutive]
\end{alemma}

\begin{proof}
    Because contextual preorder is transitive, it suffices 
    to show that we have the following equality when 
    $M$ is a computation term of type $\tau$ and $V$
    a value term of type $\sigma$ to have the 
    substitutive property on computation terms:

    \begin{equation*}
        \Gamma \vdash M[ x := V] \equiv_{ctx} (\lambda x:\sigma. M) V : \tau
    \end{equation*}

    But we have already seen that this equivalence is true for the logical 
    relation, and we know that the logical relation is included in 
    the contextual preorder, allowing us to conclude.
\qed\end{proof}


\begin{atheorem}[Contextual preorder equals the logical relation]
\end{atheorem}

\begin{proof}
    We already have one inclusion with the theorem \ref{thm:inclusionpreorder}
    and the second one is given because 
    contextual preorder is an adequate, compatible and substitutive relation,
    therefore included into the largest one (the logical relation).
\qed\end{proof}

\subsection{Meta properties}

Using the abstract equality between the contextual preorder
and the logical relation we defined, we can prove generic 
theorems about contextual equivalence independently 
of the properties of effects.

The first one is that a relation between effects is true 
at ground type if and only if they are true at all type. This 
justifies the fact that we only ask for a preorder on $\Tree_\Nat$.

\begin{atheorem}[Inequalities between effects are seen at ground type]
    Meaning that if $M$ and $M'$ are terms with only effects and free 
    variables, a polymorphic inequality between $M$ and $M'$ is 
    admissible if and only if it is admissible on natural numbers. 
\end{atheorem}

\begin{proof}
    Assume that we have $M$ and $M'$ terms constructed 
    only as trees of effect constructors with 
    a finite number of 
    place holders for leaves, and that for any type $\tau$
    we have the following admissible rule:


    \begin{prooftree}
        \AxiomC{$\forall i, \Gamma \vdash M_i \, \| \tau \|_C \, M_i'$}
        \UnaryInfC{$ \Gamma \vdash M[(M_i)] \,  \| \tau \|_C \,  M'[(M_i')]$}
    \end{prooftree}
    
    It is trivial to see that the rule can be seen 
    at type $\Nat$.

    Going the other way around, we only assume that 
    we have the following admissible rule:

    \begin{prooftree}
        \AxiomC{$\forall i, \Gamma \vdash M_i \, \| \Nat \|_C \, M_i'$}
        \UnaryInfC{$ \Gamma \vdash M[(M_i)] \,  \| \Nat \|_C \,  M'[(M_i')]$}
    \end{prooftree}

    Let $\tau$ be a given type, 
    and a set of computation terms 
    of type $\tau$ such that we have 
    derivations of $\Gamma \vdash M_i \, \| \tau \|_C \, M_i'$.

    Let $(S,S') \in \| \tau \|_C^\top$, because 
    $M$ is merely a tree, we can calculate $|S, M[(M_i)]|$:

    \begin{equation*}
        |S, M[(M_i)]| = 
        M[(i)] [ i \mapsto |S,M_i| ]
    \end{equation*}

    Now, using the given admissible rule on $\Ret i$ 
    and compositionality of the preorder we can 
    deduce that:

    \begin{equation*}
        |S, M[(M_i)]| \sqeq_b |S', M'[(M_i')] |
    \end{equation*}
\qed\end{proof}


\begin{atheorem}[Contextual preorder can be reduced to closed terms]
    Two open computation terms $M$ and $M'$ are contextually related 
    if and only if for any closing substitution of contextually related 
    values, the two closed terms obtained are related.

    This is stating that $\sqeq_{ctx}$ is the open extension of 
    $\sqeq_{ctx}$ restricted to closed terms.
\end{atheorem}

\subsection{Importance of hypothesis}

We can now ask ourselves to what extend we could change 
the hypothesis on $\sqeq_b$. The answer is that they 
are in some sense minimal because removing admissibility 
allows to "break" the behaviour of fixed-points, and 
removing compositionality allows to "break" the 
behaviour of effects.

The following results are \emph{not} using the meta-theorems 
that are extensively using compositionality and admissibility 
of $\sqeq_b$.

\begin{alemma}[Admissibility and good behaviour on $\Tree_\Nat$ implies compositionality]
    Namely, we are going to show that if $\sqeq_b$ is admissible then 
    \begin{equation*}
        \left[(M,M') \in \| \Nat \|_C \iff
        |M| \sqeq_b |M'| \right]
        \iff
        \sqeq_b \text{ is compositional }
    \end{equation*}
\end{alemma}

\begin{proof}
    \begin{itemize}
        \item Let's assume that $\sqeq_b$ is compositional. For 
            one direction it suffices to notice that
            $(Id,Id) \in \| \Nat \|_C^\top$ to conclude. For the other 
            direction, assume that $|M| \sqeq_b |M'|$, 
            we know that if $(S,S') \in \| \Nat \|_C^\top$ we have:

            \begin{equation*}
                |S,M| = |M|[i \mapsto |S, \Ret i|]
            \end{equation*}

            Now using compositionality we can conclude that $|S,M| \sqeq_b
            |S',M'|$ and therefore $(M,M') \in \| \Nat \|_C$.

        \item The other way around, we use the fact that 
            every tree is the least upper bound of it's finite approximations,
            and that any finite tree on $\Nat$ has a trivial representation as a 
            finite term\footnote{Note that it still holds if 
            we allow countable branching controled by a lambda-abstraction}.
            
            Without loss of generality (using admissibiltiy) we can 
            therefore restrict to finite trees represented by terms 
            $M,M'$.

            To prove compositionality, we are going to consider 
            two terms $M$ and $M'$ such that $|M| \sqeq_b |M'|$, two substitutions 
            $(\sigma,\sigma')$ such that $\sigma^\bullet (i)$ is always 
            a finite tree, represented by a term $N_i^\bullet$, and 
            $|N_i| \sqeq_b |N_i'|$.

            Compositionality claims that $|M|[ i \mapsto |N_i|] \sqeq_b |M'|[ i
            \mapsto |N_i'|]$, to obtain that result we are going to consider 
            stacks $(S_k,S_k') \in \| \Nat \|_C^\top$ that satisfies:

            \begin{equation*}
                | S_k^\bullet, \Ret i | = 
                \begin{cases}
                    |N_i| & \text{ when } i \leq k \\
                    \bot  & \text{ otherwise } 
                \end{cases}
            \end{equation*}

            Building theses and checking the relation is easy as it is just 
            writing a finite case disjunction and using the non terminating 
            term when we fall in the last case.

            Now, because $\| \Nat \|_C = \| \Nat \|_C^{\top\top}$ we can 
            deduce:
            \begin{equation*}
                |S_k, M| \sqeq_b |S_k', M'|
            \end{equation*}

            But we can rewrite the terms using the substitution lemma
            \ref{lem:stackcom}: 
            \begin{equation*}
                |M|[i \mapsto |S_k, \Ret i|] \sqeq_b |M'|[ i \mapsto |S_k', \Ret
                i|]
            \end{equation*}

            It is clear that the trees obtained are ascending chains in $k$
            with least upper bound $|M^\bullet|[i \mapsto |N_i^\bullet|]$ 
            giving the expected result.

            To adapt this to any tree, it suffices to see that the chain 
            is also ascending in $M$ and $N_i$, therefore allowing for 
            simultaneous approximaton on $k,M,N_i$.
    \end{itemize} 

\qed\end{proof}

