\section{Introduction}

Contextual equivalence in the style of Morris
has imposed itself as a very simple and powerful
way to express what an equivalence on programs should be. 
However
this operational notion of equivalence is rarely usable 
as-is \cite{pitts1997operationally}.

For this reason, many other forms of equivalences have 
been developed in the past years: bisimulations 
and their refinements (environmental bisimulations, 
bisimulations up-to) \cite{koutavas2011applicative}, 
game semantics \cite{abramsky1999game}, 
denotational interpretation in domains \cite{scott1982domains}, 
higher order logic on 
programs \cite{honda2005observationally} 
and logical relations \cite{Pitts2000}.

Unsurprisingly one can observe extreme variations on 
the complexity of such methods when changing  
the class of effects studied. This has to do with 
the fact that most of the operational semantics 
are deeply tied with the effects of the language, 
and adding or removing effects changes the overall 
shape of the semantics. For denotational interpretation,
the problem is less visible because it is transposed 
into domain theoretic constructions, such as powerdomains, distributive laws,
and solving domain equations.

To be able to give uniform results, the first 
step is to restrict the class of effects
to \emph{algebraic} ones, 
allowing to separate the semantics 
of the effects from the semantics of the ground language
itself by using computation trees \cite{plotkin2001adequacy}.

Following the work done by Patricia
Johann, Alex Simpson and Janis Voigtländer \cite{gom}, we then define 
a logical relation to capture the contextual 
equivalence, and apply this result to a wide class of known effects 
such as: non-determinism, probabilities, input-output, and exceptions. 
The main difference is that our approach 
is done in a call-by-value setting, as opposed 
to call-by-name.


We also continue the work done by Plotkin and Power \cite{plotkin2001adequacy}
by using their result to link our setting to the denotational approach. This 
should be considered as an empirical proof of the usefulness of this method. 
Continuing in this direction, we study a concrete example combining
non-determinism and probability, a subject that has been regularly encountered 
in denotational models \cite{tix2009semantic} \cite{JGL-mscs16}
\cite{KeimelP2016} and the study concurrent programs
\cite{Mislove2000} \cite{mislove2004axioms}.
