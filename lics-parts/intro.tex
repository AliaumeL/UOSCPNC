\section{Introduction}

Contextual equivalence in the style of Morris
has imposed itself as a very simple and powerful
way to express what an equivalence on programs should be. 
However
this operational notion of equivalence is rarely usable 
as-is \cite{pitts1997operationally}.

For this reason, many other forms of equivalences have 
been developed in the past years: bisimulations 
and their refinements (environmental bisimulations, 
bisimulations up-to) \cite{koutavas2011applicative}, 
game semantics \cite{abramsky1999game}, 
denotational interpretation in domains \cite{scott1982domains}, 
higher order logic on 
programs \cite{honda2005observationally} 
and logical relations \cite{Pitts2000}.

Unsurprisingly one can observe extreme variations on 
the complexity of such methods when changing  
the class of effects studied. This has to do with 
the fact that most of the operational semantics 
are deeply tied with the effects of the language, 
and adding or removing effects changes the overall 
shape of the semantics. For denotational interpretation,
the problem is less visible because it is transposed 
into domain theoretic constructions, such as powerdomains, distributive laws,
and solving domain equations.

To be able to give uniform results, the first 
step is to restrict the class of effects
to \emph{algebraic} ones, 
allowing to separate the semantics 
of the effects from the semantics of the ground language
itself by using computation trees \cite{plotkin2001adequacy}.


As in  \cite{gom}, where a similar programme was carried out in a call-by-name setting,
our goal is to provide generic mathematical machinery for defining and 
analysing contextual equivalence. 

Following the work done by Patricia
Johann, Alex Simpson and Janis Voigtländer \cite{gom}, we then define 
a logical relation to capture the contextual 
equivalence, and apply this result to a wide class of known effects 
such as: non-determinism, probabilities, input-output, and exceptions. 
The main difference is that our approach 
is done in a call-by-value setting, as opposed 
to call-by-name.

We use a language based on PCF \cite{plotkin1977lcf}, but with  a call-by-value 
evaluation strategy

We also continue the work done by Plotkin and Power \cite{plotkin2001adequacy}
by using their result to link our setting to the denotational approach. This 
should be considered as an empirical proof of the usefulness of this method. 
Continuing in this direction, we study a concrete example combining
non-determinism and probability, a subject that has been regularly encountered 
in denotational models \cite{tix2009semantic} \cite{JGL-mscs16}
\cite{KeimelP2016} and the study concurrent programs
\cite{Mislove2000} \cite{mislove2004axioms}.


In this section we are going to fix a specific signature $\Sigma$
containing two binary operators \texttt{pr} and \texttt{or}. The two
operators are used to model a language where both probabilistic choice 
and non-determinism coexist. Combining these specific effects has been 
the subject of numerous papers, and even when restricting ourselves to the 
denotational setting, the work of Regina Tix on powercones \cite{tix2009semantic} 
continued afterwards by Plotkin and Keimel \cite{KeimelP2016} on Kegelspitze
shows the interest of such combination.
A more functional version of theses domains can also be found in the work of Jean-Goubault Larrecq 
\cite{JGL-mscs16}.



This setting is purposely restricted, and several improvements can 
be added without technical issue. For instance, more complex types 
such as sum, products and even type polymorphism can be studied 
in this context. In fact, logical relations excel in proving parametricity 
results \cite{wadler1989theorems}.
In the spirit of simplicity 
and to allow comparison with the work on bisimulations by
Ugo Dal Lago, Francesco Gavazzo and Paul Blain Levy
\cite{Ugo2017} we take the same kind of effect signature 
as they do. Technically, it means that compared to 
the paper from Johann et al. \cite{gom} it lacks 
three of the four effect constructions, but as noticed 
in the said paper, all of the constructions share the
same pattern of proof, so that they actually treated 
only one of the four cases in their proofs.

